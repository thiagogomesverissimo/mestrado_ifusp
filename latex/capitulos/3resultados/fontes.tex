%%%%
\newpage
\section{Identificação das fontes}

Na aplicação de modelos receptores, os valores faltantes foram preenchidos com 
o valor da metade do limite de detecção (LD/2), pois supõe-se que a concentração
não detectada de um elemento que aparece frequentemente nas demais amostras 
esteja entre 0 e o limite de detecção, com igual probabilidade de ocorrência 
para cada valor. Assim, LD/2 seria a média e, portanto, o valor mais 
provável entre estes valores não detectados. Para a incerteza das concentrações,
usou-se sugestão proposta por \citet{polissar1998} de usar $(5 \cdot LD)/6$.

Até então, foram frutos dessa pesquisa dois artigos publicados: um em 2013, 
intitulado \textit{Chemical composition and sources of particle pollution in 
affluent and poor neighborhoods of Accra, Ghana} \citep{zhou2013}, e outro em 
2014, \textit{Chemical characterization and source apportionment of household 
fine particulate matter in rural, peri-urban, and urban West Africa} 
\citep{zhou2014}, nos quais, realizou-se levantamento de fontes usando PMF com 
todas as 2898 amostras ($MP_{10}$ e $MP_{2,5}$) coletadas nos cinco sítios 
ambientais de amostragem. As fontes de MP que foram associadas aos fatores 
encontrados, com seus respectivos elementos caracterizadores, foram: 
queima de lixo sólido (Br, Pb), poeira de solo e veículo (Al, Si, Ca, Fe, Zn, 
BC, Pb), solo (Al, Si, Mg, Ti, Mn, Fe), queima de biomassa (K, Cl, S, BC) e mar 
(Na, Cl, S). Estes fatores foram encontrados para os 4 bairros (5 sítios de 
coleta) do experimento e, portanto, foram escolhidos a partir de uma análise
comuns às 4 regiões, caracterizando-se como um resultado genérico sobre Acra.

O presente trabalho aprofundou análises, dentro do projeto geral, focalizando o 
bairro de Nima, uma área popular e pobre da capital ganense. Algumas diferenças 
metodológicas distinguem esta outra etapa do estudo. Primeiramente, separamos o 
MP nas frações grossa e fina, ao invés de inalável e fina, como apresentado 
naquelas primeiras publicações. Pretendeu-se desta forma melhor definir grupos 
de fontes, subtraindo os finos do inalável. Ou seja, "descolou-se" estas duas 
frações procurando apoiar-se nos seus distintos processos majoritários de 
geração, para melhor discriminação de fontes.

Por outro lado, além do PMF, utilizou-se a Análise de Fatores para dar maior 
suporte à identificação das fontes principais impactando aquela área. 
Avaliou-se, adicionalmente, o comportamento da circulação atmosférica local e 
como esta poderia apoiar a discriminação das possíveis fontes de MP coletadas 
nos amostradores. Registre-se, por fim, que neste ínterim foi publicado o novo 
censo de Gana, oferecendo atualizações no que diz respeito às fontes de energia
empregadas para preparação de alimentos em Gana e em Acra (tabela 
\ref{table:cookfuel}), abrindo possibilidades de reconsiderar algumas 
interpretações feitas nos primeiros artigos.

%%%%
\subsection{Material Particulado Fino ($MP_{2,5}$) \label{sec:pm2.5}}

Os \textit{loadings} encontrados para Análise de Fatores de $MP_{2,5}$
na área residencial estão na tabela \ref{table:AF_RFsH5} para 5 fatores retidos
e 83,58 \% de variância total explicada em 123 amotras. 
Houvéssemos incluindo os dias de Harmatão, contaríamos com 197 amostras e 
explicaríamos 90,8 \% da variância, mas extraindo praticamente um único fator 
englobando a maioria dos elementos. Sozinho explicaria mais que 57,9 \% da 
variância (tabela \ref{table:AF_RFcH5} do apêndice II).

Com 4 fatores, na tabela \ref{table:AF_RFsH5}, apesar do primeiro fator 
continuar preponderante, vemos uma melhor distribuição dos elementos e da 
variância entre os fatores, permitindo associá-los a fontes regionais.
A variabilidade dos elementos foi bem explicada, obtendo-se comunalidades 
maiores que 0,7 para os elementos, a exceção do bromo (Br) e zinco (Zn) que 
tiveram comunalidade de 0,59 e 0,44, respectivamente.

O fator predominante explica 43,78\% da variância, tem altos \textit{loadings} 
para os metais (Al, Ti, V, Fe, Mn, Ca, Mg), Si e para a massa. Tais elementos 
são comumente encontrados em diversos tipos de solos, como os disponíveis no 
banco de dados SPECIATE da EPA-US \citep{simon2010}. Podemos, portanto, 
associá-lo à re-suspensão de poeira do solo. A exceção seria o elemento Vanádio,
mas \citet{aboh2009} também encontrou-o no fator poeira de solo em Acra, 
indicando que de algum modo o V está incorporado a esta fonte.

Ainda no primeiro fator, aparece secundariamente fósforo (P) e potássio (K), 
que também pode estar presente em solos, especialmente se houver manipulação 
de fertilizantes na região. Gana não produz nenhum tipo de fertilizantes,
mas importa para uso em fazenda locais \citep{fianko2011}. 

As partículas que compõem poeira de solo estão em sua maioria na fração
grossa do Material Particulado ($MP_{2,5-10}$), pois são geradas por processo 
mecânico. Assim, é notável que mesmo no $MP_{2,5}$ ela seja o principal fator, 
provavelmente explicável pelo grande número de ruas não pavimentadas em Acra. 

O segundo fator predominante explica 12,36 \% da variância total e agrupa 
basicamente fósforo (P), potássio (K) e enxofre (S). Considerando-se o registro
de intenso uso de biomassa para cozimento (tabela \ref{table:cookfuel}), 
particularmente na área de Nima, essa seria uma associação óbvia a esta fonte
(\citet{reid2005} e tantos outros trabalhos que relacionam K,P,S e BC, 
como indicativos de queima de biomassa). Apenas o \textit{loading} nulo para BC 
é que introduz uma dúvida, que apontaria para uma associação à vegetação local, 
manipulação da terra e uso de fertilizantes, por exemplo. Mas consideramos isso 
muito improvável, já que essa não é uma atividade que ocorra próximo de Nima, 
sendo o oposto do que observa-se em relação à queima de biomassa. Restamos, 
assim, com esta associação, considerando a possibilidade de ter ocorrido algum 
artefato da modelagem ao lidar com a diversidade de fontes geradoras de BC, 
como veículos ou a queima de lixo. Ressalte-se, em particular, que o fato de 
nossas amostragens terem tido duração de 48h, reduz a habilidade da AF separar 
alguns fatores.

O terceiro fator retido pode ser associado a veículos e explica 11,24\% da 
variância total. É composto por BC, chumbo (P), zinco (Zn), potássio (K) e 
massa. \citet{aboh2009} encontrou o mesmo fator em Kwabenya, 12 km noroeste de 
Nima. BC e Zn estão associados a veículos, devido a combustão incompleta e 
desgaste das pastilhas de freio e dos pneus dos automóveis, respectivamente. 
O chumbo (Pb) foi banido da gasolina em Gana em 2003 devido a acordo 
internacional \citep{epa2015}, mas continua associado, em concentrações bem 
menores, ao processo de mineração e/ou processamento do petróleo. 
Veja-se, por exemplo, a tabela \ref{table:RFcH_descriptive} onde a concentração 
média de Pb foi $18,6 \pm 0,8 n g /m^3$, próximo de valores típicos encontrados 
em São Paulo $16 \pm 13 n g /m^3$ \citep{andrade2012}, onde desde 1992 o uso do 
tetraetilchumbo foi banido no Brasil. Quanto ao \textit{loading} significativo 
de K neste fator, reputamos novamente às dificuldades para o modelos separar 
espécies quando há elementos comuns entre fontes e má resolução temporal das 
amostragens.

O quarto fator explica 8,81\% da variância, com sódio (Na), cloro (Cl) e
conecta-se indubitavelmente à fonte mar, tendo secundariamente enxofre (S), 
possivelmente devido a emissão de dimetil-sulfeto no mar.
Partículas de Na e Cl geradas no mar estão mais concentradas na moda grossa 
($MP_{2,5-10}$), mas pela proximidade do ponto de amostragem com o mar, 
obteve-se concentrações suficientes para a extração deste fator. 
\citet{mcinnes1994} observou que em pouco de tempo de residência na atmosfera, 
o Cl do sal marinho envelhecido é substituído por $SO_4^{2-}$ como resultado 
da reação com ácido sulfúrico.

O quinto e último fator retido explicou 7,40 \% da variância e agrupou
Br e Pb, o que atribuímos à queima de lixo sólido e outros materiais a céu 
aberto. Acra, como comentado na introdução, comporta um dos maiores lixões de 
eletrônicos (\textit{e-waste}) do mundo, e dentre outros problemas, ali são 
queimados componentes (como fios antichamas feitos de plásticos compostos de Br) 
para obtenção do cobre (Cu) em Agbogbloshie, 4 kilomêtros a sudoeste de Nima. 

Como pode ser observado na rosa dos ventos para o ano de 2007 da figura 
\ref{fg:rosaCompleta} o vento predominante em Nima é de sudoeste, 
e portanto pode carregar material de Agbogbloshie para Nima. 
Assim, é provável que no fator 5, além da queima de lixo sólido local, 
também esteja representada uma porção de contaminação de Agbogbloshie.

A tabela \ref{table:AF_TFsH5} traz os resultados da Análise de Fatores do 
$MP_{2,5}$ na avenida. Os fatores principais extraídos 
foram essencialmente os mesmos que na área residencial, mudando em geral as
distribuições dos \textit{loadings} de alguns elementos, mas que não alteram 
as indicações das quatro principais fontes discutidas. Entretanto, observamos 
que o quinto fator passou a representar essencialmente o BC, enquanto que o Br, 
cuja comunalidade já não havia sido bem explicada, distribuiu-se entre outros 
três fatores. Consideramos que isso deve-se novamente às dificuldades do modelo
em separar uma espécie como o BC, originada de múltiplas fontes locais 
significativas, especialmente sendo pobre a resolução temporal das amostragens.

\newpage
\begin{table}[H]
  \centering
  \input{../outputs/beautifulFAdisplay_RFsH5.tex}
  \caption{Análise de Fatores na área residencial para $MP_{2,5}$
           excluindo dias de ocorrência de vento Harmatão. n = 123.
          \label{table:AF_RFsH5}}
\end{table}

\begin{table}[H]
  \centering
  \input{../outputs/beautifulFAdisplay_TFsH5.tex}
  \caption{Análise de Fatores na avenida para $MP_{2,5}$
           excluindo dias de ocorrência de vento Harmatão. n = 122.
          \label{table:AF_TFsH5}}
\end{table}
\newpage

A AF é uma metodologia bastante apropriada para associar as fontes que podem 
ter gerado uma base de dados de MP. Neste trabalho, entretanto, a empregamos 
apenas qualitativamente, sem estendê-la para quantificar o peso das prováveis 
fontes. Neste particular, lançamos mão das análises de PMF. Diversas 
parametrizações foram testadas nesta modelagem e as soluções estáveis, portando 
significado físico, foram retidas. Perceba-se que na AF os fatores são ordenados
segundo a fração da variância que eles explicam na base de dados. 
Mas como veremos ao fazermos o contraste com o PMF, isso não significa estar 
associado a maior massa explicada.

A tabela \ref{table:RFsH_profiles5} apresenta o perfil dos fatores para a área
residencial de $MP_{2,5}$ e o gráfico da figura \ref{fig:RFsH_contribution5}
as respectivas contribuições percentual do fatores na massa total. 

O Fator 1 tem perfil marcado por BC, Pb, Zn, K, V e Mn, representando a maior 
parcela da massa (40,0 \%) e sendo associado à fonte veicular. 
Na AF, esse era o fator 3, que explica uma variância mediana da base de dados.

Associamos o fator 2, com 73\% da massa de bromo, 15,1 \% de Pb e 15,1\% Zn à 
queima de lixo sólido e outros materiais a céu aberto, representando a menor 
contribuição para a massa total (4,65 \%). Esse era o fator 5 na AF, o que 
também representava a menor fração de variância explicada.

O fator 3 é o segundo que mais contribui para massa (22,2 \%) e agrega elementos
identificáveis como poeira de solo, Al, Si, Ti, V, Fe, Mn, Ca, Mg. 
Também aparece secundariamente fósforo (P), que poderia fazer parte da 
composição do solo, devido a materiais de construção (como gesso ou cimento, 
por exemplo) ou fertilizantes. O vanádio na AF aparecia quase que exclusivamente
no fator 1, poeira de solo, enquanto no PMF sua massa está dividida entre 
veículos (31,0\%) e solo (34,4\%), uma distribuição pode justificar-se pela 
associação frequente deste elemento a veículos movidos a óleo diesel.

Contribuindo com 11,6 \% da massa, o fator 4, é marcado por Na e Cl, além de 
representar frações menores (~16 \%) da massa do enxofre (S) e Mg, 
marcas de aerossol marinho. Corresponde a 11,6\% da massa total e também foi o 
4$\degree$ fator identificado na AF.

O fator 5, representou  21,6 \% da massa total. Além do P, K e S, também 
encontrados no fator 2 da AF, associado à queima de biomassa, ele tem 
17 \% de BC (esperado para esse processo) e recepcionou quase metade da massa 
do sódio 40,4 \% (o que também pode ser resultado da queima doméstica de biomassa).

Os resultados para $MP_{2,5}$ do PMF para a avenida estão na tabela 
\ref{table:TFsH_profiles5} e no gráfico da figura \ref{fig:TFsH_contribution5}. 

Na tabela \ref{table:pm2.5fontes} sintetiza-se os fatores, suas associações e 
os respectivos percentuais da massa total determinados pelo PMF para cada um 
dos fatores, na zona residencial e na avenida, auxiliando na articulação da 
discussão que se segue envolvendo resultados destas duas áreas, 
que apresentaram estrutura e dimensões similares.

O fator 1 na avenida, marcado principalmente por representar um alto percentual 
da massa do Bromo 68,4\%, além de frações menores de outros metais,
 que já associamos à queima de lixo sólido. Ele explica apenas 3,86 \% da 
massa total. O fator 2, de contribuição 13,6 \% relacionamos, novamente, 
à fonte marinha. O fator 3 é composto por elementos conectáveis à ressuspensão 
de poeira do solo, contribuindo com 22,2 \% da massa total. 

O fator 4, associado à queima de biomassa, registra aumento da sua fração no BC
em relação à área residencial, de 19,6 \% para 31,6\% e, também, na contribuição
sobre a massa total, de 21,6 para 28,8 \%. A região da avenida, possui comércios
e venda de alimentos prontos para consumo, como churrasco, restaurantes, 
salgados etc, que são majoritariamente produzidos por queima de biomassa. 

O fator 5, relacionado à fonte veicular, teve sua fração na massa total 
diminuída de 40,0 para 31,5 \%. Apesar da maior movimentação de veículos nesta 
área, provavelmente ainda ficou abaixo do incremento comercial devido à 
queima de biomassa.

Apesar das fontes que associamos aos fatores na PMF estarem demarcadas por 
espécies que lhes são características, pode-se observar um entrelaçamento de 
elementos que não parece usual. O Zn e o Pb, bem relacionados a veículos, 
são visíveis no que atribuímos à queima de biomassas, especialmente próximo à 
avenida. Por outro lado, o K no que seria fator veículos, chega a superar a 
parcela relativa à queima de biomassas.
Como já discutido junto aos resultados de AF, amostragens com resolução 
temporal muito larga, como as 48 h empregada neste experimento, dificultam a 
demarcação dos fatores por promoverem uma mistura relativamente homogeneizada 
daquilo que as fontes emitiram.
Mas as amostragens longas também acumulam toda a evolução evolução de gases e 
partículas na atmosfera. Isso tanto pode retirar ou agregar elementos ao MP, 
por reações químicas, condensação homogênea ou heterogênea, agregação de 
espécies sobre a superfície de partículas, coagulação, coalescência, ou 
qualquer outro processo que transforme o estado inicial de uma partícula. 
Em amostragens longas, além daquele aerossol recém emitido, há um maior espaço 
para a acumulação do MP que evolui na atmosfera. 
\citet{li2003} discutem o envelhecimento de partículas emitidas em queimadas, 
analisando-as individualmente por microscopia eletrônica de transmissão. 
Mostra diversas formas de agregados de sulfatos de cálcio e fosfatos, 
misturados com alcatrão e fuligem, registrando, também, adesões sobre partículas 
de mica muscovita, quartzo e esmectitas, que originam-se do solo. É possível que
esse "rearranjo" nas espécies químicas, portanto, também seja resultado de uma 
evolução do material particulado.

\begin{landscape}
  \begin{figure}
    \centering
    \begin{minipage}[b]{0.45\linewidth}
      \includegraphics[width=\textwidth]{../outputs/RFsH_pmf_contribution_pizza5.pdf}
      \caption{Contribuição dos fatores na massa total para $MP_{2,5}$ na área
               residencial excluindo dias de ocorrência de vento Harmatão. seed = 123 n = 123.
               \label{fig:RFsH_contribution5}}
    \end{minipage}%\hfill
    \hspace{0.5cm}
    \begin{minipage}[b]{0.45\linewidth}
      \input{../outputs/RFsH_profiles_percent_species5.tex}
      \captionof{table}{Perfis do fatores na área residencial $MP_{2,5}$ 
                 excluindo dias de ocorrência de vento Harmatão. seed=123 e n= 123. 
                \label{table:RFsH_profiles5}}
    \end{minipage}
  \end{figure}
\end{landscape}

\begin{landscape}
  \begin{figure}
    \centering
    \begin{minipage}[b]{0.45\linewidth}
      \includegraphics[width=\textwidth]{../outputs/TFsH_pmf_contribution_pizza5.pdf}
      \caption{Contribuição dos fatores na massa total para $MP_{2,5}$ na avenida
               excluindo dias de ocorrência de vento Harmatão. seed = 123 n = 122.
               \label{fig:TFsH_contribution5}}
    \end{minipage}%\hfill
    \hspace{0.5cm}
    \begin{minipage}[b]{0.45\linewidth}
      \input{../outputs/TFsH_profiles_percent_species5.tex}
      \captionof{table}{Perfis do fatores avenida $MP_{2,5}$ 
                 excluindo dias de ocorrência de vento Harmatão. seed=123 e n= 122. 
                \label{table:TFsH_profiles5}}
    \end{minipage}
  \end{figure}
\end{landscape}


%%%%
\subsection{Material Particulado Grosso $MP_{2,5-10}$}

Nas primeiras publicações referentes a este estudos, \citet{ARKU2008} e 
\citet{DIONISIO2010}, não houve separação entre $MP_{2,5}$ e $MP_{2,5-10}$. 
Nas amostras de Nima, essa separação possibilitou a identificação
de fontes na fração grossa do material particulado. Lembramos que o experimento
em paralelo com coleta de filtros de quartzo e posterior intercalibração de BC
só foi realizado para $MP_{2,5}$ e portanto não há medidas de BC para 
$MP_{2,5-10}$.

Os \textit{loadings} encontrados para Análise de Fatores de $MP_{2,5-10}$
na área residencial e na avenida estão nas tabelas \ref{table:AF_RGsH5} e 
\ref{table:AF_RGsH5}. A menos de algumas mudanças nas projeções dos 
\textit{loadings}, geralmente pequenas, percebe-se a mesma estrutura de fontes 
para os dois locais, tendo-se retido 4 fatores, que explicaram 93 e 86\% da 
variância destas duas bases de dados. A comunalidade foi maior que 0,7 para 
todos elementos.

O fator 1 explica sozinho essencialmente a metade da variância dos dados e tem 
projetados elementos componentes de poeira do solo Al, Si, Ti, Fe, Mn, Ca, 
Mg, mais $MP_{2,5-10}$ e V, cuja presença também é registrada no solo local. 
A predominância de solo é um resultado comum para o $MP_{2,5-10}$.

O fator 2 é marcado por K, Zn, Pb, S, e Br. 
\citet{aboh2009} identificou em Kwabenya um fator (explicando 17\% da 
variância) composto por Zn, BC e Pb e o relacionou a uma mistura de 
fonte de industria local e queima de biomassa. Mas os autores não explicam como 
essas fontes independentes acabaram correlacionando-se. Em nosso entender, o 
mais provável é que isso seja um efeito do envelhecimento de partículas, 
envolvendo as principais fontes locais - solo, veículos e queima de biomassa - 
já discutido ao final do item  \ref{sec:pm2.5}. Não deixa de ser possível, 
entretanto, que esse material provenha de empresas metalúrgicas situadas a
aproximadamente 2km a sudoeste de Nima (figura \ref{fg:acrasources}), 
o que poderia justificar, também as projeções não desprezíveis de Na e Cl 
neste fator.

O terceiro fator tem altos \textit{loadings} de Na, Cl e S, 
ligando-se à fonte mar. 

O quarto e último fator, marcado pelo Zn (e uma projeção de P na área 
residencial), também chama a atenção por conter \textit{loadings} homogêneos 
das principais espécies ligadas ao solo (que também tem P). Partículas grossas 
contendo Zn têm forte conexão com o desgaste de componentes de veículos 
(pneus e freios). Cremos ser possível conectar este fator à resuspensão de 
poeira da estrada. 

\newpage
\begin{table}[H]
  \centering
  \input{../outputs/beautifulFAdisplay_RGsH4.tex}
  \caption{Análise de Fatores na área residencial para $MP_{2,5-10}$
           excluindo dias de ocorrência de vento Harmatão. n = 112.
          \label{table:AF_RGsH5}}
\end{table}

\begin{table}[H]
  \centering
  \input{../outputs/beautifulFAdisplay_TGsH4.tex}
  \caption{Análise de Fatores na avenida para $MP_{2,5-10}$
           excluindo dias de ocorrência de vento Harmatão. n = 116.
          \label{table:AF_TGsH5}}
\end{table}
\newpage

A AF ofereceu elementos preliminares para a análise por PMF. Ao extrairmos 
5 fatores com esta modelagem, obtínhamos um fator que caracterizava-se como uma 
espécie de resíduo, com baixo percentual da massa (3,4\% na avenida e 
7,5\% na área residencial), sem alterar as associações de fontes que já havíamos
feito na AF. Optamos, assim pela extração de 4 fatores na análise por PMF 
(os resultados com 5 fatores estão no apêndice III). 

Os resultados na área residencial estão na tabela \ref{table:RGsH_profiles4} e
no gráfico \ref{fig:RGsH_contribution4}, enquanto para a avenida temos a 
tabela \ref{table:TGsH_profiles4} e o gráfico \ref{fig:TGsH_contribution4}. 

A tabela \ref{table:pmgrossofontes} sintetiza a individualização de fontes 
adotada e os correspondentes percentuais de massa a elas atribuídos. 
A ressuspensão de poeira de estrada e a ressuspensão de solo representam o 
maior aporte de massa no $MP_{2,5-10}$, totalizando 63,2 \% na avenida e 61,3 \%
na área residencial. 

Para o aerossol marinho o modelo estimou 16,4 \% na 
área residencial e 26,6 \% na avenida. Essa está mais próxima do oceano, 
mas o fato da PMF fechar as contas em 100 \%, também força o rateio das massas 
entre os fatores. A adoção de 5 fatores neste caso, por exemplo, colocaria o 
aerossol marinho em 21,4 \% da massa. Por fim temos o fator "envelhecimento de 
partículas" (envolvendo as principais fontes locais - solo, veículos e queima 
de biomassa), que consideramos também poder ser ligado a empresas metalúrgicas. 
O perfil dado pela PMF cremos reforçar a primeira hipótese. O modelo atribuiu 
22,3 \% a esse fator na área residencial e 10,3 \% na avenida.%

Assim, como discutido no artigo \citet{zhou2013}, medidas relativamente simples 
para limitar a ressuspensão de solo, 
reduziriam drasticamente os níveis de concentração do MP inalável local.



\begin{landscape}
  \begin{figure}
    \centering
    \begin{minipage}[b]{0.45\linewidth}
      \includegraphics[width=\textwidth]{../outputs/RGsH_pmf_contribution_pizza4.pdf}
      \caption{Contribuição dos fatores na massa total para $MP_{2,5-10}$ na área
               residencial excluindo dias de ocorrência de vento Harmatão. seed = 123 n = 123.
               \label{fig:RGsH_contribution4}}
    \end{minipage}%\hfill
    \hspace{0.5cm}
    \begin{minipage}[b]{0.45\linewidth}
      \input{../outputs/RGsH_profiles_percent_species4.tex}
      \captionof{table}{Perfis do fatores na área residencial $MP_{2,5-10}$ 
                 excluindo dias de ocorrência de vento Harmatão. seed=123 e n= 112. 
                \label{table:RGsH_profiles4}}
    \end{minipage}
  \end{figure}
\end{landscape}

\begin{landscape}
  \begin{figure}
    \centering
    \begin{minipage}[b]{0.45\linewidth}
      \includegraphics[width=\textwidth]{../outputs/RGsH_pmf_contribution_pizza5.pdf}
      \caption{Contribuição dos fatores na massa total para $MP_{2,5-10}$ na área
               residencial excluindo dias de ocorrência de vento Harmatão. seed = 123 n = 123.
               \label{fig:RGsH_contribution5}}
    \end{minipage}%\hfill
    \hspace{0.5cm}
    \begin{minipage}[b]{0.45\linewidth}
      \input{../outputs/RGsH_profiles_percent_species5.tex}
      \captionof{table}{Perfis do fatores na área residencial $MP_{2,5-10}$ 
                 excluindo dias de ocorrência de vento Harmatão. seed=123 e n= 112. 
                \label{table:RGsH_profiles5}}
    \end{minipage}
  \end{figure}
\end{landscape}

%%%%

\begin{landscape}
  \begin{figure}
    \centering
    \begin{minipage}[b]{0.45\linewidth}
      \includegraphics[width=\textwidth]{../outputs/TGsH_pmf_contribution_pizza4.pdf}
      \caption{Contribuição dos fatores na massa total para $MP_{2,5-10}$ na avenida
               excluindo dias de ocorrência de vento Harmatão. seed = 123 n = 116.
               \label{fig:TGsH_contribution4}}
    \end{minipage}%\hfill
    \hspace{0.5cm}
    \begin{minipage}[b]{0.45\linewidth}
      \input{../outputs/TGsH_profiles_percent_species4.tex}
      \captionof{table}{Perfis do fatores avenida $MP_{2,5-10}$ 
                 excluindo dias de ocorrência de vento Harmatão. seed=123 e n= 116. 
                \label{table:TGsH_profiles4}}
    \end{minipage}
  \end{figure}
\end{landscape}

\begin{landscape}
  \begin{figure}
    \centering
    \begin{minipage}[b]{0.45\linewidth}
      \includegraphics[width=\textwidth]{../outputs/TGsH_pmf_contribution_pizza5.pdf}
      \caption{Contribuição dos fatores na massa total para $MP_{2,5-10}$ na avenida
               excluindo dias de ocorrência de vento Harmatão. seed = 123 n = 116.
               \label{fig:TGsH_contribution5}}
    \end{minipage}%\hfill
    \hspace{0.5cm}
    \begin{minipage}[b]{0.45\linewidth}
      \input{../outputs/TGsH_profiles_percent_species5.tex}
      \captionof{table}{Perfis do fatores avenida $MP_{2,5-10}$ 
                 excluindo dias de ocorrência de vento Harmatão. seed=123 e n= 116. 
                \label{table:TGsH_profiles5}}
    \end{minipage}
  \end{figure}
\end{landscape}


