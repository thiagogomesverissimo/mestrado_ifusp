%%%%
\section{Identificação das fontes}

No artigo publicado em 2013 intitulado
\textit{Chemical composition and sources of particle pollution in affluent and
poor neighborhoods of Accra, Ghana} \citep{zhou2013} e \citep{zhou2014} 
realizou-se levantamento de 
fontes usando PMF com todas as 2898 amostras coletadas no XX sítios de 
amostragem. As fonte encotradas com seus repectivos elementos foram: 
queima de lixo sólido (Br, Pb), poeira de solo e veículo (Al, Si, Ca, Fe, Zn, 
BC, Pb), solo (Al, Si, Mg, Ti, Mn, Fe), queima de biomassa (K, Cl, S, BC) e mar 
(Na, Cl, S). Estas fontes foram encontradas para os 4 bairros (XX sítios de 
coleta) do experimento e portanto a escolha desses fatores foi realizadas de 
forma que fosse comuns as 4 regiões, portanto é um resultado genérico sobre 
Acra. Segue-se análise detalhada de um dos bairros, Nima, que além do PMF
também utiliza-se Análise de Fatores para levantamento de fontes principais. 

\citet{ARKU2008} e \citet{DIONISIO2010} foram pioneiros em conduzir  
levantamento dos níveis de poluição, composição elementar e 
distribuição espacial e temporal de poluentes em Gana.

Identificação de possíveis fontes poluídoras locais de Acra estão na figura 
\ref{fg:acrasources} bem como a aeroporto
(onde foram coletados os dados meteorológicos).

\begin{figure}[H]
  \centering
  \includegraphics[width=0.9\textwidth]{../outputs/accra_sources.pdf}
  \caption{Levantamento de algumas fontes poluídora de Accra \label{fg:acrasources}}
\end{figure}

%%%%
\subsection{Harmatão}

A discriminação e identificação de fontes por análises estatísticas 
multivariadas é um processo complexo, e no caso de Gana, mais ainda, 
pois há mistura fonte global, poeira do deserto do Saara trazido 
pelo Harmatão e as fontes antropogênicas locais.

No Harmatão praticamente dobra a quantidade de 
material particulado total coletado na amostragem, isso é inclusive refletido na 
legislação ambiental de Gana que tem limites mais permissivos durante esse 
período. 

Estudos como \citet{aboh2009}, \citet{ofosu2013}, \citet{ofosu2012} sugerem que
quando possível, isto é, com número de amostras suficientementes grande, excluir
dias do Harmatão nas análises estatísticas multivariadas, pois quando 
presentes, o Fator representante do Harmatão ofusca as fatores das fontes
locais.

O alto número de amostras coletadas neste experimento 
possibilitou a exclusão dos dias de 
ocorrência do Harmatão nas tabelas de concentrações sem comprometer as análises
multivariadas

O critério de exclusão não é simples, pois a poeira do Harmatão não é contínua 
e portanto não são todos os dias entre novembro e março que ocorre o Harmatão, 
assim, usando esse critério corre-se o risco de perde informações da fontes
locais durante esse período. 
Também não é possivel classificar os dias como Harmatão usando a velocidade do 
vento, já que o fenômeno ocorre em altas altitudes. 

Uso-se o critério sugerido por \citet{aboh2009}, que considera os dias de 
ocorrência do Harmatão como os que tem concentrações de Silício maiores que 
10 $\mu g/m^3$ e estão entre Novembro e Março.

Resultados da Análise de Fatores e PMF para as tabelas de concentrações 
completas (com Harmatão) podem ser consultadas no apêndice II. 

%%%%
\subsection{Material Particulado Fino ($MP_{2,5}$)}

Os \textit{loadings} encontrados para Análise de Fatores de $MP_{2,5}$
na área residencial estão na tabela \ref{table:AF_RFsH5}, onde 83,58 \% 
da variância total dos dados foi explicada com 5 fatores retidos 
(o ideal é que se explique ao menos 80 \%) para uma base de dados de 123 amotras. 
O mesmo resultado, mas incluindo os dias de Harmatão, elevando para 197 amostras, 
disponível no apêndice II, tabela \ref{table:AF_RFcH5}, 90,8 \% da variância
foi explicada, mas há praticamente um fator predominante 
englobando a maioria dos elementos e explicando sozinho mais que 57,9 \% da 
variância total.

Na tabela \ref{table:AF_RFsH5}, apesar do primeiro fator continuar prepoderante,
ao menos os elementos estão melhor distribuídos entre os demais fatores.  

A comunalidade, variável que indica o quão o ajuste explicou a variabilidade por 
espécie, foi maior que $0,7$ para a maioria dos elementos,
com exceção do bromo (Br) e zinco (Zn) que tiveram comunalidade de $0,59$ e 
$0,44$, respectivamente.

No fator predominante, isto é, que explica sozinho a maior parte da variância 
43,78\%, tem altos \textit{loadings} para os metais (Al, Si, Ti, V, Fe, Mn, Ca, 
Mg) e massa. São oriundos de re-suspensão de poeira do solo e com exceção do 
Vanádio, são comumente encontrados em diversos tipos de solos, 
como os disponíveis no banco de dados SPECIATE da EPA-US \citep{simon2010}.
Ainda no primeiro fator, aparece secundariamente fósforo (P) e potássio (K), 
indicando possível sobreposição de fontes ou solo contaminado. 
Fósforo (P) e potássio (K), segundo SPECIATE, podem ser indicadores de 
fertilizantes. Gana não produz nenhum tipo de feritizantes, mas importa para 
uso em fazenda locais \citep{fianko2011}. 
Vanádio, normalmente associado a queima de óleos pesados como o diesel, ao 
aparecer na fonte solo, indica sobreposição de fontes com temporalidade próximas.
\citet{aboh2009} também encontrou Vanádio no fator de poeira solo.

As partículas que compõem poeira de solo estão em sua maioria na fração
grossa do Material Particulado ($MP_{2,5-10}$), pois são geradas por processo 
mecânico. Assim, é notável que mesmo no $MP_{2,5}$ é o principal fator, 
provavelmente explicável pelo grande número de ruas não pavimentadas em Acra. 

O segundo fator predominante explica 12,36 \% da variância total. 
Agrupa basicamente fósforo (P), potássio (K) e enxofre (S) e está relacionado
vegetação local, manipulação da terra e uso fertizantes. 
\citet{reid2005} aponta que o aparecimento de K,P,S e BC indicam queima de
biomassa, entretanto, o \textit{loading} do BC neste fator foi zero, o que nos
impede de supor que este fator esteja relacionado a queima de biomassa. 

O terceiro fator retido representa fonte veicular e explica 11,24\% da variância
total. É composto por Black Carbon (BC), chumbo (P), zinco (Zn), potássio (K) e 
massa. \citet{aboh2009} encontrou o mesmo fator em Kwabenya, 12km noroeste de 
Nima. BC e Zn estão associados a veículos, devido a combustão 
incompleta e desgaste das pastilhas de freio e dos pneus dos automóveis, 
respectivamente. O chumbo (Pb) foi banido da gasolina em Gana em 2003 devido 
a acordo internacional \citep{epa2015}.  
Na tabela \ref{table:RFcH_descriptive} a concentração média de $Pb$ 
foi $18,6 \pm 0,8 n g /m^3$, próximo de valores típicos encontrados em São Paulo 
$16 \pm 13 n g /m^3$ \citep{andrade2012}.

O quarto fator explica 8,81\% da variância, com Sódio (Na), Cloro (Cl) e
secundariamente enxofre (S) representa indubitavelmente a fonte mar. 
Partículas de Sódio (Na) e Cloro (Cl) geradas no mar estão mais concentradas 
na moda grossa ($MP_{2,5-10}$), mas pela proximidade do ponto de amostragem 
com mar, obteu-se concentrações suficientes que permitiram a geração de um fator. 
Em pouco tempo de residência na atmosfera, o Cloro do sal marinho envelhecido 
é substituído por $SO_4^{2-}$ como resultado da reação com ácido sulfúrico e 
ácido nítrico \citep{mcinnes1994}, o que explica a projeção do enxofre neste
fator (\textit{loading} de $0,29$).

O quinto e último fator retido explicou 7,40 \% da variância e agrupou
Bromo e Chumbo e representa queima de lixo sólido e outros materiais a céu 
aberto. 

Acra comporta um dos maiores lixões de eletrônicos (\textit{e-waste}) 
do mundo, recebendo grande parte dos equipamentos descartados como lixo na 
Europa e que são derretidos para obtenção do cobre (Cu) em Agbogbloshie, 
4 kilometros sudoeste de Nima.  

O zinco (Zn) é o principal composto de baterias de eletrônicos e altas 
concentrações de Al, Co, Cu, Zn, Cd, In, Sb, Ba, e Pb foram encontradas
no solo do \textit{e-waste} de Agbogbloshie por \citet{asante2012}. 

Como pode ser observado na rosa dos ventos para o ano de 2007 
da figura \ref{fg:rosa2007} o vento predominante em Nima é de sudoeste, 
e portanto pode carregar material de Agbogbloshie para Nima. 
Assim, é possível que no fator 5, além da queima de lixo sólido local, 
também esteja representada pequena porção de contaminação do 
\textit{e-waste} de Agbogbloshie.

\begin{figure}[H]
  \centering
  \includegraphics[width=0.5\textwidth]{../outputs/windRose2007.pdf}
  \caption{Rosa do ventos para dados horários de 2007 do 
           Kotoka International Airport em Acra 
           \label{fg:rosa2007}}
\end{figure}%

%TODO: avaliar se usaremos o factor score para alguma coisa. 
%\newpage
%\begin{figure}[H]
%  \centering
%  \includegraphics[width=\textwidth]{../outputs/scores_RFsH5.pdf}
%  \caption{RGsH factor scores}
%\end{figure}

Na avenida, os resultados da Análise de Fatores de $MP_{2,5}$ estão na 
tabela \ref{table:AF_TFsH5}. Os fatores principais extraídos 
foram essencialmente os mesmos que na área residencial, apenas com mudanças nas
distribuições dos \textit{loadings} de alguns elementos, mas que não alteram 
as indicações de fontes já discutidas. Há um leve aumento do \textit{loading} 
do BC de 0,81 para 0,94 no fator de veículos, possivelmente relacionado a
maior movimentação de veículos na avenida.

\newpage
\begin{table}[H]
  \centering
  \input{../outputs/beautifulFAdisplay_RFsH5.tex}
  \caption{Análise de Fatores na área residencial para $MP_{2,5}$
           excluindo dias de ocorrência de vento Harmatão. n = 123.
          \label{table:AF_RFsH5}}
\end{table}

\begin{table}[H]
  \centering
  \input{../outputs/beautifulFAdisplay_TFsH5.tex}
  \caption{Análise de Fatores na avenida para $MP_{2,5}$
           excluindo dias de ocorrência de vento Harmatão. n = 122.
          \label{table:AF_TFsH5}}
\end{table}
\newpage

Para melhorar o apuramento de fontes de $MP_{2,5}$ apresenta-se também os
resultados das análises de PMF. Diversas parametrizações foram testadas no PMF 
e as soluções estáveis com fatores de significados físicos foram retidas.

A tabela \ref{table:RFsH_profiles5} apresenta o perfil dos fatores para a área
residencial de $MP_{2,5}$ e o gráfico da figura \ref{fig:RFsH_contribution5}
as respectivas contribuições do fatores na massa total. 

Assim como na Análise de Fatores, quando inclusos dias do Harmatão, 
o fator com elementos de poeira de solo se destaca de todos outros, 
pois sozinho contribuiu para 55,05\% da massa 
e carrega em seu perfil quase todos elementos, como pode ser observado 
no apendice I tabela \ref{table:RFcH_profiles5} e gráfico 
\ref{fig:RFcH_contribution5}.

Removendo-se o Harmatão o Fator 1 de perfil composto Black Carbon (BC), 
chumbo (P), zinco (Zn), potássio (K), $MP_{2,5}$, vanádio (V) e manganês (Mn).
Assim como na Análise de Fatores, representa fonte veicular, porém é que 
contribui mais para massa total 39,98 $\pm$1,93. É o fator de maior peso, 
o que é mais coerente do que se espera em $MP_{2,5}$, que o fator mais 
significativo seja de fontes de processos de combustão, que geram partículas
finas, e não solo, como encotrado na Análise de Fatores.

O fator 2 com 73\% da massa de bromo, 15,1\% de Pb e 15,1\% Zn representa 
queima de lixo sólido e outros materiais a céu aberto, não tendo contribuição
muito expressiva na massa total 4,65 $\pm$ 0,43 \%.

O fator 3 é o segundo que mais contribui para massa (22,18 $\pm$ 2,84 \%) 
e agrega os elementos identificados como poeira de solo Al, Si, Ti, V, Fe, Mn, 
Ca, Mg. Também aparece secundariamente fósforo (P), possível identificador de
fertilizantes. Enquanto que na Análise de Fatores o vanádio aparecia
quase que exclusivamente no fator poeira de solo no PMF sua massa está dividida
entre veículos (31,0\%) e solo (34,4\%), o que mostra que a distribuição
feita pelo PMF no caso do vanádio faz mais sentido com a realidade, já que
é normal encontrar vanánio em veículos movido a óleo pesado.

Contribuindo com 11,64 $\pm$ 0,89 \% da massa, o fator 4, composto por Na, Cl e
16,7 \% da massa do enxofre (S) corresponde a fonte marinha. 

O fator 5, mas o terceiro em contribuição na massa 21,55 $\pm$ 1,66), depois
de veículo e poeira de solo, além do fósforo (P), potássio (K) e enxofre (S) 
encontrados no fator vegetação da Análise de Fatores, possui expressiva 
participação 



\begin{landscape}
  \begin{figure}
    \centering
    \begin{minipage}[b]{0.45\linewidth}
      \includegraphics[width=\textwidth]{../outputs/RFsH_pmf_contribution_pizza5.pdf}
      \caption{Contribuição dos fatores na massa total para $MP_{2,5}$ na área
               residencial excluindo dias de ocorrência de vento Harmatão. seed = 123 n = 123.
               \label{fig:RFsH_contribution5}}
    \end{minipage}%\hfill
    \hspace{0.5cm}
    \begin{minipage}[b]{0.45\linewidth}
      \input{../outputs/RFsH_profiles_percent_species5.tex}
      \captionof{table}{Perfis do fatores na área residencial $MP_{2,5}$ 
                 excluindo dias de ocorrência de vento Harmatão. seed=123 e n= 122. 
                \label{table:RFsH_profiles5}}
    \end{minipage}
  \end{figure}
\end{landscape}

\begin{landscape}
  \begin{figure}
    \centering
    \begin{minipage}[b]{0.45\linewidth}
      \includegraphics[width=\textwidth]{../outputs/TFsH_pmf_contribution_pizza5.pdf}
      \caption{Contribuição dos fatores na massa total para $MP_{2,5}$ na avenida
               excluindo dias de ocorrência de vento Harmatão. seed = 123 n = 123.
               \label{fig:TFsH_contribution5}}
    \end{minipage}%\hfill
    \hspace{0.5cm}
    \begin{minipage}[b]{0.45\linewidth}
      \input{../outputs/TFsH_profiles_percent_species5.tex}
      \captionof{table}{Perfis do fatores avenida $MP_{2,5}$ 
                 excluindo dias de ocorrência de vento Harmatão. seed=123 e n= 122. 
                \label{table:TFsH_profiles5}}
    \end{minipage}
  \end{figure}
\end{landscape}
 





%%%%
\subsection{Material Particulado Grosso $MP_{2,5-10}$}

Fonte de energia para cozimento de alimentos em Gana \ref{table:cookfuel}.
\begin{table}[H]
 \centering
  \input{../outputs/census_cookfuel.tex}
  \caption{Fontes de energia usadas para cozimento de alimentos em 
           Gana \citep{ghanacensus2013} \label{table:cookfuel}}
\end{table}

Os perfis dos fatores para $MP_{2,5-10}$ nos dois sítios de amostragem, 
residencial e avenida, estão na tabela \ref{}.

A contribuição por fator e a respectiva associação com fontes poluídoras
estão no gráfico da figura \ref{}. 

\citep{asante2012} analisou nível de elementos traços nas urinas de trabalhadores 
do e-waste de Agbogbloshie e Fe, Sb, and Pb tiveram concentrações alta se comparada
a um pessoa de referência, que não tem contato com \textbf{e-waste}.

O Harmatão foi presente tanto em MP10 quanto em MP2.5.

\begin{table}[H]
  \input{../outputs/beautifulFAdisplay_RGsH4.tex}
  \caption{Análise de Fatores para $MP_{2,5-10}$ na área residencial
           excluindo-se dias de ocorrência do Harmatão.
           Rotação varimax - 5 fatores retidos (n=).
           (\textcolor{red}{h} : Comunalidade; 
           \textcolor{red}{S=1-h} : Singularidade; 
           \textcolor{red}{C} : Complexidade.)
           \label{table:beautifulFAdisplay_RGsH4}}
\end{table}

\begin{table}[H]
  \input{../outputs/beautifulFAdisplay_TGsH4.tex}
  \caption{Análise de Fatores para $MP_{2,5-10}$ na avenida
           excluindo-se dias de ocorrência do Harmatão.
           Rotação varimax - 5 fatores retidos (n=).
           (\textcolor{red}{h} : Comunalidade; 
           \textcolor{red}{S=1-h} : Singularidade; 
           \textcolor{red}{C} : Complexidade.)
           \label{table:beautifulFAdisplay_TGsH4}}
\end{table}

\begin{table}[H]
  \centering
    \input{../outputs/RGsH_profiles_percent_species4.tex}
    \caption{residencial $MP_{2,5-10}$ removendo-se os dias do Harmatão 
              seed=123; n=. 
             \label{table:RGsH_profiles4}}
\end{table}

\begin{table}[H]
  \centering
    \input{../outputs/TGsH_profiles_percent_species4.tex}
    \caption{avenida $MP_{2,5-10}$ removendo-se os dias do Harmatão 
              seed=123; n= . 
             \label{table:TGsH_profiles4}}
\end{table}


\begin{figure}[H]
  \centering
  \begin{subfigure}[b]{0.45\textwidth}
    \includegraphics[width=\textwidth]{../outputs/RGsH_pmf_contribution_pizza4.pdf}
    \caption{Residencial}
  \end{subfigure}%
  \begin{subfigure}[b]{0.45\textwidth}
    \includegraphics[width=\textwidth]{../outputs/TGsH_pmf_contribution_pizza4.pdf}
    \caption{Avenida}
  \end{subfigure}
  \caption{Contribuição $MP_{2,5-10}$ \label{qqq}}
\end{figure}

%Início de meteorologia a transferir

Para avaliação do perfil dos parâmetros meteorológicos locais
utilizou-se dados horários de Setembro de 2006 à Junho de 2008 
coletados na estação meteorológica do aeroporto de Acra 
(\textbf{Kotoka International Airport}) cadastrado na \textbf{NOAA}, 2015 (\textbf{National Oceanic 
and Atmospheric Administration} - United States Department of Commerce)%
%Você está fazendo uma tabela com acrônimos ou siglas. Isso é interessante para eliminar a inserção deles no texto e facilitar compreendê-los aonde quer que apareçam. Veja que corrigi a NOAA, de acordo como eles se referenciam.
%Precisa pinças todos os acrônimos espalhados pelo texto, eliminar o nome por extenso e colocar nesta tabela. Atenção para o ajuste fino da frase quando fizer esta mudança. 
. Esta mantém um banco de dados com parâmetros 
meteorológicos do mundo inteiro, enviados por estações meteorológicas 
cadastradas.

A figura \ref{fg:rosaCompleta}, 
mostra a distribuição de frequência da direção dos ventos bem, como a 
intensidade. Verifica-se que a direção predominante de origem dos ventos de superfície
é de sudoeste. 

\begin{figure}[H]
  \centering
  \includegraphics[width=0.5\textwidth]{../outputs/windRoseNoaaHarvard.pdf}
  \caption{Rosa do ventos entre
           Setembro de 2006 e Junho de 2008. Utilizou-se dados 
           do \textbf{Kotoka International Airport} de Acra 
           \label{fg:rosaCompleta}}
\end{figure}%
%veja que você não está considerando ciclos anuais completos, ou seja, de junho a junho, setembro a setembro. Isso pode introduzir vícios na avaliação do vento local em termos sazonais. Precisamos refletir um pouco como tratar isso. 

No gráfico da figura \ref{fg:rosaCompleta}%
%precisa de legenda. É necessário explicitar se o período do ano é o mesmo
 a observação horária da direção e 
intensidade dos ventos, identifica-se um forte componente regional para o vento local, associável à brisa marinha, com ventos de sul (do oceano) quando o sol encontra-se alto,%

%sol a pino significa sol no zênit, o que ocorre duas vezes ao ano naquela região e uma vez ao ano em SP.
%Precisa colocar um mapa, ou referir-se ao mapa colocado em outro lugar da tese, para que o leitor possa ver que ao sul fica o oceano
 mas deslocando-se à oeste na medida que as horas avançam (deslocamento à direita do sentido do movimento - ação típica do efeito de Coriolis no hemisfério Norte). 

No gráfico de rosa dos ventos mensal \ref{fig:windRose_mensal}
 nota-se 
perceptível diferença entre os dois períodos climáticos, 
pois no Verão temos maior quantidade de radiação solar, fortalecendo a 
formação de brisa marinha e, consequentemente, havendo maior tempo para a velocidade do vento intensificar-se e para processar-se um deslocamento para oeste.

Na maior parte do verão Ganha localiza-se no hemisfério sul em termos de padrão global 
de circulação. Neste período as intensidades de vento são maiores e são menores as 
frequências de calmarias. No período do inverno Ghana posiciona-se no hemisfério
 norte em termos da circulação global. 
Observa-se com isso maior incidência de vento norte (observe-se particularmente o mês 
de janeiro), com velocidades médias do vento um pouco menores e maior percentual de 
calmaria. É nesta época que Gana e o Saara situam-se no mesmo sistema de circulação 
global, ocorrendo o Harmatão. 

\begin{figure}[H]
  \centering
  \includegraphics[width=1\textwidth]{../outputs/windRose_horaria.pdf}
  \caption{ \citep{carslaw2012} \label{fig:windRose_horaria}}%
  %precisa colocar legenda e período a que se refere
\end{figure}

\begin{figure}[H]
  \centering
  \includegraphics[width=1\textwidth]{../outputs/windRose_mensal.pdf}
  \caption{ \citep{carslaw2012} \label{fig:windRose_mensal}}
\end{figure}

Nos meses de ocorrência do Harmatão (inverno), há um pequeno aumento da frequência de ventos de nordeste ao nível do solo (direção do Saara). Mas a maior parte do particulado que este vento transporta passa por Gana 
em altitudes superiores \citep{breuning2005}, interferindo pouco no predomínio da brisa marinha na circulação local.%
%
%Aqui termina o trecho sobre meteorologia que deve ir para a discussão sobre modelos receptores. 
Esse elementos representam atividades de indústrias locais. 



%
%
%%Zheng: 
%%Possíveis fontes: ressuspensão da 
%%biomassa: K,Cl,S
%
%%massa,Al,Si,K,Ca,Fe
%%K e cl se relacionam
%%K e S no período não harmathan 
%%Na e Cl se descorrelacionam no harmathan
%%Br, Zn, Pb no não harmathan (gasolina-Br e Pb) Pneu:Zn
