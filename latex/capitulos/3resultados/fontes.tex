%%%%
\newpage
\section{Identificação das fontes}

Na aplicação de modelos receptores, os valores faltantes foram preenchidos com 
o valor da metade do limite de detecção (LD/2), pois supõe-se que a concentração
não detectada de um elemento que aparece frequentemente nas demais amostras 
esteja entre 0 e o limite de detecção, com igual probabilidade de ocorrência 
para cada valor. Assim, LD/2 seria a média e, portanto, o valor mais 
provável entre estes valores não detectados. Para a incerteza das concentrações,
usou-se sugestão proposta por \citet{polissar1998} de usar $(5 \cdot LD)/6$.

Até então, foram frutos dessa pesquisa dois artigos publicados: um em 2013, 
intitulado \textit{Chemical composition and sources of particle pollution in 
affluent and poor neighborhoods of Accra, Ghana} \citep{zhou2013}, e outro em 
2014, \textit{Chemical characterization and source apportionment of household 
fine particulate matter in rural, peri-urban, and urban West Africa} 
\citep{zhou2014}, nos quais, realizou-se levantamento de fontes usando PMF com 
todas as 2898 amostras ($MP_{10}$ e $MP_{2,5}$) coletadas nos cinco sítios 
ambientais de amostragem. As fontes de MP que foram associadas aos fatores 
encontrados, com seus respectivos elementos caracterizadores, foram: 
queima de lixo sólido (Br, Pb), poeira de solo e veículo (Al, Si, Ca, Fe, Zn, 
BC, Pb), solo (Al, Si, Mg, Ti, Mn, Fe), queima de biomassa (K, Cl, S, BC) e mar 
(Na, Cl, S). Estes fatores foram encontrados para os 4 bairros (5 sítios de 
coleta) do experimento e, portanto, foram escolhidos a partir de uma análise
comuns às 4 regiões, caracterizando-se como um resultado genérico sobre Acra.

O presente trabalho aprofundou análises, dentro do projeto geral, focalizando o 
bairro de Nima, uma área popular e pobre da capital ganense. Algumas diferenças 
metodológicas distinguem esta outra etapa do estudo. Primeiramente, separamos o 
MP nas frações grossa e fina, ao invés de inalável e fina, como apresentado 
naquelas primeiras publicações. Pretendeu-se desta forma melhor definir grupos 
de fontes, subtraindo os finos do inalável. Ou seja, "descolou-se" estas duas 
frações procurando apoiar-se nos seus distintos processos majoritários de 
geração, para melhor discriminação de fontes.

Por outro lado, além do PMF, utilizou-se a Análise de Fatores para dar maior 
suporte à identificação das fontes principais impactando aquela área. 
Avaliou-se, adicionalmente, o comportamento da circulação atmosférica local e 
como esta poderia apoiar a discriminação das possíveis fontes de MP coletadas 
nos amostradores. Registre-se, por fim, que neste ínterim foi publicado o novo 
censo de Gana, oferecendo atualizações no que diz respeito às fontes de energia
empregadas para preparação de alimentos em Gana e em Acra (tabela 
\ref{table:cookfuel}), abrindo possibilidades de reconsiderar algumas 
interpretações feitas nos primeiros artigos.

%%%%
\subsection{Material Particulado Fino ($MP_{2,5}$)}

Os \textit{loadings} encontrados para Análise de Fatores de $MP_{2,5}$
na área residencial estão na tabela \ref{table:AF_RFsH5}, com 5 fatores retidos
e 83,58 \% de variância total explicada para uma base de 123 amotras. 
Houvéssemos incluindo os dias de Harmatão, contaríamos com 197 amostras e 
explicaríamos 90,8 \% da variância, mas extraindo praticamente um único fator 
e englobando a maioria dos elementos. Sozinho explicaria mais que 57,9 \% da 
variância (apêndice II, tabela \ref{table:AF_RFcH5}).

Com 4 fatores, na tabela \ref{table:AF_RFsH4}, apesar do primeiro fator 
continuar preponderante,
vemos uma melhor distribuição dos elementos e da variância entre os fatores, 
permitindo associá-los a fontes regionais.
A variabilidade dos elementos foi bem explicada, obtendo-se comunalidades 
maiores que 0,7 para os elementos, a exceção do bromo (Br) e zinco (Zn) que 
tiveram comunalidade de $0,59$ e 
$0,44$, respectivamente.
%
%Você deve ter explicado o que é comunalidade no método. Não é para repetir aqui. Quem quiser saber deve procurar em métodos.
%

No fator predominante explica 
43,78\% da variância, tem altos \textit{loadings} para os metais (Al, Ti, V, Fe, Mn, Ca, Mg), Si e para a massa. Tais elementos são comumente encontrados em diversos tipos de solos, como os disponíveis no banco de dados SPECIATE da EPA-US \citep{simon2010}. Podemos, portanto, associá-lo à re-suspensão de poeira do solo. A exceção seria o elemento Vanádio, mas \citet{aboh2009} também encontrou-o no fator poeira de solo em Acra, 
indicando que de algum modo o V está incorporado a esta fonte.
%
%Este artigo diz que o V é uma contaminação? Se não, corte este último trecho da frase. 
%
Ainda no primeiro fator, aparece secundariamente fósforo (P) e potássio (K), que também pode estar presente em solos, especialmente se houver manipulação de fertilizantes na região. Gana não produz nenhum tipo de fertilizantes, mas importa para 
uso em fazenda locais \citep{fianko2011}. 


As partículas que compõem poeira de solo estão em sua maioria na fração
grossa do Material Particulado ($MP_{2,5-10}$), pois são geradas por processo 
mecânico. Assim, é notável que mesmo no $MP_{2,5}$ ela seja o principal fator, 
provavelmente explicável pelo grande número de ruas não pavimentadas em Acra. 

O segundo fator predominante explica 12,36 \% da variância total. 
Agrupa basicamente fósforo (P), potássio (K) e enxofre (S). Considerando-se o registro de intenso uso de biomassa para cozimento (Tabela 2.1
%
%acertar codificação da tabela
%
, particularmente na área de Nima, essa seria uma associação óbvia a esta fonte (veja-se por exemplo \citet{reid2005} e tantos outros trabalhos que relacionam K,P,S e BC, como indicativos de queima de
biomassa). Apenas o \textit{loading} nulo para BC é que introduz uma dúvida, que apontaria para uma associação à vegetação local, manipulação da terra e uso de fertilizantes, por exemplo. Mas consideramos isso muito improvável, já que essa não é uma atividade que ocorra próximo de Nima, sendo o oposto do que observa-se em relação à queima de biomassa. Restamos, assim, com esta associação, considerando a possibilidade de ter ocorrido algum artefato da modelagem ao lidar com a diversidade de fontes geradoras de $BC$, como veículos ou a queima de lixo. Ressalte-se, em particular, que o fato de nossas amostragens terem tido duração de 48h, reduz a habilidade da AF separar alguns fatores.

O terceiro fator retido pode ser associado a veículos e explica 11,24\% da variância
total. É composto por Black Carbon (BC), chumbo (P), zinco (Zn), potássio (K) e 
massa. \citet{aboh2009} encontrou o mesmo fator em Kwabenya, 12km noroeste de 
Nima. BC e Zn estão associados a veículos, devido a combustão incompleta e desgaste das pastilhas de freio e dos pneus dos automóveis, 
respectivamente. O chumbo (Pb) foi banido da gasolina em Gana em 2003 devido 
a acordo internacional \citep{epa2015}, mas continua associado, em concentrações bem menores, ao processo de mineração e/ou processamento do petróleo. Veja-se, por exemplo, a tabela \ref{table:RFcH_descriptive} onde a concentração média de $Pb$ 
foi $18,6 \pm 0,8 n g /m^3$, próximo de valores típicos encontrados em São Paulo 
$16 \pm 13 n g /m^3$ \citep{andrade2012}, onde desde 1992 o uso do tetraetilchumbo foi banido no Brasil. Quanto ao \textit{loading} significativo de K neste fator, reputamos novamente às dificuldades para o modelos separar espécies quando há elementos comuns entre fontes e má resolução temporal das amostragens.

O quarto fator explica 8,81\% da variância, com Sódio (Na), Cloro (Cl) e
secundariamente enxofre (S) conecta-se indubitavelmente à fonte mar. 
Partículas de Sódio (Na) e Cloro (Cl) geradas no mar estão mais concentradas 
na moda grossa ($MP_{2,5-10}$), mas pela proximidade do ponto de amostragem 
com o mar, obteve-se concentrações suficientes para a extração deste fator. 
Em pouco tempo de residência na atmosfera, o Cloro do sal marinho envelhecido 
é substituído por $SO_4^{2-}$ como resultado da reação com ácido sulfúrico e 
ácido nítrico \citep{mcinnes1994}, o que poderia ampliar a projeção do enxofre neste
fator (\textit{loading} de $0,29$).

O quinto e último fator retido explicou 7,40 \% da variância e agrupou
Bromo e Chumbo, o que atribuímos à queima de lixo sólido e outros materiais a céu 
aberto. Acra, como comentamos na introdução, comporta um dos maiores lixões de eletrônicos (\textit{e-waste}) 
do mundo, e dentre otros problemas, ali são queimados componentes (como fios antichamas) para obtenção do cobre (Cu) em Agbogbloshie, 4 kilometros a sudoeste de Nima. O Br compõe plásticos antichama.

Como pode ser observado na rosa dos ventos para o ano de 2007 da figura \ref{fg:rosa2007} o vento predominante em Nima é de sudoeste, 
e portanto pode carregar material de Agbogbloshie para Nima. 
Assim, é provável que no fator 5, além da queima de lixo sólido local, 
também esteja representada uma porção de contaminação do 
\textit{e-waste} de Agbogbloshie.
%
%acho que esta figura da rosa dos ventos está repetida. Pode fazer referência à primeira vez em que ela aparece.
%

\begin{figure}[H]
  \centering
  \includegraphics[width=0.5\textwidth]{../outputs/windRose2007.pdf}
  \caption{Rosa do ventos para dados horários de 2007 do 
           Kotoka International Airport em Acra 
           \label{fg:rosa2007}}
\end{figure}%

%TODO: avaliar se usaremos o factor score para alguma coisa. 
%\newpage
%\begin{figure}[H]
%  \centering
%  \includegraphics[width=\textwidth]{../outputs/scores_RFsH5.pdf}
%  \caption{RGsH factor scores}
%\end{figure}

A tabela \ref{table:AF_TFsH5} traz os resultados da Análise de Fatores do $MP_{2,5}$ na avenida. Os fatores principais extraídos 
foram essencialmente os mesmos que na área residencial, mudando em geral as
distribuições dos \textit{loadings} de alguns elementos, mas que não alteram 
as indicações das quatro principais fontes discutidas. Entretanto, observamos que o quinto fator passou a representa essencialmente o BC, enquanto que o Br, cuja comunalidade já não havia sido bem explicada, distribuiu-se entre outros três fatores. Consideramos que isso deve-se novamente às dificuldades do modelo em separar uma espécie como o BC, originada de múltiplas fontes locais significativas, especialmente sendo pobre a resolução temporal das amostragens.

\newpage
\begin{table}[H]
  \centering
  \input{../outputs/beautifulFAdisplay_RFsH5.tex}
  \caption{Análise de Fatores na área residencial para $MP_{2,5}$
           excluindo dias de ocorrência de vento Harmatão. n = 123.
          \label{table:AF_RFsH5}}
\end{table}

\begin{table}[H]
  \centering
  \input{../outputs/beautifulFAdisplay_TFsH5.tex}
  \caption{Análise de Fatores na avenida para $MP_{2,5}$
           excluindo dias de ocorrência de vento Harmatão. n = 122.
          \label{table:AF_TFsH5}}
\end{table}
\newpage

A AF é uma metodologia bastante apropriada para associar as fontes que podem ter gerado uma base de dados de MP. Neste trabalho, entretanto, a empregamos apenas qualitativamente, sem estendê-la para quantificar o peso das prováveis fontes. Neste particular lançamos mão das análises de PMF. Diversas parametrizações foram testadas nesta modelagem e as soluções estáveis, portando significado físico, foram retidas. Perceba-se que na AF os fatores são ordenados segundo a fração da variância que eles explicam na base de dados. Mas como veremos ao fazermos o contraste com o PMF, isso não significa estar associado a maior massa explicada.

A tabela \ref{table:RFsH_profiles5} apresenta o perfil dos fatores para a área
residencial de $MP_{2,5}$ e o gráfico da figura \ref{fig:RFsH_contribution5}
as respectivas contribuições percentual do fatores na massa total. 

%
%O Harmartão já deve ter sido analisado e as peculiaridades da seleção dos dias também devem ter sido comentadas anteriormente. Deixo o trecho seguinte comentado para um eventual aproveitamento nesta parte sobre Harmartão.
%
%Assim como na Análise de Fatores, quando inclusos dias do Harmatão, o fator com elementos de poeira de solo se destaca de todos outros, pois sozinho contribuiu para 55,05\% da massa e carrega em seu perfil quase todos elementos, como pode ser observado no apendice I tabela \ref{table:RFcH_profiles5} e gráfico \ref{fig:RFcH_contribution5}.%
%
%

O Fator 1 tem perfil marcado por Black Carbon (BC), chumbo (Pb), zinco (Zn), potássio (K), vanádio (V) e manganês (Mn), representando a maior parcela da massa (40,0 \%) e sendo associado à fonte veicular. Na AF,esse era o fator 3, que explica uma variância mediana da base de dados.

Associamos o fator 2, com 73\% da massa de bromo, 15,1 \% de Pb e 15,1\% Zn à queima de lixo sólido e outros materiais a céu aberto, representando a menor contribuição para a massa total (4,65 \%). Esse era o fator 5 na AF, o que também representava a menor fração de variância explicada.

O fator 3 é o segundo que mais contribui para massa (22,2 \%) e agrega elementos identificáveis como poeira de solo Al, Si, Ti, V, Fe, Mn, Ca, Mg. Também aparece secundariamente fósforo (P), que poderia fazer parte da composição do solo, vir de materiais de construção (gesso ou cimento por exemplo) ou fertilizantes. O Vanádio (V) na AF aparecia quase que exclusivamente no fator 1, poeira de solo, enquanto no PMF sua massa está dividida entre veículos (31,0\%) e solo (34,4\%), uma distribuição pode justificar-se pela associação frequente deste elemento a veículos movidos a óleo diesel.

Contribuindo com 11,6 \% da massa, o fator 4, é marcado por Na, Cl, além de representar frações menores (~16 \%) da massa do enxofre (S) e Mg, marcas de aerossol marinho. Corresponde a 11,6\% da massa total e também foi o 4º fator identificado na AF.

O fator 5, representou  21,6 \% da massa total. Além do fósforo (P), potássio (K) e enxofre (S), que também encontramos no fator 2 da AF, associado à queima de biomassa, ele tem 17 \% de BC (esperado para esse processo) e recepcionou quase metade da massa do sódio 40,4 \% (o que também pode ser resultado da queima doméstica de biomassa).

Os resultados para $MP_{2,5}$ do PMF para a avenida estão na tabela 
\ref{table:TFsH_profiles5} e no gráfico da figura \ref{fig:TFsH_contribution5}. 
%
%fazer uma tabela sintetizando os fatores e os percentuais - associação,fator bairro, fator avenida,(%) Bairro, (%) avenida, Tabela xxxx
%
Construímos, também, a tabela xxxx, onde sintetizamos  os fatores, suas associações e os respectivos percentuais da massa total determinados pelo PMF para cada um dos fatores, na zona residencial e na avenida. Ela pode auxiliar na articulação da discussão que se segue envolvendo resultados destas duas áreas, que apresentaram estrutura e dimensões similares.

O fator 1 na avenida, marcado principalmente por representar um alto percentual da massa do Bromo 68,4\%, além de frações menores de outros metais, que já associamos à queima de lixo sólido. Ele explica apenas 3,86 \% da massa total.  

O fator 2, de contribuição 13,6 \% relacionamos, novamente, à fonte marinha. 

O fator 3 é composto por elementos conectáveis à ressuspensão de poeira do 
solo, contribuindo com 22,2 \% da massa total. 

O fator 4, associado à queima de biomassa, registra aumento da sua fração no BC em relação à área residencial, de 19,6 \% para 31,6\% e, também, na contribuição sobre a massa total, de 21,6 para 28,8 \%. A região da avenida, possui comércios e venda de alimentos prontos para consumo, como churrasco, restaurantes, salgados etc, que são majoritariamente produzidos por queima de biomassa. 

O fator 5, relacionado à fonte veicular, teve sua fração na massa total diminuída de 40,0 para 31,5 \%. Apesar da maior movimentação de veículos nesta área, provavelmente ainda ficou abaixo do incremento comercial devido à queima de biomassa.

Apesar das fontes que associamos aos fatores na PMF estarem demarcadas por espécies que lhes são características, pode-se observar um entrelaçamento de elementos que não parece usual. O Zn e o Pb, bem relacionados a veículos, são visíveis no que atribuímos à queima de biomassas, especialmente próximo à avenida. Por outro lado, o K no que seria fator veículos, chega a superar a parcela relativa à queima de biomassas.
Como já discutimos junto aos resultados de AF, amostragens com resolução temporal muito larga, como as 48 h empregada neste experimento, dificultam a demarcação dos fatores por promoverem uma mistura relativamente homogeneizada daquilo que as fontes emitiram. Mas as amostragens longas também acumulam toda a evolução evolução de gases e partículas na atmosfera. Isso tanto pode retirar ou agregar elementos ao MP, por reações químicas, condensação homogênea ou heterogênea, agregação de espécies sobre a superfície de partículas, coagulação, coalescência, ou qualquer outro processo que transforme o estado inicial de uma partícula. Em amostragens longas, além daquele aerossol recém emitido, há um maior espaço para a acumulação do MP que evolui na atmosfera. Li et al., 2003 discutem o envelhecimento de partículas emitidas em queimadas, analisando-as individualmente por microscopia eletrônica de transmissão. Mostra diversas formas de agregados de sulfatos de cálcio e fosfatos, misturados com alcatrão e fuligem, registrando, também, adesões sobre partículas de mica muscovita, quartzo e esmectitas, que originam-se do solo. É possível que esse "rearranjo" nas espécies químicas, portanto, também seja resultado de uma evolução do material particulado.
%
%Li et al., 2003
%Li, Jia; Po´sfai, Miha´ly; Hobbs, Peter V.; Buseck, Peter R., 2003. Individual aerosol particles from biomass burning in southern Africa: 2. Compositions and aging of inorganic particles, JOURNAL OF GEOPHYSICAL RESEARCH, VOL. 108, NO. D13, 8484, doi:10.1029/2002JD002310.
%


Os \textit{loadings} encontrados para Análise de Fatores de $MP_{2,5}$
na área residencial estão na tabela \ref{table:AF_RFsH5}, onde 83,58 \% 
da variância total dos dados foi explicada com 5 fatores retidos 
(o ideal é que se explique ao menos 80 \%) para uma base de dados de 123 amotras. 
O mesmo resultado, mas incluindo os dias de Harmatão, elevando para 197 amostras, 
disponível no apêndice II, tabela \ref{table:AF_RFcH5}, 90,8 \% da variância
foi explicada, mas há praticamente um fator predominante 
englobando a maioria dos elementos e explicando sozinho mais que 57,9 \% da 
variância total.

Na tabela \ref{table:AF_RFsH5}, apesar do primeiro fator continuar prepoderante,
ao menos os elementos estão melhor distribuídos entre os demais fatores.  

A comunalidade, variável que indica o quão o ajuste explicou a variabilidade por 
espécie, foi maior que $0,7$ para a maioria dos elementos,
com exceção do bromo (Br) e zinco (Zn) que tiveram comunalidade de $0,59$ e 
$0,44$, respectivamente.

No fator predominante, isto é, que explica sozinho a maior parte da variância 
43,78\%, tem altos \textit{loadings} para os metais (Al, Si, Ti, V, Fe, Mn, Ca, 
Mg) e massa. São oriundos de re-suspensão de poeira do solo e com exceção do 
Vanádio, são comumente encontrados em diversos tipos de solos, 
como os disponíveis no banco de dados SPECIATE da EPA-US \citep{simon2010}.
Ainda no primeiro fator, aparece secundariamente fósforo (P) e potássio (K), 
indicando possível sobreposição de fontes ou solo contaminado. 
Fósforo (P) e potássio (K), segundo SPECIATE, podem ser indicadores de 
fertilizantes. Gana não produz nenhum tipo de fertilizantes, mas importa para 
uso em fazenda locais \citep{fianko2011}. 
Vanádio, normalmente associado a queima de óleos pesados como o diesel, ao 
aparecer na fonte solo, indica sobreposição de fontes com temporalidade próximas.
\citet{aboh2009} também encontrou vanádio no fator de poeira solo em Acra, 
indicação que o solo é contamindado por vánadio.

As partículas que compõem poeira de solo estão em sua maioria na fração
grossa do Material Particulado ($MP_{2,5-10}$), pois são geradas por processo 
mecânico. Assim, é notável que mesmo no $MP_{2,5}$ é o principal fator, 
provavelmente explicável pelo grande número de ruas não pavimentadas em Acra. 

O segundo fator predominante explica 12,36 \% da variância total. 
Agrupa basicamente fósforo (P), potássio (K) e enxofre (S) e está relacionado
vegetação local, manipulação da terra e uso fertizantes. 
\citet{reid2005} aponta que o aparecimento de K,P,S e BC indicam queima de
biomassa, entretanto, o \textit{loading} do BC neste fator foi zero, o que nos
impede de supor que este fator esteja relacionado a queima de biomassa. 

O terceiro fator retido representa fonte veicular e explica 11,24\% da variância
total. É composto por Black Carbon (BC), chumbo (P), zinco (Zn), potássio (K) e 
massa. \citet{aboh2009} encontrou o mesmo fator em Kwabenya, 12km noroeste de 
Nima. BC e Zn estão associados a veículos, devido a combustão 
incompleta e desgaste das pastilhas de freio e dos pneus dos automóveis, 
respectivamente. O chumbo (Pb) foi banido da gasolina em Gana em 2003 devido 
a acordo internacional \citep{epa2015}.  
Na tabela \ref{table:RFcH_descriptive} a concentração média de $Pb$ 
foi $18,6 \pm 0,8 n g /m^3$, próximo de valores típicos encontrados em São Paulo 
$16 \pm 13 n g /m^3$ \citep{andrade2012}.

O quarto fator explica 8,81\% da variância, com Sódio (Na), Cloro (Cl) e
secundariamente enxofre (S) representa indubitavelmente a fonte mar. 
Partículas de Sódio (Na) e Cloro (Cl) geradas no mar estão mais concentradas 
na moda grossa ($MP_{2,5-10}$), mas pela proximidade do ponto de amostragem 
com mar, obteu-se concentrações suficientes que permitiram a geração de um fator. 
Em pouco tempo de residência na atmosfera, o Cloro do sal marinho envelhecido 
é substituído por $SO_4^{2-}$ como resultado da reação com ácido sulfúrico e 
ácido nítrico \citep{mcinnes1994}, o que explica a projeção do enxofre neste
fator (\textit{loading} de $0,29$).

O quinto e último fator retido explicou 7,40 \% da variância e agrupou
Bromo e Chumbo e representa queima de lixo sólido e outros materiais a céu 
aberto. 

Acra comporta um dos maiores lixões de eletrônicos (\textit{e-waste}) 
do mundo, recebendo grande parte dos equipamentos descartados como lixo na 
Europa e que são derretidos para obtenção do cobre (Cu) em Agbogbloshie, 
4 kilometros sudoeste de Nima.  

O zinco (Zn) é o principal composto de baterias de eletrônicos.Altas 
concentrações de Al, Co, Cu, Zn, Cd, In, Sb, Ba, e Pb foram encontradas
no solo do \textit{e-waste} de Agbogbloshie por \citet{asante2012},
que também observou altos nível de Fe, Sb, and Pb nas urinas de trabalhadores 
do de Agbogbloshie quando comparada a um pessoa de referência
(que não tenha tido contato com \textit{e-waste}).

Como pode ser observado na rosa dos ventos para o ano de 2007 
da figura \ref{fg:rosa2007} o vento predominante em Nima é de sudoeste, 
e portanto pode carregar material de Agbogbloshie para Nima. 
Assim, é possível que no fator 5, além da queima de lixo sólido local, 
também esteja representada pequena porção de contaminação do 
\textit{e-waste} de Agbogbloshie.

\begin{figure}[H]
  \centering
  \includegraphics[width=0.5\textwidth]{../outputs/windRose2007.pdf}
  \caption{Rosa do ventos para dados horários de 2007 do 
           Kotoka International Airport em Acra 
           \label{fg:rosa2007}}
\end{figure}%

%TODO: avaliar se usaremos o factor score para alguma coisa. 
%\newpage
%\begin{figure}[H]
%  \centering
%  \includegraphics[width=\textwidth]{../outputs/scores_RFsH5.pdf}
%  \caption{RGsH factor scores}
%\end{figure}

Na avenida, os resultados da Análise de Fatores de $MP_{2,5}$ estão na 
tabela \ref{table:AF_TFsH5}. Os fatores principais extraídos 
foram essencialmente os mesmos que na área residencial, apenas com mudanças nas
distribuições dos \textit{loadings} de alguns elementos, mas que não alteram 
as indicações de fontes já discutidas. Há um leve aumento do \textit{loading} 
do BC de 0,81 para 0,94 no fator de veículos, possivelmente relacionado a
maior movimentação de veículos na avenida.

\newpage
\begin{table}[H]
  \centering
  \input{../outputs/beautifulFAdisplay_RFsH5.tex}
  \caption{Análise de Fatores na área residencial para $MP_{2,5}$
           excluindo dias de ocorrência de vento Harmatão. n = 123.
          \label{table:AF_RFsH5}}
\end{table}

\begin{table}[H]
  \centering
  \input{../outputs/beautifulFAdisplay_TFsH5.tex}
  \caption{Análise de Fatores na avenida para $MP_{2,5}$
           excluindo dias de ocorrência de vento Harmatão. n = 122.
          \label{table:AF_TFsH5}}
\end{table}
\newpage

Para melhorar o apuramento de fontes de $MP_{2,5}$ apresenta-se também os
resultados das análises de PMF. Diversas parametrizações foram testadas no PMF 
e as soluções estáveis com fatores de significados físicos foram retidas. 
Foi adicionado 8\% da concentração na incerteza devido a amostragem paralela. 

A tabela \ref{table:RFsH_profiles5} apresenta o perfil dos fatores para a área
residencial de $MP_{2,5}$ e o gráfico da figura \ref{fig:RFsH_contribution5}
as respectivas contribuições do fatores na massa total. 

Assim como na Análise de Fatores, quando inclusos dias do Harmatão, 
o fator com elementos de poeira de solo se destaca de todos outros, 
pois sozinho contribuiu para 55,05\% da massa 
e carrega em seu perfil quase todos elementos, como pode ser observado 
no apendice I tabela \ref{table:RFcH_profiles5} e gráfico 
\ref{fig:RFcH_contribution5}.

Removendo-se o Harmatão o Fator 1 de perfil composto Black Carbon (BC), 
chumbo (Pb), zinco (Zn), potássio (K), $MP_{2,5}$, vanádio (V) e manganês (Mn).
Assim como na Análise de Fatores, representa fonte veicular, porém é que 
contribui mais para massa total 39,98 $\pm$1,93. É o fator de maior peso, 
o que é mais coerente do que se espera em $MP_{2,5}$, que o fator mais 
significativo seja de fontes de processos de combustão, que geram partículas
finas, e não solo, como encotrado na Análise de Fatores.

O fator 2, com 73\% da massa de bromo, 15,1\% de Pb e 15,1\% Zn representa 
queima de lixo sólido e outros materiais a céu aberto, não tendo contribuição
muito expressiva na massa total 4,65 $\pm$ 0,43 \%.

O fator 3 é o segundo que mais contribui para massa (22,18 $\pm$ 2,84 \%) 
e agrega os elementos identificados como poeira de solo Al, Si, Ti, V, Fe, Mn, 
Ca, Mg. Também aparece secundariamente fósforo (P), possível identificador de
fertilizantes. Enquanto que na Análise de Fatores o vanádio aparecia
quase que exclusivamente no fator poeira de solo no PMF sua massa está dividida
entre veículos (31,0\%) e solo (34,4\%), o que mostra que a distribuição
feita pelo PMF no caso do vanádio faz mais sentido com a realidade, já que
é normal encontrar vanánio em veículos movido a óleo pesado.

Contribuindo com 11,64 $\pm$ 0,89 \% da massa, o fator 4, composto por Na, Cl e
16,7 \% da massa do enxofre (S) corresponde a fonte marinha. 

O fator 5, mas o terceiro em contribuição na massa 21,55 $\pm$ 1,66 \%, depois
de veículo e poeira de solo, além do fósforo (P), potássio (K) e enxofre (S) 
encontrados no fator vegetação da Análise de Fatores, possui quase metade da 
massa do sódio 40,4 \%.

Os resultados para $MP_{2,5}$ do PMF para a avenida estão na tabela 
\ref{table:TFsH_profiles5} e no gráfico da figura \ref{fig:TFsH_contribution5}.

O fator 1 na avenida, quase que exclusivamente com toda massa do Bromo 68,4\%,
representa queima de lixo sólido, explicando apenas 3,86$\pm$0,37 \% da massa 
total.  

O fator 2, de contribuição 13,64 $\pm$ 1,06 \% corresponde a fonte marinha. 

O fator 3 é composto por elementos associados a ressuspensão de poeira de 
solo e contribuiu 22,23 $\pm$ 2,14) \% para massa total, praticamente
a mesma contribuição da área residencial. 

O fator 4, associado anteriormente a vegetação, teve aumento da massa de BC, 
de 19,6 \% para 31,6\% e também na contribuição da massa total, 
de 21,55 $\pm$ 1,66 para 28,82 $\pm$ 1,86) \%. A avenida XX, possui comércios
de venda de alimentos prontos para consumo, como churrasco, restaurantes, 
salgados, etc, que são majoritariamento produzidos por queima de biomassa. 
Para \citet{reid2005} K,P,S e BC segerem queima de biomassa, neste caso, 
acreditamos que o fator 4 representação vegetação e queima de biomassa.   

O fator 5, fonte veícular, teve sua contribuição na massa total relativa 
dimunuída de 39,98 $\pm$ 1,93 para 31,45 $\pm$ 1,83 \%, mesmo a avenida contendo 
movimentação de veículos superior a área residêncial,
dimunuição esta provavelmente devido ao aparecimento da fonte queima de biomassa. 

\begin{landscape}
  \begin{figure}
    \centering
    \begin{minipage}[b]{0.45\linewidth}
      \includegraphics[width=\textwidth]{../outputs/RFsH_pmf_contribution_pizza5.pdf}
      \caption{Contribuição dos fatores na massa total para $MP_{2,5}$ na área
               residencial excluindo dias de ocorrência de vento Harmatão. seed = 123 n = 123.
               \label{fig:RFsH_contribution5}}
    \end{minipage}%\hfill
    \hspace{0.5cm}
    \begin{minipage}[b]{0.45\linewidth}
      \input{../outputs/RFsH_profiles_percent_species5.tex}
      \captionof{table}{Perfis do fatores na área residencial $MP_{2,5}$ 
                 excluindo dias de ocorrência de vento Harmatão. seed=123 e n= 123. 
                \label{table:RFsH_profiles5}}
    \end{minipage}
  \end{figure}
\end{landscape}

\begin{landscape}
  \begin{figure}
    \centering
    \begin{minipage}[b]{0.45\linewidth}
      \includegraphics[width=\textwidth]{../outputs/TFsH_pmf_contribution_pizza5.pdf}
      \caption{Contribuição dos fatores na massa total para $MP_{2,5}$ na avenida
               excluindo dias de ocorrência de vento Harmatão. seed = 123 n = 122.
               \label{fig:TFsH_contribution5}}
    \end{minipage}%\hfill
    \hspace{0.5cm}
    \begin{minipage}[b]{0.45\linewidth}
      \input{../outputs/TFsH_profiles_percent_species5.tex}
      \captionof{table}{Perfis do fatores avenida $MP_{2,5}$ 
                 excluindo dias de ocorrência de vento Harmatão. seed=123 e n= 122. 
                \label{table:TFsH_profiles5}}
    \end{minipage}
  \end{figure}
\end{landscape}
 
%%%%
\subsection{Material Particulado Grosso $MP_{2,5-10}$}

Nas publicações referentes a este estudos \citet{ARKU2008} e 
\citet{DIONISIO2010} não houve separação entre $MP_{2,5}$ e $MP_{2,5-10}$. 
Nas amostras de Nima, essa separação foi feita, possibilitando a identificação
de fontes na fração grossa do material partículado. Lembra-se que o experimento
em paralelo com coleta de filtros de quartzo e posterior intercalibração de BC
só foi realizado para $MP_{2,5}$ e portanto não há medidas de BC para 
$MP_{2,5-10}$.

Os \textit{loadings} encontrados para Análise de Fatores de $MP_{2,5-10}$
na área residencial estão na tabela \ref{table:AF_RGsH5}. Com 112 
amostras explicou-se 93,36 \% da variância total dos dados, mas o número 
fatores retidos foi 4 e não 5 como no caso do fino. A comunalidade foi maior 
que $0,7$ para todos elementos.

O fator 1 explica sozinho quase metade da variância 49,07\% e tem 
projetados elementos componentes de poeira do solo Al, Si, Ti, Fe, Mn, Ca, 
Mg, mais $MP_{2,5-10}$ e contaminação de vanádio. Como é comum em $MP_{2,5-10}$, 
o solo é a fonte predominante. Nota-se que a poeira de solo foi presente 
tanto em $MP_{2,5-10}$ quanto em $MP_{2,5}$.

O fator 2 explica 24,25 \% da variância total e é formado por K, Zn, Pb, S, e 
Br. \citet{aboh2009} identificou em Kwabenya um fator (explicando 17\% da 
variância) composto por Zn, BC e Pb e o relacionou a uma mistura de 
fonte de industria local e queima de biomassa. O fator 2 aqui também será 
classificado como industria local e queima de biomassa. 

O terceiro fator explica 13,55\% da variância com altos \textit{loadings} de 
Na, Cl e S, representam se dúvida a fonte mar. 

O quarto e último fator composto por P e Zn representa industria local. 

Os resultados para Análise de Fatores de $MP_{2,5-10}$ na avenida expostos
na tabela \ref{table:AF_RGsH5} não diferem quase em nada daqueles encontrados
na área residencial, com pequenas mudaças nas projeções dos \textit{loadings}, 
como o caso do fósforo que passou do fator industria para o fator solo. 

\newpage
\begin{table}[H]
  \centering
  \input{../outputs/beautifulFAdisplay_RGsH4.tex}
  \caption{Análise de Fatores na área residencial para $MP_{2,5-10}$
           excluindo dias de ocorrência de vento Harmatão. n = 112.
          \label{table:AF_RGsH5}}
\end{table}

\begin{table}[H]
  \centering
  \input{../outputs/beautifulFAdisplay_TGsH4.tex}
  \caption{Análise de Fatores na avenida para $MP_{2,5-10}$
           excluindo dias de ocorrência de vento Harmatão. n = 116.
          \label{table:AF_TGsH5}}
\end{table}
\newpage

Os resultados de PMF para $MP_{2,5-10}$ na área residencial com extração de 
quatro fatores estão na tabela \ref{table:RGsH_profiles4} e gráfico 
\ref{fig:RGsH_contribution4}. Rápida olhada nos elementos dos fatores 1 e 2 
percebe-se que o PMF fez a separação do solo em dois, o primeiro com elementos
da crosta terrestre e um segundo solo contaminado com Zn, Pb e P. O terceiro 
fator é exclusivamente mar e o último mistura de queima de biomassa 
com industria. 

Repetindo a análise PMF mas com extração de 5 fatores obtém-se a tabela 
\ref{table:RGsH_profiles5} e o gráfico \ref{fig:RGsH_contribution5}.
Com extração de 5 fatores foi possível separar a queima de biomassa (fator 5)
das demais fontes e queima de lixo sólido (fator 1) das demais fontes. 
Nota-se que o solo continua separado em dois (fator 3 e 4) e a fonte mar está 
no fator 2. 

Na avenida, com extração de 4 fatores (tabela \ref{table:TGsH_profiles4} 
e gráfico \ref{fig:TGsH_contribution4}) apenas um solo é encontrados
(fator 3). Assim como na área residencial, é possível discriminar melhor 
os fatores relacionados a queima de biomassa ou lixo sólido se extrairmos 5
fatores no PMF. 

A tabela \ref{table:TGsH_profiles5} e e gráfico \ref{fig:TGsH_contribution5} 
mostram os resultados para o caso de 5 fatores extraídos na avenida. 
No fator 1 o mar ficou
bem definido, no fator 2, Mg, Fe e V, representa industria local. 

O fator 3 também reprenta alguma industria local (Zn, Pb e P).

No fator 4 estão juntos queima de biomassa e queima de lixo sólido a céu aberto. 
Por fim, o fator 5 representam poeira de ressuspensão de solo.  

\begin{landscape}
  \begin{figure}
    \centering
    \begin{minipage}[b]{0.45\linewidth}
      \includegraphics[width=\textwidth]{../outputs/RGsH_pmf_contribution_pizza4.pdf}
      \caption{Contribuição dos fatores na massa total para $MP_{2,5-10}$ na área
               residencial excluindo dias de ocorrência de vento Harmatão. seed = 123 n = 123.
               \label{fig:RGsH_contribution4}}
    \end{minipage}%\hfill
    \hspace{0.5cm}
    \begin{minipage}[b]{0.45\linewidth}
      \input{../outputs/RGsH_profiles_percent_species4.tex}
      \captionof{table}{Perfis do fatores na área residencial $MP_{2,5-10}$ 
                 excluindo dias de ocorrência de vento Harmatão. seed=123 e n= 112. 
                \label{table:RGsH_profiles4}}
    \end{minipage}
  \end{figure}
\end{landscape}

\begin{landscape}
  \begin{figure}
    \centering
    \begin{minipage}[b]{0.45\linewidth}
      \includegraphics[width=\textwidth]{../outputs/RGsH_pmf_contribution_pizza5.pdf}
      \caption{Contribuição dos fatores na massa total para $MP_{2,5-10}$ na área
               residencial excluindo dias de ocorrência de vento Harmatão. seed = 123 n = 123.
               \label{fig:RGsH_contribution5}}
    \end{minipage}%\hfill
    \hspace{0.5cm}
    \begin{minipage}[b]{0.45\linewidth}
      \input{../outputs/RGsH_profiles_percent_species5.tex}
      \captionof{table}{Perfis do fatores na área residencial $MP_{2,5-10}$ 
                 excluindo dias de ocorrência de vento Harmatão. seed=123 e n= 112. 
                \label{table:RGsH_profiles5}}
    \end{minipage}
  \end{figure}
\end{landscape}

%%%%

\begin{landscape}
  \begin{figure}
    \centering
    \begin{minipage}[b]{0.45\linewidth}
      \includegraphics[width=\textwidth]{../outputs/TGsH_pmf_contribution_pizza4.pdf}
      \caption{Contribuição dos fatores na massa total para $MP_{2,5-10}$ na avenida
               excluindo dias de ocorrência de vento Harmatão. seed = 123 n = 116.
               \label{fig:TGsH_contribution4}}
    \end{minipage}%\hfill
    \hspace{0.5cm}
    \begin{minipage}[b]{0.45\linewidth}
      \input{../outputs/TGsH_profiles_percent_species4.tex}
      \captionof{table}{Perfis do fatores avenida $MP_{2,5-10}$ 
                 excluindo dias de ocorrência de vento Harmatão. seed=123 e n= 116. 
                \label{table:TGsH_profiles4}}
    \end{minipage}
  \end{figure}
\end{landscape}

\begin{landscape}
  \begin{figure}
    \centering
    \begin{minipage}[b]{0.45\linewidth}
      \includegraphics[width=\textwidth]{../outputs/TGsH_pmf_contribution_pizza5.pdf}
      \caption{Contribuição dos fatores na massa total para $MP_{2,5-10}$ na avenida
               excluindo dias de ocorrência de vento Harmatão. seed = 123 n = 116.
               \label{fig:TGsH_contribution5}}
    \end{minipage}%\hfill
    \hspace{0.5cm}
    \begin{minipage}[b]{0.45\linewidth}
      \input{../outputs/TGsH_profiles_percent_species5.tex}
      \captionof{table}{Perfis do fatores avenida $MP_{2,5-10}$ 
                 excluindo dias de ocorrência de vento Harmatão. seed=123 e n= 116. 
                \label{table:TGsH_profiles5}}
    \end{minipage}
  \end{figure}
\end{landscape}


