%%%%
\section{Identificação das fontes}

O alto número de amostras coletadas possibilitou a exclusão dos dias de 
ocorrência do Harmatão nas tabelas de concentrações sem comprometer as análises
multivariadas. A alta concentração de poeira provinda do Harmatão dificulta a 
identificação de fontes locais.

Nem todos os dias nos meses de ocorrência do Harmatão (Novembro-Março) 
ocorrem altas concentrações, assim, se usássemos como critério de exclusão para 
influência do Harmatão todas amostras entre Novembro e Março, poderíamos
perder informações de fontes locais nos dias em que a poeira do Harmatão fosse
tão intensa. 

Uso-se o critério sugerido por \cite{aboh2009}, que considera os dias de 
ocorrência do Harmatão como os que tem concentrações de silício maiores que 
10 $\mu g/m^3$ e estão entre Novembro e Março.

%%# Fontes encontradas em Harvard:
%%#   Solid waste burning: Br. 
%%#   Road dust & vehicle: Al, Si, Ca, Fe, Zn, BC.  
%%#   Crustal: Al, Si, Mg, Ti, Mn, Fe.
%%#   Aged biomass particles: K, Cl, S, BC
%%   Fresh biomass burning: K, Cl, S, BC
%%#   Sea salt: Na, Cl, S
%%# Sal marinho, solo, emissões veiculares e combustão de biomassa
%%# Zn/Cu podem representa veículos por causa do mencanismo interno do carro. 
%%# Zn: Freio, pneu, peças.

%%%%
\subsection{Material Particulado Fino ($MP_{2,5}$)}

Os \textbf{loadings} encontrados para \textbf{Análise de Fatores} de $MP_{2,5}$
na área residencial estão na tabela \ref{table:loadings_RFcH5}. 
Nota-se que o primeiro fator sozinho explica 58 \% da variância e quase todos 
elementos tem \textbf{loadings} altos nesse fator. 

\begin{table}[H]
  \input{../outputs/loadings_RFcH5.tex}
  \caption{Análise de Fatores para $MP_{2,5}$ na região residencial.
           Rotação varimax - 5 fatores retidos (n=197).
           (\textcolor{red}{h} : Comunalidade; 
           \textcolor{red}{S=1-h} : Singularidade; 
           \textcolor{red}{C} : Complexidade.)
           \label{table:loadings_RFcH5}}
\end{table}

Os perfis de fontes da análise \textbf{PMF} para a mesma base de dados é 
apresentado na tabela \ref{table:RFcH_profiles5} e a contribuição na massa
de cada fator é apresentada no gráfico da figura \ref{table:RFcH_contribution5}.
Assim como na \textbf{Análise de Fatores}, há um fator se destacando de todos
outros, pois sozinho contribuiu para 51,84\% da massa e carrega em seu perfil
quase todos elementos. O fator que está ofuscando os demais fatores
tanto no \textbf{PMF} quanto na \textbf{Análise de Fatores} está relacionado
a poeira do Harmatão. 
Assim, para melhor identificar as fontes locais, daqui em diante basearemos 
nossa investigação de fontes excluindo-se os dias de ocorrência de ventos do 
Harmatão, como proposto por \cite{aboh2009}. Entretanto, análise \textbf{PMF} e 
\textbf{Análise de Fatores} para as tabelas de concentrações completas podem
ser consultadas no apêndice II. 

\begin{figure}[H]
\centering
  \includegraphics[width=0.5\textwidth]{../outputs/RFcH_pmf_contribution_pizza5.pdf}
  \caption{Contribuição dos fatores na massa total para $MP_{2,5}$ na área
           residencial. 5 fatores extraídos.
          \label{table:RFcH_contribution5}}
\end{figure}

\begin{table}[H]
  \centering
    \input{../outputs/RFcH_profiles_percent_species5.tex}
    \caption{Perfis do fatores na área residencial $MP_{2,5}$ 
             (seed=123, n= 197). 
             \label{table:RFcH_profiles5}}
\end{table}



Refazendo-se a \textbf{Análise de Fatores} de $MP_{2,5}$ na área residencial, 
porém excluindo-se os dias de Harmatão (número de amostra cai de 197 para 123), 
obtém-se os \textbf{loadings} da tabela \ref{table:loadings_RFsH6}. 

\begin{table}[H]
  \input{../outputs/loadings_RFsH5.tex}
  \caption{Análise de Fatores para $MP_{2,5}$ na região residencial
           excluindo-se dias de ocorrência do Harmatão.
           Rotação varimax - 5 fatores retidos (n=123).
           (\textcolor{red}{h} : Comunalidade; 
           \textcolor{red}{S=1-h} : Singularidade; 
           \textcolor{red}{C} : Complexidade.)
           \label{table:loadings_RFsH5}}
\end{table}

Apesar do fator com elementos de poeira continuar prepoderante, agora, os 
elementos estão melhor distribuídos entre os demais fatores.  

A \textbf{Análise de Fatores} para $MP_{2,5}$ na área residencial explicou 
$84,0\%$ da variância total dos dados com 5 fatores retidos.
A comunalidade, variável que indica o quão o ajuste explicou a variabilidade por 
espécie, foi maior que $0,7$ para a maioria dos elementos,
com exceção do bromo (Br) e zinco (Zn) que tiveram comunalidade de $0,59$ e 
$0,44$, respectivamente.

O fator predominante, isto é, que explica sozinho a maior parte da variância, 
tem autovalor $7,88$ ou $44,0\%$ da variância total.
Os elementos com maiores \textbf{loadings} neste fator são essencialmente 
metais (Al, Si, Ti, V, Fe, Mn, Ca, Mg) e massa (\textbf{mass}).
São oriundos de re-suspensão de poeira do solo e com exceção do Vanádio,
são comumente encontrados em diversos tipos de solos, como os disponíveis no 
banco de dados \textbf{SPECIATE} da \textbf{EPA-US} \citep{simon2010}.
Neste fator, aparece secundariamente fósforo (P) e potássio (K), indicando 
possível sobreposição de fontes ou solo contaminado. 
Vanádio, normalmente associado a queima de óleos pesados como o diesel, ao 
aparecer na fonte solo, indica sobreposição de fontes com temporalidade próximas.
\cite{aboh2009} também encontrou Vanádio no fator de poeira solo.
As partículas que compõem poeira de solo estão em sua maioria na moda grossa 
($MP_{2,5-10}$) do material particulado, pois são geradas por processo mecânico,
entretanto, tamanha a quantidade de poeira em Acra que mesmo no $MP_{2,5}$ o 
principal fator encontrado representa solo.  

O segundo fator predominante, de autovalor $2,22$, contém alto \textbf{loadings}
de fósforo (P), potássio (K) e enxofre (S), espécies típicas de queima de 
biomassa \citep{reid2005}, ainda hoje pratica comum em Acra para preparação 
de alimentos. 

O terceiro fator retido, de autovalor $2,02$, é composto por Black Carbon (BC), 
chumbo (P), zinco (Zn), potássio (K) e massa.
\cite{aboh2009} encontrou o mesmo fator em Kwabenya e o considerou como fonte
veicular, pois $BC$ e $Zn$ estão associados a veículos, devido a combustão 
incompleta e desgaste das pastilhas de freio e dos pneus dos automóveis, 
respectivamente.

O chumbo (Pb) foi banido da gasolina em Gana em 2003 devido a acordo 
internacional \citep{epa2015}.  
Na tabela \ref{table:RFcH_descriptive} a concentração média de $Pb$ 
foi $18,6 \pm 0,8 n g /m^3$, próxima de valores típicos de São Paulo 
$16 \pm 13 n g /m^3$ \citep{andrade2012}.

Entretanto, Acra tem um dos maiores lixões de eletrônicos (\textbf{e-waste}) 
do mundo, recebendo grande parte dos equipamentos descartados como lixo na 
Europa e que são derretidos para obtenção do cobre (Cu) em \textbf{Agbogbloshie},
bairro de Acra a 4 kilometros sudoeste de Nima.  

Altas concentrações de Al, Co, Cu, Zn, Cd, In, Sb, Ba, e Pb foram encontradas
no solo do \textbf{e-waste} de \textbf{Agbogbloshie} \citep{asante2012}. 
O zinco (Zn) é o principal composto de baterias de eletrônicos. 

O vento predominante em Nima é de sudoeste e portanto pode carregar material 
de \textbf{Agbogbloshie} para Nima, distante apenas 4 kilômetros. 
Assim, é possível que no fator 3 esteja representada duas fontes, a veicular e 
e queima de lixo eletrônico, pois as duas emitem grande quantidade de partículas
na atmosfera do mesmo tipo. 

O quarto fator predominante, de autovalor $1,59$, com Sódio (Na) e Cloro (Cl)
representa indubitavelmente a fonte mar. 
Partículas de Sódio (Na) e Cloro (Cl) geradas no mar estão mais concentradas 
na moda grossa, mas pela proximidade do ponto de amostragem com mar, obteu-se
concentrações suficientes que permitiram a geração de um fator. 
Em pouco tempo de residência na atmosfera, o Cloro do sal marinho envelhecido 
é substituído por $SO_4^{2-}$ como resultado da reação com ácido sulfúrico e 
ácido nítrico \citep{mcinnes1994}, o que explica o \textbf{loading} $0,29$ 
neste fator.

O quinto e último fator predominante retido teve autovalor $1,33$. 
Os elementos com altos \textbf{loadings} foram Vanádio, Sódio, Bromo e Chumbo.
Esse elementos representam atividades de indústrias locais. 

Na avenida, o resultado da \textbf{Análise de Fatores} de $MP_{2,5}$ está na 
tabela \ref{table:loadings_TFsH5}. Os 4 primeiros fatores principais extraídos 
foram essencialmente os mesmos que na área residencial, com mudanças nos 
\textbf{loadings} de alguns elementos. O último fator, entretanto, de autovalor
$1,29$ possui BC e K, diferente do quinto fator da área residencial. 
Na avenida há muitos comércios   


\begin{table}[H]
  \input{../outputs/loadings_TFsH5.tex}
  \caption{Análise de Fatores para $MP_{2,5}$ na avenida
           excluindo-se dias de ocorrência do Harmatão.
           Rotação varimax - 5 fatores retidos (n=123).
           (\textcolor{red}{h} : Comunalidade; 
           \textcolor{red}{S=1-h} : Singularidade; 
           \textcolor{red}{C} : Complexidade.)
           \label{table:loadings_TFsH5}}
\end{table}

---
\begin{table}[H]
  \input{../outputs/loadings_TGsH4.tex}
  \caption{Análise de Fatores para $MP_{2,5-10}$ na avenida
           excluindo-se dias de ocorrência do Harmatão.
           Rotação varimax - 5 fatores retidos (n=123).
           (\textcolor{red}{h} : Comunalidade; 
           \textcolor{red}{S=1-h} : Singularidade; 
           \textcolor{red}{C} : Complexidade.)
           \label{table:loadings_TGsH5}}
\end{table}

The major sources of phosphorus air pollution are the

industries engaged in the production of phosphate fertilizers,
phosphoric acid, phosphorus pentoxide, and phosphorus chemicals

for industrial uses.

 Recently, various organic phosphorus compounds have been

sed as combustion chamber deposit modifiers as well as corro-
sion inhibitors for motor gasolines and aviation fuels.  Thus
there is a possibility that automobile emissions may contain
toxic phosphorus compounds.
     Phosphorus concentrations in the ambient air of Los

Angeles (1954) and Cincinnati (1966) were found to average
1.43 |J,g/m3 , and 0.22 |ag/m3 , respectively.  Measurements on
national or regional scales are not available.



\begin{table}[H]
  \centering
    \input{../outputs/RFsH_profiles_percent_species5.tex}
    \caption{residencial $MP_{2,5}$ removendo-se os dias do Harmatão 
              seed=123; n= 123. 
             \label{table:RFsH_profiles5}}
\end{table}

\begin{figure}[H]
\centering
  \includegraphics[width=0.6\textwidth]{../outputs/RFsH_pmf_contribution_pizza5.pdf}
  \caption{Contribuição dos fatores na massa total para $MP_{2,5}$ na área
           residencial . 5 fatores extraídos.
          \label{table:RFsH_contribution5}}
\end{figure}

-------------------------

\begin{table}[H]
  %\label{my-label} 
  \caption{Análise de Fatores para $MP_{2,5}$ na região residencial.
           Rotação varimax - 5 fatores retidos.
           (\textcolor{red}{h} : Comunalidade; 
           \textcolor{red}{S=1-h} : Singularidade; 
           \textcolor{red}{C} : Complexidade.)}
  \input{../outputs/loadings_RFsH5.tex}
\end{table}


\begin{table}[H]
  %\label{my-label} 
  \caption{\textbf{Análise de Fatores com rotação varimax - 4 fatores retidos} 
             para $MP_{2,5}$ na avenida.
           (\textcolor{red}{h} : Comunalidade; 
           \textcolor{red}{S=1-h} : Singularidade; 
           \textcolor{red}{C} : Complexidade.)}
  \input{../outputs/loadings_TFsH4.tex}
\end{table}

Ainda para $MP_{2,5}$, mas na avenida com intenso movimento de veículos,
a Análise de Fatores tamém explicou $84,0\%$ da variância total 
dos dados com 4 fatores retidos.

O fator predominante, de autovalor $9,87$, ressuspensão de solo, continua 
como os mesmos elementos (Al, Si, Ti, V, Fe, Mn, Ca, Mg, P, K, Cl, massa).

O segundo fator predominante, de autovalor $1,87$, além dos elementos que 
apareceram no barro residencial - fósforo (P), potássio (K) e enxofre (S) -
também contém black carbon (BC) com alto loading $(0,72)$. 
Esse fator continua caracterizando queima de biomassa.

O terceiro fator predominante, autovalor $1,75$, composto por
potássio (K), enxofre (S), bromo (Br), chumbo (P) e zinco (Zn). 
Esse fator está associado a emissões veiculares. 

O quarto e último fator predominante, autovalor $1,65$, 
Nódio (Na), cloro (Cl) e bromo (Br) e enxofre (S).

A associação entre fatores e fontes poluídoras pode ser resumida
assim: \textbf{solo, queima de biomassa, mar e lixo.}

\begin{table}[H]
  \centering
  \caption{Associação de fonte de poluídoras na \textbf{Análise de Fatores}
         para $MP_{2,5}$ na região residencial}
  \input{../outputs/briefFA_RFsH4.tex}
\end{table}

Na Análise de Fatores por exemplo, as estimativas de fatores 
mantendo os dias de ocorrência do Harmatão, resultam em geral em apenas 
1 fator predominante, ficando os outros fatores com autovalores
menores do que 1 e degenerados. Neste cenário, a fonte 
\textbf{poeira de solo} predomina e fica praticamte impossível
detectar as outras fontes. Assim, só apresentaremos aqui os resultados 
removendo-se os dias do Harmatão. Mas a Análise de Fatores com os
dias do Harmatão inclusos estão no estão no Apêndice I.

Diversas parametrizações foram testadas na \textbf{PMF}
e aquelas que resultaram em fatores que se associavam com fontes poluidoras
reais foram escolhidas. A seguir estão os resultados para 
\textbf{Análise de Fatores} nos dois pontos de amostragem 
(residencial e avenida) e nas duas modas
fina ($MP_{2,5}$) e grossa ($MP_{2,5-10}$).

%%%%
\subsection{$MP_{2,5-10}$ na região residencial e avenida}

\begin{table}[H]
  %\label{my-label} 
  \caption{\textbf{Análise de Fatores com rotação varimax - 4 fatores retidos} 
             para $MP_{2,5}$ na avenida.
           (\textcolor{red}{h} : Comunalidade; 
           \textcolor{red}{S=1-h} : Singularidade; 
           \textcolor{red}{C} : Complexidade.)}
  \input{../outputs/loadings_TFsH4.tex}
\end{table}

\begin{table}[H]
  %\label{my-label} 
  \caption{\textbf{Análise de Fatores com rotação varimax - 4 fatores retidos} 
             para $MP_{2,5}$ na avenida.
           (\textcolor{red}{h} : Comunalidade; 
           \textcolor{red}{S=1-h} : Singularidade; 
           \textcolor{red}{C} : Complexidade.)}
  \input{../outputs/loadings_TFsH4.tex}
\end{table}

%%%%
\section{Positive Matrix Factorization}
Avaliou-se o perfil de parâmetros meteorológicos locais, como direção e intensidade dos ventos, 
identificando-se um regime de brisa marinha bem definido, com ventos de sul quando sol a pino, mas se deslocando a oeste com a entrada da noite. 

I assume you are talking about the paper "Chemical composition and sources of particle pollution in affluent and poor

neighborhoods of Accra, Ghana". Both Pb and Zn were included in the PMF analysis. Based on our analysis, they showed up in two factors: Road dust and traffic particles and solid waste burning. As you may already know, the choice of the number of PM sources and labeling PM sources involve subjectivity. Since we were comparing four neighborhoods, we kept sources more or less consistent across neighborhoods. In your case, it might be reasonable to further break down the sources.

\begin{figure}[H]
  \centering
  \includegraphics[width=1\textwidth]{../outputs/windRose_horaria.pdf}
  \caption{ \citep{carslaw2012} \label{fig:windRose_horaria}}
\end{figure}

\begin{figure}[H]
  \centering
  \includegraphics[width=1\textwidth]{../outputs/windRose_mensal.pdf}
  \caption{ \citep{carslaw2012} \label{fig:windRose_mensal}}
\end{figure}

%Nota-se também uma perceptível diferença entre os dois períodos climáticos, 
%pois no Verão temos maior quantidade de radiação solar, fortalecendo a 
%formação de brisa marinha e, consequentemente, deslocando mais para o 
%leste a distribuição de frequência nas direções dos ventos.

\begin{figure}[H]
  \centering
  \includegraphics[width=0.5\textwidth]{../outputs/windRoseNoaaHarvard.pdf}
  \caption{Rosa do ventos entre
           Setembro de 2006 e Junho de 2008. Utilisou-se dados 
           do \textbf{Kotoka International Airport} de Acra \label{fg:rosaCompleta}}
\end{figure}

Fonte de energia para cozimento de alimentos em Gana \ref{table:cookfuel}.
\begin{table}[H]
 \centering
  \input{../outputs/census_cookfuel.tex}
  \caption{Fontes de energia usadas para cozimento de alimentos em 
           Gana \citep{ghanacensus2013} \label{table:cookfuel}}
\end{table}

No caso de Accra a discriminação de fontes é complexa pois
há uma mistura de elementos do deserto com fontes antropogênicas.

Apesar dos altos índices de poluição em Accra pouca pesquisa tem sido 
realizada na tentativa de enteder a composição química do ar na região
(tanto antropogênica quando natural - poeira mineral do deserto). 

\subsection{$MP_{2,5}$ na região residencial e avenida}

Os perfis dos fatores para $MP_{2,5-10}$ nos dois sítios de amostragem, 
residencial e avenida, estão na tabela \ref{table:grosso_profiles_percent_species}.

A contribuição por fator e a respectiva associação com fontes poluídoras
estão no gráfico da figura \ref{figure:grosso_pmf_contribution_pizza}. 

\citep{asante2012} analisou nível de elementos traços nas urinas de trabalhadores 
do e-waste de Agbogbloshie e Fe, Sb, and Pb tiveram concentrações alta se comparada
a um pessoa de referência, que não tem contato com \textbf{e-waste}.



\citep{kaku2016}

\citep{prospero2002} mostra países afetados por harmatão. 

\citep{engelbrecht2009a} e \citep{engelbrecht2009b} estudaram a composição 
da poeira do deserto.




\begin{table}[H]
  \centering
  %\label{my-label} 
  \caption{RGsH}
  \input{../outputs/RGsH_profiles_percent_species4.tex}
\end{table}

\begin{table}[H]
  \centering
  %\label{my-label} 
  \caption{TGsH}
  \input{../outputs/TGsH_profiles_percent_species4.tex}
\end{table}


\begin{table}[H]
  \centering
  %\label{my-label} 
  \caption{TFsH}
  \input{../outputs/TFsH_profiles_percent_species4.tex}
\end{table}
O Harmatão foi presente tanto em MP10 quanto em MP2.5.


%%# Fontes encontradas em Harvard:
%%#   Solid waste burning: Br. 
%%#   Road dust & vehicle: Al, Si, Ca, Fe, Zn, BC.  
%%#   Crustal: Al, Si, Mg, Ti, Mn, Fe.
%%#   Aged biomass particles: K, Cl, S, BC
%%   Fresh biomass burning: K, Cl, S, BC
%%#   Sea salt: Na, Cl, S
%%# Sal marinho, solo, emissões veiculares e combustão de biomassa
%%# Zn/Cu podem representa veículos por causa do mencanismo interno do carro. 
%%# Zn: Freio, pneu, peças.
%
%%\begin{figure}[H]
%%\centering
%%\includegraphics[width=\textwidth]{../outputs/AFCH_factor_scores.pdf}
%%\caption{AFCH factor scores}
%%\label{fig:AFCH_factor_scores}
%%\end{figure}
%
%%Aumento do silicio no harmatan.
%
%%Zheng: 
%%Possíveis fontes: ressuspensão da ruas não pavimentadas
%%biomassa: K,Cl,S
%
%%massa,Al,Si,K,Ca,Fe
%%K e cl se relacionam
%%K e S no período não harmathan 
%%Na e Cl se descorrelacionam no harmathan
%%Br, Zn, Pb no não harmathan (gasolina-Br e Pb) Pneu:Zn
