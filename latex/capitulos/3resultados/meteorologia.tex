%%%%
\section{Meteorologia local e Harmatão}

Para melhor entendimento das fontes poluídoras, analisa-se a seguir a 
dinâmica de ventos a partir de dados da estação meteorológia do 
aeroporto de Acra (Kotoka International Airport) disponibilizados no site da 
National Oceanic and Atmospheric Administration - United States Department of 
Commerce (NOAA). As rosas dos ventos foram plotadas usando a biblioteca 
openair desenvolvida por \citet{carslaw2012}.

\begin{figure}[H]
  \centering
  \includegraphics[width=0.5\textwidth]{../outputs/windRose2007.pdf}
  \caption{Rosa do ventos para Acra/2007. \label{fg:rosaCompleta}}
\end{figure}

A figura \ref{fg:rosaCompleta} mostra a distribuição de frequência da direção 
dos ventos em conjunto como a intensidade para o ano de 2007, sendo a direção 
predominante de origem dos ventos de superfície sudoeste.

Acra é uma região litorânea e na observação horária da direção e intensidade 
dos ventos, gráfico da figura \ref{fig:windRose_horaria}, identifica-se 
forte componente regional para o vento local, associável à brisa marinha, 
com ventos próximos da direção sul (do oceano) quando o sol encontra-se alto, 
mas deslocando-se 
à oeste na medida que as horas avançam (deslocamento à direita do sentido do 
movimento - ação típica do efeito de Coriolis no hemisfério Norte).

\begin{figure}[H]
  \centering
  \includegraphics[width=\linewidth]{../outputs/windRose_horaria.pdf}
  \caption{Rosa do ventos horária para Acra/2007. \label{fig:windRose_horaria}}
\end{figure}

Mensalmente, figura \ref{fig:windRose_mensal}, nota-se perceptível diferença 
entre os dois períodos climáticos, pois no verão temos maior quantidade de 
radiação solar, fortalecendo a formação de brisa marinha e, consequentemente, 
havendo maior tempo para a velocidade do vento intensificar-se e para 
processar-se um deslocamento para oeste (Coriolis).

Na maior parte do verão, Gana localiza-se, em termos de padrão global de 
circulação, no hemisfério sul, ou seja, abaixo da Zona de Convergência 
Intertropical (ZCIT). Neste período as intensidades de vento são maiores e 
são menores as frequências de calmarias. No período do inverno, Gana 
posiciona-se a norte da ZCIT, e observa-se com isso maior incidência de 
vento norte (particularmente o mês de janeiro), 
com velocidades médias do vento um pouco menores e maior percentual
de calmaria. É nesta época que Gana e o deserto do Saara situam-se no mesmo 
sistema de circulação global, ocorrendo o Harmatão. 

%%%%
\subsection{Circulação Global}

\begin{figure}[H]
  \centering
  \includegraphics[width=\linewidth]{../outputs/windRose_mensal.pdf}
  \caption{Rosa do ventos mensal para Acra/2007. \label{fig:windRose_mensal}}
\end{figure}

\begin{figure}[H]
  \centering
  \begin{subfigure}[b]{0.5\linewidth}
    \includegraphics[width=\linewidth]{../inputs/grads/gimp/875hPa/DEZ_2007.pdf}
    \caption{Dezembro de 2007}
  \end{subfigure}%
%  \hspace{0.5cm}
  \begin{subfigure}[b]{0.5\linewidth}
    \includegraphics[width=\linewidth]{../inputs/grads/gimp/875hPa/JAN_2008.pdf}
    \caption{Janeiro de 2008}
  \end{subfigure}
  \caption{Intensidade e direção do vento médio na altitude de 1000 metros 
           sobre o continente Africano. \label{fig:ECMWF1000}}
\end{figure}

\begin{figure}[H]
  \centering
  \begin{subfigure}[b]{0.5\linewidth}
    \includegraphics[width=\linewidth]{../inputs/grads/gimp/1000hPa/DEZ_2007.pdf}
    \caption{Dezembro de 2007}
  \end{subfigure}%
%  \hspace{0.5cm}
  \begin{subfigure}[b]{0.5\linewidth}
    \includegraphics[width=\linewidth]{../inputs/grads/gimp/1000hPa/JAN_2008.pdf}
    \caption{Janeiro de 2008}
  \end{subfigure}
  \caption{Intensidade e direção do vento médio na superfície (10 metros) no 
           continente Africano. \label{fig:ECMWF10} }
\end{figure}

Na rosa do ventos para o ano de 2007 (figura \ref{fg:rosaCompleta}) nota-se 
que praticamente não há ventos de norte e nordeste, mesmo no inverno, 
figura \ref{fig:windRose_mensal}, quando Gana está no hemisfério norte em 
termos da circulação global, a frequência de ventos nordeste na superfície 
continua baixa. Como descrito por \citet{breuning2005}, o Harmatão
é um vento nordeste de altas altitudes (~1000 m) que interfere pouco no
no predomínio da brisa marinha da circulação local.  

Para contraste dos ventos médios em altas altidudes com os de superfície, 
dados de re-análise disponibilizados pela ECMWF de dezembro de 
2007 e janeiro de 2008 foram utilizados. Com recorte na longitude entre 2
0 oeste e 60 leste e latitude de 30 sul a 40 norte, selecionou-se assim o 
continente Africano e parte dos oceanos atlântico e índico. 
Os mapas da figura \label{fig:ECMWF10} mostram
ventos próximos a superfície e os da \label{fig:ECMWF1000} os ventos na altitude 
próxima a 1000 metros para o mesmo período. 
Na superfície, é possível notar o regime de brisa marinha, com entrada de vento 
do mar. Em 1000 metros, os ventos de nordeste são muitos mais intensos e 
frequentes que na superfície, caracterizando a presença do Harmatão. 
