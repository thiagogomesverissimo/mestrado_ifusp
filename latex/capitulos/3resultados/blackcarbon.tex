%%%%
\section{Black Carbon}

Medidas de BC pela técnica de refletância apresentam a vantagem de poderem ser 
realizadas sobre os mesmos filtros empregados para análises gravimétricas, 
XRF ou mesmo em técnicas destrutivas que posteriormente possam vir a ser 
realizadas sobre eles, como cromatografia iônica ou espectroscopia de massa. 
Isso promove a redução das incertezas provenientes de análises que pedem 
diferentes amostradores coletados em paralelo. Também pode diminuir 
significativamente a dimensão da instrumentação em uma pesquisa. Entretanto, 
a refletância é uma medida indireta, cujo resultado depende das características 
do aerossol amostrado. Utilizando o método absoluto TOT para algumas amostras, 
foi possível intercalibrar as medidas de refletância com esse método absoluto, 
permitindo a quantificação do BC nas demais amostras.

%%%%
\subsection{Calibração da refletância usando BC Monarch 71}

Alvos padrões produzidos no laboratório do antigo GEPA-IFUSP (Grupo de Estudos 
de Poluição do Ar), hoje Laboratório de Física Atmosférica (LFA-IFUSP), 
com concentrações conhecidas do BC de referência Monarch 71 (M71) 
\citep{clarke1986}, foram usados por décadas em equipamentos de refletância 
alocados nestes laboratórios e no LAPAt, para a calibração de refletância versus
densidade superficial de BC em filtros que coletaram aerossol atmosférico.

A tabela \ref{table:monarch71} mostra um conjunto de valores medidos em 2007 
para estes padrões (com respectivas médias e incertezas), sobre os quais foi 
feita uma calibração no Lapat, seguindo exatamente os mesmos procedimentos do 
GEPA/LFA-IFUSP. Os novos parâmetros então ajustados ofereciam resultados 
apenas 2,3\% menores que a calibração anterior (1994), do GEPA/LFA-IFUSP. 
Isso revela equipamentos uniformes e grande estabilidade destes padrões. 
Entretanto, essa sistemática de calibração padecia de três problemas.

O primeiro deles refere-se à estreita faixa de valores oferecidas por estes 
padrões que, como pode ser visto na referida tabela, variam entre 30\% e 85\%, 
não cobrindo a região acima de 85\% ou abaixo de 30\%, esta última onde 
posicionam-se os filtros carregados que encontramos no experimento de Gana. 
Mesmo assim, se fosse considerada a linearidade que predomina na relação 
entre densidade de BC e log da refletância, equação \ref{m/a_2}, e acrescida 
no ajuste a massa zero, para qual a refletância é 100\% ($log_{10}$100 = 2), 
poder-se-ia estender o ajuste para uma faixa apreciável de valores.
Mas a função adotada para os ajustes foi uma parábola. Esse foi um segundo 
problema para esta calibração, como se pode verificar na figura 
\ref{fig:monarch71}, essa função distancia-se bastante de um ajuste linear 
conforme nos deslocamos em direção a $log_{10}$ 10, ponto em que a relação da 
massa com o log da refletância começa a perder a linearidade. O gráfico da 
figura \ref{fig:razaoTOTM71} mostra que a margem de erro em alvos carregados 
cresce com o ajuste de 2$\degree$ grau, quando confrontado com a calibração 
baseada em TOT, para filtros coletados em Gana. Note-se, entretanto, que na 
zona de refletância onde havia alvos padrões, as diferenças com o método 
absoluto, por óbvio, são semelhantes. 

A calibração do refletômetro do LAPAt para 2007 usando alvos padrões M71 com 
ajuste de primeiro grau está no gráfico da figura \ref{fig:monarch71}, com 
respectivos dados apresentados na tabela \ref{table:monarch71}.

\begin{figure}[H]
  \centering
  \includegraphics[width=0.9\textwidth]{../outputs/BC_monarch71.pdf}
  \caption{Calibração do refletômetro do LAPAt em 2007 usando alvos padrões M71.
         \label{fig:monarch71}}
\end{figure}

\newpage
\begin{table}[H]
  \centering
  \small
    \input{../outputs/BC_monarch71.tex}
    \caption{Calibração do refletômetro do LAPAt em 2007 usando alvos padrões 
             M71. \label{table:monarch71}}
\end{table}

\newpage
Procuramos construir uma calibração mais consistente, produzindo alvos de 
calibração que cobrissem toda a faixa de valores entre 0 a 100\% de refletância. 
Mas não encontramos mais o BC M71 para comercializar, assim usamos o BC de 
referência (ASTM-N762) do IPT com o qual produzimos 27 padrões depositados em 
filtros de policarbonato (nuclepore fino de 37 mm, orifícios de 0,4 $\mu m$) 
usando amostrador dicotômico (diâmetro de corte em 2,5 $\mu m$) na câmara de 
ressuspensão do laboratórios da Divisão de Qualidade do Ar da CETESB.

Os valores de refletância nesses filtros variaram de 3,1\% até 98,7\%, mais um 
"zero" de massa teórico com refletância 100\%. As massas foram medidas em 
microbalança analítica $\pm$ 1 g. A figura \ref{fig:bc_cetesb} mostra o ajuste 
entre massa de BC depositada versus o log da refletância medida. Pode-se ver 
que o ajuste linear foi excelente,
indicando que a calibração deve perder linearidade apenas em refletâncias 
inferiores a 5\%.

\begin{figure}[H]
	\centering
	\includegraphics[width=0.7\linewidth]{../outputs/BC_cetesb.pdf}
	\caption{Reflêtancia e pesagem dos padrões produzidos na CETESB com BC 
                 de referência ASTM-N762. \label{fig:bc_cetesb}}
\end{figure}

\newpage
\begin{table}[H]
	\centering
	\small
	\input{../outputs/cetesb2012.tex}
	\caption{Reflêtancia e pesagem dos padrões produzidos na CETESB com BC 
                 de referência ASTM-N762. \label{table:bc_cetesb}}
\end{table} 



As barras de erros nas massas correspondem ao erro efetivo $\sigma_{efetivo}$, 
o qual é a adição (soma dos quadrados) do erro propagado da refletância (eixo x)
com o erro de medida da massa (eixo y). Como a relação entre x e y é linear, 
a incerteza em y devido a x pode ser escrita como:

\begin{equation}
  {\sigma_y}_x = \frac{\partial y}{\partial x} \cdot \sigma_x
\end{equation} 

Usada na soma dos quadrados para calcular a incerteza efetiva em y, 
$\sigma_{efetiva}$: 

\begin{equation}
  \sigma_{efetiva} = \sqrt{{\sigma_y}^2 + {{\sigma_y}_x}^2}
\end{equation} 

Como a menor medida na leitura do aparelho digital de refletância é 0,1\%
(o \% aqui é a unidade de medida para refletância) é muito baixa, estimamos 
a incerteza na refletância a partir da oscilação na alvura do substrato de 
amostragem calculando o desvio padrão para sete amostras 
brancas ($\sigma$=0,957 \%). Aplicamos correção de fator 1,11 devido ao 
espaço amostral pequeno \citep{helene1981}, resultando em $\sigma$=1,06 \%.

Infelizmente, como veremos a seguir, no confronto com TOT, o BC utilizado 
mostrou ter características bastante diversas do BC encontrado regularmente 
na atmosfera, introduzindo um erro sistemático da ordem de um fator 5.  


%%%%
\newpage
\subsection{Calibração com TOT - Experiência nos túneis Jânio Quadros e Rodoanel}

Em maio de 2011, simultaneamente ao período de análises das amostras de Gana, 
conduzia-se experimento para analisar as emissões veiculares na cidade de São 
Paulo. As medidas foram feitas no interior dos túneis Jânio Quadros e Rodoanel,
que têm características de circulação predominante de veículos leves e pesados 
(ou diesel), respectivamente. 

Tomou-se amostragens simultâneas de $MP_{2,5}$, por amostrador MiniVol com 
filtros de quartzo e por amostrador Partisol com filtros de policarbonato, 
sendo o primeiro para medida absoluta de EC por TOT e o segundo para análises
de XRF-ED, medida relativa de BC por refletância e para cromatografia iônica.

Tamanha era a poluição dos caminhões que os filtros coletados no Rodoanel 
ficaram saturados para medidas de BC por refletância (valores próximos a zero), 
o que impossibilitou usá-los para fins de calibração. No túnel Jânio Quadros as 
refletâncias variaram de 4,9 a 57,1\%, permitindo a calibração da refletância 
por TOT.

A determinação BC por TOT nos filtros de quartzo foi conduzida pelo Dr. Pierre 
Herckes, do Departamento de Química e Bioquímica da Universidade Estadual do 
Arizona. Contou-se com 16 filtros coletados no túnel Jânio Quadros. 
O gráfico da figura \ref{table:interJQ} apresenta a intercalibração com ajuste 
linear entre as medidas de TOT nas amostras dos filtros de quartzo e a 
refletância das 16 amostras correspondentes nos filtros de policarbonato.
Nota-se a boa qualidade do ajuste linear de calibração da densidade superficial 
de massa de BC versus o log da refletância por TOT no túnel Jânio Quadros. 

\begin{figure}[H]
  \centering
  \includegraphics[width=0.5\linewidth]{../outputs/JQ_TOT_Refletancia.pdf}
  \caption{Intercalibração TOT e Reflêtancia para amostragem paralela no 
           túnel Jânio Quadros. \label{table:interJQ}}
\end{figure}

Buscou-se, também, confrontar a massa obtida pela calibração com os  
alvos de BC comercial elaborados no laboratório da CETESB, com a massa 
obtida pela calibração via TOT, gráfico da figura \ref{fig:JQ} e tabela
\ref{table:JQ}. Apesar da excelente relação linear de ambas calibrações, os 
valores de massa calculados diferiram em aproximadamente um fator 5, sendo as 
massas calculadas com a calibração da CETESB maiores que a por TOT.  
Como já mencionamos, isso significa que o BC comercial (ASTM-N762) tem 
propriedades diferentes daquele BC que compõe o aerossol coletado no túnel do 
Jânio Quadros, provavelmente, sua granulometria seria mais grossa, o que gera 
menos superfície de absorção por muito mais massa coletada. Neste sentido, ele 
não seria indicado para calibrações realistas de equipamentos de refletância. 

\begin{figure}[H]
  \centering
  \begin{minipage}[b]{0.5\linewidth}
    \includegraphics[width=\textwidth]{../outputs/BC_janio_quadros.pdf}
    \caption{Comparação da massa de BC das amostras do túnel Jânio Quadros 
             calculada usando calibração por TOT versus calibração a partir dos 
             alvos padrões produzidos na CETESB. \label{fig:JQ}}
  \end{minipage}
  \hspace{0.5cm}
  \begin{minipage}[b]{0.45\linewidth}
    \begin{small}
      \input{../outputs/BC_janio_quadros.tex}
    \end{small}
    \captionof{table}{Comparação da massa de BC das amostras do túnel Jânio Quadros 
             calculada usando calibração por TOT(1) versus calibração a partir dos 
             alvos padrões produzidos na CETESB(2). \label{table:JQ}}
  \end{minipage}
\end{figure}

\newpage
%%%%
\subsection{Intercalibração de TOT e refletância para Acra}

Como discutido na descrição da metodologia analítica para BC, as características
de tamanho, forma geométrica e seção de choque de absorção de luz das 
partículas de BC variam de região para região.
Por esses motivos, é importante fazer a comparação de método indireto com método
direto \citep{quincey2007}. \citet{taha2007} também aponta que há uma limitação 
da técnica de refletância quando filtros de teflon são usados, pois em altas 
concentrações (refletâncias inferiores a 20\%), ocorre saturação, e a 
relação exponencial da equação \ref{m/a_2} passa a não valer 
mais. 

Dos 2898 filtros PTFE coletados no experimento de Gana, 52 tiveram medidas 
paralelas em filtros de quartzo para $MP_{2,5}$, analisados por TOT.
Os filtros coletados em Acra apresentaram refletância pequena, pois além da 
alta poluição atmosférica o tempo de amostragem de 48 horas contribuiu para que
as amostras ficassem muito carregadas. Buscou-se então proceder como no 
experimento do túnel Jânio Quadros em São Paulo, intercalibrando a refletância 
com as medidas de TOT disponíveis, de forma a estimar os valores de BC em todos 
filtros PTFE de $MP_{2,5}$. O gráfico da figura \ref{fig:interGanaBC} e a 
tabela \label{table:interGanaBC} mostram o ajuste realizado, já contabilizando 
um ponto de massa zero no ajuste. A incerteza efetiva considera o erro do TOT 
e da refletância, além dos 8\% de amostragem paralela. 

Assumimos como incerteza para os valores de refletância (medido em porcentagem)
o desvio padrão corrigido das medições de 7 filtros em branco 
(média = 0,96\% x 1,11 = 1,06\%), corrigidos para as estatísticas de poucos 
dados, utilizando 7 valores). %A incerteza em massa acumulada foi de $\sqrt{2}$.

O gráfico da figura \ref{fig:razaoTOTM71} mostra a relação entre a calibração do 
BC por TOT e o resultado que teríamos se empregássemos a calibração 
tradicionalmente feita com M71. Neste último caso o ajuste foi feito com uma
reta, como diz a relação derivada para absorção, e uma curva parabólica, 
como usualmente empregada no laboratório. Na região onde existem alvos padrões, 
os ajustes de primeiro e segundo grau apresentam concordância com os valores 
obtidos por TOT, mas em altas concentrações, o ajuste linear proveria resultados
mais próximos daqueles obtidos por calibração TOT, mesmo assim com diferenças 
em torno de 25\%. 

Fazendo-se a mesma comparação para filtros analisados em Recife \citep{santos2014},
as diferenças oscilavam entre um fator 1,4 e 2,4 (figura \ref{fig:BC_compara_recife}).

O uso de refletância mostrou-se bastante eficiente para determinação de BC, 
sendo aconselhável, entretanto, prover a calibração de um conjunto de filtros 
com um sistema de medida absoluto como o TOT.


\begin{figure}[H]
	\begin{center}
		\includegraphics[width=0.9\textwidth]{../outputs/Gana_TOT_Refletancia.pdf}
		\caption{Intercalibração entre TOT e refletância em Acra. \label{fig:interGanaBC}}
	\end{center}
\end{figure}

\begin{figure}[H]
	\centering
	\begin{subfigure}[b]{0.43\linewidth}
		\includegraphics[width=\linewidth]{../outputs/BC_compara_calibs.pdf}
		\caption{Acra \label{fig:razaoTOTM71}}
	\end{subfigure}
		\hspace{0.3cm}
	\begin{subfigure}[b]{0.43\linewidth}
		\includegraphics[width=\linewidth]{../outputs/BC_compara_calibs_recife.pdf}
		\caption{Recife \label{fig:BC_compara_recife}}
	\end{subfigure}%

	\caption{Razão dos valores de BC medidos por refletância e calibrados por 
		TOT e M71 para Recife e Acra. \label{fig:BC_compara}}
\end{figure}

\newpage
\begin{table}[H]
	\centering
	\footnotesize 
	\input{../outputs/Gana_TOT_Refletancia.tex}
	\caption{Intercalibração entre TOT e refletância para Acra. \label{table:interGanaBC}} 
\end{table} 
\newpage


