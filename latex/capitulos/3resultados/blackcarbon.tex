%%%%
\section{Black Carbon}


O texto da EPA
é bem ilustrado. Em geral o nome do arquivo dá uma idéia do que tem
dentro. No caso dos arquivos da Atmospheric Environment eu os indico com
a&e e ano (trabalhos do Quincey, por exemplo). Na discussão sobre a
qualidade deste modo de calibração que adotamos, é importante você
destacar aquela calibração antiga do IAG, usando o Monarch21, apontando
as particularidades que mudam para as medidas de refletância - o filtro
usado, características regionais e temporais do BC etc. Do túnel do
Jânio para Ghana também há muita diferença - e ambos envolvem calibração
com TOT.

citar WHO.

Etapas:

1) usualmente medido por refletência comparando com monarca (artigo que indidica
ele para ser usado em amostras ambientais)

2) recebemos amostras super carregadas, mas os monarca tinha pouca carga.

3) Tentamos encontrar monarca 21 comercial mais carregado. Na cetesb,
usando material astm, produzimos filtro com alta e baixa concentração.
Neste caso a massa e o log da refletância tiveram relação linear. 

Equações entre massa e refletância e evolução hostórica do perfil de BC:
% Quincey, Paul; 2007. A relationship between Black Smoke Index and Black Carbon concentration, Atmospheric Environment, 41 7964-7968.
% Quincey, Paul; Butterfield, David; Green, David; Coyle, Mhairi; Capre Neil J.; 2009. An evaluation of measurement methods for organic, elemental and black carbon in ambient air monitoring sites, Atmospheric Environment, 43 5085-5091.

4) com aumento da massa (artigo do quince) a linearidade se perde. 

5) Na experiência do tunel (tb carregado) houve medida em quartzo para medida
em TOT em paralelo a policarbonato. Os policarbonato foram medidos por reflectancia
calibrados com os Monarca. O TOT e a refletancia estava correlacionados, mas
os valores abosolutos da refle. era 4 vezes maiores.
Particula grossa tem menor superficie de absorção (ASTM -N762 Black). 
explicação: a reflectance mede outras coisas que não BC, assim é importante
fazer a comparação de metodo indireto com metodo direto, pois BC tem caracteristicas
diferentes dependendo da região.

calibraça; filtro, material.

6) Em Gana, só TOT para amostras muito carregadas, o ponto zero (teórico)
ajudou no ajuste.

O padrão de calibração usado até então no laboratório usando Monarca 21 cobre refletância de 30\% a 100\%. 
Os alvos de Ghana, por serem extremamente carregados, chegaram a quase 100\% o que nos impossibilitou de usar 
essa curva de calibração.  

Para tentar contornar e recalibrar o refletância foi produzido 27 filtros (cetesb) com camera de ressuspensão
usando ASTM -N762 Black Carbon, com amostrador dicotomico, com refletância de 3,1 até 95,9\% (mais um de 100\%).

A massa foi medida em balança microanalítica +/- 1ug.

%Discordância entre os alvos padrões antigos e novos foram entre 2 e 5, mas chegou 
%até 18 em alvos poucos carregados. Problemas com alvos padrões BC são cohecidos e 
%neste caso, provavelmente associado com diferentes distriições de tamanho das partículas.

%Experimento realizado em túneis (rodoanel - saturou - e janio quadros) em 2010 são paulo também colotou amostras com 100\%
%em alvos de policarbonato (amostrador Partisol)  e quartzo em paralelo usando amostrador Minivol . 
%O filtros de quartzo foram analisados usando thermal/optical transmittance (TOT) (método absoluto) o que
%possibilitou uma calibração informal entre o TOT e os filtros de policarbonato na refletometro. 
%Os filtros do rodoanel foram saturado e os do janio quados teve resultados de 4.9 to 57,1\% (mais o branco de 100\%)
%o que possibilitou a comparação e o ajuste dos dois métodos.  

%A intercalibração de TOT em refletância Janio quadros feita em 4/Maio/2011 até 13/Maio/2011.

%quartzo TOT foi medido pelo Dr. Pierre Herckes
%Department of Chemistry and Biochemistry in the Arizona State University

%Em gana também foi feito amostragem em paralelo com 100 filtros de quartzo e teflon. 
%Os dados mostram uma combinação linear entre BC obtido da refletência e TOT. 

%impor o ponto 0 no ajuste.

%Na refletância, considerou-se a incerteza como a desvio padrão da medidas de 10 alvos brancos de laboratórios, 
%assim incorporamos a incerteza dos brancos e da variabilidade do equipamento  (0,1). 

%A incerteza do ajuste foi feita usando ajuste dos minimos quadrados, considerando
%variâncias e co-variâncias. 
%Obrigar a linha a passar em zero, ou seja, em 100\% de reflêtancia há zero de massa. 

%A dependência linear entre Black Carbon (ASTM -N762) do log da refletância indica
%que a reflectance é um bom método para avaliar a massa de BC no filtro. 
%O problema é na refletância é que ela depende do tipo de BC e de filtro. 

%Inter-calibrou-se a curva obtida pela refletância com um equipamento %
%Sunset para determinação de carbono orgânico e elementar, 
%por processo Térmico/Transmitância Óptica (EPA, 2012).

%incerteza na medida do black carbon: calculado com os brancos e tirar o sd. 
%erro absoluto , a mesma para todos .

\begin{table}[H]
  \centering
  \begin{scriptsize}
    \input{../outputs/TOTcalibration}
  \end{scriptsize}
\end{table}

\begin{figure}[H]
\begin{center}
  \includegraphics[width=0.5\textwidth]{../outputs/CalibracaoRefletancia2007Lapat.pdf}
  \caption{Lapat }
\end{center}
\end{figure}

\begin{figure}[H]
\begin{center}
  \includegraphics[width=0.5\textwidth]{../outputs/TOTrefletanciaCalibration.pdf}
  \caption{}
\end{center}
\end{figure}
  


