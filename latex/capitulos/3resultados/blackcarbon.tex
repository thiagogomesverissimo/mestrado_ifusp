%%%%
\section{Black Carbon}

%%%%
\subsection{Calibração da refletância usando BC Monarch 71}

Alvos padrões comerciais com concentrações conhecidas de BC de referência 
Monarch 71 (M71) suspendidos em $CH_2Cl_2$ \citep{clarke1986} são usados na 
refletância do LAPAt para calibração entre as medidas em porcentagem e a massa.
Na tabela \ref{table:monarch71} estão os valores de 3 medidas de refletância  
realizadas nos alvos padrões Monarch 71 (M71) pertecentes ao IFUSP.
Observa-se que os valores de refletância variam entre 30\% e 85\%, não cobrindo
os extremos, ou seja, alvos carregados, refletâncias próximas de 0\%, e alvos
pouco carregados, refletâncias próximas de 100\%.

O logarítmo da média das 3 medidas apresenta relação aproximadamente linear 
com a densidade ($\mu g /cm^2$) informada pelo fabricante dos filtros 
conforme pode ser verificado no gráfico da figura \ref{fig:monarch71}.

Ajustando-se equação do primeiro grau em \ref{fig:monarch71} obtém-se 
$a + bx$, possibilitando o cálculo dos novos valores de densidade para os 
alvos padrões a partir da calibração, apresentados na última coluna da tabela 
\ref{table:monarch71}.

\begin{figure}
  \centering
  \includegraphics[width=0.5\textwidth]{../outputs/BC_cetesb.pdf}
  \caption{Calibração refletômetro do \textbf{LAPAt} usando alvos padrões Monarch 21
         \label{fig:monarch71}}
\end{figure}

\newpage
\begin{table}
  \centering
  \small
    \input{../outputs/BC_monarch71.tex}
    \caption{Reflêtancia de filtros padrões tipo Monarch 21 \citep{clarke1986} 
           do IFUSP usados na calibração do refletometro do 
           LAPAt 2007  erro de 0,25 ug/cm2 \label{table:monarch71}}
\end{table} 

%%%%
\subsection{Experiência nos túneis Jânio Quadros e Rodoanel}

Em maio de 2011, simultaneamente ao período de análises das amostras de Gana,
conduzia-se experimento para analisar as emissões veiculares na cidade de São 
Paulo no interior dos túneis Jânio Quadros e Rodoanel, com as características
de circulação de veículos leves e pesados, respectivamente. 

Com amostragem simultâneas de $MP_{2,5}$ por amostrador MiniVol com 
filtros de quartzo e por amostrador Partisol com filtros de policarbonato, 
sendo o primeiro para medida de absoluta de BC por TOT e o segundo para medida
relativa de BC por refletância.

Tamanha a poluição dos caminhões que os filtros coletados no Rodoanel 
foram saturado e impossibilitados de serem usados. No túnel Jânio Quadros
as refletâncias variaram de 4,9 à 57,1\% e possibilitou comparação e o 
ajuste dos dois métodos: TOT e refletância.

Porém, a curva de calibração criada com alvos comerciais de BC Monarch 71
do gráfico \ref{fig:mocarch21calib} não engloba valores de refletâncias 
nos extremos.
Assim, optou-se por produzir filtro padrões que abrangem-se os extremos.   
Foi depositado então BC de referência (ASTM-N762) em 27 filtros de policarbonato
(nuclepore fino de 37 mm (0,4 µm) usando amostrador dicotômico em câmara de 
ressuspensão nos laboratórios da 
Companhia Ambiental Do Estado De São Paulo (CETESB). A deposição foi feita por 
nebulização sobre a área total, sem controle do diâmetro de partícula.

Os valores de refletância para esses filtros variaram de 3,1\% até 95,9\%, 
mais um branco de 100\%, as respectivas massas foram medidas em microbalança 
($\pm$ 1 g). A seguir a correlação entre BC e o log da reflectância encontrado.
As barras de erros no gráfico correspondem a coluna de erro efetivo na tabela, 
que é a tranposição do erro da refletância (eixo x) na massa (eixo y). 

\newpage
\begin{table}[H]
  \centering
  \small
    \input{../outputs/cetesb2012.tex}
    \caption{Reflêtancia de filtros produzidos na cetesb}
\end{table} 

\begin{figure}[H]
  \centering
  \includegraphics[width=0.5\textwidth]{../outputs/BC_cetesb.pdf}
  \caption{Reflêtancia de filtros produzidos na cetesb}
\end{figure}
\newpage

Os filtros de quartzo coletados no túnel Jânio Quadros foram analisados usando
método absoluto TOT e possibilitou a comparação com os filtros correspondente 
de policarbonato analisados por refletância usando calibração dos filtros
produzidos na CETESB. 

A determinação BC por TOT nos filtros de quartzo foi conduzido pelo 
Dr. Pierre Herckes, do Departamento de Química e Bioquímica da 
Universidade Estadual do Arizona. A seguir, para os 16 filtros coletados no 
túnel Jânio Quadros, foi realizado ajuste da regressão linear
entre medidas de TOT e refletância usando calibração dos filtros produzidos na 
CETESB.

\begin{figure}
  \centering
  \begin{minipage}[b]{0.5\linewidth}
    \includegraphics[width=\textwidth]{../outputs/BC_janio_quadros.pdf}
    \caption{JQ}
  \end{minipage}%\hfill
  \hspace{0.5cm}
  \begin{minipage}[b]{0.45\linewidth}
    \begin{small}
      \input{../outputs/BC_janio_quadros.tex}
    \end{small}
    \captionof{table}{túnel Jânio Quadros)}
  \end{minipage}
\end{figure}

Percebe-se que apesar da calibração realizada nos filtros produdidos na CETESB
ter relação linear entre o log da refletância e massa medida na balança, quando
confrontada com medidas abosolutas de TOT, os resultados são correlacionados, 
entretanto diferem de até um fator 5. 

O TOT e a refletancia estão correlacionados, mas os valores obtidos via 
reflectância é de até 5 vezes maiores que TOT.

\citet{taha2007} aponta que há uma limitação da técnica de refletância quando 
filtros de teflon são usados, pois em altas concentrações (refletâncias
inferiores a 20\%), ocorre saturação, e a relação exponencial da equação 
\ref{eq:iso9835refletancia} passa a não valer mais. Assim, a técnica de
refletância não é adequada para alvos muito carregados.

Possível causa para saturação pode estar relacionado a suferficie das partículas 
de BC, pois BC de referência como o ASTM-N762 ou Monarch 71 contém partículas
de BC mais grossas que as encontradas na atmosfera, o que influência a 
absorção de luz, técnica usada na reflectância, pois a superficie de absorção
de uma partícula grossa é menor do que a superficie de absorção de diversas
particulas menores, mas de massa equivalente. Outra motivo é que a refletância
mede outras coisas que não BC, pois se baseia na propriedade de abosorção de luz.
Por fim, as caracteristicas de tamanho, forma geométrica e seção de choque
de abosorção de luz das partículas de BC variam de região para região.
Por esses motivos, é importante fazer a comparação de metodo indireto com 
metodo direto.

%%%%
\subsection{Intercalibração entre TOT e refletância em Acra}

Foram coletados 2898 filtros de teflon, analisados por refletância,
sendo que destes, 52 tiveram medidas correspondentes em filtros de quartzo 
para $MP_{2,5}$, analisados por TOT.

Assim como no experimento do túnel Janio Quadros, os filtros de teflon 
coletados em Acra apresentaram refletância próximas de 0\%, pois além 
da alta poluição atmosférica o tempo de amostragem de 48 horas contribuiu 
para coleta de amostras muitos carregadas. 
Portanto, como constatado no túnel Janio Quadros, não é possível utilizar 
curva de calibração com alvos BC de referência, pois em altas concentrações
não hã relação linear entre massa e logarítimo da refletância.

Entretanto, como constatado, medidas de refletância e TOT são correlacionadas 
a menos de um fator, assim é possível utilizar 
Utilizou-se o resultado das medidas abosolutas do 52 filtros de quartzo
analisada por TOT e seus correspondetes em teflon para criar uma curva de 
calibração que permitisse converter as refletâncias em medidas de BC.

Dada a saturação observada para filtros carregados e a correlação encontrada entre medidas TOT 

impor o ponto 0 no ajuste.

%Na refletância, considerou-se a incerteza como a desvio padrão da medidas de
%10 alvos brancos de laboratórios, assim incorporamos a incerteza dos brancos 
%e da variabilidade do equipamento  (0,1). 

 

Inter-calibrou-se a curva obtida pela refletância com um equipamento %
Sunset para determinação de carbono orgânico e elementar, 
por processo Térmico/Transmitância Óptica (EPA, 2012).

incerteza na medida do black carbon: calculado com os brancos e tirar o sd. 
erro absoluto , a mesma para todos .

Assumimos como incerteza para os valores de reflectância (medido em percentagem),
o desvio padrão corrigido das medições fez para 7 filtros em branco 
(0,96\% x 1,11 = 1,06\%, corrigidos para as estatísticas poucos dados, 
utilizando 7 valores). 
A incerteza em massa acumulada foi de (2) 1/2. 

Uma variação eficaz para medidas BC foi interativamente ajustada, adicionando, às próprias incertezas BC, a transposição linear das incertezas da reflectância, até que a convergência de crises sucessivas.

Os parâmetros para este ajuste linear foram:






 

As medidas de refletância das amostras coletadas nos três experimentos, 
túnel Rodoanel, túnel Jânio Quadros e Gana, oscilavam em torno dos 0\% e como
a curva de calibração produzida a partir dos alvos com BC Monarch 71 não

experiência para avaliação da poluição em túneis de São Paulo, 
Rodoanel e Jânio Quadros, com coleta de   em que amostras super carregadas foram coletadas, 
assim como as de Gana, porém em filtros de policarbonato.


Discordância entre os alvos padrões antigos e novos foram entre 2 e 5, mas chegou 
até 18 em alvos poucos carregados. Problemas com alvos padrões BC são cohecidos e 
neste caso, provavelmente associado com diferentes distriições de tamanho das partículas.



%O texto da EPA
%é bem ilustrado. Em geral o nome do arquivo dá uma idéia do que tem
%dentro. No caso dos arquivos da Atmospheric Environment eu os indico com
%&e e ano (trabalhos do Quincey, por exemplo). Na discussão sobre a
%qualidade deste modo de calibração que adotamos, é importante você
%destacar aquela calibração antiga do IAG, usando o Monarch21, apontando
%as particularidades que mudam para as medidas de refletância - o filtro
%usado, características regionais e temporais do BC etc. Do túnel do
%Jânio para Ghana também há muita diferença - e ambos envolvem calibração
%com TOT.

%citar WHO.

Equações entre massa e refletância e evolução hostórica do perfil de BC:
% Quincey, Paul; 2007. A relationship between Black Smoke Index and Black Carbon concentration, Atmospheric Environment, 41 7964-7968.
% Quincey, Paul; Butterfield, David; Green, David; Coyle, Mhairi; Capre Neil J.; 2009. An evaluation of measurement methods for organic, elemental and black carbon in ambient air monitoring sites, Atmospheric Environment, 43 5085-5091.

4) com aumento da massa (artigo do quince) a linearidade se perde. 



6) Em Gana, só TOT para amostras muito carregadas, o ponto zero (teórico)
ajudou no ajuste.

O padrão de calibração usado até então no laboratório usando Monarca 21 cobre refletância de 30\% a 100\%. 
Os alvos de Ghana, por serem extremamente carregados, chegaram a quase 100\% o que nos impossibilitou de usar 
essa curva de calibração.  

Para tentar contornar e recalibrar o refletância foi produzido 27 filtros (cetesb) com camera de ressuspensão
usando ASTM -N762 Black Carbon, com amostrador dicotomico, com refletância de 3,1 até 95,9\% (mais um de 100\%).

A massa foi medida em balança microanalítica +/- 1ug.







1.2. The disagreement between the old and new standards reflectances are not less than 1.5 times, typically being around 2 to 5, but arriving till 18 times. Problems with BC standards are known and, in this case, probably was associated to different particles size distributions, other than the not precise and variable characteristics of what is called BC.


%TODO: citar o clássico  \citet{quincey2007}, mas onde?



\begin{table}[H]
  \centering
  \begin{scriptsize}
    \input{../outputs/TOTcalibration}
  \end{scriptsize}
\end{table}


\begin{figure}[H]
\begin{center}
  \includegraphics[width=0.5\textwidth]{../outputs/TOTrefletanciaCalibration.pdf}
  \caption{}
\end{center}
\end{figure}