%%%%
\section{Black Carbon}

Alvos padrões comerciais com concentrações conhecidas de BC de referência 
Monarch 71 (M71) suspendidos em $CH_2Cl_2$ \citep{clarke1986} são usados na 
refletância do LAPAt para calibração entre as medidas em porcentagem e a massa.
Na tabela \ref{table:monarch71} estão os valores de 3 medidas de refletância  
realizadas nos alvos padrões Monarch 71 (M71) pertecentes ao IFUSP.
Observa-se que os valores de refletância variam entre 30\% e 85\%, não cobrindo
os extremosm, ou seja, alvos carregados, refletâncias próximas de 0\%, e alvos
pouco carregados, refletâncias próximas de 100\%.

O logarítmo da média das 3 medidas apresenta relação aproximadamente linear 
com a densidade ($\mu g /cm^2$) informada pelo fabricante dos filtros 
conforme pode ser verificado no gráfico da figura \ref{fig:monarch71}.

Ajustando-se equação do primeiro grau em \ref{fig:monarch71} obtém-se 
$a + bx$, possibilitando o cálculo dos novos valores de densidade para os 
alvos padrões a partir da calibração, apresentados na última coluna da tabela 
\ref{table:monarch71}.

\begin{figure}[H]
  \centering
  \includegraphics[width=0.5\textwidth]{../outputs/BC_cetesb.pdf}
  \caption{Calibração refletômetro do \textbf{LAPAt} usando alvos padrões Monarch 21
         \label{fig:monarch71}}
\end{figure}

\newpage
\begin{table}[H]
  \centering
  \small
    \input{../outputs/BC_monarch71.tex}
    \caption{Reflêtancia de filtros padrões tipo Monarch 21 \citep{clarke1986} 
           do IFUSP usados na calibração do refletometro do 
           LAPAt 2007  erro de 0,25 ug/cm2 \label{table:monarch71}}
\end{table} 
\newpage

%TODO: citar o clássico  \citet{quincey2007}, mas onde?

Para ampliar a curva de calibração de gráfico \ref{fig:mocarch21calib} 
englobando refletâncias nos extremos, próximas de 0\% ou próximas de 100, 
produziu-se 27 filtros de policarbonato com BC de referência (ASTM-N762) 
usando amostrador dicotômico em câmara de ressuspensão nos laboratórios da 
Companhia Ambiental Do Estado De São Paulo (CETESB). 
A deposição foi feita por nebulização sobre a área total, 
sem controle do diâmetro de partícula.

A refletância desses filtros foi realizada no refletômetro e a massa foi medida 
em microbalança (± 1 g). A seguir a correlação entre BC e o log da reflectância
encontrado.

\begin{figure}[H]
  \centering
  \includegraphics[width=0.5\textwidth]{../outputs/BC_cetesb.pdf}
  \caption{Reflêtancia de filtros produzidos na cetesb}
\end{figure}

\citet{taha2007} aponta que há uma limitação da técnica de refletância quando 
filtros de teflon são usados, pois em altas concentrações, isto é refletâncias
inferiores a 20\%, ocorre saturação, e a relação exponencial da equação 
\ref{eq:iso9835refletancia} passa a não valer mais. Assim, a técnica de
refletância não é adequada para alvos muito carregados.

\newpage
\begin{table}[H]
  \centering
  \small
    \input{../outputs/cetesb2012.tex}
    \caption{Reflêtancia de filtros produzidos na cetesb}
\end{table} 
\newpage



No mesmo período das análises das amostras de Gana (2010), foi realizada 
experiência para avaliação da poluição em túneis de São Paulo, 
Rodoanel e Jânio Quadros, em que amostras super carregadas foram coletadas, 
assim como as de Gana, porém em filtros de policarbonato.


Experimento dirigido para analisar as emissões veiculares no interior do 
"Jânio Quadros" túnel na cidade de São Paulo, no Brasil, de 04 de maio 2011 
até que May, 13, 2011. Amostras simultâneas de PM2.5 foram coletadas em 
em filtros de quartz por amostrador MiniVol e em filtros de policarbonato 
po amostrador Partisole. 


A determinação BC por TOT (TOTºBC) nos filtros de quartzo foi conduzido pelo Dr. Pierre Herckes, no Departamento de Química e Bioquímica da Universidade Estadual do Arizona. No próximo artigo vamos mostrar como BC foi avaliada por reflectância e, depois disso, fizemos o ajuste entre TOTºBC e BC reflectância (RºBC). As incertezas de ambos os métodos foram levados em conta.

%Assumimos como incerteza para os valores de reflectância (medido em percentagem),
%o desvio padrão corrigido das medições fez para 7 filtros em branco 
%(0,96\% x 1,11 = 1,06\%, corrigidos para as estatísticas poucos dados, 
%utilizando 7 valores). 
%A incerteza em massa acumulada foi de (2) 1/2. 

Uma variação eficaz para medidas BC foi interativamente ajustada, adicionando, às próprias incertezas BC, a transposição linear das incertezas da reflectância, até que a convergência de crises sucessivas.

Os parâmetros para este ajuste linear foram:

%O texto da EPA
%é bem ilustrado. Em geral o nome do arquivo dá uma idéia do que tem
%dentro. No caso dos arquivos da Atmospheric Environment eu os indico com
%&e e ano (trabalhos do Quincey, por exemplo). Na discussão sobre a
%qualidade deste modo de calibração que adotamos, é importante você
%destacar aquela calibração antiga do IAG, usando o Monarch21, apontando
%as particularidades que mudam para as medidas de refletância - o filtro
%usado, características regionais e temporais do BC etc. Do túnel do
%Jânio para Ghana também há muita diferença - e ambos envolvem calibração
%com TOT.

%citar WHO.

%Etapas:


2) recebemos amostras super carregadas, mas os monarca tinha pouca carga.

3) 

Equações entre massa e refletância e evolução hostórica do perfil de BC:
% Quincey, Paul; 2007. A relationship between Black Smoke Index and Black Carbon concentration, Atmospheric Environment, 41 7964-7968.
% Quincey, Paul; Butterfield, David; Green, David; Coyle, Mhairi; Capre Neil J.; 2009. An evaluation of measurement methods for organic, elemental and black carbon in ambient air monitoring sites, Atmospheric Environment, 43 5085-5091.

4) com aumento da massa (artigo do quince) a linearidade se perde. 

5) Na experiência do tunel (tb carregado) houve medida em quartzo para medida
em TOT em paralelo a policarbonato. Os policarbonato foram medidos por reflectancia
calibrados com os Monarca. O TOT e a refletancia estava correlacionados, mas
os valores abosolutos da refle. era 4 vezes maiores.
Particula grossa tem menor superficie de absorção (ASTM -N762 Black). 
explicação: a reflectance mede outras coisas que não BC, assim é importante
fazer a comparação de metodo indireto com metodo direto, pois BC tem caracteristicas
diferentes dependendo da região.

calibraça; filtro, material.

6) Em Gana, só TOT para amostras muito carregadas, o ponto zero (teórico)
ajudou no ajuste.

O padrão de calibração usado até então no laboratório usando Monarca 21 cobre refletância de 30\% a 100\%. 
Os alvos de Ghana, por serem extremamente carregados, chegaram a quase 100\% o que nos impossibilitou de usar 
essa curva de calibração.  

Para tentar contornar e recalibrar o refletância foi produzido 27 filtros (cetesb) com camera de ressuspensão
usando ASTM -N762 Black Carbon, com amostrador dicotomico, com refletância de 3,1 até 95,9\% (mais um de 100\%).

A massa foi medida em balança microanalítica +/- 1ug.




Este intercalibração veio de um experimento dirigido para analisar as emissões veiculares no interior do "Jânio Quadros" túnel na cidade de São Paulo, no Brasil, a partir de maio, 04, 2011, até que May, 13, 2011. amostras simultâneas de PM2.5 foram coletados por um MiniVol amostrador, em filtros de quartzo, e por um amostrador Partisol, em filtros de policarbonato. O MiniVol amostrados períodos 12h, com início às 08:00 eo Partisol amostrados sequencialmente dois períodos 6h, com início às 08:00 e outro período de 12 h com início às 20:00. As medidas nos dois primeiros períodos Partisol 6h foram calculados como sendo equivalente às medidas no primeiro período Mini Vol.
A determinação BC por TOT (TOTºBC) nos filtros de quartzo foi conduzido pelo Dr. Pierre Herckes, no Departamento de Química e Bioquímica da Universidade Estadual do Arizona. No próximo artigo vamos mostrar como BC foi avaliada por reflectância e, depois disso, fizemos o ajuste entre TOTºBC e BC reflectância (RºBC). As incertezas de ambos os métodos foram levados em conta.

%Discordância entre os alvos padrões antigos e novos foram entre 2 e 5, mas chegou 
%até 18 em alvos poucos carregados. Problemas com alvos padrões BC são cohecidos e 
%neste caso, provavelmente associado com diferentes distriições de tamanho das partículas.

%Experimento realizado em túneis (rodoanel - saturou - e janio quadros) em 2010 são paulo também colotou amostras com 100\%
%em alvos de policarbonato (amostrador Partisol)  e quartzo em paralelo usando amostrador Minivol . 
%O filtros de quartzo foram analisados usando thermal/optical transmittance (TOT) (método absoluto) o que
%possibilitou uma calibração informal entre o TOT e os filtros de policarbonato na refletometro. 
%Os filtros do rodoanel foram saturado e os do janio quados teve resultados de 4.9 to 57,1\% (mais o branco de 100\%)
%o que possibilitou a comparação e o ajuste dos dois métodos.  

%A intercalibração de TOT em refletância Janio quadros feita em 4/Maio/2011 até 13/Maio/2011.

%quartzo TOT foi medido pelo Dr. Pierre Herckes
%Department of Chemistry and Biochemistry in the Arizona State University

%Em gana também foi feito amostragem em paralelo com 100 filtros de quartzo e teflon. 
%Os dados mostram uma combinação linear entre BC obtido da refletência e TOT. 

%impor o ponto 0 no ajuste.

%Na refletância, considerou-se a incerteza como a desvio padrão da medidas de 10 alvos brancos de laboratórios, 
%assim incorporamos a incerteza dos brancos e da variabilidade do equipamento  (0,1). 

%A incerteza do ajuste foi feita usando ajuste dos minimos quadrados, considerando
%variâncias e co-variâncias. 
%Obrigar a linha a passar em zero, ou seja, em 100\% de reflêtancia há zero de massa. 

%A dependência linear entre Black Carbon (ASTM -N762) do log da refletância indica
%que a reflectance é um bom método para avaliar a massa de BC no filtro. 
%O problema é na refletância é que ela depende do tipo de BC e de filtro. 

%Inter-calibrou-se a curva obtida pela refletância com um equipamento %
%Sunset para determinação de carbono orgânico e elementar, 
%por processo Térmico/Transmitância Óptica (EPA, 2012).

%incerteza na medida do black carbon: calculado com os brancos e tirar o sd. 
%erro absoluto , a mesma para todos .

\begin{table}[H]
  \centering
  \begin{scriptsize}
    \input{../outputs/TOTcalibration}
  \end{scriptsize}
\end{table}


\begin{figure}[H]
\begin{center}
  \includegraphics[width=0.5\textwidth]{../outputs/TOTrefletanciaCalibration.pdf}
  \caption{}
\end{center}
\end{figure}
  




