%%%%
\section{Black Carbon}

%%%%
\subsection{Calibração da refletância usando BC Monarch 71}

Nas medidas de Black Carbon (BC), a técnica de refletância pode ser realizada nos mesmos filtros analisados por XRF, diminuindo significativamente os custos da pesquisa. Entretanto, a refletância apresentou problemas para filtros super carregados. Utilizando o método absoluto TOT para inter-calibrar as medidas de refletância, possibilitou a quantificação do BC em todas amostras.

Alvos padrões comerciais com concentrações conhecidas de BC de referência 
Monarch 71 (M71) suspendidos em $CH_2Cl_2$ \citep{clarke1986} são usados na 
refletância do LAPAt para calibração entre as medidas em porcentagem e a massa.
Na tabela \ref{table:monarch71} estão os valores de 3 medidas de refletância  
realizadas nos alvos padrões Monarch 71 (M71) pertecentes ao IFUSP.
Observa-se que os valores de refletância variam entre 30\% e 85\%, não cobrindo
os extremos, ou seja, alvos carregados, refletâncias próximas de 0\%, e alvos
pouco carregados, refletâncias próximas de 100\%.

O logarítmo da média das 3 medidas apresenta relação aproximadamente linear 
com a densidade ($\mu g /cm^2$) informada pelo fabricante dos filtros 
conforme pode ser verificado no gráfico da figura \ref{fig:monarch71}.

Ajustando-se equação do primeiro grau em \ref{fig:monarch71} obtém-se 
$a + bx$, possibilitando o cálculo dos novos valores de densidade para os 
alvos padrões a partir da calibração, apresentados na última coluna da tabela 
\ref{table:monarch71}.

\begin{figure}[H]
  \centering
  \includegraphics[width=0.5\textwidth]{../outputs/BC_monarch71.pdf}
  \caption{Calibração refletômetro do LAPAt usando alvos padrões Monarch 71.
         \label{fig:monarch71}}
\end{figure}

\newpage
\begin{table}[H]
  \centering
  \small
    \input{../outputs/BC_monarch71.tex}
    \caption{Reflêtancia de 2007 dos filtros padrões tipo Monarch 71 
             do IFUSP usados na calibração do refletômetro do 
             LAPAt.  \label{table:monarch71}}
\end{table} 

%%%%
\subsection{Experiência nos túneis Jânio Quadros e Rodoanel}

Em maio de 2011, simultaneamente ao período de análises das amostras de Gana,
conduzia-se experimento para analisar as emissões veiculares na cidade de São 
Paulo no interior dos túneis Jânio Quadros e Rodoanel, com as características
de circulação de veículos leves e pesados, respectivamente. 

Com amostragem simultâneas de $MP_{2,5}$ por amostrador MiniVol com 
filtros de quartzo e por amostrador Partisol com filtros de policarbonato, 
sendo o primeiro para medida de absoluta de BC por TOT e o segundo para medida
relativa de BC por refletância.

Tamanha a poluição dos caminhões que os filtros coletados no Rodoanel 
foram saturado e impossibilitados de serem usados. No túnel Jânio Quadros
as refletâncias variaram de 4,9 à 57,1\% e possibilitou comparação e o 
ajuste dos dois métodos: TOT e refletância.

Porém, a curva de calibração criada com alvos comerciais de BC Monarch 71
do gráfico \ref{fig:mocarch21calib} não engloba valores de refletâncias 
nos extremos.
Assim, optou-se por produzir filtro padrões que abrangem-se os extremos.   
Foi depositado então BC de referência (ASTM-N762) em 27 filtros de policarbonato
(nuclepore fino de 37 mm (0,4 µm) usando amostrador dicotômico em câmara de 
ressuspensão nos laboratórios da 
Companhia Ambiental Do Estado De São Paulo (CETESB). A deposição foi feita por 
nebulização sobre a área total, sem controle do diâmetro de partícula.

Os valores de refletância para esses filtros variaram de 3,1\% até 95,9\%, 
mais um branco de 100\%, as respectivas massas foram medidas em microbalança 
($\pm$ 1 g). A seguir a correlação entre BC e o log da reflectância encontrado.
As barras de erros no gráfico correspondem a coluna de erro efetivo na tabela, 
que é a tranposição do erro da refletância (eixo x) na massa (eixo y). 

\newpage
\begin{table}[H]
  \centering
  \small
    \input{../outputs/cetesb2012.tex}
    \caption{Reflêtancia de filtros produzidos na cetesb}
\end{table} 

\begin{figure}[H]
  \centering
  \includegraphics[width=0.5\textwidth]{../outputs/BC_cetesb.pdf}
  \caption{Reflêtancia de filtros produzidos na cetesb}
\end{figure}
\newpage

Os filtros de quartzo coletados no túnel Jânio Quadros foram analisados usando
método absoluto TOT e possibilitou a comparação com os filtros correspondente 
de policarbonato analisados por refletância usando calibração dos filtros
produzidos na CETESB. 

A determinação BC por TOT nos filtros de quartzo foi conduzido pelo 
Dr. Pierre Herckes, do Departamento de Química e Bioquímica da 
Universidade Estadual do Arizona. A seguir, para os 16 filtros coletados no 
túnel Jânio Quadros, foi realizado ajuste da regressão linear
entre medidas de TOT e refletância usando calibração dos filtros produzidos na 
CETESB.

\begin{figure}[H]
  \centering
  \begin{minipage}[b]{0.5\linewidth}
    \includegraphics[width=\textwidth]{../outputs/BC_janio_quadros.pdf}
    \caption{Comparação da massa de BC das amostras do túnel Jânio Quadros 
             calculada usando calibração por TOT(1) versus calibração a partir dos 
             alvos produzidos na CETESB(2).}
  \end{minipage}%\hfill
  \hspace{0.5cm}
  \begin{minipage}[b]{0.45\linewidth}
    \begin{small}
      \input{../outputs/BC_janio_quadros.tex}
    \end{small}
    \captionof{table}{Comparação da massa de BC das amostras do túnel Jânio Quadros 
             calculada usando calibração por TOT versus calibração a partir dos 
             alvos produzidos na CETESB.}
  \end{minipage}
\end{figure}

Percebe-se que apesar da calibração realizada nos filtros produdidos na CETESB
ter relação linear entre o log da refletância e massa medida na balança, quando
confrontada com medidas abosolutas de TOT, os resultados são correlacionados, 
entretanto diferem de até um fator 5. 

O TOT e a refletancia estão correlacionados, mas os valores obtidos via 
refletância é de 2 até 5 vezes maiores que TOT em alvos muitos carregados e 
chegou até 18 vezes em alvos poucos carregados.

\citet{taha2007} aponta que há uma limitação da técnica de refletância quando 
filtros de teflon são usados, pois em altas concentrações (refletâncias
inferiores a 20\%), ocorre saturação, e a relação exponencial da equação 
\ref{eq:iso9835refletancia} passa a não valer mais. Assim, a técnica de
refletância não é adequada para alvos muito carregados.

Problemas com alvos padrões BC são conhecidos e neste caso, 
provavelmente associado com diferentes distriições de tamanho das partículas.
Possível causa para saturação pode estar relacionado a suferficie das partículas 
de BC, pois BC de referência como o ASTM-N762 ou Monarch 71 contém partículas
de BC mais grossas que as encontradas na atmosfera, o que influência a 
absorção de luz, técnica usada na reflectância, pois a superficie de absorção
de uma partícula grossa é menor do que a superficie de absorção de diversas
particulas menores, mas de massa equivalente. Outra motivo é que a refletância
mede outras coisas que não BC, pois se baseia na propriedade de abosorção de luz.
Por fim, as caracteristicas de tamanho, forma geométrica e seção de choque
de abosorção de luz das partículas de BC variam de região para região.
Por esses motivos, é importante fazer a comparação de metodo indireto com 
metodo direto \citep{quincey2007}.

\newpage
%%%%
\subsection{Intercalibração entre TOT e refletância em Acra}

Foram coletados 2898 filtros de teflon, analisados por refletância,
sendo que destes, 52 tiveram medidas correspondentes em filtros de quartzo 
para $MP_{2,5}$, analisados por TOT.

Assim como no experimento do túnel Janio Quadros, os filtros de teflon 
coletados em Acra apresentaram refletância próximas de 0\%, pois além 
da alta poluição atmosférica o tempo de amostragem de 48 horas contribuiu 
para coleta de amostras muitos carregadas, nesta situação não é possível 
utilizar curva de calibração com alvos BC de referência, pois
não hã relação linear entre a massa e logarítimo da refletância.

Entretanto, como constatado, medidas de refletância e TOT são correlacionadas 
a menos de um fator, assim optou-se por intercalibrar as medidas de 
TOT e refletância, de forma a estimar os valores de BC em todos filtros
de teflon. Curva mostrada na figura a seguir com imposição do ponto 0 no ajuste.
A incerteza efetiva considera o erro do TOT e da refletância, além dos 8\% de 
amostragem paralela. 

Assumimos como incerteza para os valores de reflectância (medido em porcentagem)
o desvio padrão corrigido das medições de 7 filtros em branco 
(0,96\% x 1,11 = 1,06\%, corrigidos para as estatísticas de poucos dados, 
utilizando 7 valores). A incerteza em massa acumulada foi de $sqrt(2)$.

\begin{figure}[H]
  \centering
  \begin{subfigure}[b]{0.45\linewidth}
    \includegraphics[width=\linewidth]{../outputs/BC_compara_calibs_recife.pdf}
    \caption{Recife}
  \end{subfigure}%
  \hspace{0.3cm}
  \begin{subfigure}[b]{0.45\linewidth}
    \includegraphics[width=\linewidth]{../outputs/BC_compara_calibs.pdf}
    \caption{Gana}
  \end{subfigure}
  \caption{}
\end{figure}

\newpage
\begin{figure}[H]
\begin{center}
  \includegraphics[width=0.5\textwidth]{../outputs/Gana_TOT_Refletancia.pdf}
  \caption{Intercalibração entre TOT e refletância em Acra}
\end{center}
\end{figure}

\begin{table}[H]
  \centering
  \small
   \input{../outputs/Gana_TOT_Refletancia.tex}
   \caption{Intercalibração entre TOT e refletância em Acra}
\end{table} 


