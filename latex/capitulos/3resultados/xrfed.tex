%%%%
\section{Fluorescência de Raiox X}

As análises de \textbf{PMF} são ponderadas pelas incertezas, assim 
tentou-se incluir todas fontes possíveis de erro, além do erro instrínseco 
do método analítico. Um erro percentual fixo de 8\% atribuído ao método 
de amostragem foi incluído 
(erro médio observado em amostragens em paralelo com o amostrador Harvard)
\citep{santos2014}.

%%%%
\subsection{Calibração}

Durante o período de análise das amostra de Acra foram realizadas 
três calibrações na \textbf{ED-XRF}: Maio de 2010, Novembro de 2010
e Abril de 2011. 

Com um conjunto de alvos padrões da marca \textbf{Micromatter} 
(incerteza 5\%) calculou-se os fatores de respostas correspondentes
usando-se a equação \ref{eq:fator_de_resposta}, bem como as respectivas
incertezas analíticas usando-se \ref{eq:erro_fator_de_resposta}.

Um ajuste polinomial foi suficiente para descrever os pontos experimentais. 
Na linha K foi necessário fazer dois ajustes polinomiais de grau 3, um do 
número atômico 11 até 26 e o outro de 22 até 42. Para linha L foi 
feito um ajuste em todos pontos com um polinômio de grau 5.
Os ajustes e os pontos experimentais para Maio de 2010 estão plotados no 
gráfico da figura \ref{fig:edx_calib_KLmaio2010}.

\begin{figure}[H]
  \begin{subfigure}[b]{0.5\textwidth}
    \includegraphics[width=\textwidth]{../outputs/CalibrationK2010MaiAkerr.pdf}
    \caption{Linha K}
  \end{subfigure}%
  \begin{subfigure}[b]{0.5\textwidth}
    \includegraphics[width=\textwidth]{../outputs/CalibrationL2010MaiAkerr.pdf}
    \caption{Linha L}
  \end{subfigure}
  \caption{Calibração da Fluorescência de Raiox X - Maio de 2010 \label{fig:edx_calib_KLmaio2010}}
\end{figure}

%\begin{figure}[H]
%  \caption{Calibração da Fluorescência de Raiox X - Novembro de 2010}
%  \begin{subfigure}[b]{0.5\textwidth}
%   \includegraphics[width=\textwidth]{../outputs/CalibrationK2010NovAkerr.pdf}
%    \caption{Linha K}
%  \end{subfigure}%
%  \begin{subfigure}[b]{0.5\textwidth}
%    \includegraphics[width=\textwidth]{../outputs/CalibrationL2010NovAkerr.pdf}
%    \caption{Linha L}
%  \end{subfigure}
%\end{figure}

%\begin{figure}[H]
%  \caption{Calibração da Fluorescência de Raiox X - Abril de 2011}
%  \begin{subfigure}[b]{0.5\textwidth}
%    \includegraphics[width=\textwidth]{../outputs/CalibrationK2011AbrAkerr.pdf}
%    \caption{Linha K}
%  \end{subfigure}%
%  \begin{subfigure}[b]{0.5\textwidth}
%    \includegraphics[width=\textwidth]{../outputs/CalibrationL2011AbrAkerr.pdf}
%    \caption{Linha L}
%  \end{subfigure}
%\end{figure}
 
As tabelas \ref{table:edx_calib_Kmaio2010} e \ref{table:edx_calib_Lmaio2010}
apresentam os valores finais da calibração de Maio de 2010. 
Percebe-se que a incerteza percentual ajustada é menor que 
a incerteza percentual medida. 

\begin{table}[H]
  \centering
  \begin{scriptsize} 
    \input{../outputs/edxCalibrationnov2010K.tex}
  \end{scriptsize}
  \caption{Calibração da Fluorescência de Raiox X - Maio de 2010 - Linha K
  \label{table:edx_calib_Kmaio2010}}
\end{table}

\begin{table}[H]
  \centering
  \begin{scriptsize} 
    \input{../outputs/edxCalibrationnov2010L.tex}
  \end{scriptsize}
  \caption{Calibração da Fluorescência de Raiox X - Maio de 2010 - Linha L 
  \label{table:edx_calib_Lmaio2010}}
\end{table}

Usando \textbf{Ajuste dos Mínimos Quadrados Matricial} foi possível
reduzir a incerteza percentual. 
A incerteza é dada pela diagonal da matriz de covariância 
de $[\tilde{Y}]$, $[V_{\tilde{Y}}]$ em \ref{eq:matrizcovarianciaY}.
Onde $[Y]$ são os fatores de respostas $[R]$ e $[X]$ os números atômicos $[Z]$.

%TODO: inclir comparações entre calibrações
%\begin{table}[H]
%  \begin{footnotesize} %small
%  \input{../outputs/comparaCalibrationK.tex}
%  \end{footnotesize}
%\end{table}

%\begin{table}[H]
%  \begin{footnotesize} %small
%  \input{../outputs/comparaCalibrationK.tex}
%  \end{footnotesize}
%\end{table}

O queda de desempenho no uso da fluorescência de Raios X pode ser observada 
na figura \ref{fig:compara_calibracao}. 
Possíveis motivos para queda de desempenho: desgaste do tubo ou desgaste do detector. 

\begin{figure}[H]
  \begin{subfigure}[b]{0.5\textwidth}
    \includegraphics[width=\textwidth]{../outputs/CalibrationKcomparacao.pdf}
    \caption{Linha K}
  \end{subfigure}%
  \begin{subfigure}[b]{0.5\textwidth}
    \includegraphics[width=\textwidth]{../outputs/CalibrationLcomparacao.pdf}
    \caption{Linha L}
  \end{subfigure}
  \caption{Calibrações de Fluorescência de Raiox X em 3 períodos \label{fig:compara_calibracao}}
\end{figure}

%%%%
\subsection{Limite de Detecçção}

\begin{figure}[H]
  \caption{}
  \begin{subfigure}[b]{0.5\textwidth}
    \includegraphics[width=\textwidth]{../outputs/limitDetectionK.pdf}
    \caption{k}
  \end{subfigure}%
  \begin{subfigure}[b]{0.5\textwidth}
    \includegraphics[width=\textwidth]{../outputs/limitDetectionL.pdf}
    \caption{l}
  \end{subfigure}
\end{figure}

O limite de detecção é pior para elementos de baixo número atômico e há
altos valores de incerteza quando a medida está próxima do limite de detecção.
