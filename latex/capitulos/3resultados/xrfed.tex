%%%%
\section{Calibração da Fluorescência de Raiox X}

Alvos padrões comerciais da Micromatter com incertezas de 5\% nas densidades 
foram utilizados no ajuste funcional da calibração do sistema XRF-ED, 
permitindo medição de elementos com alvos padrões inexistentes. Três
calibrações foram realizadas no período de análises, maio de 2010 até julho de 
2011, sendo a primeira antes do início das irradiações, em maio de 2010, 
uma intermediária, em novembro de 2010 e a última próximo do fim, em abril de 
2011.

Após irradiação dos alvos de calibração, calculou-se os fatores de respostas 
correspondentes a partir da equação \ref{eq:fator_de_resposta}, bem como as 
respectivas incertezas analíticas com a equação \ref{eq:erro_fator_de_resposta}.
Um ajuste polinomial foi suficiente para descrever os pontos experimentais,
sendo que na linha K foi necessário fazer dois ajustes de grau 3, de Z 
(número atômico) 11 até 26 e de 22 até 42. Para linha L um polinômio de grau 
5 foi suficiente para descrever todos pontos.

Os fatores de resposta medidos para cada elemento com alvo padrão 
correspondente, as respectivas incertezas propagadas para os mesmos, 
os fatores de respostas calculados ajustados e as respectivas incertezas 
ajustadas com MQM podem ser conferidos nos 
gráficos da figura \ref{fig:edx_calib3} e nas tabelas 
\ref{table:edxAllCalibrationK} e \ref{table:edxAllCalibrationL}, que 
apresentam tmabém a incerteza percentual de cada medida. 
Elementos com Z entre 29 e 42 apresentam picos no espectro tanto na linha K 
quanto L, porém são mais intensos e melhores
definidos na linha K e só foram incluídos nos gráficos da linha L no intuito de 
aumentar a quantidade de pontos para o ajuste polinomial.

A incerteza do ajuste foi estimada usando MQM e é dada pela diagonal da matriz 
de covariância de $[\tilde{Y}]$, $[V_{\tilde{Y}}]$, da equação 
\ref{eq:matrizcovarianciaY}, onde [Y] são os fatores de respostas [R] e 
[X] os números atômicos [Z]. 

Mesmo para os elementos com alvo padrão correspondente, as incertezas 
percentuais ajustadas com MQM foram em muitos casos, menores que a 
do próprio fabricante, por exemplo, na linha K em maio de 2010 a 
incerteza percentual para o Cálcio (20) no alvo padrão, depois de propagada,
foi de 3,5\%, enquanto que a ajustada com MQM foi 1,4\%. Para o Ferro (26), os
mesmo valores foram 5,0\% e 2,6\%, respectivamente. Elementos com número atômico
muito baixos e portanto alto limite de detecção neste equipamento, a diferença 
na incerteza percentual foi baixa, no caso do Sódio (11), aumento 0,1\% e no 
do Magnésio (12), diminuiu 0,5\%. Esse padrão de comportamento foi repetido nas
outras duas calibrações para linhas K e L. Uma vez que o PMF pondera sua 
estimativa pelas incertezas das concentrações a
qualidade das incertezas interfere na determinação da contribuição das fontes.

\begin{figure}[H]
  \begin{subfigure}[b]{0.5\textwidth}
    \includegraphics[width=\textwidth]{../outputs/CalibrationKcomparacao.pdf}
    \caption{linha K}
  \end{subfigure}%
  \begin{subfigure}[b]{0.5\textwidth}
    \includegraphics[width=\textwidth]{../outputs/CalibrationLcomparacao.pdf}
    \caption{linha L}
  \end{subfigure}
  \caption{Comparação das calibrações da XRF-ED nos 3 períodos. 
          \label{fig:compara_calibracao}}
\end{figure}

Calibrações periódicas permitem identificar variações na eficiência 
do equipamento ou mesmo possíveis avarias. Observou-se queda do desempenho da 
XRF-ED durante esta análise como pode ser observado nos gráficos da figura 
\ref{fig:compara_calibracao} que compara as três calibrações. 
A diferença média no fator de resposta em relação a maio de 2010 foi de 
-4,41(0,43) para novembro 2010 e -9,22(0,36) para abril de 2011 na linha K, 
e -2,02(0,22) e -7,48(0,18) na linha L, respectivamente. Desgaste no tubo 
de Rh ou no detector são as possíveis causas na queda no fator de respota 
e poderão ser confirmadas no futuro quando das trocas dessas peça e realização
de nova curva de calibração.

\begin{landscape}
\begin{figure}
    \centering
    \includegraphics[width=0.33\linewidth]{../outputs/CalibrationK2010MaiAkerr.pdf}
    \includegraphics[width=0.33\linewidth]{../outputs/CalibrationK2010NovAkerr.pdf}
    \includegraphics[width=0.33\linewidth]{../outputs/CalibrationK2011AbrAkerr.pdf}
    \includegraphics[width=0.33\linewidth]{../outputs/CalibrationL2010MaiAkerr.pdf}
    \includegraphics[width=0.33\linewidth]{../outputs/CalibrationL2010NovAkerr.pdf}
    \includegraphics[width=0.33\linewidth]{../outputs/CalibrationL2011AbrAkerr.pdf}
    \caption{Comparação das calibrações do XRF-ED nos 3 períodos. 
            \label{fig:edx_calib3}}
\end{figure}
\end{landscape}

\begin{landscape}
  \input{../outputs/edxAllCalibrationK.tex}
\end{landscape}

\begin{landscape}
  \input{../outputs/edxAllCalibrationL.tex}
\end{landscape}

%%%%
\subsection{Limite de detecçção}

O limite de detecção (LD) foi estimado para uma amostra branca e outra carregada
a partir da contagem de fundo de cada pico nos respectivos espectros.
Os LDs encontrados estão apresentados nos gráficos da figura \ref{table:ld} 
em termos de concentrações típicas ($\mu g / m^3$) para o volume médio 
de 13,9 $m^3$. Na linha K o LD é alto para elementos de baixo número atômico, 
dificultando a detecção desses elementos, mas melhora com o aumento de Z. 
Em amostras muito carregadas, o fundo do espectro aumenta, aumentando o LD para 
todos elementos, atrapalhando a detecção de elementos com baixas concetrações
na mesma amostra. 

\begin{figure}[H]
  \begin{subfigure}[b]{0.5\textwidth}
    \includegraphics[width=\textwidth]{../outputs/limitDetectionK.pdf}
    \caption{linha K}
  \end{subfigure}%
  \begin{subfigure}[b]{0.5\textwidth}
    \includegraphics[width=\textwidth]{../outputs/limitDetectionL.pdf}
    \caption{linha L}
  \end{subfigure}
  \caption{Limite de detecção em termos de concentrações típicas 
           ($\mu g / m^3$) para amostra branca e carregada.
           \label{table:ld}}
\end{figure}
