%%%%
\section{Calibração da Fluorescência de Raiox X}

Alvos padrões comerciais da Micromatter com incertezas de 5\% nas densidades 
superficiais foram utilizados no ajuste funcional da calibração do sistema XRF-ED, 
permitindo medição mesmo de elementos com alvos padrões inexistentes. Três
calibrações foram realizadas no período de análises, maio de 2010 até julho de 
2011, sendo a primeira antes do início das irradiações, em maio de 2010, 
uma intermediária, em novembro de 2010 e a última próximo do fim, em abril de 
2011.

Após irradiação dos alvos de calibração, calculou-se os fatores de respostas 
correspondentes a partir da equação \ref{eq:fator_de_resposta}, bem como as 
respectivas incertezas analíticas com a equação \ref{eq:erro_fator_de_resposta}.

Polinômios oferecem uma boa versatilidade para a realização de ajustes 
funcionais empíricos a um conjunto de dados. Ao mesmo tempo facilitam a 
utilização de MQM, que pondera o peso do dado experimental pelo inverso de sua 
incerteza e permite obter incertezas confiáveis para os valores calculados. 
Para a linha K obtivemos melhores resultados realizando dois ajustes de grau 3: 
um de Z=11 até 26 e outro de Z=22 até 42 (figura \ref{fig:edx_calib3}). 
Na zona de sobreposição entre estas duas curvas, empregou-se o valor médio como
fator de resposta. Para a linha L, um polinômio de grau 5 ofereceu um bom 
ajuste funcional para todos os pontos.

Os gráficos da figura \ref{fig:edx_calib3} e as tabelas \ref{table:edxAllCalibrationK} e 
\ref{table:edxAllCalibrationL} apresentam os fatores de resposta medidos para 
cada elemento com alvo padrão disponível, as respectivas incertezas propagadas,
os ajustes funcionais, os correspondentes fatores de respostas calculados e 
respectivas incertezas resultantes do ajuste por MQM.

Elementos com Z entre 29 e 42 apresentam picos no espectro tanto para linhas K 
quanto L. Porém, as linhas K são mais intensas e melhor definidas. Eles somente 
foram incluídos nos gráficos da linha L no intuito de aumentar a quantidade de 
pontos para o ajuste polinomial.

A incerteza do ajuste por MQM é dada pela diagonal da matriz de covariância de 
$[\tilde{Y}]$, $[V_{\tilde{Y}}]$ (equação \ref{eq:matrizcovarianciaY}), onde 
a matriz [Y] é a matriz dos fatores de respostas [R] e [X] são os números 
atômicos [Z]. 

As incertezas percentuais ajustadas com MQM são tipicamente menores que aquelas 
declaradas pelo fabricante dos alvos de calibração. Isso é uma consequência 
natural de uma metodologia que busca os melhores valores a se ajustarem ao 
conjunto dos pontos experimentais ao invés de focar cada ponto individualmente -
o que demandaria um grande número de alvos para cada elemento a analisar. 
Vejamos dois exemplos na calibração feita em maio de 2010: 
1) para a linha K do Ca (Z=20), para o qual havia dois alvos de calibração, 
obteve-se um R medido com incerteza de 3,5\%, contra 1,4\% para o valor ajustado; 
2) para o Ferro (26), com um alvo de calibração, esses valores foram 5,0\% e 
2,6\%, respectivamente. Apenas para o Na (Z=11), a incerteza do R ajustado foi 
maior que o medido, porque se trata de um ponto extremo da curva em uma 
região onde o limite de detecção do sistema é muito alto. Por conseguinte, esse 
padrão de comportamento repete-se nas outras duas calibrações para as linhas K 
e L.
Vê-se, portanto, que essa metodologia, ademais de fornecer a calibração mesmo 
para alguns elementos para os quais não se dispõe de alvos de calibração, 
permite uma significativa redução nas incertezas dos valores de concentração 
calculados. Além do valor intrínseco que isso tem nestas determinações, 
também tem papel importante para métodos estatísticos que em seus ajustes 
usam ponderação pela incertezas, caso do PMF, utilizado neste trabalho.

\begin{landscape}
\begin{figure}
    \centering
    \includegraphics[width=0.33\linewidth]{../outputs/CalibrationK2010MaiAkerr.pdf}
    \includegraphics[width=0.33\linewidth]{../outputs/CalibrationK2010NovAkerr.pdf}
    \includegraphics[width=0.33\linewidth]{../outputs/CalibrationK2011AbrAkerr.pdf}
    \includegraphics[width=0.33\linewidth]{../outputs/CalibrationL2010MaiAkerr.pdf}
    \includegraphics[width=0.33\linewidth]{../outputs/CalibrationL2010NovAkerr.pdf}
    \includegraphics[width=0.33\linewidth]{../outputs/CalibrationL2011AbrAkerr.pdf}
    \caption{Comparação das calibrações do XRF-ED nos 3 períodos. 
            \label{fig:edx_calib3}}
\end{figure}
\end{landscape}

\begin{landscape}
  \input{../outputs/edxAllCalibrationK.tex}
\end{landscape}

\begin{landscape}
  \input{../outputs/edxAllCalibrationL.tex}
\end{landscape}

Ressaltamos, ainda, que calibrações periódicas são necessárias para atualizar os
fatores de resposta devido às variações na eficiência do equipamento. Podem, 
ainda, vir a sinalizar eventuais avarias no sistema.
A figura \ref{fig:compara_calibracao} mostra uma gradual queda do desempenho 
da XRF-ED durante esta análise, ao longo das três calibrações realizadas. 
A diferença média no fator de resposta em relação a maio de 2010 foi de 
-4,41 $\pm$ 0,43 para novembro 2010 e -9,22 $\pm$ 0,36 para abril de 2011, 
na linha K, enquanto na linha L foi de -2,02 $\pm$ 0,22 e -7,48 $\pm$ 0,18, 
respectivamente. Essa é uma tendência esperada, devido ao envelhecimento do tubo
de raios X (alvo de Rh no nosso caso) e à perda de eficiência no detector. 
Por conseguinte, a substituição de um destes componentes pede sempre uma 
imediata recalibração do equipamento.

\begin{figure}[H]
  \begin{subfigure}[b]{0.5\textwidth}
    \includegraphics[width=\textwidth]{../outputs/CalibrationKcomparacao.pdf}
    \caption{linha K}
  \end{subfigure}%
  \begin{subfigure}[b]{0.5\textwidth}
    \includegraphics[width=\textwidth]{../outputs/CalibrationLcomparacao.pdf}
    \caption{linha L}
  \end{subfigure}
  \caption{Comparação das calibrações da XRF-ED nos 3 períodos. 
          \label{fig:compara_calibracao}}
\end{figure}

%%%%
\subsection{Comparação interlaboratorial com a US-EPA}

Entre as 2898 amostras enviadas para serem analisadas por XRF no LAPAt, 
92 foram previamente quantificadas por XRF na US-EPA.  O Prof. Dr. Majid Ezzati, 
no intuito de fazer um teste cego sobre a qualidade das nossas medidas, 
para verificação da exatidão e precisão dos resultados, não nos passou tal 
informação. Somente depois que enviamos os resultados de um primeiro pacote de 
análises, é que fomos chamados a discutir esta intercomparação, que mostrou uma 
qualidade muito boa. 

Esse tipo de procedimento representa uma boa prática analítica e tem sido 
particularmente recomendada para pesquisas ambientais. \citet{kang2014} aponta 
que a precisão de medidas elementares usando XRF tornou-se importante nos 
últimos anos porque as concentrações de MP ambiente estão diminuído, devido a 
leis mais rigorosas e ao desenvolvimento tecnológico, chegando próximas aos 
limites de detecção para algumas cidades. Sugere a intercomparação com outros 
laboratórios como medida necessária para avaliar a qualidade dos resultados. 
Considera necessário não apenas intercomparação entre laboratórios que 
trabalham com XRF, mas intercomparação com equipamentos de outros laboratórios 
que funcionam com princípios físicos diferentes, caso de \citet{nejedly1998}, 
que encontrou concordância entre medidas de XRF e PIXE 
(Particle-Induced X-ray Emission) intra-laboratoriais.

A intercomparação dos resultados das medidas das 92 amostras da XRF do LAPAt e 
da XRF da US-EPA estão expostos nos gráficos da figura \ref{fig:epa_lapat}. 
Verifica-se boa concordância, isto é, pontos bem alinhados
em relação à linha 1:1 (linha vermelha) para os elementos com 
concentrações adequadamente definidas acima do limite de detecção: Al, S, 
K, Ca, Sr e Zn e Pb a menos de um \textit{outlier}, mostrando que as medidas 
estão correlacionadas e concordantes em valores absolutos. Outro elementos como
Si, Ti, Fe, Mn e Br provavelmente tiveram problemas de ajuste do espectro, por 
isso a leve flutuação. 

Os valores para V e P mostraram-se mais dispersos por estarem próximos aos 
respectivos limites de detecção do equipamento de XRF da US-EPA. A linha K do Cl 
se sobrepõe a linha L do Rh (tubo de raios X usado no equipamento do LAPAt), 
dificultando a medida do mesmo.   

Foi realizada uma regressão linear simples, sem barra de erro para o ajuste, 
pois os valores enviados pela US-EPA não continham incertezas, se incluída as 
incertezas, seria possível refinar essa comparação.  A interlaboratorial 
possibilitou a validação do método de 
calibração desenvolvido no LAPAt, em confronto com uma instituição com o nível 
de reconhecimento da US-EPA.

\newpage
\begin{figure}[H]
  \centering
    \includegraphics[width=0.32\textwidth]{../outputs/epa_iag_Al.pdf}
    \includegraphics[width=0.32\textwidth]{../outputs/epa_iag_Si.pdf}
    \includegraphics[width=0.32\textwidth]{../outputs/epa_iag_P.pdf}
    \includegraphics[width=0.32\textwidth]{../outputs/epa_iag_S.pdf}
    \includegraphics[width=0.32\textwidth]{../outputs/epa_iag_Cl.pdf}
    \includegraphics[width=0.32\textwidth]{../outputs/epa_iag_K.pdf}
    \includegraphics[width=0.32\textwidth]{../outputs/epa_iag_Ca.pdf}
    \includegraphics[width=0.32\textwidth]{../outputs/epa_iag_Ti.pdf}
    \includegraphics[width=0.32\textwidth]{../outputs/epa_iag_V.pdf}
    \includegraphics[width=0.32\textwidth]{../outputs/epa_iag_Mn.pdf}
    \includegraphics[width=0.32\textwidth]{../outputs/epa_iag_Fe.pdf}
    \includegraphics[width=0.32\textwidth]{../outputs/epa_iag_Zn.pdf}
    \includegraphics[width=0.32\textwidth]{../outputs/epa_iag_Br.pdf}
    \includegraphics[width=0.32\textwidth]{../outputs/epa_iag_Sr.pdf}
    \includegraphics[width=0.32\textwidth]{../outputs/epa_iag_Pb.pdf}
  \caption{Comparação das análises de XRF no LAPAt e na US-EPA. 
           \label{fig:epa_lapat}}
\end{figure}
