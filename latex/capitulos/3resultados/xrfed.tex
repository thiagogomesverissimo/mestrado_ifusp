%%%%
\section{Calibração da Fluorescência de Raiox X}

Alvos padrões comerciais (Micromatter) com incertezas 5\% foram utilizados para 
realização do ajuste funcional da calibração do sistema XRF-ED, estratégia esta, 
que permitiu a medição de elementos com alvos padrões inexistentes. Calibrações 
periódicas permitem identificar variações na eficiência do equipamento conforme 
o uso, em particular, no tubo de raios X e no detector. Durante o período de 
irradiação das amostras, maio de 2010 até julho de 2011, uma primeira calibração 
foi feita antes do início das análises, em maio de 2010, uma intermediária, 
em novembro de 2010 e a última próximo do fim, em abril de 2011, totalizando três. 

Depois de irradiar os alvos de calibração, calculou-se os fatores de respostas 
correspondentes a partir da equação \ref{eq:fator_de_resposta}, bem como as 
respectivas incertezas analíticas com a equação \ref{eq:erro_fator_de_resposta}.
Um ajuste polinomial foi suficiente para descrever os pontos experimentais,
sendo que na linha K foi necessário fazer dois ajustes de grau 3, um de Z 
(número atômico) 11 até 26 e o outro de 22 até 42. Para linha L um único 
polinômio de grau 5 foi suficiente para descrever todos pontos.

As curvas de calibração das linhas K e L para maio de 2010 estão plotada no 
gráfico da figura \ref{fig:edx_calib_KLmaio2010}, bem como os respectivos 
ajustes e coeficientes encontrados. Elementos com Z 29 entre 42 apresentam
picos no espectro tanto na linha K quanto L, porém são mais intensos e melhores
definidos na linha K e só foram incluídos no gráfico da linha L no intuito de 
aumentar a quantidade de pontos do ajuste polinomial.

\newpage
\begin{figure}[H]
  \begin{subfigure}[b]{0.5\textwidth}
    \includegraphics[width=\textwidth]{../outputs/CalibrationK2010MaiAkerr.pdf}
    \caption{linha K}
  \end{subfigure}%
  \begin{subfigure}[b]{0.5\textwidth}
    \includegraphics[width=\textwidth]{../outputs/CalibrationL2010MaiAkerr.pdf}
    \caption{linha L}
  \end{subfigure}
  \caption{Calibração XRF-ED em maio de 2010 \label{fig:edx_calib_KLmaio2010}}
\end{figure}

\begin{figure}[H]
  \begin{subfigure}[b]{0.5\textwidth}
   \includegraphics[width=\textwidth]{../outputs/CalibrationK2010NovAkerr.pdf}
    \caption{linha K}
  \end{subfigure}
  \begin{subfigure}[b]{0.5\textwidth}
    \includegraphics[width=\textwidth]{../outputs/CalibrationL2010NovAkerr.pdf}
    \caption{linha L}
  \end{subfigure}
  \caption{Calibração XRF-ED em novembro de 2010 \label{fig:edx_calib_KLnov2010}}
\end{figure}

\newpage
\begin{figure}[H]
  \begin{subfigure}[b]{0.5\textwidth}
    \includegraphics[width=\textwidth]{../outputs/CalibrationK2011AbrAkerr.pdf}
    \caption{linha K}
  \end{subfigure}
  \begin{subfigure}[b]{0.5\textwidth}
    \includegraphics[width=\textwidth]{../outputs/CalibrationL2011AbrAkerr.pdf}
    \caption{linha L}
  \end{subfigure}
\caption{Calibração XRF em  abril de 2011 \label{fig:edx_calib_KLabr2011}}
\end{figure}

As tabelas \ref{fig:edx_calib_KLnov2010} e \ref{fig:edx_calib_KLabr2011}
apresentam a calibração de novembro de 2010 e abril de 2011, respectivamente.

Os fatores de resposta para cada elemento disponível nos alvos padrões, as 
respectivas incertezas propagadas para os mesmos, os fatores de respostas 
calculados a partir das curvas de calibração ajustadas e as respectivas 
incertezas ajustadas usando MQM podem ser conferidos na tabela 
\ref{table:edxAllCalibration}, que mostra também a incerteza percentual de 
cada medida. A incerteza do ajuste é dada pela diagonal da matriz de covariância 
de $[\tilde{Y}]$, $[V_{\tilde{Y}}]$ em \ref{eq:matrizcovarianciaY}, onde [Y] 
são os fatores de respostas [R] e [X] os números atômicos [Z].

O uso do MQM, além de permitir estimar a incerteza do ajuste, ofereceu incerteza
percentual para os alvos padrões comercias, em muitos casos, menores que a 
do próprio fabricante. Na calibração da linha K de maio de 2010 a 
incerteza percentual para o Cálcio (20) no alvo padrão, depois de propagada,
foi de 3,5\%, enquanto que a ajustada com MQM foi 1,4\%. Para o Ferro (26), os
mesmo valores foram 5,0\% e 2,6\%, respectivamente. Elementos com número atômico
muito baixos e portanto alto limite de detecção neste equipamento, a diferença 
na incerteza percentual foi baixa, no caso do Sódio (11), aumento 0,1\% e no 
do Magnésio (12), diminuiu 0,5\%. Esse comportamento foi repetido nas outras 
duas calibrações.

Na linha L...

Uma vez que o PMF pondera sua estimativa pelas incertezas das concentrações a
qualidade das incertezas interfere na determinação da contribuição das fontes.

\begin{landscape}
\begin{table}[H]
  \small
  \centering
    \input{../outputs/edxAllCalibrationK.tex}
  \caption{Calibração da Fluorescência de Raiox X linha K
  \label{table:edxAllCalibration}}
\end{table}
\end{landscape}

\begin{landscape}
%\begin{table}
%  \small
%  \centering
    \input{../outputs/edxAllCalibrationL.tex}
%  \caption{Calibração da Fluorescência de Raiox X  linha L
%  \label{table:edxAllCalibration}}
%\end{table}
\end{landscape}


O queda de desempenho no uso da fluorescência de Raios X pode ser observada 
na figura \ref{fig:compara_calibracao}. 
Possíveis motivos para queda de desempenho: desgaste do tubo ou desgaste do detector. 

\begin{figure}[H]
  \begin{subfigure}[b]{0.5\textwidth}
    \includegraphics[width=\textwidth]{../outputs/CalibrationKcomparacao.pdf}
    \caption{linha K}
  \end{subfigure}%
  \begin{subfigure}[b]{0.5\textwidth}
    \includegraphics[width=\textwidth]{../outputs/CalibrationLcomparacao.pdf}
    \caption{linha L}
  \end{subfigure}
  \caption{Calibrações de Fluorescência de Raiox X em 3 períodos \label{fig:compara_calibracao}}
\end{figure}

%%%%
\subsection{Limite de Detecçção}

\begin{figure}[H]
  \caption{}
  \begin{subfigure}[b]{0.5\textwidth}
    \includegraphics[width=\textwidth]{../outputs/limitDetectionK.pdf}
    \caption{k}
  \end{subfigure}%
  \begin{subfigure}[b]{0.5\textwidth}
    \includegraphics[width=\textwidth]{../outputs/limitDetectionL.pdf}
    \caption{l}
  \end{subfigure}
\end{figure}

A incerteza do ajuste foi calculada usando MQM e foram menores que as garantidas
pelo fabricante nos alvos padrões.  precisão dos elementos com alvo padrão. 

O limite de detecção é pior para elementos de baixo número atômico e há
altos valores de incerteza quando a medida está próxima do limite de detecção.

Pegue no endereço do site os textos e artigos sobre XRF e BC.
Para o XRF há revisões nos textos gerais, como na tese sobre o Epsilon5.
As análises de \textbf{PMF} são ponderadas pelas incertezas, assim 
tentou-se incluir todas fontes possíveis de erro, além do erro instrínseco 
do método analítico. Um erro percentual fixo de 8\% atribuído ao método 
de amostragem foi incluído 
(erro médio observado em amostragens em paralelo com o amostrador Harvard)
\citep{santos2014}.





