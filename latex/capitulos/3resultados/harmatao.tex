\newpage
\section{Composição elementar da poeira do Harmatão}

A poeira do deserto do Saara trazida pelo Harmatão faz aumentar as concentrações
de partículas em mais de 400 \% para alguns dias do inverno, casos extremos, 
que além de ofuscar fontes poluídoras antropogênicas locais, prejudicam 
análises estatísticas multivariadas, pois estas não realizam bem o ajuste 
quando há presença de eventos extremos (\textit{outliers}). Estudos como 
\citet{aboh2009}, \citet{ofosu2013} e \citet{ofosu2012} conduzidos em Gana 
sugerem que quando possível, isto é, com número de amostras suficientementes 
grande, pode-se separar dias do Harmatão nas análises estatísticas multivariadas. 
Caso contrário o Fator representante do Harmatão ofusca as fatores das fontes 
locais. 

O alto número de amostras coletadas em Nima possibilitou a exclusão dos dias de 
ocorrência do Harmatão nas tabelas de concentrações sem comprometer as análises
estatísticas multivariadas, assim como permitiu avaliação exclusiva dos dias
identificados com eventos de Harmatão. 

Como o PMF tentará encontrar solução para qualquer número de fatores inseridos, 
iniciar as análises com AF nos possibilita investigar um número de fontes 
razoáveis, usadas então como primeiras chutes nas análises de PMF. Devido 
ao método de resolução diferente da AF e a ponderação pelas incertezas, os
resultados do PMF confrontados com os da AF permitem avaliação da 
razoabilidade dos ajustes encontrados. 

Na Sam Road, área residencial, 49 amostras de $MP_{2,5-10}$ foram identificadas 
como impactadas pelo Harmatão, e na Nima Road, avenida, 58, números possíveis 
para rodadas de AF ou PMF, pois não são muito pequenos. Os resultados 
da AF rotacionados do $MP_{2,5-10}$ na área residencial e avenida para cinco 
fatores extraídos 
estão nas tabelas \label{table:AF_RGeH5} e \label{table:AF_TGeH5}, 
respectivamente. A variaciância total explicada foi maior que 90\% nos dois  
casos e quase todos elementos foram bem explicado pelo ajuste, com comunalidades 
próximas de um, com exceção do Br e Zn. 

Nos dois casos há um fator predominante explicando boa parte da variância,
43,50 \% e 62,50 \%, apresentado \textit{loadings} altos para quase todos  
elementos - $MP_{2,5-10}$, Na, Mg, Al, Si, K, Ca, Ti, Mn, Fe - 
e claramente associado a poeira do Harmantão. 
O Fator 2, nas duas bases, contém mistura de elementos que podem ser associados 
a veículos, como Zn, Pb e S, mas como se trata de partículas grossas e não finas, 
é possível que dada a quantidade de poeira, do Harmatão e do solo local, 
partículas finas se agreguem os depositem nas superfícies de partículas maiores. 
Os demais fatores praticamente isolaram o P, S e Cl. 

No PMF, os resultados dos perfis dos fatores estão na tabela 
\ref{table:RGeH_profiles5} e contribuição na massa no gráfico da figura
\ref{fig:RGeH_contribution5} para área residencial e tabela 
\ref{table:TGeH_profiles5} e figura \ref{fig:TGeH_contribution5} para 
avenida. 

Na área residencial, o fator predominate é o Fator 2, que contribui
54,19 \% para massa total, associado ao Harmatão. O Fator 1 (19,39 \% da massa)
contém em seu perfil elementos de solo, como Al, Si e Fe, e elementos associados
a veículos, como S, Zn e Pb. Como na área residencial é circundada por vias 
não pavimentadas, muito provavelmente esse fator é resultado de um solo local
contaminado por fontes veículares. Assim como na AF, no PMF ou outros fatores 
além de isolarem alguns elementos, não são significativos, pois contribuiem 
pouco para massa. 

Para $MP_{2,5}$, 51 e 59 amostras foram identificadas como do Harmatão, 
na área residencial e avenida, respectivamente. Novamente extraiu-se cinco
fatores, tanto no PMF quanto na AF. No fino, a AF separou bem o fator da 
poeira do Harmatão do fator veicular, mas continuou não encotrando nada para 
os demais fatores. Já o PMF distinguiu melhor os fatores, pois na área 
residencial, separou o fator associado a solo local, Fator 1, do Harmatão, 
Fator 5, além do Fator 4, com BC, Pb, Zn e S, associado a fonte veicular. 
O mesmo se repetiu na avenida.    

Como observado por \citet{aboh2009} e \citet{ofosu2013} a separação do Harmantão
é importante para melhorar os ajustes da AF e PMF, pois quando presente dias de
 Harmatão, o fator associado ao mesmo, se soprepõem 
aos demais, seja em massa total (PMF), ou em variabilidade (AF). Assim, 
o levantamento de fontes, será realizado com a separação dos dias de Harmantão.     

\newpage
\begin{table}[H]
  \centering
  \input{../outputs/beautifulFAdisplay_RFeH5.tex}
  \caption{Análise de Fatores na área residencial para $MP_{2,5}$
           somente do dias de ocorrência de vento Harmatão. n = 51.
          \label{table:AF_RFeH5}}
\end{table}

\begin{table}[H]
  \centering
  \input{../outputs/beautifulFAdisplay_TFeH5.tex}
  \caption{Análise de Fatores na avenida para $MP_{2,5}$
           somente dias de ocorrência de vento Harmatão. n = 59.
          \label{table:AF_TFeH5}}
\end{table}

\newpage
\begin{table}[H]
  \centering
  \input{../outputs/beautifulFAdisplay_RGeH4.tex}
  \caption{Análise de Fatores na área residencial para $MP_{2,5-10}$
           somente dos dias de ocorrência de vento Harmatão. n = 49.
          \label{table:AF_RGeH4}}
\end{table}

\begin{table}[H]
  \centering
  \input{../outputs/beautifulFAdisplay_TGeH4.tex}
  \caption{Análise de Fatores na avenida para $MP_{2,5-10}$
           somente dos dias de ocorrência de vento Harmatão. n = 58.
          \label{table:AF_TGeH4}}
\end{table}

\begin{landscape}
  \begin{figure}
    \centering
    \begin{minipage}[b]{0.45\linewidth}
      \includegraphics[width=\textwidth]{../outputs/RGeH_pmf_contribution_pizza4.pdf}
      \caption{Contribuição dos fatores na massa total para $MP_{2,5-10}$ na área
               residencial somente nos dias de ocorrência de vento Harmatão. seed = 123 n = 49.
               \label{fig:RGeH_contribution4}}
    \end{minipage}%\hfill  
    \hspace{0.5cm}
    \begin{minipage}[b]{0.45\linewidth}
      \input{../outputs/RGeH_profiles_percent_species4.tex}
      \captionof{table}{Perfis do fatores para $MP_{2,5-10}$ na área residencial,
                 somente nos de ocorrência de vento Harmatão. seed=123 e n= 49. 
                \label{table:RGeH_profiles4}}
    \end{minipage}
  \end{figure}
\end{landscape}

\begin{landscape}
  \begin{figure}
    \centering
    \begin{minipage}[b]{0.45\linewidth}
      \includegraphics[width=\textwidth]{../outputs/TGeH_pmf_contribution_pizza4.pdf}
      \caption{Contribuição dos fatores na massa total para $MP_{2,5-10}$ na avenida
               somente nos dias de ocorrência de vento Harmatão. seed = 123 n = 58.
               \label{fig:TGeH_contribution4}}
    \end{minipage}%\hfill
    \hspace{0.5cm}
    \begin{minipage}[b]{0.45\linewidth}
      \input{../outputs/TGeH_profiles_percent_species4.tex}
      \captionof{table}{Perfis do fatores na avenida $MP_{2,5-10}$ 
                 somente nos de ocorrência de vento Harmatão. seed=123 e n= 58. 
                \label{table:TGeH_profiles4}}
    \end{minipage}
  \end{figure}
\end{landscape}


\begin{landscape}
  \begin{figure}
    \centering
    \begin{minipage}[b]{0.45\linewidth}
      \includegraphics[width=\textwidth]{../outputs/RFeH_pmf_contribution_pizza5.pdf}
      \caption{Contribuição dos fatores na massa total para $MP_{2,5}$ na área
               residencial somente nos dias de ocorrência de vento Harmatão. seed = 123 n = 51.
               \label{fig:RFeH_contribution5}}
    \end{minipage}%\hfill
    \hspace{0.5cm}
    \begin{minipage}[b]{0.45\linewidth}
      \input{../outputs/RFeH_profiles_percent_species5.tex}
      \captionof{table}{Perfis do fatores para $MP_{2,5}$ na área residencial,
                 somente nos de ocorrência de vento Harmatão. seed=123 e n= 51. 
                \label{table:RFeH_profiles5}}
    \end{minipage}
  \end{figure}
\end{landscape}

\begin{landscape}
  \begin{figure}
    \centering
    \begin{minipage}[b]{0.45\linewidth}
      \includegraphics[width=\textwidth]{../outputs/TFeH_pmf_contribution_pizza5.pdf}
      \caption{Contribuição dos fatores na massa total para $MP_{2,5}$ na avenida
               somente nos dias de ocorrência de vento Harmatão. seed = 123 n = 59.
               \label{fig:TFeH_contribution5}}
    \end{minipage}%\hfill
    \hspace{0.5cm}
    \begin{minipage}[b]{0.45\linewidth}
      \input{../outputs/TFeH_profiles_percent_species5.tex}
      \captionof{table}{Perfis do fatores na avenida $MP_{2,5}$ 
                 somente nos de ocorrência de vento Harmatão. seed=123 e n= 59. 
                \label{table:TFeH_profiles5}}
    \end{minipage}
  \end{figure}
\end{landscape}

