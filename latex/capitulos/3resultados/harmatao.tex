\newpage
\section{Composição elementar da poeira do Harmatão}

A poeira do deserto do Saara trazida pelo Harmatão faz aumentar as concentrações
de partículas em mais de 400 \% para alguns dias do inverno. Casos extremos 
que, além de ofuscar fontes poluídoras antropogênicas locais, prejudicam 
análises estatísticas multivariadas, pois estas não realizam bem o ajuste 
quando há presença de eventos extremos (\textit{outliers}).

\citet{aboh2009} e \citet{ofosu2013} sugerem a separação dos dias de Harmatão, 
quando o número de amostras coletadas permitem, para melhora dos ajustes da AF e PMF. 
Quando não removidos, o fator a eles associado sobrepõe-se aos demais, seja em 
massa total (PMF), ou em variabilidade (AF).

O alto número de amostras coletadas em Nima possibilitou analisar separadamente 
o período do Harmatão. Estudaremos nesta seção somente os dias classificados 
com ocorrência de vento Harmatão, aplicando-se AF e PMF. 
Na Sam Road, área residencial, 51 amostras de $MP_{2,5}$ foram identificadas 
como Harmatão, e na Nima Road, avenida, 58. 

Para buscar contornar o obscurecimento das fontes locais, impusemos o mesmo 
número de fatores que foram discriminados sem o Harmatão. Ou seja, consideramos 
que se algo deveria emergir da análise, não seria diverso do que ali estaria sem
esse evento.

Os resultados de AF para $MP_{2,5}$ na área residencial e avenida para cinco 
fatores extraídos estão nas tabelas \ref{table:AF_RFeH5} e \ref{table:AF_TFeH5},
respectivamente.  A variância total explicada foi maior que 90\% nos dois  
casos e quase todas espécies foram bem explicadas pelo ajuste, com comunalidades 
próximas de 1. 

Apesar do ótimo ajuste da AF, uma rápida olhada nos \textit{laodings} extraídos
revela que o primeiro fator agrupa praticamente sozinho todas as espécies, 
mas trazendo com mais intensidade os elementos característicos da poeira do solo.
Entendemos, assim, que ele representa especialmente o Harmatão e o solo local. 
O Br aparece isolado nos dois locais (fator 4 e 5, residencial e avenida, 
respectivamente), que caracterizamos anteriormente como queima de lixo sólido. 
Já nos demais fatores emergem apenas sinais confusos das fontes detectadas sem 
o Harmatão, alternando \textit{laodings} de espécies que indicam ligações tanto 
com queima de biomassa, quanto veículos (BC, S, P), e mais delineado nos 
fatores 3 e 5 (Zn e Pb), que associamos a veículos, acompanhado com 
\textit{loadings} altos de Na.

No PMF, os resultados para área residencial dos perfis dos fatores estão na 
tabela \ref{table:RFeH_profiles5} e a contribuição para massa total no gráfico 
da figura \ref{fig:RFeH_contribution5}. Para a avenida, os perfis estão na 
tabela \ref{table:TFeH_profiles5} e a contribuição no gráfico da figura 
\ref{fig:TFeH_contribution5}. Diferente da AF, o PMF conseguiu distinguir melhor
os fatores, com a mesma associações de fontes que já fizemos: solo, veículos, 
queima de biomassa, mar, queima de lixo, mas que apresentam peculiaridades e 
devem ser vistos com cuidado. Na área residencial os fatores 1 e 5 seriam 
associáveis a solo, considerando-se o primeiro mais ligado ao Harmatão e o 
último, a um solo local (por apresentar V, Zn e P). As espécies típicas de solo 
também aparecem, em menor percentual, nos fatores 3 e 4, mas neles encontramos 
entrelaçadas espécies traçadoras de queima de biomassa e veículos, bem como Na 
que pode vir do aerossol marinho. 
Na avenida os correspondentes para solo seriam os fatores 1 e 4, o mar aparece 
separado (fator 2) e o fator 3 representa a mistura veículos, queima de biomassa
e frações de solo. O Br domina o fator 2 na área residencial e o 5 na avenida, 
que relacionaríamos à queima de lixo sólido. Acreditamos que esse mix - 
veículos, biomassa, solo - possa ser resultado de processo de agregação de 
partículas finas, entre si, e/ou com partículas de solo, como resultado da 
evolução desse conjunto de partículas naquela atmosfera. Mas resta sempre a 
possibilidade deste ser um resultado recepcionando Zn, que podem originar-se do 
desgaste de componentes veiculares apenas das imprecisões que acompanham o 
predomínio do Harmatão.

Na AF para $MP_{2,5-10}$, tabelas \ref{table:AF_RGeH4} e \ref{table:AF_TGeH4}, 
os 4 fatores extraídos apresentam problemas de definição. Teríamos na área 
residencial o fator 1, associável ao Harmartão, e fator 4, ao solo local 
(mas contendo também Zn, S que podem originar-se de veículos), que na avenida 
seriam os fatores 4 e 3. O mar estaria marcando os fatores 2 e 1 - residencial 
e avenida - respectivamente. O fator 3 da área residencial e o 2 na avenida, 
assemelham-se a uma espécie de sopa de finos e grossos em envelhecimento naquela
atmosfera.

 No PMF, tabelas 
\ref{table:RGeH_profiles4} e \ref{table:TGeH_profiles4}, e gráficos 
\ref{fig:RGeH_contribution4} e \ref{fig:TGeH_contribution4}, na avenida e na
área residencial o fator solo contribuiu em mais de 60 \% para massa total, 
sendo que agora o fator mar somente apareceu na avenida (mais próxima do mar).

Assim, apesar do esforço para identificarmos o conjunto de fontes no período 
Harmatão, temos uma percepção consciente, ao observar os resultados dos modelos 
receptores, de que mesmo as fontes locais mais ponderáveis, tendem a ficar 
obscurecidas por esse fenômeno. Nestas circunstâncias, cuidado especial deve 
ser tomado com o PMF, que tenta encontrar solução, qualquer que seja o número 
de fatores inseridos. O processo numérico não
é capaz de avaliar se isso faz sentido no problema atmosférico estudado. 
Devemos ter especial cautela, portanto, com os resultados sintetizados na 
tabela, no que se refere ao período do Harmatão. Mesmo com as limitações da 
AF neste caso, a PMF essas modelagens utilizam metodologias distintas para 
extrair fatores de uma base de dados, e a combinação destas duas 
metodologias também auxilia na avaliação da razoabilidade dos ajustes 
encontrados. A síntese das associações dos fatores de AF e PMF para o período do Harmatão 
estão nas tabelas \ref{sintese_grosso_harmatao} e \ref{sintese_fino_harmatao}.

\newpage
\begin{table}[H]
  \centering
  \caption{Análise de Fatores na área residencial para $MP_{2,5}$
           somente do dias de ocorrência de vento Harmatão. n = 51.
          \label{table:AF_RFeH5}}
  \input{../outputs/beautifulFAdisplay_RFeH5.tex}

\end{table}

\begin{table}[H]
  \centering
  \caption{Análise de Fatores na avenida para $MP_{2,5}$
           somente dias de ocorrência de vento Harmatão. n = 59.
          \label{table:AF_TFeH5}}
  \input{../outputs/beautifulFAdisplay_TFeH5.tex}

\end{table}

\begin{landscape}
  \begin{figure}
    \centering
    \begin{minipage}[b]{0.45\linewidth}
      \includegraphics[width=\textwidth]{../outputs/RFeH_pmf_contribution_pizza5.pdf}
      \caption{Contribuição dos fatores na massa total para $MP_{2,5}$ na área
               residencial somente nos dias de ocorrência de vento Harmatão. seed = 123 n = 51.
               \label{fig:RFeH_contribution5}}
    \end{minipage}%\hfill
    \hspace{0.5cm}
    \begin{minipage}[b]{0.45\linewidth}
      \captionof{table}{Perfis (\%) do fatores para $MP_{2,5}$ na área residencial,
                 somente nos de ocorrência de vento Harmatão. seed=123 e n= 51. 
                \label{table:RFeH_profiles5}}
      \input{../outputs/RFeH_profiles_percent_species5.tex}

    \end{minipage}
  \end{figure}
\end{landscape}

\begin{landscape}
  \begin{figure}
    \centering
    \begin{minipage}[b]{0.45\linewidth}
      \includegraphics[width=\textwidth]{../outputs/TFeH_pmf_contribution_pizza5.pdf}
      \caption{Contribuição dos fatores na massa total para $MP_{2,5}$ na avenida
               somente nos dias de ocorrência de vento Harmatão. seed = 123 n = 59.
               \label{fig:TFeH_contribution5}}
    \end{minipage}%\hfill
    \hspace{0.5cm}
    \begin{minipage}[b]{0.45\linewidth}
      \captionof{table}{Perfis (\%) do fatores na avenida $MP_{2,5}$ 
                 somente nos de ocorrência de vento Harmatão. seed=123 e n= 59. 
                \label{table:TFeH_profiles5}}
      \input{../outputs/TFeH_profiles_percent_species5.tex}

    \end{minipage}
  \end{figure}
\end{landscape}


\newpage
\begin{table}[H]
  \centering
  \caption{Análise de Fatores na área residencial para $MP_{2,5-10}$
           somente dos dias de ocorrência de vento Harmatão. n = 49.
          \label{table:AF_RGeH4}}
  \input{../outputs/beautifulFAdisplay_RGeH4.tex}

\end{table}

\begin{table}[H]
  \centering
  \caption{Análise de Fatores na avenida para $MP_{2,5-10}$
           somente dos dias de ocorrência de vento Harmatão. n = 58.
          \label{table:AF_TGeH4}}
  \input{../outputs/beautifulFAdisplay_TGeH4.tex}
\end{table}

\begin{landscape}
  \begin{figure}
    \centering
    \begin{minipage}[b]{0.45\linewidth}
      \includegraphics[width=\textwidth]{../outputs/RGeH_pmf_contribution_pizza4.pdf}
      \caption{Contribuição dos fatores na massa total para $MP_{2,5-10}$ na área
               residencial somente nos dias de ocorrência de vento Harmatão. seed = 123 n = 49.
               \label{fig:RGeH_contribution4}}
    \end{minipage}%\hfill  
    \hspace{0.5cm}
    \begin{minipage}[b]{0.45\linewidth}
      \captionof{table}{Perfis (\%) do fatores para $MP_{2,5-10}$ na área residencial,
                 somente nos de ocorrência de vento Harmatão. seed=123 e n= 49. 
                \label{table:RGeH_profiles4}}
      \input{../outputs/RGeH_profiles_percent_species4.tex}
    \end{minipage}
  \end{figure}
\end{landscape}

\begin{landscape}
  \begin{figure}
    \centering
    \begin{minipage}[b]{0.45\linewidth}
      \includegraphics[width=\textwidth]{../outputs/TGeH_pmf_contribution_pizza4.pdf}
      \caption{Contribuição dos fatores na massa total para $MP_{2,5-10}$ na avenida
               somente nos dias de ocorrência de vento Harmatão. seed = 123 n = 58.
               \label{fig:TGeH_contribution4}}
    \end{minipage}%\hfill
    \hspace{0.5cm}
    \begin{minipage}[b]{0.45\linewidth}
      \captionof{table}{Perfis (\%) do fatores na avenida $MP_{2,5-10}$ 
                 somente nos de ocorrência de vento Harmatão. seed=123 e n= 58. 
                \label{table:TGeH_profiles4}}
      \input{../outputs/TGeH_profiles_percent_species4.tex}
    \end{minipage}
  \end{figure}
\end{landscape}

\begin{landscape}
\begin{table}[H]
  \centering
  \caption{Síntese das associações dos fatores extraídos na AF e PMF com fontes 
           poluidoras para $MP_{2,5}$ somente para os dias de Harmartão. 
           \label{sintese_fino_harmatao}}
  \begin{tabular}{lll|lll|ll|lll}
\hline
                                                                         & \multicolumn{5}{c|}{Residencial (massa média 231,42 $\mu g / m^3$)} & \multicolumn{5}{c}{Avenida (massa média 174,93 $\mu g / m^3$)}    \\
                                                                          & \multicolumn{2}{c}{AF}      & \multicolumn{3}{c|}{PMF}              & \multicolumn{2}{c}{AF}                      & \multicolumn{3}{c}{PMF}          \\
\hline
  & & & \multicolumn{3}{c|}{contribuição na massa} & & & \multicolumn{3}{c}{contribuição na massa} \\
Fonte associada                                   & $NF^1$   & VE$^2$ (\%)               & $NF^1$   & (\%)   & $\mu g / m^3$  & $NF^1$       & VE$^2$  (\%)            & $NF^1$ & (\%)  & $\mu g / m^3$ \\
\hline
\textbf{Solo}                                     &  1   & 59,21                 & 5   & 36,8          & \textbf{85,16}   & 1       & 67,14                  & 1 & 41,50         & \textbf{78,89}  \\
\textbf{Queima biomassa}                          &  2   & 10,53                 & 4   & 15,75        & \textbf{36,44}      & 2       & 7,98                 & 3 & 21,68         & \textbf{37,92 }  \\
\textbf{Veículos}                                 &  3   & 8,54                  & 3   & 19,12           & \textbf{44,25}  & 4       & 6,69                  & 4 & 30,66          & \textbf{53,63 } \\
\textbf{Mar}                                      &  4   & 7,38                  & 1   & 23,39          & \textbf{53,90}   & 3       & 6,82                  & 2 & 3,59           & \textbf{6,30}  \\
\textbf{Queima lixo}                              &  5   & 6,28                  & 2   & 4,94         & \textbf{11,43}   & 5       & 6,31                    & 5 & 2,58           & \textbf{4,51} \\

\hline
\multicolumn{11}{l}{$^1$ NF: Número do Fator} \\
\multicolumn{11}{l}{$^2$ VE: Variância Explicada} \\
\hline
\end{tabular}

\end{table}

\begin{table}[H]
  \centering
  \caption{Síntese das associações dos fatores extraídos na AF e PMF com fontes 
           poluidoras para $MP_{2,5-10}$ somente para os dias de Harmartão. 
           \label{sintese_grosso_harmatao}}
  \begin{tabular}{lll|lll|ll|lll}
\hline
                                                                          & \multicolumn{5}{c|}{Residencial (massa média 112,24 $\mu g / m^3$)} & \multicolumn{5}{c}{Avenida (massa média 112,09 $\mu g / m^3$)}    \\
                                                                          & \multicolumn{2}{c}{AF}      & \multicolumn{3}{c|}{PMF}              & \multicolumn{2}{c}{AF}                    & \multicolumn{3}{c}{PMF}          \\
\hline
 & & &  \multicolumn{3}{c|}{contribuição na massa} & & & \multicolumn{3}{c}{contribuição na massa} \\
Fonte associada                                    & $NF^1$   & VE$^2$ (\%)               & $NF^1$   & (\%)   & $\mu g / m^3$  & $NF^1$       & VE$^2$  (\%)            & $NF^1$ & (\%)  & $\mu g / m^3$ \\
\hline
\textbf{Solo}                                                    & 1  & 60,30               & 1      & 60,21 & \textbf{67,57} & 1  & 57,70               & 4      & 64,43 & \textbf{72,22} \\
\textbf{Mar}                                                     & 2  & 11,60               & 3      & 7,49  & \textbf{8,41} & 3  & 12,16                & 2      & 0,04  & \textbf{0,05}  \\
\textbf{Poeira de estrada}                                       & 3  & 9,04                & 2      & 14,26 & \textbf{16,01} & 4  & 12,01               & 1      & 7,63  & \textbf{8,55}   \\ 
\textbf{Envelhecido}                                             & 4  & 7,71                & 4      & 18,04 & \textbf{20,25} & 2  & 9,79                & 3      & 27,9  & \textbf{31,27}  \\


\hline
\multicolumn{11}{l}{$^1$ NF: Número do Fator} \\
\multicolumn{11}{l}{$^2$ VE: Variância Explicada} \\
\hline
\end{tabular}

\end{table}

\end{landscape}
