\newpage
\section{Composição elementar da poeira do Harmatão}


\citet{aboh2009} e \citet{ofosu2013} sugerem a separação do Harmantão, quando 
o número de amostras coletadas permitir, para melhora dos ajustes da AF e PMF, 
pois quando não removidos, o fator associado a ele, se soprepõem 
aos demais, seja em massa total (PMF), ou em variabilidade (AF).

O alto número de amostras coletadas em Nima possibilitou analisar separadamente 
o período do Harmatão. Analisa-se nesta seção somentes os dias classificados 
com ocorrência de vento Harmatão, aplicando-se AF e PMF. 
Na Sam Road, área residencial, 51 amostras de 
$MP_{2,5}$ foram identificadas como Harmatão, e na Nima Road, avenida, 58. 

Os resultados de AF para $MP_{2,5}$ na área residencial e avenida para cinco 
fatores extraídos estão nas tabelas \label{table:AF_RFeH5} e \label{table:AF_TFeH5}, 
respectivamente.  A variaciância total explicada foi maior que 90\% nos dois  
casos e quase todos elementos foram bem explicado pelo ajuste, com comunalidades 
próximas de um. 

Apesar do ótimo ajuste da AF, uma rápida olhada nos \textit{laodings} extraídos
percebe-se que o primeiro fator praticamente agrupa todos elementos sozinho, com 
exceção do fator 2 da área residencial que apareceu \textit{laodings} de 0,80 
e 0,89 para BC e S, respectivamente, associado provavelmente a veículos.

No PMF, os resultados para área residencial dos perfis dos fatores estão na 
tabela \ref{table:RFeH_profiles5} e a contribuição para massa total no gráfico 
da figura \ref{fig:RFeH_contribution5}. Para a avenida, os perfis estão na 
tabela \ref{table:TFeH_profiles5} e a contribuição no gráfico da figura 
\ref{fig:TFeH_contribution5}. Diferente da AF, o PMF conseguiu distinguir melhor
os fatores, com fontes associadas iguais as já identificados: solo, veículos, 
queima de biomassa, mar, queima de lixo, com a diferença de que os elementos 
Mg, Al, Si, Ca, Ti e Fe estão distríbuidos em todos fatores.

Na AF para $MP_{2,5-10}$, tabelas \label{table:AF_RGeH4} e \label{table:AF_TGeH4}, 
com 4 fatores extraídos, além do fator 1, associado a solo, o fator 2, com 
altos \textit{laodings} de Na e Cl, nenhum outro fator tem significado físico, 
tanto na avenida quanto na área residencial. No PMF, tabelas 
\ref{table:RGeH_profiles4} e \ref{table:TGeH_profiles4}, e gráficos 
\ref{fig:RGeH_contribution4} e \ref{fig:TGeH_contribution4}, na avenida e na
área residencial o fator solo contribuiu em mais de 60 \% para massa total, 
sendo que agora o fator mar somente apareceu na avenida (mais próxima do mar).

Assim, temos uma percepção consciente, ao observar os resultados dos modelos 
receptores, de que mesmo as fontes locais mais ponderáveis, tendem a ficar 
obscurecidas por esse fenômeno.
Para buscar, delineá-las, impusemos o mesmo número de fatores que foram 
discriminados sem o Harmatão. Ou seja, consideramos que se algo deveria emergir 
da análise, não seria diverso do que ali estaria sem esse evento. 
Isso é particularmente importante para o modelo PMF, que tenta encontrar 
solução qualquer que seja o número de fatores inseridos. O processo numérico não
é capaz de avaliar se isso faz sentido no problema atmosférico estudado. 
Já a AF possibilitou-nos investigar um número razoável de fatores, uma vez que 
agrupamos espécies que apresentam-se correlacionadas na base de dados. 
Como a AF e a PMF utilizam metodologias 
distintas para extrair fatores de uma base de dados, a combinação destas duas 
metodologias também auxilia na avaliação da razoabilidade dos ajustes encontrados. 

\newpage
\begin{table}[H]
  \centering
  \input{../outputs/beautifulFAdisplay_RFeH5.tex}
  \caption{Análise de Fatores na área residencial para $MP_{2,5}$
           somente do dias de ocorrência de vento Harmatão. n = 51.
          \label{table:AF_RFeH5}}
\end{table}

\begin{table}[H]
  \centering
  \input{../outputs/beautifulFAdisplay_TFeH5.tex}
  \caption{Análise de Fatores na avenida para $MP_{2,5}$
           somente dias de ocorrência de vento Harmatão. n = 59.
          \label{table:AF_TFeH5}}
\end{table}

\begin{landscape}
  \begin{figure}
    \centering
    \begin{minipage}[b]{0.45\linewidth}
      \includegraphics[width=\textwidth]{../outputs/RFeH_pmf_contribution_pizza5.pdf}
      \caption{Contribuição dos fatores na massa total para $MP_{2,5}$ na área
               residencial somente nos dias de ocorrência de vento Harmatão. seed = 123 n = 51.
               \label{fig:RFeH_contribution5}}
    \end{minipage}%\hfill
    \hspace{0.5cm}
    \begin{minipage}[b]{0.45\linewidth}
      \input{../outputs/RFeH_profiles_percent_species5.tex}
      \captionof{table}{Perfis (\%) do fatores para $MP_{2,5}$ na área residencial,
                 somente nos de ocorrência de vento Harmatão. seed=123 e n= 51. 
                \label{table:RFeH_profiles5}}
    \end{minipage}
  \end{figure}
\end{landscape}

\begin{landscape}
  \begin{figure}
    \centering
    \begin{minipage}[b]{0.45\linewidth}
      \includegraphics[width=\textwidth]{../outputs/TFeH_pmf_contribution_pizza5.pdf}
      \caption{Contribuição dos fatores na massa total para $MP_{2,5}$ na avenida
               somente nos dias de ocorrência de vento Harmatão. seed = 123 n = 59.
               \label{fig:TFeH_contribution5}}
    \end{minipage}%\hfill
    \hspace{0.5cm}
    \begin{minipage}[b]{0.45\linewidth}
      \input{../outputs/TFeH_profiles_percent_species5.tex}
      \captionof{table}{Perfis (\%) do fatores na avenida $MP_{2,5}$ 
                 somente nos de ocorrência de vento Harmatão. seed=123 e n= 59. 
                \label{table:TFeH_profiles5}}
    \end{minipage}
  \end{figure}
\end{landscape}


\newpage
\begin{table}[H]
  \centering
  \input{../outputs/beautifulFAdisplay_RGeH4.tex}
  \caption{Análise de Fatores na área residencial para $MP_{2,5-10}$
           somente dos dias de ocorrência de vento Harmatão. n = 49.
          \label{table:AF_RGeH4}}
\end{table}

\begin{table}[H]
  \centering
  \input{../outputs/beautifulFAdisplay_TGeH4.tex}
  \caption{Análise de Fatores na avenida para $MP_{2,5-10}$
           somente dos dias de ocorrência de vento Harmatão. n = 58.
          \label{table:AF_TGeH4}}
\end{table}

\begin{landscape}
  \begin{figure}
    \centering
    \begin{minipage}[b]{0.45\linewidth}
      \includegraphics[width=\textwidth]{../outputs/RGeH_pmf_contribution_pizza4.pdf}
      \caption{Contribuição dos fatores na massa total para $MP_{2,5-10}$ na área
               residencial somente nos dias de ocorrência de vento Harmatão. seed = 123 n = 49.
               \label{fig:RGeH_contribution4}}
    \end{minipage}%\hfill  
    \hspace{0.5cm}
    \begin{minipage}[b]{0.45\linewidth}
      \input{../outputs/RGeH_profiles_percent_species4.tex}
      \captionof{table}{Perfis (\%) do fatores para $MP_{2,5-10}$ na área residencial,
                 somente nos de ocorrência de vento Harmatão. seed=123 e n= 49. 
                \label{table:RGeH_profiles4}}
    \end{minipage}
  \end{figure}
\end{landscape}

\begin{landscape}
  \begin{figure}
    \centering
    \begin{minipage}[b]{0.45\linewidth}
      \includegraphics[width=\textwidth]{../outputs/TGeH_pmf_contribution_pizza4.pdf}
      \caption{Contribuição dos fatores na massa total para $MP_{2,5-10}$ na avenida
               somente nos dias de ocorrência de vento Harmatão. seed = 123 n = 58.
               \label{fig:TGeH_contribution4}}
    \end{minipage}%\hfill
    \hspace{0.5cm}
    \begin{minipage}[b]{0.45\linewidth}
      \input{../outputs/TGeH_profiles_percent_species4.tex}
      \captionof{table}{Perfis (\%) do fatores na avenida $MP_{2,5-10}$ 
                 somente nos de ocorrência de vento Harmatão. seed=123 e n= 58. 
                \label{table:TGeH_profiles4}}
    \end{minipage}
  \end{figure}
\end{landscape}
