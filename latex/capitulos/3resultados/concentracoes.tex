%%%%
\newpage
\section{Concentrações Ambientais}

Medidas de concetrações da massa de MP são usadas para avaliação da qualidade do
ar, as quais relacionam-se à saúde, mas que também associam-se a danos a 
vegetais, materiais e ao meio ambiente em geral. 
As legislações fundamentam-se sobre estudos de impactos à saúde, atribuíveis a 
poluentes, em curto ou longo espaço de tempo. Procura-se definir, em função 
disso, os limites que permitem boa qualidade de vida ao cidadão.

Fruto de reuniões de trabalho internacionais, a OMS \citeyearpar{who} definiu 
limites para os principais poluentes antropogênicos da atualidade, as quais são 
diretrizes indicadas para todos os países. Mas, pela própria constituição deste
organismo internacional, cada país tem autonomia para definir sua própria 
legislação, podendo ou não acatá-las. De qualquer modo, ao sustentar padrões 
assentados sobre estudos de riscos e danos à saúde, atuais e de amplo 
reconhecimento internacional, ela cria um referencial para instituições e 
movimentos sociais que pressionam por melhores níveis de qualidade de vida em 
cada país.

Em estudos mais detalhados, é possível aprofundar a análise de dados com 
identificação e quantificação de espécies que compõem o MP (elementos, compostos,
misturas, ou fatores diversos que interferem em sua concentração). 
Além de revelar os fatores mais significativos nas concentrações determinadas, 
ou nos efeitos sobre a saúde, eles possibilitam ajustes de métodos estatísticos 
para identificar e estimar perfis das principais fontes poluidoras. 
Assim, nesta seção serão expostos inicialmente os resultados das concentrações 
para comparação com limites de legislações e com padrões sugeridos pela OMS, 
e na próxima seção, serão aplicados métodos 
estatísticos multivariados, AF e PMF, para levantamento de fontes. 

A campanha de amostragem nos quatro bairros, conforme já registramos, resultou 
em um total de 2898 amostras que foram analisadas por refletância e XRF-ED no 
LAPAt. Concentrações médias, análises estatísticas gerais e levantamento de 
fontes, para os onze locais em conjunto, foram apresentados em 2011 no 
\textit{23th Congress of the International Society for Environmental 
Epidemiology} \citep{zhou2011} e publicadas em \citep{zhou2013} e 
\citep{zhou2014}. Mas estes trabalhos tinham um olhar mais voltado às questões 
de saúde pública, tratando o  $PM_{10}$ e $PM_{2,5}$, sem separar o 
$PM_{2,5-10}$. Isso limita a capacidade de discernir o MP originário de 
processos mecânicos ou de conversão gás-partícula, bem como de possíveis 
interações entre eles.

Nesta pesquisa em particular, aprofundou-se os estudos exclusivamente no bairro
de Nima, em dois locais, Sam Road e Nima Road, com 791 amostras válidas obtidas
entre 11 de novembro de 2006 e 15 de agosto de 2008. Além da separação do MP 
grosso, informações sobre a circulação regional de ventos nas discussões foram 
incoporadas, o que contribue no processo de discernimento de prováveis fontes 
que chegam aos amostradores.

Daqui em diante, medidas realizadas na Sam Road serão referidas como área 
residencial e na Nima Road como avenida, para melhor refletir as características
de ambas.

%%%%
%\newpage
\subsection{Material Particulado Inalável $MP_{10}$}

O presumível interesse na saúde pública como elemento central na definição de 
padrões de qualidade do ar nas legislações específicas, termina dependendo de
aspectos políticos, sociais e econômicos de cada país. Em virtude disso, há entre
estes, uma multiplicidade de limites para as espécies legisladas.

O padrão de qualidade do ar para $MP_{10}$ ambiental vigente em Gana, referendado
pela EPA-GH \citeyearpar{epa2015}, é apresentado na tabela 
\ref{table:padroesgana}. Nota-se que há limites específicos para os meses em que 
ocorre o Harmatão, refletindo o efeito deste fenômeno no MP local. 
Regiões consideradas como zonas industriais ou de comércio também têm limites 
mais permissivos, revelando uma priorização do Governo aos interesses negociais
no país, em detrimento da saúde e bem estar da população.

\begin{table}[H]
\centering
  % http://www.epa.gov.gh/ghanalex/policies/EPAguidelines%20Report.pdf
\begin{tabular}{cccc}
\hline
                              &   regiões  &        zonas       &         zonas               \\
                              & sensíveis  & residenciais e rurais & insdustriais e comerciais      \\
Tipo da média                 & $\mu / m^3$ & $\mu / m^3$ & $\mu / m^3$      \\
\hline
diária                    & 110             & 150                      & 260         \\       
anual geométrica          & 70              & 100                      & 200                   \\
mensal (durante Harmatão) & 100             & 200                      & 500                     \\
\hline
\end{tabular}

\caption{Padrões de Qualidade do Ar para $MP_{10}$ Ambiental em Gana
         \cite{epa2015} \label{table:padroesgana}}
\end{table}

Na tabela \ref{table:pm10standards} compara-se os padrões de $MP_{10}$ entre 
Brasil, Gana e OMS, com a observação de que em Gana a média anual é calculada 
geometricamente, o que a torna tanto menor que a média aritmética, quanto 
mais extremos forem os valores registrados.

\begin{table}[H]
\centering
  \begin{tabular}{cccc}
\hline
              & Brasil & Gana (residencial e rural) & OMS \\
Tipo da média & $\mu / m^3$ & $\mu / m^3$ & $\mu / m^3$          \\
\hline
diária   & 150              & 150              &  50             \\
anual    &  50 (aritmética) & 100 (geométrica) &  20 (aritmética) \\
\hline
\end{tabular}

  \caption{Padrões para média anual de $MP_{10}$ no Brasil \citep{conama1990}, 
           Gana \citep{epa2015} e OMS \citep{who}. \label{table:pm10standards}}
\end{table}

O gráfico da figura \ref{fig:massa_temporal_mp10} mostra as concetrações medidas
(com distâncias do eixo em escala logaritímica para melhor vizualização dos 
dados) nos dois sítios em função do tempo, sinalizando os padrões diários 
da tabela anterior. 

Observa-se, que para verificação do atendimento ou não de padrão anual previstos
na legislação de Gana, será necessário remover as amostras de 2006 e 2008, 
mantendo somente as de 2007, pois foi o único ano deste experimento com medidas 
uniformemente distribuídas em todos meses. A confrontação com parâmetros legais 
será realizada somente para dar noção relativa e qualitativa dos resultados 
obtidos na campanha e não para observância estrita de atendimento ou não da lei.

Sendo assim, o índice de ultrapassagens do padrão diário foi 16,24 \% na área 
residencial e 19,60 \% na avenida, quando considerado o padrão de 150 
$\mu g / m^3$ praticado pela EPA-GH e no Brasil. Em relação
ao padrão de média diária da OMS, 50 $\mu g / m^3$, mais restritivo, 
o índice de ultrapassagens foi alto para os dois sítios: 59,90 \% 
na área residencial e 90,95 \% na avenida.

\begin{figure}[H]
  \centering
  \begin{subfigure}[b]{0.45\textwidth}
    \includegraphics[width=\textwidth]{../outputs/massa_temporal_RIcH.pdf}
    \caption{Área residencial (Sam Road)}
  \end{subfigure}%
  \begin{subfigure}[b]{0.45\textwidth}
    \includegraphics[width=\textwidth]{../outputs/massa_temporal_TIcH.pdf}
    \caption{Avenida (Nima Road)}
  \end{subfigure}
  \caption{Concentrações de $MP_{10}$ ao longo da campanha de amostragem.
           \label{fig:massa_temporal_mp10}}
\end{figure}

Regiões localizadas próximas ao deserto do Saara sofrem mais diretamente 
eventos extremos de poluição. \citet{kaku2016} ao estudar partículas no 
Golfo Pérsico, a leste do deserto do Saara, relatou que em períodos de chegada 
de poeira, 76$\pm$7\% da massa total de MP é comprometida com partículas da 
crosta terrestre, ou seja, de solo.

A separação dos dias em que ocorrem ventos do Harmatão dos outros dias é 
importante, pois além de ser previsto na lei local, permite realizar a 
comparação separando uma dinâmica mais antropogênica de outra natural, 
ligada a esse fenômeno.

O critério de identificação dos dias de Harmatão, entretanto, não é simples, 
pois o fenômeno não é ininterrupto durante os meses em que ocorre. 
Classificando todos dias de novembro até março como Harmantão, corre-se o risco 
de perder informações importantes de fontes locais com atividades nesse período, 
comprometendo análises estatísticas de AF e PMF. 

Considerando o fato dos silicatos serem os principais constituintes da crosta
terrestre, e tendo em vista o convívio com esse fenômeno, \citet{aboh2009} 
sugere como critério para identificação dos dias de ocorrência do Harmatão os de
concentrações de silício no $MP_{10}$ maiores que 10 $\mu g/m^3$, entre os meses
de novembro e março. 

Quando desconsidera-se o Harmatão, seguindo esse critério,
não há mais ultrapassagens do padrão diário de 150 $\mu g / m^3$ na área 
residencial e na avenida o índice de ultrapassagem cai de 19,60 \% para 0,88 \%. 
Para o padrão OMS de 50 $\mu g / m^3$ na área residencial as 
ultrapassagens diminuem para 31,25 \% (antes 59,90 \%) e na avenida para 
84,21 \% (antes 90,95 \%).

\begin{table}[H]
  \centering
  \input{../outputs/descriptive_inalavel_harmatao}
  \caption{Estatística descritiva das concentrações de $MP_{10}$ conjunta
           (área residencial e avenida) somente para os dias de ocorrência 
           de vento do Harmatão. 54 amostras na área residencial e 59 na avenida 
          \label{table:descriptive_inalavel_harmatao}}
\end{table}

A tabela \ref{table:descriptive_inalavel_harmatao} apresenta a média, 
desvio padrão da média, mediana, mínimo e máximo para as concentrações de 
MP Inalável, $MP_{10}$ para dias de ocorrência do Harmatão, classificados
segundo sugestão de \citet{aboh2009}. A massa total média foi 269,2 $\pm$ 23,0
$\mu g/ m^3$, tendo como elemento mais representativo o Si, 11,3 \% da massa total. 
A ambiguidade (talvez estratégica) da lei define um intervalo de média mensal 
de $MP_{10}$ no Harmantão de 100 à 500 $\mu g/ m^3$, dependendo de como a região
é classificada: sensível, residencial, rural, industrial ou comercial, abrindo
brechas que permitem definir regiões baseado em interesses puramente econômicos,
pois limites de 500 $\mu g/ m^3$ mesmo com a poeira pesada do Harmatão serão
dificilmente alcançados. A mediana, permite ter uma ideia de assimetrias na 
distribuição de um conjunto de dados. Neste caso ela foi de 154,7 $\mu g/ m^3$, 
indicando que houve extremos com valores altos, já que a média foi de 
269,2 $\mu g/ m^3$. O máximo alcançou 1255,8 $\mu g/ m^3$, mas no geral apenas 
alguns dias tiveram concentrações extremamente elevadas. %

Excluíndo-se as concentrações de $MP_{10}$ medidas nos dias de ocorrência do 
Harmantão, obtém-se as estatísticas descritivas apresentadas na tabela 
\ref{table:descriptive_inalavel_sH}.

\begin{table}[H]
  \centering
    \input{../outputs/descriptive_inalavel_sH}
  \caption{Estatística descritiva das concentrações de $MP_{10}$ conjunta, 
           (Sam Road e Nima Road) excluindo-se os dias do Harmantão.
            \label{table:descriptive_inalavel_sH}}
\end{table}

Entretanto, para melhor comparação com os padrões anuais, e inclusão de todas 
sazonalidades de um ano, na tabela \ref{table:mediasmp10} apresenta-se médias 
aritméticas e geométricas exclusivas para 2007, que teve medidas ao longo de 
todo ano.

\begin{table}[H]
  \centering
  % latex table generated in R 3.2.4 by xtable 1.7-1 package
% Mon May 16 17:18:28 2016
\begin{tabular}{cccccc|ccccc}
  \hline
  &  \multicolumn{5}{c|}{com Harmatão} & \multicolumn{5}{c}{sem Harmatão} \\
 & n & $\overline{x}^*$ & $\overline{x}_g^{**}$ & $\sigma^{***}$ & $\overline{\sigma}^{***}$
 & n & $\overline{x}^*$ & $\overline{x}_g^{**}$ & $\sigma^{***}$ & $\overline{\sigma}^{***}$ \\
                       \hline & \multicolumn{10}{c}{$\mu g \cdot m^{-3}$} \\  \hline
residencial & 87 & 44,91 & 43,14 & 11,87 & 1,12 & 136 & 115,12 & 67,72 & 186,39 & 13,28 \\ 
  avenida & 89 & 61,57 & 60,55 & 11,52 & 1,08 & 138 & 131,65 & 88,99 & 183,62 & 13,02 \\ 
  ambas & 176 & 53,33 & 51,21 & 14,34 & 0,95 & 274 & 123,45 & 77,70 & 184,85 & 9,29 \\ 
   \hline
\multicolumn{11}{l}{$^{*}$ Média Aritimética} \\
\multicolumn{11}{l}{$^{**}$ Média Geométrica} \\
\multicolumn{11}{l}{$^{***}$ Desvio Padrão} \\
\multicolumn{11}{l}{$^{****}$ Desvio Padrão da Média} \\
   \hline
\end{tabular}

  \caption{Médias de $MP_{10}$ para o ano de 2007. \label{table:mediasmp10}}
\end{table}

A média aritmética da concentração de $MP_{10}$ durante a campanha de amostragem 
foi de 57,1 $\pm$ 2,5 $\mu g/ m^3$, maior que a média anual para 2007 de 
53,33 $\pm$ 0,95 $\mu g/ m^3$. A média geométrica, 51,21 $\pm$ 0,95 $\mu g/ m^3$, 
ficou abaixo da média anual no país em qualquer tipo de zona 
(70, 100 ou 200 $\mu g/ m^3$). 

Já a média aritmética foi de 57,1 $\pm$ 2,5 $\mu g/ m^3$, foi quase três vezes 
do recomendado pela OMS, 20 $\mu g/m^3$. Nota-se que a média agora está próxima
da mediana, 53,7 $\mu g/ m^3$, pois os dias do Harmatão, eventos extremos, 
foram excluídos. 

Podemos comparar estes resultados com a RMSP, por exemplo. Estudo de 2008, 
realizado por \citet{souza2014}, quantificou e caracterizou a composição química
das partículas na RMSP, reportando 64 $\mu g / m^3$ para média de $MP_{10}$ 
durante o período do estudo, com variações entre 30 e 122 $\mu g / m^3$.

Por outro lado, o acompanhamento da Companhia Ambiental do Estado de São Paulo 
(CETESB), que possui rede de monitoramento de qualidade do ar em todo estado 
\citep{cetesb2014}, a média de $MP_{10}$ tem caído nas últimas décadas, saindo 
de 58 $\mu g / m^3$ em 2003 para 40 $\mu g / m^3$ em 2012. Em 2013, a estação da
rede que registrou maior média foi a de Parelheiros, 44 $\mu g / m^3$, e a menor 
Ibirapuera, 29 $\mu g / m^3$. Assim, quando excluímos os dias de Harmatão, 
as concetrações de massa total de $MP_{10}$ de Acra são similares às encontradas
na RMSP.

Outro estudo realizado em Acra por \citet{aboh2009}, no distrito de 
Kwabenya, região periférica e semi-rural, 11 $km$ a noroeste de Nima, realizou 
coleta de MP em 2006 e 2007 e obteve concentração média de 179 $\mu g / m^3$ 
sobre todas as amostras de $MP_{10}$ (portanto, média não anual). 
Entretanto, quando o autor removeu as medidas dos dias de Harmatão, a média 
diminui para 77 $\mu g / m^3$, valor um pouco maior do que os 53,33 $\mu g/ m^3$
encontrados em Nima.

\begin{table}[H]
  \centering
    \input{../outputs/inalavel_2sitios}
  \caption{Estatística descritiva da área residencial (Sam Road) e avenida (Nima) 
           para $MP_{10}$. \label{table:inalavel_2sitios}}
\end{table}

Por fim, a tabela \ref{table:inalavel_2sitios} apresenta as médias em separado
para a área residencial (Sam Road) e a avenida (Nima). As diferenças 
entre o dois locais são mínimas, mas quando há presença de Harmatão, 
elas são levementes camufladas. Enquanto na presença do fenômeno, 
a massa média de $MP_{10}$ residencial equivale a 0,84 \% da avenida, 
sem ele, essa equivalência é de 0,70 \%. Com a exclusão do Harmatão, as 
diferenças entre os dois pontos são melhores expressadas, pois o mesmo ocorre 
com os demais elementos. A avenida, como esperado devido ao tráfego de 
veículos e ao comércio, a carrega as maiores concentrações.

%%%%
%\newpage
\subsection{Material Particulado Fino ($MP_{2,5}$)}

\citet{ARKU2008} e \citet{DIONISIO2010} foram pioneiros em conduzir levantamento
da concentração de $MP_{2,5}$ em Acra. A recomendação da OMS para $MP_{2,5}$ é 
de 10 $\mu g/m^3$ para média anual e 25 $\mu g/m^3$ para média diária, sendo que
em um ano o padrão diário não deve ser ultrapassado em mais que 1\% das 
amostragens. Assim como no Brasil (a menos do Estado de São Paulo), não há 
regulamentação para $MP_{2,5}$ em Gana. 

A figura \ref{fig:massa_temporal_mp2.5} apresenta as concentrações de $MP_{2,5}$
(eixo com distâncias logarítmicas), com a sinalização do padrão diário da OMS. 
Os índices de ultrapassagens foram de 66,49 e 92,0 \% para área residencial e 
avenida, respectivamente. Removido o Harmatão, os índices caem para 48,8 e 86,9 \%,
respectivamente, queda muito menor do que a observada no $MP_{10}$, pois o 
Harmatão tem origem em processos mecânicos, impactando menos no particulado fino.
A participação média de $MP_{2,5}$ no $MP_{10}$ foi de 64,41 \% e de 51,99 \% 
quando excluído o Harmatão.

\begin{figure}[H]
  \centering
  \begin{subfigure}[b]{0.45\textwidth}
    \includegraphics[width=\textwidth]{../outputs/massa_temporal_RFcH.pdf}
    \caption{Área residencial (Sam Road)}
  \end{subfigure}%
  \begin{subfigure}[b]{0.45\textwidth}
    \includegraphics[width=\textwidth]{../outputs/massa_temporal_TFcH.pdf}
    \caption{Avenida (Nima Road)}
  \end{subfigure}
  \caption{Concentrações de $MP_{2,5}$ ao longo da campanha de amostragem. 
           Padrão da OMS para média diária de $MP_{2,5}$ 25 $\mu g/m^3$ sinalizado.
           \label{fig:massa_temporal_mp2.5}}
\end{figure}

Na tabela \ref{table:descriptive_Fino_sH} encontra-se a estatística 
descritiva conjunta para as concentrações de $MP_{2,5}$ (residencial e avenida)
e a tabela \ref{table:medias_fino} as respectivas médias exclusivas para o ano
de 2007.

\begin{table}[H]
  \centering
  % latex table generated in R 3.2.4 by xtable 1.7-1 package
% Mon May 16 17:18:28 2016
\begin{tabular}{cccccc|ccccc}
  \hline
  &  \multicolumn{5}{c|}{com Harmatão} & \multicolumn{5}{c}{sem Harmatão} \\
 & n & $\overline{x}^*$ & $\overline{x}_g^{**}$ & $\sigma^{***}$ & $\overline{\sigma}^{***}$
 & n & $\overline{x}^*$ & $\overline{x}_g^{**}$ & $\sigma^{***}$ & $\overline{\sigma}^{***}$ \\
                       \hline & \multicolumn{10}{c}{$\mu g \cdot m^{-3}$} \\  \hline
residencial & 87 & 44,91 & 43,14 & 11,87 & 1,12 & 136 & 115,12 & 67,72 & 186,39 & 13,28 \\ 
  avenida & 89 & 61,57 & 60,55 & 11,52 & 1,08 & 138 & 131,65 & 88,99 & 183,62 & 13,02 \\ 
  ambas & 176 & 53,33 & 51,21 & 14,34 & 0,95 & 274 & 123,45 & 77,70 & 184,85 & 9,29 \\ 
   \hline
\multicolumn{11}{l}{$^{*}$ Média Aritimética} \\
\multicolumn{11}{l}{$^{**}$ Média Geométrica} \\
\multicolumn{11}{l}{$^{***}$ Desvio Padrão} \\
\multicolumn{11}{l}{$^{****}$ Desvio Padrão da Média} \\
   \hline
\end{tabular}

  \caption{Médias de $MP_{10}$ para o ano de 2007. \label{table:medias_fino}}
\end{table} 

A participação do BC na massa foi de 10,49 \% e, se considerado 
o Harmatão, diminui para 4,20 \%, que dilui a contribuição das fontes locais 
de combustão.
Na RMSP, \citet{andrade2012} encontrou 35,7 \% de BC na massa em medidas
de 2007 e 2008, bem maior que em Nima. O BC é um indicador de fontes de 
combustão, principalmente de veículos pesados, chegando a valores de 50\% ou 
mais, em cidades com muito tráfego, caso de RMSP.

\begin{table}[H]
  \centering
    \input{../outputs/descriptive_Fino_sH}
  \caption{Estatística descritiva das concentrações de $MP_{2,5}$ conjunta 
           (residencial e avenida) excluindo-se os dias do Harmantão
            \label{table:descriptive_Fino_sH}}
\end{table}

Devido ao impacto na saúde causado pelo o $MP_{2,5}$, acrescido recente facilidade 
 de acesso a tecnologias para medida e caracterização do mesmo, 
na última década, cientistas do mundo inteiro realizaram pesquisas de 
levantamento dos níveis de concentrações elementares de $MP_{2,5}$, permitindo
comparações importantes entre cidades, com características distintas em 
relação a fontes de poluição naturais ou antropogênicas.

Reuniu-se dados de concentrações médias elementares de $MP_{2,5}$ de oito 
países considerados em desenvolvimento pelo Fundo Monetário Internacional (FMI),
pertencentes a América do Sul, África (norte e sul), Ásia e Europa e que tiveram
campanhas de amostragem em períodos próximos ao deste trabalho. As cidades
com suas respectivas médias e períodos de medidas estão relacionadas na 
tabela \ref{table:fino_in_the_world} composta por  
Cidade do México \citep{diaz2014},
Cairo  \citep{boman2013},
Pequim  \citep{yang2011},
Nairóbi   \citep{gaita2014},
Brasil \citep{andrade2012urban},
Rijeka  \citep{ivovsevic2015}, 
Córdoba  \citep{achad2014} e mais duas pesquisas realizadas em 
outros distritos de Acra, Kwabenya \citep{aboh2009} e Ashaiman \citep{ofosu2012}. 

Em Kwabenya e Ashaiman, regiões de Acra, foram reportadas concentrações 
levemente inferiores às de Nima. Neste caso também dispúnhamos de resultados 
com e sem Harmatão. A diminuição da massa média de todos elementos nestas duas 
situações foi similar entre Nima e Kwabenya. A redução do Si em Kwabenya, 
por exemplo, baixou de 4500 para 425 $\mu g/m^3$ 
($\pm$ 10 vezes) e de 7011 para 1453 $\mu g/m^3$ ($\pm$ 5 vezes) em Nima. 
As medidas de Ashaiman, 22 km a nordeste de Nima, foram feitas entre 
fevereiro e agosto, e portanto, não afetadas pelo Harmatão.

Considerando o Harmatão, as concentrações médias observadas para a maioria dos 
elementos do $MP_{2,5}$ em Nima foram superiores a todas outras cidades, com 
exceção de Pequim. Por outro lado, removendo-se o Harmatão, apesar de ainda 
haver concentrações altas, os valores tornam-se mais equiparados. 

O estudo realizado por \citet{diaz2014} em Cuajimalpa de Morelos, distrito 
da Cidade do México, capital do México, apresenta as menores concentrações, 
pois o local, segundo os autores, apesar de próximo de uma grande zona industrial, 
não é afetado diretamente por ela, em função da altitude (2760 m do nível do mar)
e das condições climáticas favoráveis à dispersão de poluentes, com massa média 
menor que a metade da encontrada em Nima, 12 e 29,7 $\mu g/m^3$, respectivamente.

Com relação ao Cairo, capital do Egito, Nima obteve concentrações menores, 
em torno da metade, para a maioria dos elementos. O Egito está localizado 
no norte da África e na parte leste do deserto do Saara, sendo assim,
sofre influência direta e constante do mesmo. Daí suas altas concentrações. 
Diferente do Harmatão, que é sazonal, neste caso não é trivial excluir o impacto 
do deserto (fonte natural) da poluição das fontes locais \citet{boman2013}.

A China é conhecida mundialmente por sua poluição, resultado de condições
climáticas desfavoráveis à dispersão de poluentes, pouca chuva e pouco vento, 
em conjunto com desenfreada industrialização. As inúmeras indústrias
de diversas naturezas, usinas de carvão, tráfego de veículos, entre outras, 
explicam índices altíssimos, como o de BC, 8,19 $\mu g/m^3$, 
medido em Pequim, capital do país, por \citet{yang2011}. Mesmo considerando 
os dias de Harmatão, as concentrações de Nima foram inferiores às de Pequim.

Quênia, assim como Gana, faz parte da SSA, e sua capital, Nairóbi, tem condições
econômicas e sociais similares às de Acra, com recente industrialização, 
transporte público precário, frota veicular velha, queima de biomassa, 
entre outros.  
\citet{gaita2014} realizou estudo detalhado de Nairóbi comparando regiões 
periféricas e centrais. Encontrou concentrações pouco menores que as de Nima, 
com porcentagem de participação de BC na massa de 15\%, contra 10,49 \% em Acra,
indicando atividade de combustão, em especial, queima de biomassa e veículos, 
como principais fontes na cidade 
(reportado no levantamento de fontes com PMF realizado no artigo). 

Nima apresentou concentrações ao menos três vezes maiores que as reportadas 
para o Brasil, em estudo que avaliou seis cidades brasileiras: São Paulo, 
Rio de Janeiro, Belo Horizonte, Curitiba, Recife e Porto Alegre
\citep{andrade2012urban}. Para São Paulo, em particular, \citet{andrade2012}
reportou média de $MP_{2,5}$ durante seu estudo de 28 $\mu g / m^3$, 
próximo da média de 29,7 $\mu g / m^3$ encontrada em Acra. 

A cidade de Rijeka, na Croácia, é pequena, e possui apenas 130 mil habitantes, 
porém aloca polo industrial que inclui terméletrica e refinaria de óleo
e no estudo de \citet{ivovsevic2015} apresentou concentração de enxofre
(789 $ng / m^3$) e BC (3,4 $\mu g / m^3$) maiores que os encontrados em Nima, 
tanto em absoluto, quanto relativamente à massa média de $MP_{2,5}$ 
(20,6 $\mu g / m^3$). 
Por fim, \citet{achad2014} detalhou o $MP_{2,5}$ em Córdoba, Argentina, 
encontrando massa média de 50,1 $\mu g / m^3$, maior que em Nima, devido a 
região ser zona industrial e de alta movimentação de caminhões.   

\begin{landscape}
  \begin{table}[H]
    \centering
    %  & \multicolumn{2}{c}{maio 2010} & \multicolumn{2}{c}{novembro 2010} & \multicolumn{2}{c}{abril 2011} \\
% Z & medido & ajustado & medido & ajustado & medido & ajustado \\

\begin{tabular}{ccccccccccccc}
\hline
& \multicolumn{2}{c}{Este estudo} & \multicolumn{2}{c}{Acra} & Acra & México & Egito & China & Quênia & Brasil $^d$ & Croácia & Argentina \\

& \multicolumn{2}{c}{Nima$^{a,b}$} & \multicolumn{2}{c}{Kwabenya$^{a,b}$} & Ashaiman & Cidade do México & Cairo  & Pequim & Nairóbi &   & Rijeka & Córdoba\\

& \multicolumn{2}{c}{2006-2008} & \multicolumn{2}{c}{2006-2007} & 2008 & 2004-2005 & 2010-2011 & 2008-2009 &  2008-2010 & 2007-2008 & 2013-2015 & 2010-2011\\

\hline
\multicolumn{13}{c}{$ng / m^3$} \\
\hline
$MP_{2,5} {^c}$&29,7   & 79,8       & 22,9             & 96,5 &21,6   & 12     & 51    & 118,5 & 18    &        & 20,6    & 50,1  \\
BC $^c$   & 3,1        & 3,4        & 1,7              & 2,5  &2,07   &        & 3,7   & 8,19  & 2,7   &        & 3,4     &       \\
Na        & 336        & 272        &                  &      &743    &        & 610   &       &       & 117    & 117     &       \\
Mg        & 108        & 436        &                  &      &94,4   &        &       & 290   &       &        & 22      & 73    \\
Al        & 720        & 2897       & 342              & 1430 &806    &        &       & 790   &       & 43,9   & 44      & 383   \\
Si        & 1453       & 7011       & 425              & 4500 &1159   &        &       & 1790  &       & 125,3  & 110     & 1509  \\
P         & 14         & 25         &                  &      &       &        &       &       &       & 10,9   & 2,8     &       \\
S         & 605        & 793        & 442              & 524  &391    & 600    & 1200  &       & 640   & 496,6  & 789     & 336   \\
Cl        & 458        & 603        &                  &      &145    & 37     & 2200  & 2300  & 480   & 66,6   & 54      &       \\
K         & 858        & 1545       & 260              & 731  &487    & 96     & 470   & 3520  & 310   & 225,3  & 194     & 628   \\
Ca        & 353        & 1248       & 41,7             & 432  &287    & 130    & 2900  & 900   & 310   & 64     & 88      & 308   \\
Ti        & 47,2       & 205        & 12,6             & 100  &59     & 9      & 100   & 80    & 54    & 5,5    & 3,4     & 22    \\
V         & 1,8        & 4,4        &                  &      &2,9    & 10     & 9,6   & 30    & 41    & 1,53   & 3,4     & 0,5   \\
Mn        & 9,3        & 36         & 3,5              & 19   &27,4   & 3      & 24    & 90    &       & 11,71  & 4,4     & 12    \\
Fe        & 446        & 1853       & 109              & 845  &987    & 100    & 1000  & 1130  & 350   & 108,3  & 93      & 301   \\
Zn        & 31         & 41,0       & 5,5              & 9,5  &164    & 12     & 200   & 530   & 91    & 29,7   & 14      & 17    \\
Br        & 23,5       & 27,4       & 5,7              & 6,8  &32,4   & 5      & 21    & 30    & 12    & 3,75   & 2,6     &       \\
Pb        & 15,5       & 19,3       & 2,1              & 3,5  &43,9   & 9      & 86    & 240   & 22    & 8,39   & 6,8     &       \\
\hline
\multicolumn{13}{l}{$^a$ Removendo Harmatão} \\
\multicolumn{13}{l}{$^b$ Incluindo Harmatão} \\
\multicolumn{13}{l}{$^c$ Dimensão de $\mu g / m^3$ } \\
\multicolumn{13}{l}{$^d$ Média para seis cidades: São Paulo, Rio de Janeiro, 
Belo Horizonte, Curitiba, Recife e Porto Alegre. } \\
\hline
\end{tabular}


    \caption{Médias elementares e média da massa de $MP_{2,5}$ encontradas
             em Nima e comparadas com outras regiões do mundo:
             Kwabenya (Acra) \citep{aboh2009},
             Ashaiman (Acra) \citep{ofosu2012},
             Cidade do México (México) \citep{diaz2014},
             Cairo (Egito) \citep{boman2013},
             Pequim (China) \citep{yang2011},
             Nairóbi (Quênia)  \citep{gaita2014},
             Brasil $^d$ \citep{andrade2012urban},
             Rijeka (Croácia) \citep{ivovsevic2015} e
             Córdoba (Argentina) \citep{achad2014}.
             \label{table:fino_in_the_world}}
  \end{table} 
\end{landscape}

Por fim, a tabela \ref{table:fino_2sitios} apresenta as médias elementares e
da massa de $MP_{2,5}$ em separado para a área residencial (Sam Road) e
a avenida (Nima). Apesar das diferenças pequenas entre os dois sítios, 
diferente do $MP_{10}$, a média da massa de $MP_{2,5}$, quando inclusos os dias
de Harmatão, apresenta maior valor na área residencial, 83,3 $\mu g / m^3$,
que na avenida, 76,4 $\mu g / m^3$. Entretanto, corrigindo o efeito do Harmatão,
a avenida volta a ter maior concentração, 31,9 $\mu g / m^3$, contra 27,5 
$\mu g / m^3$  da área residencial.

\begin{table}[H]
  \centering
    \input{../outputs/fino_2sitios}
  \caption{Estatística descritiva da área residencial (Sam Road) e avenida (Nima) 
           para $MP_{2,5}$. \label{table:fino_2sitios}}
\end{table}
