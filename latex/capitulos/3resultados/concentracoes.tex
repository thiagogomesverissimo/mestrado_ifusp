%%%%
\newpage
\section{Concentrações Ambientais}

%
%Modelos receptores não medem concentrações. Concentrações medidas são usadas em modelos receptores. Os níveis de concentração indicam a necessidade de se adotar providências de controle. Os modelos receptores são usados para identificar responsabilidades de fontes por uma dada situação ambiental. Isso orienta providências de controle que devem ser adotadas para mitigar problemas ambientais.  Em áreas remotas, os modelos receptores permitem conhecer o papel que diferentes fontes têm na composição do aerossol atmosférico, que por sua vez também participa de ciclos naturais (chuvas, transferência de nutrientes, fecundação de vegetais etc).
%
Medidas de concetrações da massa de MP são usadas para avaliação da qualidade do
ar, as quais relacionam-se à saúde, mas que também associam-se a danos a 
vegetais, materiais e ao meio ambiente em geral. As legislações fundamentam-se 
sobre estudos de impactos à saúde, atribuíveis a poluentes, em curto ou longo 
espaço de tempo. Procura-se definir, em função disso, os limites que permitem 
boa qualidade de vida ao cidadão.

Fruto de reuniões de trabalho internacionais neste sentido, a OMS 
\citeyearpar{who} definiu limites para os principais poluentes antropogênicos 
da atualidade, as quais são diretrizes indicadas para todos os países. Mas, 
pela própria constituição deste organismo internacional, cada país tem autonomia
para definir sua própria legislação, podendo ou não acatá-las. De qualquer modo,
ao sustentar padrões assentados sobre estudos de riscos e danos à saúde, atuais 
e de amplo reconhecimento internacional, ela cria um referencial para 
instituições e movimentos sociais que pressionam por melhores níveis de 
qualidade de vida em cada país.

Em estudos mais detalhados, é possível aprofundar a análise de dados com 
identificação e quantificação de espécies que compõem o MP (elementos, 
compostos, misturas, ou fatores diversos que interferem em sua concentração). 
Além de revelar os fatores mais significativos nas concentrações determinadas, 
ou nos efeitos sobre a saúde, eles possibilitam ajustes de métodos estatísticos 
para identificar e estimar perfis das principais fontes poluidoras. Assim, nesta
seção serão expostos inicialmente os resultados das concentrações para 
comparação com limites de legislações e com padrões sugeridos pela OMS, e na 
próxima seção, serão aplicados métodos 
estatísticos multivariados, AF e PMF, para levantamento de fontes. 

A campanha de amostragem nos quatro bairros, conforme já registramos, 
resultou em um total de 2898 amostras que foram analisadas por refletância e 
XRF-ED no LAPAt. Concentrações médias, análises estatísticas gerais e 
levantamento de fontes, para os onze locais em conjunto, foram apresentados em 
2011 no \textit{23th Congress of the International Society 
for Environmental Epidemiology} \citep{zhou2011} e publicadas em \cite{zhou2013} 
e \cite{zhou2014}. Mas estes trabalhos tinham um olhar mais voltado às questões
de saúde pública, tratando o  $PM_{10}$ e $PM_{2,5}$, sem separar o 
$PM_{2,5-10}$. Isso limita a capacidade de discernir o MP originário de 
processos mecânicos ou de conversão gás-partícula, bem como de possíveis 
interações entre eles. Incorporamos também informações sobre a circulação 
regional de ventos nas discussões, o que contribuem no processo de discernimento
de prováveis fontes que chegam aos amostradores.

Nesta pesquisa em particular, aprofundou-se os estudos exclusivamente no bairro
de Nima, em dois locais, Sam Road e Nima Road, com 791 amostras válidas obtidas
entre 11 de novembro de 2006 e 15 de agosto de 2008. Daqui em diante, 
medidas realizadas na Sam Road serão referidas como área residencial e na 
Nima Road como avenida, para melhor refletir as características de ambas.

%%%%
\newpage
\subsection{Material Particulado Inalável $MP_{10}$}

O presumível interesse na saúde pública como elemento central na definição 
de padrões de qualidade do ar nas legislações específicas, termina dependendo 
de aspectos políticos, sociais e econômicos de cada país. Em função disso há 
entre estes uma multiplicidade de limites para as espécies legisladas.

O padrão de qualidade do ar para $MP_{10}$ ambiental vigente em Gana, referendado
\ref{table:padroesgana}. Nota-se que há limites específicos para os meses em que 
ocorre o Harmatão, refletindo o efeito deste fenômeno no MP local. 
Regiões consideradas como zonas industriais ou de comércio também têm limites mais
permissivos, revelando uma priorização do Governo aos interesses negociais no 
País, em detrimento da saúde e bem estar da população.

\begin{table}[H]
\centering
  % http://www.epa.gov.gh/ghanalex/policies/EPAguidelines%20Report.pdf
\begin{tabular}{cccc}
\hline
                              &   regiões  &        zonas       &         zonas               \\
                              & sensíveis$^1$  & residenciais e rurais & industriais e comerciais      \\
Tipo da média                 & $\mu / m^3$ & $\mu / m^3$ & $\mu / m^3$      \\
\hline
diária                    & 110             & 150                      & 260         \\       
anual geométrica          & 70              & 100                      & 200                   \\
mensal (durante Harmatão) & 100             & 200                      & 500                     \\
\hline
\multicolumn{4}{l}{$^1$ A determinação se uma região é sensível ou não é feita
                   pela EPA de Gana, bairros com hospitais, por exemplo, são classificados como sensíveis.} \\
\hline
\end{tabular}

\caption{Padrões de Qualidade do Ar para $MP_{10}$ Ambiental em Gana
         \cite{epa2015} \label{table:padroesgana}}
\end{table}



Na tabela \ref{table:pm10standards} compara-se os padrões de $MP_{10}$ entre 
Brasil, Gana e OMS, com a observação de que em Gana a média anual é calculada 
geometricamente, o que a torna tanto menor que a média aritmética, 
quanto mais extremos forem os valores registrados.

\begin{table}[H]
\centering
  \begin{tabular}{cccc}
\hline
              & Brasil & Gana (residencial e rural) & OMS \\
Tipo da média & $\mu / m^3$ & $\mu / m^3$ & $\mu / m^3$          \\
\hline
diária   & 150              & 150              &  50             \\
anual    &  50 (aritmética) & 100 (geométrica) &  20 (aritmética) \\
\hline
\end{tabular}

  \caption{Padrões para média anual de $MP_{10}$ no Brasil \citep{conama1990}, 
           Gana \citep{epa2015} e OMS \citep{who}. \label{table:pm10standards}}
\end{table}

O gráfico da figura \ref{fig:massa_temporal_mp10} mostra as concetrações medidas
(em escala logaritímica para melhor vizualização dos dados) 
nos dois sítios em função do tempo, sinalizando os padrões de médias 
diárias da tabela anterior. 

Observa-se, que para verificação do atendimento ou não de padrões diários, 
mensais ou anuais previstos na legislação seria necessário remover as amostras 
de 2006 e 2008, mantendo somente as 2007, pois foi o único ano deste experimento
com medidas em todos meses, além de outras exigências da norma, como o 
número máximo de filtros faltantes possível para considerar essas comparações
válidas. O mesmo procedimento deveria ser realizado para comparação com 
limites Brasileiros ou da OMS, seguindo as respectivas normas. 
Entretanto, a intenção é verificar
relativamente os níveis de poluição durante o período de 
coleta, sendo a confrontação com parâmetros legais realizada somente para dar
noção relativa e qualitativa dos resultados obtidos na campanha, 
e não para observância estrita de atendimento ou não da lei.

Sendo assim, o índice de ultrapassagens foi 16,24 \% na área 
residencial e 19,60 \% na avenida, quando considerado o padrão de 150 
$\mu g / m^3$ praticado pela EPA-GH e no Brasil. Em relação
ao padrão de média diária da OMS, 50 $\mu g / m^3$, mais restritivo, 
O índice de ultrapassagens foi alto para os dois sítios 59,90 \% 
na área residencial e 90,95 \% na avenida, sendo que nos meses do Harmatão 
os dois padrões são ultrapassados todos dias. 

\begin{figure}[H]
  \centering
  \begin{subfigure}[b]{0.45\textwidth}
    \includegraphics[width=\textwidth]{../outputs/massa_temporal_RIcH.pdf}
    \caption{Área residencial (Sam Road)}
  \end{subfigure}%
  \begin{subfigure}[b]{0.45\textwidth}
    \includegraphics[width=\textwidth]{../outputs/massa_temporal_TIcH.pdf}
    \caption{Avenida (Nima Road)}
  \end{subfigure}
  \caption{Concentrações de $MP_{10}$ ao longo da campanha de amostragem.
           \label{fig:massa_temporal_mp10}}
\end{figure}

Regiões localizadas próximas do deserto do Saara sofrem mais diretamente 
eventos extremos de poluição. \citet{kaku2016} ao estudar partículas no 
Golfo Pérsico, a leste do deserto do Saara, relatou que em períodos de chegada 
de poeira, 76$\pm$7\% da massa total de MP é comprometida com partículas da 
crosta terrestre, ou seja, de solo.

A separação dos dias em que ocorrem ventos do Harmatão dos outros dias é 
importante, pois além de ser previsto na lei local, permite realizar a 
comparação separando uma dinâmica mais antropogênica de outra natural, 
ligada a esse fenômeno.
  
O critério de identificação dos dias de Harmatão, entretanto, não é simples, 
pois o fenômeno não é ininterrupto durante os meses em que ocorre. 
Classificando todos dias de novembro até março como Harmantão, corre-se o risco
de perde informações importantes de fontes locais com atividades nesse período, 
comprometendo análises estatísticas de AF e PMF. 

Considerando o fato dos silicatos serem os principais constituintes da crosta
terrestre, e tendo em vista o convívio com esse fenômeno, \citet{aboh2009} 
sugere como critério para identificação dos dias de ocorrência do Harmatão 
os de concentrações de silício no $MP_{10}$ maiores que 10 $\mu g/m^3$, 
entre os meses de novembro e março. 

Quando desconsidera-se o Harmatão, seguindo esse critério,
não há mais ultrapassagens do padrão de 150 $\mu g / m^3$ na área residencial 
e na avenida o índice de ultrapassagem cai de 19,60 \% para 0,88 \%. 
Para o padrão diário da OMS de 50 $\mu g / m^3$ na área residencial as 
ultrapassagens diminuem para 31,25 \% (antes 59,90 \%) e na avenida para 
84,21 \% (antes 90,95 \%).

\begin{table}[H]
  \centering
  % latex table generated in R 3.2.4 by xtable 1.7-1 package
% Mon May 16 17:18:28 2016
\begin{tabular}{cccccc|ccccc}
  \hline
  &  \multicolumn{5}{c|}{sem Harmatão} & \multicolumn{5}{c}{com Harmatão} \\
 & n & $\overline{x}^*$ & $\overline{x}_g^{**}$ & $\sigma^{***}$ & $\overline{\sigma}^{***}$
 & n & $\overline{x}^*$ & $\overline{x}_g^{**}$ & $\sigma^{***}$ & $\overline{\sigma}^{***}$ \\
                       \hline & \multicolumn{10}{c}{$\mu g \cdot m^{-3}$} \\  \hline
residencial & 87 & 44,91 & 43,14 & 11,87 & 1,12 & 136 & 115,12 & 67,72 & 186,39 & 13,28 \\ 
  avenida & 89 & 61,57 & 60,55 & 11,52 & 1,08 & 138 & 131,65 & 88,99 & 183,62 & 13,02 \\ 
  ambas & 176 & 53,33 & 51,21 & 14,34 & 0,95 & 274 & 123,45 & 77,70 & 184,85 & 9,29 \\ 
   \hline
\multicolumn{11}{l}{$^{*}$ Média Aritimética} \\
\multicolumn{11}{l}{$^{**}$ Média Geométrica} \\
\multicolumn{11}{l}{$^{***}$ Desvio Padrão} \\
\multicolumn{11}{l}{$^{****}$ Desvio Padrão da Média} \\
   \hline
\end{tabular}

  \caption{Médias de $MP_{10}$ para o ano de 2007. \label{ddd}}
\end{table}

\begin{table}[H]
  \centering
  \input{../outputs/descriptive_inalavel_harmatao}
  \caption{Estatística descritiva das concentrações de $MP_{10}$ conjunta
           (área residencial e avenida) somente para os dias de ocorrência 
           de vento do Harmatão. 54 amostras na área residencial e 59 na avenida 
          \label{table:descriptive_inalavel_harmatao}}
\end{table}

A tabela \ref{table:descriptive_inalavel_harmatao} apresenta a média, 
desvio padrão da média, mediana, mínimo e máximo para as concentrações de 
MP Inalável, $MP_{10}$ para dias de ocorrência do Harmatão, classificados
segundo sugestão de \citet{aboh2009}. A massa total média foi 269,2 $\pm$ 23,0
$\mu g/ m^3$, tendo como elemento mais representativo o Si, 11,3 \% da massa total. 
A ambiguidade (talvez estratégica) da lei define um intervalo de média diária 
de $MP_{10}$ no Harmantão de 100 à 500 $\mu g/ m^3$, dependendo de como a região
é classificada: sensível, residencial, rural, industrial ou comercial, abrindo
brechas que permitem definir regiões baseado em interesses puramente econômicos,
pois limites de 500 $\mu g/ m^3$ mesmo com a poeira pesade do Harmatão serão
dificilmente alcançados. A mediana, que representa melhor um conjunto de dados
com valores no extremo, foi 154,7 $\mu g/ m^3$, e indica que apesar da média 
de 269,2 $\mu g/ m^3$ e máximo 1255,8 $\mu g/ m^3$, no geral apenas alguns
dias tiveram concentrações extremamentes altas. 

Excluíndo-se as concentrações de $MP_{10}$ medidas nos dias de 
ocorrência do Harmantão obtém-se as estatísticas descritivas apresentadas na 
tabela \ref{table:descriptive_inalavel_sH}.

\begin{table}[H]
  \centering
    \input{../outputs/descriptive_inalavel_sH}
  \caption{Estatística descritiva das concentrações de $MP_{10}$ conjunta 
           (Sam Road e Nima Road) excluíndo-se os dias do Harmantão
            \label{table:descriptive_inalavel_sH}}
\end{table}

A média aritmética da massa total de $MP_{10}$ durante a campanha de amostragem 
foi de 57,1 $\pm$ 2,5 $\mu g/ m^3$, abaixo da média 
anual do no país em qualquer tipo de zona (70, 100 ou 200 $\mu g/ m^3$), 
mas quase três vezes do recomendado pela OMS, 20 $\mu g/m^3$.
A média agora está próxima da mediana, 53,7 $\mu g/ m^3$, 
pois os dias do Harmatão, eventos extremos, foram excluídos. 

Comparando com RMSP, em um estudo de 2008 realizado por \citet{souza2014}
que quantificou e caracterizou a composição química das partículas na RMSP,
reportou 64 $\mu g / m^3$ para média de $MP_{10}$ durante o período do estudo, 
com variações entre 30 e 122 $\mu g / m^3$. Por outro lado, o acompanhamento 
da Companhia Ambiental Do Estado De São Paulo (CETESB), que possui 
rede de monitoramento de qualidade do ar em todo estado \citep{cetesb2014}, 
a média de $MP_{10}$ tem despencado nas últimas décadas, saindo de 58 
$\mu g / m^3$ em 2003 para 40 $\mu g / m^3$ em 2012. Em 2013, a estação da rede
que registrou maior média foi a de Parelheiros, 44 $\mu g / m^3$, e a menor 
Ibirapuera, 29 $\mu g / m^3$. Assim, quando excluímos o efeito do Harmatão, 
as concetrações de massa total de $MP_{10}$ de Acra são similares as encontradas
na RMSP no mesmo período.

Em outro estudo também realizado em Acra por \citet{aboh2009} no distrito de 
Kwabenya, região periférica e semi-rural, a 11 $km$ a noroeste de Nima, com 
medidas no mesmo período, 2006 e 2007, obteve 179 $\mu g / m^3$ 
na média da massa total de $MP_{10}$ quando considerou todas amostras. 
Entretanto, quando o autor removeu as medidas dos dias de Harmatão, 
a média diminui para 77 $\mu g / m^3$,
valor um pouco maior do que os 57,1 $\mu g/ m^3$ encontrados em Nima.

\begin{table}[H]
  \centering
    \input{../outputs/inalavel_2sitios}
  \caption{Estatística descritiva da área residencial (Sam Road) e avenida (Nima) 
           para $MP_{10}$. \label{table:inalavel_2sitios}}
\end{table}

Por fim, a tabela \ref{table:inalavel_2sitios} apresenta as médias em separado
para a área residencial (Sam Road) e a avenida (Nima). As diferenças 
entre o dois locais são mínimas, mas quando há presença de Harmatão, 
elas são levementes camufladas, pois enquanto que na presença do fenômeno, 
a massa média de $MP_{10}$ na Sam Road equivale a 0,84 \% da Nima Road, 
sem ele, essa equivalência é de 0,70 \%. Com a exclusão do Harmatão, as 
diferenças entre os dois pontos são melhores expressadas, pois o mesmo ocorre 
com os demais elementos, sendo a avenida, como esperado devido ao tráfego de 
veículos e ao comércio, a que carrega as maiores concentrações.

%%%%
\newpage
\subsection{Material Particulado Fino ($MP_{2,5}$)}

\begin{table}[H]
  \centering
    % latex table generated in R 3.2.4 by xtable 1.7-1 package
% Mon May 16 17:08:28 2016
\begin{tabular}{cccccc|ccccc}
  \hline
  &  \multicolumn{5}{c|}{sem Harmatão} & \multicolumn{5}{c}{com Harmatão} \\
 & n & $\overline{x}^*$ & $\overline{x}_g^{**}$ & $\sigma^{***}$ & $\overline{\sigma}^{***}$
 & n & $\overline{x}^*$ & $\overline{x}_g^{**}$ & $\sigma^{***}$ & $\overline{\sigma}^{***}$ \\
                       \hline & \multicolumn{10}{c}{$\mu g \cdot m^{-3}$} \\  \hline
residencial & 94 & 27,25 & 25,40 & 13,40 & 1,21 & 134 & 67,17 & 36,77 & 120,61 & 8,59 \\ 
  avenida   & 96 & 31,67 & 30,85 & 7,86 & 0,71 & 139 & 78,80 & 44,60 & 157,11 & 11,11 \\ 
  ambas     & 190 & 29,49 & 28,02 & 11,15 & 0,71 & 273 & 73,09 & 40,57 & 140,25 & 7,04 \\ 
   \hline
\multicolumn{11}{l}{$^{*}$ Média Aritimética} \\
\multicolumn{11}{l}{$^{**}$ Média Geométrica} \\
\multicolumn{11}{l}{$^{***}$ Desvio Padrão} \\
\multicolumn{11}{l}{$^{****}$ Desvio Padrão da Média} \\
\hline
\end{tabular}

  \caption{
            \label{ddd}}
\end{table}



Em termos de massa total, \citet{ARKU2008} e \citet{DIONISIO2010} 
foram pioneiros em conduzir levantamento da massa total de $MP_{2,5}$ em 
Acra. A recomendação da OMS para $MP_{2,5}$ é de 10 $\mu g/m^3$ para média anual
e 25 $\mu g/m^3$ para média diária, com adendo de que as concentrações de 
25 $\mu g/m^3$ não devem ultrapassar em mais que 1\% no ano. Assim como no
Brasil, não há regulamentação para $MP_{2,5}$ em Gana. 

O gráfico da figura \ref{fig:massa_temporal_mp2.5} apresenta as concentrações
de $MP_{2,5}$ (escala logaritímica) medidas nos dois locais 
(área residencial e avenida) com a sinalização do padrão da OMS de 
25 $\mu g/m^3$. Os índices de ultrapassagens foram de 66,49 e 92,0 \% 
para área residencial e avenida, respectivamente. Quando removido o Harmantão, 
os índices se tornam 48,8 e 86,9 \%, respectivamente, queda muito menor do 
que a observada no $MP_{10}$, pois o $MP_{2,5}$ caracteriza a poluição local, 
sendo gerado principalmente de processos de combustão e industriais.
A participação média de $MP_{2,5}$ no $MP_{10}$ foi de 64,41 \% e de 
51,99 \% quando excluído o Harmantão.

\begin{figure}[H]
  \centering
  \begin{subfigure}[b]{0.45\textwidth}
    \includegraphics[width=\textwidth]{../outputs/massa_temporal_RFcH.pdf}
    \caption{Área residencial (Sam Road)}
  \end{subfigure}%
  \begin{subfigure}[b]{0.45\textwidth}
    \includegraphics[width=\textwidth]{../outputs/massa_temporal_TFcH.pdf}
    \caption{Avenida (Nima Road)}
  \end{subfigure}
  \caption{Concentrações de $MP_{2,5}$ ao longo da campanha de amostragem. 
           Padrão da OMS para média diária de $MP_{2,5}$ 25 $\mu g/m^3$ sinalizado.
           \label{fig:massa_temporal_mp2.5}}
\end{figure}

Na tabela \ref{table:descriptive_Fino_sH} encontra-se a estatística 
descritiva para as concentrações de $MP_{2,5}$ conjunta, Sam Road e Nima Road. 
A participação do BC na massa foi de 10,49 \% e se considerado 
o Harmatão, diminui para 4,20 \%, minimizando a contribuição de fontes locais.
Em São Paulo (SP), \citet{andrade2012} encontrou 35,7 \% de BC na massa em medidas
de 2007 e 2008, maior que em Nima, pois o BC é um indicador de fontes de 
combustão, principalmente de veículos pesados, chegando a valores de 50\% ou 
mais, em cidade com muito tráfego, caso de SP.

\begin{table}[H]
  \centering
    \input{../outputs/descriptive_Fino_sH}
  \caption{Estatística descritiva das concentrações de $MP_{2,5}$ conjunta 
           (Sam Road e Nima Road) excluíndo-se os dias do Harmantão
            \label{table:descriptive_Fino_sH}}
\end{table}

Devido ao impacto na sáude causado pelo o $MP_{2,5}$, acoplado a facilidade 
recente de acesso a tecnologias para medida e caracterização do mesmo, 
na última década, cientistas do mundo inteiro realizaram pesquisas de 
levantamento de níveis concentrações elementares de $MP_{2,5}$, permitindo
comparações importantes entre cidades, com características distintas com 
relação a fontes, naturais ou antropológicas, de partículas.

Reuniu-se dados de concentrações médias elementares de $MP_{2,5}$ de oito 
países considerados em desenvolvimento pelo Fundo Monetário Internacional (FMI),
pertencentes a América do Sul, África (norte e sul), Ásia e Europa e que tiveram
campanhas de amostragem próximas a deste trabalho. As cidades
com suas respectivas médias e períodos de medidas estão relacionadas na 
tabela \ref{table:fino_in_the_world} composta por  
Cidade do México (México) \citep{diaz2014},
Cairo (Egito) \citep{boman2013},
Pequim (China) \citep{yang2011},
Nairóbi (Quênia)  \citep{gaita2014},
Brasil (média de São Paulo, Rio de Janeiro, Belo Horizonte, Curitiba, 
Recife e Porto Alegre)  \citep{andrade2012urban},
Rijeka (Croácia) \citep{ivovsevic2015}, 
Córdoba (Argentina) \citep{achad2014} e mais duas pesquisas realizadas em 
outros distritos de Acra, Kwabenya (semi-rual) \citep{aboh2009} e Ashaiman
(região industrial) \citep{ofosu2012}. 

Em Kwabenya e Ashaiman ocorreram medidas quase que paralelamente a esta 
pesquisa e reportaram concentrações levementes inferiores as de Nima, como
em Kwabenya o mesmo procedimento de remoção do Harmatão foi utilizado, os dois 
resultados foram apresentados. A dimuição da massa média de todos elementos,
depois da remoção, foram similares em Nima e em Kwabenya, 
por exemplo, o sílicio, abaixou de 425 para 4500 $nu g/m^3$ 
(9,4 \%) em Kwabenya e de 1453 para 7011 $nu g/m^3$ (20,7 \%) em Nima. 
As medidas de Ashaiman, 22 $km$ a nordeste de Nima, foram feitas entre 
Fevereiro e Agosto, e portanto, não afetadas pelo Harmatão. 

As concentrações médias elementares de $MP_{2,5}$ observadas em Nima foram 
superiores a todas outras cidades, com exceção de Pequim, 
para a maioria dos elementos, quando consideradas amostras do Harmatão.
Por outro lado, removendo-se o Harmatão, apesar de ainda haver concentrações
altas, as comparações se tornam mais equiparadas. 

O estudo realizado por \citet{diaz2014} em Cuajimalpa de Morelos, distrito 
da Cidade do México, capital do México, 
apresenta as menores concentrações, pois o local, apesar de próximo
de uma grande zona industrial, não é afetado diretamente por ela, por conta
da altitude (2760 m do nível do mar) e das condições climáticas favoráveis a
dispersão de poluentes, com massa média menor que a metade da encotrada em Nima, 
12 e 29,7 $\mu g/m^3$, respectivamente.

Com relação a Cairo, capital do Egito, Nima obteve concentrações menores, 
em torno da metade, para maioria dos elementos. O Egito está localizado 
no norte da África e ao leste do deserto do Saara, compondo o mesmo, sendo assim,
sofre influência direta e constante do mesmo, por isso as altas concentraçoes. 
Direfente do Harmatão, que é sazonal, não é trivial excluir o impacto 
do deserto (fonte natural) da poluição das fontes locais \citet{boman2013}.

A China é conhecida mundialmente por sua poluição, resultado de condições
climáticas desfavoráveis a dispersão de poluentes, pouca chuva e pouco vento, 
em conjunto com desenfreada industrialização, que resulta em inúmeras indústrias
de diversas naturezas, usinas de carvão, tráfego de veículos, entre outras, 
que explicam índices altíssimos, como o de BC, 8,19 $\mu g/m^3$, 
medido em Pequim, capital do país, por \citet{yang2011}. Mesmo considerando 
os dias de Harmatão, as concentrações de Nima foram inferiores as de Pequim.

Quênia, assim como Gana, faz parte da SSA, e Nairóbi, a capital tem condições
econômicas e sociais próximas das de Acra, com recente industrização, transporte
público precário, frota veicular velha, queima de biomassa, entre outros.  
\citet{gaita2014} realizou estudo detalhado de Nairóbi compararando regiões 
periféricas e centrais. Encontrou concentrações pouco menores que as Nima, 
com porcentagem de participação de BC na massa de 15\%, contra 10,49 \% de Acra,
indicando atividade de combustão, em especial, queima de biomassa e veículos, 
como principais na cidade (reportada no levantamento de fontes com PMF realizado
no artigo). 

Nima apresentou concetrações de ao menos três vezes maiores que as reportadas 
para o Brasil, em estudo que avaliou seis cidades brasileiras: São Paulo, 
Rio de Janeiro, Belo Horizonte, Curitiba, Recife e Porto Alegre
\citep{andrade2012urban}. Para São Paulo, em particular, \citet{andrade2012}
reportou média de $MP_{2,5}$ durante seu estudo de 28 $\mu g / m^3$, 
próximo da média de 29,7 $\mu g / m^3$ encontrada em Acra. 

A cidade de Rijeka, na Croácia, é pequena, e possui apenas 130 mil habitantes, 
porém aloca polo industrial que inclui terméletrica e refinaria de óleo
e no estudo de \citep{ivovsevic2015} apresentou concentração de enxofre
(789 $ng / m^3$) e BC (3,4 $\mu g / m^3$) maiores que os encontrados em Nima, 
tanto em abosoluto, quanto relativamente a massa média de $MP_{2,5}$ 
(20,6 $\mu g / m^3$). 
Por fim, \citet{achad2014} detalhou o $MP_{2,5}$ em Córdoba, Argentina, 
encontrando massa média de 50,1 $\mu g / m^3$, maior que Nima, devido a região
ser zona industria e de alta movimentação de caminhões.   

\begin{landscape}
  \begin{table}[H]
    \centering
    %  & \multicolumn{2}{c}{maio 2010} & \multicolumn{2}{c}{novembro 2010} & \multicolumn{2}{c}{abril 2011} \\
% Z & medido & ajustado & medido & ajustado & medido & ajustado \\

\begin{tabular}{ccccccccccccc}
\hline
& \multicolumn{2}{c}{Acra Nima} & \multicolumn{2}{c}{Acra kwabenya} & México & kenia & China & India & Brasil & Croácia & Argentina & Acra \\
& sem H & com H & sem H & com H & mexico & kenia & china & india & brasil & croacia & argentina & ashaiman \\
\hline
$MP_{2,5}$    & 29,7       & 79,8       & 22,9             & 96,5         &        & 30    &       & 41,78 &        &         & 50,055    & 21,6     \\
Na       & 336,0      & 270,9      &                  &              & 12     &       &       & 1920  &        & 117     &           & 743      \\
Mg       & 108,1      & 436,4      &                  &              &        &       & 935   & 650   &        & 22      & 73        & 94,4     \\
Al       & 719,6      & 2897,1     & 342              & 1430         &        &       &       & 1520  & 43,9   & 44      & 383       & 806      \\
Si       & 1453,1     & 7011,3     & 425              & 4500         &        &       &       & 460   & 125,3  & 110     & 1509      & 1159     \\
P        & 14,3       & 25,3       &                  &              &        &       &       &       & 10,9   & 2,8     &           &          \\
S        & 604,6      & 793,1      & 442              & 524          & 600    & 1300  &       & 247   & 496,6  & 789     & 336       & 391      \\
Cl       & 457,7      & 603,2      &                  &              & 370    &       &       & 2210  & 66,6   & 54      &           & 145      \\
K        & 857,6      & 1544,8     & 260              & 731          & 96     & 730   &       & 1550  & 225,3  & 194     & 628       & 487      \\
Ca       & 353,2      & 1248,1     & 41,7             & 432          & 130    & 70    &       & 2300  & 64     & 88      & 308       & 287      \\
Ti       & 47,2       & 205,3      & 12,6             & 100          & 9      & 8,7   &       & 45    & 5,5    & 3,4     & 22        & 59       \\
V        & 1,8        & 4,4        &                  &              & 10     & 3,2   &       &       & 1,53   & 3,4     & 0,5       & 2,9      \\
Mn       & 9,3        & 36,0       & 3,5              & 19           & 3      & 12    &       &       & 11,71  & 4,4     & 12        & 27,4     \\
Fe       & 445,6      & 1852,9     & 109              & 845          & 100    & 130   &       & 438   & 108,3  & 93      & 301       & 987      \\
Zn       & 30,7       & 41,0       & 5,5              & 9,5          & 12     & 100   &       & 73    & 29,7   & 14      & 17        & 164      \\
Br       & 23,5       & 27,4       & 5,7              & 6,8          & 5      & 36    &       &       & 3,75   & 2,6     &           & 32,4     \\
Pb       & 15,5       & 19,3       & 2,1              & 3,5          & 9      & 76    &       & 3     & 8,39   & 6,8     &           & 43,9     \\
BC       & 3,1        & 3,4        & 1,7              & 2,5          &        & 4,8   &       &       &        &         &           & 2,07    \\
\hline
\end{tabular}


    \caption{Médias elementares e média da massa de $MP_{2,5}$ encontradas
             em Nima e comparadas com outras regiões do mundo:
             Kwabenya (Acra) \citep{aboh2009},
             Ashaiman (Acra) \citep{ofosu2012},
             Cidade do México (México) \citep{diaz2014},
             Cairo (Egito) \citep{boman2013},
             Pequim (China) \citep{yang2011},
             Nairóbi (Quênia)  \citep{gaita2014},
             Brasil $^d$ \citep{andrade2012urban},
             Rijeka (Croácia) \citep{ivovsevic2015} e
             Córdoba (Argentina) \citep{achad2014}.
             \label{table:fino_in_the_world}}
  \end{table} 
\end{landscape}

\begin{table}[H]
  \centering
    \input{../outputs/fino_2sitios}
  \caption{Estatística descritiva da área residencial (Sam Road) e avenida (Nima) 
           para $MP_{2,5}$. \label{table:fino_2sitios}}
\end{table}

A tabela \ref{table:fino_2sitios} apresenta as médias elementares e
da massa de $MP_{2,5}$ em separado para a área residencial (Sam Road) e
a avenida (Nima). Apesar das diferenças pequenas entre os dois sítios, 
diferente do $MP_{10}$, a média da massa de $MP_{2,5}$, quando inclusos os dias
de Harmatão, apresenta maior valor na área residencial, 83,3 $\mu g / m^3$,
que na avenida, 76,4 $\mu g / m^3$. Entretanto, corrigindo o efeito do Harmatão,
a avenida volta a ter maio concentração, 31,9 $\mu g / m^3$, contra 27,5 
$\mu g / m^3$  da área residencial.


