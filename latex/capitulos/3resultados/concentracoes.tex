%%%%
\section{Concentrações Ambientais}

Em estudos do aerossol atmosférico usando modelos receptores, 
pode-se direcionar a análise em duas direções. 
A primeira, é a avaliação ambiental das concentrações, considerando-se a massa total,
que possibilita estimar de maneira genérica a carga de poluição total na atmosfera 
durante o período da medida, usada principalmente para comparação com padrões
de qualidade do ar definidos por agências ambientais ou de sáude. 

Na segunda abordagem, mais detalhada, utiliza-se a composição química 
das partículas para estimar perfis de fontes. 

Inicialmente, será exposta a análise ambiental e posteriormente a 
avaliaçao dos perfis de fontes.  

No projeto global, envolvendo 11 pontos de amostragem em Acra, 
coletou-se 2898 amostras, todas analisadas na refletância e 
Fluorescência de Raios X (XRF) do 
Laboratório de Análise dos Processos Atmosféricos - LAPAt
do Instituto de Astronomia, Geofísica e Ciências Atmosféricas - IAG-USP.

Os resultados gerais foram apresentados em 2011 no
\textit{23th Congress of the International Society for Environmental 
Epidemiology} \citep{zhou2011} e publicados em \cite{zhou2013} e \cite{zhou2014}. 

\begin{table}[H]
  \centering
    \input{../outputs/samples}
  \caption{Quantificação total das amostras analisadas no LAPAt porcentagem
          refletância e XRF-ED}
\end{table}

Nesta pesquisa, em particular, aprofundou-se os estudos exclusivamente no bairro
de Nima, onde coletou-se 791 amostras válidas obtidas em dois pontos de 
amostragem entre 11 de Novembro de 2006 e 15 de Agosto de 2008.

Será referido como residencial a rua só com residências e como avenida 
o local de amostagrem com presença de comércio e alto tráfego de veículos.

%%%%
\subsection{Material Particulado Inalável $MP_{10}$}

Os padrões de qualidade do ar são fixados em níveis que quando ultrapassados 
podem afetar a saúde da população. 
A tabela \ref{table:padroesgana} lista os valores do Padrão de Qualidade do Ar 
para $MP_{10}$ ambiental vigente em Gana. Nota-se que há limites diferentes para 
meses de ocorrências do Harmantão. 

% Artigos sobre deserto
\cite{prospero2002} mostra países afetados por harmatão. 

\cite{engelbrecht2009a} e \citep{engelbrecht2009b} estudaram a composição 
da poeira do deserto.

\citep{kaku2016} Golfo Pérsico apresenta altas concentrações devido a poeira do 
deserto do Saara e de industrias locais, como petroquímica. 

Regiões consideradas como zonas
industriais ou de comércio possuem limites mais permissivos, revelando 
uma priorização do Governo com o desenvolvimento econômico do País em 
detrimento da saúde da população.

% http://www.epa.gov.gh/ghanalex/policies/EPAguidelines%20Report.pdf
\begin{table}[H]
\centering
\begin{tabular}{cccc}
\hline
Condição                      &   regiões  &        regiões       &         regiões               \\
Tipo da média                 & sensíveis & residenciais e rurais & insdustriais e comerciais      \\
\hline
Média anual geométrica        & 70              & 100                      & 200                   \\
Média mensal durante Harmatão & 100             & 200                      & 500                     \\
Média diária                  & 110             & 150                      & 260         \\               
\hline
\end{tabular}
\caption{Padrões de Qualidade do Ar para $MP_{10}$ Ambiental em Gana
         \cite{epa2015} \label{table:padroesgana}}
\end{table}

A tabela \ref{table:RIcH_descriptive} mostra a média, desvio padrão da média,
mediana, mínimo e máximo para as concentrações de $MP_{10}$ (Inalável) 
encontradas na área residencial. Esses resultados possibilitam comparações com 
padrões de qualidade do ar para avaliação de atendimento ou não dos mesmos.    

\begin{table}[H]
  \centering
 % \begin{scriptsize}
    \input{../outputs/descriptive_RIcH}
 % \end{scriptsize}
  \caption{Estatística descritiva das concentrações de  $MP_{10}$ na área 
           \textbf{residencial} \label{table:RIcH_descriptive}}
\end{table}

Na tabela \ref{table:pm10standards} estão agrupados os limite da média anual de 
concentrações ambientais para $MP_{10}$ entre Brasil \citep{conama1990}, 
Gana \citep{epa2015} e Organização Mundial de Sáude \citep{who}. 
Em Gana a média anual é geométrica e no Brasil aritmética.
Outra recomendação da OMS é que a concentração não ultrapasse $25 \mu g/m^3$ 
em mais que 1\% das amostragens durante um ano. 

\begin{table}[H]
  \centering
      \begin{tabular}{cccc}
     \hline
   Tipo & Brasil & Gana (residencial e rural) & Organização Mundial de Saúde \\
     \hline
   diaria & 150 & 150 &  50 \\
     anual &  50 & 100 &  20 \\
      \hline
  \end{tabular}
  \caption{Padrões para média anual de $MP_{10}$ no Brasil \citep{conama1990}, 
           Gana \citep{epa2015} e 
           Organização Mundial de Sáude \citep{who}
           \label{table:pm10standards}}
\end{table}

Na área residencial de Nima, a concentração média da massa total de $MP_{10}$ 
durante este estudo foi $114\pm 11 \mu g / m^3$ e na avenida 
$134\pm 12 \mu g / m^3$. Na avenida, como esperado, a concentração média 
foi maior que na região residencial, já que possui maior movimentação de 
veículos, que além de emitirem grande quantidades de poluentes, levantam 
poeira local. 

A CETESB (Companhia Ambiental Do Estado De São Paulo) possui uma rede de 
monitoramento de qualidade do Ar no estado de São Paulo e em seu relatório 
de 2014 \citep{cetesb2014} a estação com menor média anual da RMSP 
de concentração média da massa total de $MP_{10}$ foi a Ibirapuera 
(29 $\mu g / m^3$) e a maior em Parelheiros 44 $\mu g / m^3$.  

Em estudo realizado em XXXX \cite{souza2014} encontrou 64 $\mu g / m^3$.

Em termos gerais, em Nima a concentração é de 2 a 3 vezes maior do que 
tipicamente encontrado em São Paulo. 

Excluindo-se os dias de ocorrência do Harmatão da tabela 
\ref{table:RIcH_descriptive} obtém-se a tabela \ref{table:RIsH_descriptive}.
Assim, é possível avaliar com melhor precisão o impacto de fontes locais.  

\begin{table}[H]
  \centering
 % \begin{scriptsize}
    \input{../outputs/descriptive_RIsH}
 % \end{scriptsize}
  \caption{Estatística descritiva das concentrações de  $MP_{10}$ na área 
           \textbf{residencial} excluíndo-se os dias do Harmantão
            \label{table:RIsH_descriptive}}
\end{table}

Removendo-se o Harmatão, as médias são próximas das tipicamente
encontradas em São Paulo. 

Em Kwabenya, região periférica e semi-rural de Acra 11 $km$ a noroeste de Nima, 
medidas realizadas durante 2006 e 2007 resultaram em 179 $\mu g / m^3$ de massa 
de $MP_{10}$ e 77 $\mu g / m^3$ quando removidos os dias de ocorrências
de ventos do Harmatão \citep{aboh2009}.

\begin{table}[H]
  \centering
 % \begin{scriptsize}
    \input{../outputs/descriptive_TIcH}
 % \end{scriptsize}
  \caption{Estatística descritiva das concentrações de $MP_{10}$ na 
           \textbf{avenida} \label{table:TIcH_descriptive}}
\end{table}

\begin{table}[H]
  \centering
 % \begin{scriptsize}
    \input{../outputs/descriptive_TIsH}
 % \end{scriptsize}
  \caption{Estatística descritiva das concentrações de $MP_{10}$ na 
           \textbf{avenida} removendo-se o Harmatão
           \label{table:TIsH_descriptive}}
\end{table}

No gráfico da figura \ref{fig:plot_RFcH_massa} percebe-se que no período 
do harmatão tanto o padrão nacional de Gana quanto a recomendação da OMS 
são ultrapassados.

\begin{figure}[H]
  \centering
  \begin{subfigure}[b]{0.45\textwidth}
    \includegraphics[width=\textwidth]{../outputs/massa_temporal_RFcH.pdf}
    \caption{Residencial}
  \end{subfigure}%
  \begin{subfigure}[b]{0.45\textwidth}
    \includegraphics[width=\textwidth]{../outputs/massa_temporal_TFcH.pdf}
    \caption{Avenida}
  \end{subfigure}
  \caption{Massa total de $MP_{2,5}$ na área \textbf{residencial} \label{fig:massa_temporal_fino}}
\end{figure}

O índice de ultrapassagens foi alto para os dois os casos, 
mas principalmente na avenida

%%%%
\subsection{Material Particulado Fino ($MP_{2,5}$)}

As recomendações da Organização Mundial de Sáude (OMS) para média diária de 
25 $\mu g/m^3$ para $MP_{2,5}$ e 50 $\mu g/m^3$ 

Na tabela \ref{table:RFcH_descriptive} encontra-se a média, desvio padrão da média, 
mediana, mínimo e máximo das concentrações na área residencial. 

A concentração média da massa total ($83\pm 18 \mu g / m^3$) é aproximadamente
3 vezes maior que a relatada para São Paulo ($28\pm 13 \mu g / m^3$) reportada 
em estudo que mediu as concentrações elementares de $MP_{2,5}$ em 6 cidades 
brasileiras (São Paulo, Rio de Janeiro, Belo Horizonte, Curitiba, Recife e 
Porto Alegre) \cite{andrade2012}. 

Em Kwabenya, região periférica e semi-rural de Acra 11 $km$ a noroeste de Nima, 
medidas realizadas durante 2006 e 2007 resultaram em 40,8 $g / m^3$ de massa de 
$MP_{2,5}$ e 1,9 $g / m^3$ (ou $4,6\%$) de $BC$ \citep{aboh2009}.

Em outro estudo realizado em Aishaiman (região industrial), também em Acra, 
mas a 22 $km$ a nordeste de Nima, entre Fevereiro e Agosto (portanto não 
incluindo os meses de incidência do Harmatão) reportou 21,6 $g / m^3$ de massa 
de $MP_{2,5}$ e 2,07 $g / m^3$ (ou $9,5\%$) de $BC$ \citep{ofosu2012}.

O $BC$ é um indicador de fontes de combustão, principalmente de veículos pesados.
Cidades onde veículos representam a principal fonte poluídora o $BC$ pode chegar 
a $50\%$ da massa total de $MP_{2,5}$.  

Neste estudo $BC$ representou $3,7 \%$ ($3,13\pm 0,05 \mu g / m^3$) da massa 
total, enquanto que em São Paulo esse valor foi de $35,7 \%$ \citep{andrade2012}.
Apesar de muito baixa quando comparada com São Paulo, a porcentagem de $BC$ 
encontrada em Nima concorda com outros estudos de Acra, mesmo Nima tendo alto
tráfego de veículo e frota envelhecida. Isso explica-se, pois, partículas 
oriundas de poeira do solo local e do Harmatão ofuscam a contribuição de outros
elementos, no caso o $BC$. 

\begin{table}[H]
  \centering
 % \begin{scriptsize}
    \input{../outputs/descriptive_RFcH}
 % \end{scriptsize}
  \caption{Estatística descritiva para $MP_{2,5}$ na área \textbf{residencial}
            \label{table:RFcH_descriptive}}
\end{table}

\begin{table}[H]
  \centering
 % \begin{scriptsize}
    \input{../outputs/descriptive_RFsH}
 % \end{scriptsize}
  \caption{Estatística descritiva para $MP_{2,5}$ na área \textbf{residencial}
           removendo-se harmatão \label{table:RFsH_descriptive}}
\end{table}

\begin{table}[H]
  \centering
 % \begin{scriptsize}
    \input{../outputs/descriptive_TFcH}
 % \end{scriptsize}
  \caption{Estatística descritiva para $MP_{2,5}$ na \textbf{avenida}
            \label{table:TFcH_descriptive}}
\end{table}

\begin{table}[H]
  \centering
 % \begin{scriptsize}
    \input{../outputs/descriptive_TFsH}
 % \end{scriptsize}
  \caption{Estatística descritiva para $MP_{2,5}$ na \textbf{avenida}
           removendo-se harmatão \label{table:TFsH_descriptive}}
\end{table}


Já na avenida \ref{table:TFcH_descriptive} o \textbf{Black Carbon} 
representa 4.7 \% da massa total.

%\begin{table}[H]
% \centering
%  \begin{scriptsize}
%    \input{../outputs/tabela_descritiva_com_harmatan.tex}
%  \end{scriptsize} 
%  \caption{Média, desvio padrão e mediana da massa total e ultrapassagens das 
%           recomendações da Organização Mundial de Sáude (OMS) para média diária de 
%           25 $\mu g/m^3$ para $MP_{2,5}$ e 50 $\mu g/m^3$ para $MP_{10}$
%           \label{table:descritiva}}
%\end{table}

%%%%
\subsection{Comparação dos resultados com os da EPA-US}

Entre as 2898 amostras enviadas para serem analisadas na USP, 92 foram previamente 
analisadas por Fluorescência de Raios X na \textbf{United States Environmental 
Protection Agency (USEPA)}. Ao fazermos as análises no Lapat, não sabíamos
dessa informação, pois o Prof. Dr. Majid Ezzati executou um teste a cega das 
nossas medidas para verificação da exatidão e precisão dos resultados. 
Só depois que enviamos os resultados das nossas análises, nos foi informando, 
que parte das amostra já tinham sido analisadas no EPA. Os resultados do Lapat
e EPA tiveram concordância. 

Os elementos com concentrações muito acima do limite de detecção tiveram ótima
correlação, com pode ser observado no gráfico da figura \ref{fig:epa} 

\begin{figure}[H]
  \centering
    \includegraphics[width=0.3\textwidth]{../outputs/EPA_Si.pdf}
    \includegraphics[width=0.3\textwidth]{../outputs/EPA_Fe.pdf}
    \includegraphics[width=0.3\textwidth]{../outputs/EPA_P.pdf}
  \caption{Comparação das concentrações com análise da USEPA \label{fig:epa}.}
\end{figure}

Verificação da qualidade dos resultado obtidos por XRF.
