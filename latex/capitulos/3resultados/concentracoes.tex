%%%%
\newpage
\section{Concentrações Ambientais}

Concetrações da massa total de MP medidas usando metodologia de modelos 
receptores são usadas para avaliação do impacto no 
meio ambiente, materiais e saúde, usando como parâmetro a legislação, que
define limites que não deveriam ser atigindos, e sugestões de órgãos
internacionais, como da OMS, que estipula padrões de concentrações tomando como
critério a preservação da saúde humana. Em estudos mais detalhados, é possível 
aprofundar a análise com identificação e quantificação de espécies químicas, 
que além de revelar os elementos significativos na massa, possibilitam ajustes 
de métodos estatísticos para identificar e estimar perfis das principais 
fontes poluídoras. Assim, nesta seção serão expostos inicialmente os 
resultados das concentrações para comparação com limites de legislações e 
com padrões sugeridos pela OMS, e na próxima seção, serão aplicados métodos 
estatísticos multivariados, AF e PMF, para levantamento das fontes. 

A campanha de amostragem nos quatro bairros resultou em um total de 2898 
amostras, analisadas por refletância e XRF-ED no LAPAt. Concentrações médias, 
análises estatísticas gerais e levantamento de fontes, para os onze locais em 
conjuto, isto é, sem considerar as especifidades de cada bairro, foram 
apresentados em 2011 no \textit{23th Congress of the International Society 
for Environmental Epidemiology} \citep{zhou2011} e publicadas em 
\cite{zhou2013} e \cite{zhou2014}. 

Nesta pesquisa em particular, aprofundou-se os estudos exclusivamente no bairro
de Nima, em dois locais, Sam Road e Nima Road, com 791 amostras válidas obtidas
entre 11 de novembro de 2006 e 15 de agosto de 2008. Daqui em diante, 
medidas realizadas na Sam Road serão referidas como área residencial e na 
Nima Road como avenida, para melhor refletir as características de ambas.

%%%%
\newpage
\subsection{Material Particulado Inalável $MP_{10}$}

Os padrões de qualidade do ar são fixados em níveis que quando 
ultrapassados podem afetar o meio ambiente ou a saúde da população, porém, 
como são definidos na legislação, dependem também de aspectos políticos, 
sociais e econômicos da realidade de cada país e por isso variam são diferentes.

O padrão de qualidade do ar para $MP_{10}$ ambiental vigente em Gana referendado
e monitorado pela EPA-GH \citeyearpar{epa2015} é apresentado na tabela
\ref{table:padroesgana}, possuindo limites diferentes para os meses em que
ocorre o Harmantão, refletindo o efeito que o fenômeno no MP local. 
Regiões consideradas como zonas industriais ou de comércio possuem limites mais
permissivos, revelando uma priorização do Governo com o desenvolvimento 
econômico do País em detrimento da saúde da população.

\begin{table}[H]
\centering
  % http://www.epa.gov.gh/ghanalex/policies/EPAguidelines%20Report.pdf
\begin{tabular}{cccc}
\hline
                              &   regiões  &        zonas       &         zonas               \\
                              & sensíveis  & residenciais e rurais & insdustriais e comerciais      \\
Tipo da média                 & $\mu / m^3$ & $\mu / m^3$ & $\mu / m^3$      \\
\hline
diária                    & 110             & 150                      & 260         \\       
anual geométrica          & 70              & 100                      & 200                   \\
mensal (durante Harmatão) & 100             & 200                      & 500                     \\
\hline
\end{tabular}

\caption{Padrões de Qualidade do Ar para $MP_{10}$ Ambiental em Gana
         \cite{epa2015} \label{table:padroesgana}}
\end{table}

Na tabela \ref{table:pm10standards} compara-se os padrões de $MP_{10}$ entre 
Brasil, Gana e OMS, sendo que em Gana, a média anual é calculada geometricamente
e é portanto matematicamente menor que a aritmética. 

\begin{table}[H]
\centering
  \begin{tabular}{cccc}
\hline
              & Brasil & Gana (residencial e rural) & OMS \\
Tipo da média & $\mu / m^3$ & $\mu / m^3$ & $\mu / m^3$          \\
\hline
diária   & 150              & 150              &  50             \\
anual    &  50 (aritmética) & 100 (geométrica) &  20 (aritmética) \\
\hline
\end{tabular}

  \caption{Padrões para média anual de $MP_{10}$ no Brasil \citep{conama1990}, 
           Gana \citep{epa2015} e OMS \citep{who}. \label{table:pm10standards}}
\end{table}

A tabela \ref{table:RIcH_descriptive} apresenta a média, desvio padrão da média,
mediana, mínimo e máximo para as concentrações de MP Inalável, $MP_{10}$,  
encontradas na área residencial. Esses resultados possibilitam comparações com 
padrões de qualidade do ar para avaliação de atendimento ou não dos mesmos.    

\newpage
\begin{table}[H]
  \centering
    \input{../outputs/descriptive_fino_sH}
  \caption{Estatística descritiva das concentrações de  $MP_{10}$ na área 
           residencial \label{table:descriptive_fino_sH}}
\end{table}

\newpage
\begin{table}[H]
  \centering
    \input{../outputs/fino_2sitios}
  \caption{}
\end{table}



\citep{kaku2016} Golfo Pérsico apresenta altas concentrações devido a poeira do 
deserto do Saara e de indústrias locais, como petroquímica. 

Mn (65), Ti (196+35), V(16)


\cite{engelbrecht2009a} e \citep{engelbrecht2009b} estudaram a composição 
da poeira do deserto.

%The average PM 2.5 /PM 10 aerosol mass fraction for the period of study
%was 0.28




Na área residencial de Nima, a concentração média da massa total de $MP_{10}$ 
durante este estudo foi $114\pm 11 \mu g / m^3$ e na avenida 
$134\pm 12 \mu g / m^3$. Na avenida, como esperado, a concentração média 
foi maior que na região residencial, já que possui maior movimentação de 
veículos, que além de emitirem grande quantidades de poluentes, levantam 
poeira local. 

A CETESB (Companhia Ambiental Do Estado De São Paulo) possui uma rede de 
monitoramento de qualidade do ar no estado de São Paulo e em seu relatório 
de 2014 \citep{cetesb2014} a estação com menor média anual da RMSP 
de concentração média da massa total de $MP_{10}$ foi a Ibirapuera 
(29 $\mu g / m^3$) e a maior em Parelheiros 44 $\mu g / m^3$.  

Em estudo realizado em XXXX \cite{souza2014} encontrou 64 $\mu g / m^3$.

Em termos gerais, em Nima a concentração é de 2 a 3 vezes maior do que 
tipicamente encontrado em São Paulo. 

Excluindo-se os dias de ocorrência do Harmatão da tabela 
\ref{table:RIcH_descriptive} obtém-se a tabela \ref{table:RIsH_descriptive}.
Assim, é possível avaliar com melhor precisão o impacto de fontes locais.  

\begin{table}[H]
  \centering
 % \begin{scriptsize}
    \input{../outputs/descriptive_RIsH}
 % \end{scriptsize}
  \caption{Estatística descritiva das concentrações de  $MP_{10}$ na área 
           residencial excluíndo-se os dias do Harmantão
            \label{table:RIsH_descriptive}}
\end{table}

Removendo-se o Harmatão, as médias são próximas das tipicamente
encontradas em São Paulo. 

Em Kwabenya, região periférica e semi-rural de Acra 11 $km$ a noroeste de Nima, 
medidas realizadas durante 2006 e 2007 resultaram em 179 $\mu g / m^3$ de massa 
de $MP_{10}$ e 77 $\mu g / m^3$ quando removidos os dias de ocorrências
de ventos do Harmatão \citep{aboh2009}.

\begin{table}[H]
  \centering
    \input{../outputs/descriptive_TIcH}
  \caption{Estatística descritiva das concentrações de $MP_{10}$ na 
           avenida \label{table:TIcH_descriptive}}
\end{table}

\begin{table}[H]
  \centering
    \input{../outputs/descriptive_TIsH}
  \caption{Estatística descritiva das concentrações de $MP_{10}$ na 
           avenida removendo-se o Harmatão
           \label{table:TIsH_descriptive}}
\end{table}

No gráfico da figura \ref{fig:plot_RFcH_massa} percebe-se que no período 
do harmatão tanto o padrão nacional de Gana quanto a recomendação da OMS 
são ultrapassados.

\begin{figure}[H]
  \centering
  \begin{subfigure}[b]{0.45\textwidth}
    \includegraphics[width=\textwidth]{../outputs/massa_temporal_RFcH.pdf}
    \caption{Residencial}
  \end{subfigure}%
  \begin{subfigure}[b]{0.45\textwidth}
    \includegraphics[width=\textwidth]{../outputs/massa_temporal_TFcH.pdf}
    \caption{Avenida}
  \end{subfigure}
  \caption{Massa total de $MP_{2,5}$ na área residencial \label{fig:massa_temporal_fino}}
\end{figure}

O índice de ultrapassagens foi alto para os dois os casos, 
mas principalmente na avenida

%%%%
\subsection{Material Particulado Fino ($MP_{2,5}$)}

Para $MP_{2,5}$ Acrescenta-se ainda outra 
recomendação da OMS de que a concentração de  não ultrapasse 25 $\mu g/m^3$ 
em mais que 1\% das amostragens durante um ano.

As recomendações da Organização Mundial de Sáude (OMS) para média diária de 
25 $\mu g/m^3$ para $MP_{2,5}$ e 50 $\mu g/m^3$ 

\begin{landscape}
  \begin{table}[H]
    \centering
    %  & \multicolumn{2}{c}{maio 2010} & \multicolumn{2}{c}{novembro 2010} & \multicolumn{2}{c}{abril 2011} \\
% Z & medido & ajustado & medido & ajustado & medido & ajustado \\

\begin{tabular}{ccccccccccccc}
\hline
& \multicolumn{2}{c}{Este estudo} & \multicolumn{2}{c}{Acra} & Acra & México & Egito & China & Quênia & Brasil $^d$ & Croácia & Argentina \\

& \multicolumn{2}{c}{Nima$^{a,b}$} & \multicolumn{2}{c}{Kwabenya$^{a,b}$} & Ashaiman & Cidade do México & Cairo  & Pequim & Nairóbi &   & Rijeka & Córdoba\\

& \multicolumn{2}{c}{2006-2008} & \multicolumn{2}{c}{2006-2007} & 2008 & 2004-2005 & 2010-2011 & 2008-2009 &  2008-2010 & 2007-2008 & 2013-2015 & 2010-2011\\

\hline
\multicolumn{13}{c}{$ng / m^3$} \\
\hline
$MP_{2,5} {^c}$&29,7   & 79,8       & 22,9             & 96,5 &21,6   & 12     & 51    & 118,5 & 18    &        & 20,6    & 50,1  \\
BC $^c$   & 3,1        & 3,4        & 1,7              & 2,5  &2,07   &        & 3,7   & 8,19  & 2,7   &        & 3,4     &       \\
Na        & 336        & 272        &                  &      &743    &        & 610   &       &       & 117    & 117     &       \\
Mg        & 108        & 436        &                  &      &94,4   &        &       & 290   &       &        & 22      & 73    \\
Al        & 720        & 2897       & 342              & 1430 &806    &        &       & 790   &       & 43,9   & 44      & 383   \\
Si        & 1453       & 7011       & 425              & 4500 &1159   &        &       & 1790  &       & 125,3  & 110     & 1509  \\
P         & 14         & 25         &                  &      &       &        &       &       &       & 10,9   & 2,8     &       \\
S         & 605        & 793        & 442              & 524  &391    & 600    & 1200  &       & 640   & 496,6  & 789     & 336   \\
Cl        & 458        & 603        &                  &      &145    & 37     & 2200  & 2300  & 480   & 66,6   & 54      &       \\
K         & 858        & 1545       & 260              & 731  &487    & 96     & 470   & 3520  & 310   & 225,3  & 194     & 628   \\
Ca        & 353        & 1248       & 41,7             & 432  &287    & 130    & 2900  & 900   & 310   & 64     & 88      & 308   \\
Ti        & 47,2       & 205        & 12,6             & 100  &59     & 9      & 100   & 80    & 54    & 5,5    & 3,4     & 22    \\
V         & 1,8        & 4,4        &                  &      &2,9    & 10     & 9,6   & 30    & 41    & 1,53   & 3,4     & 0,5   \\
Mn        & 9,3        & 36         & 3,5              & 19   &27,4   & 3      & 24    & 90    &       & 11,71  & 4,4     & 12    \\
Fe        & 446        & 1853       & 109              & 845  &987    & 100    & 1000  & 1130  & 350   & 108,3  & 93      & 301   \\
Zn        & 31         & 41,0       & 5,5              & 9,5  &164    & 12     & 200   & 530   & 91    & 29,7   & 14      & 17    \\
Br        & 23,5       & 27,4       & 5,7              & 6,8  &32,4   & 5      & 21    & 30    & 12    & 3,75   & 2,6     &       \\
Pb        & 15,5       & 19,3       & 2,1              & 3,5  &43,9   & 9      & 86    & 240   & 22    & 8,39   & 6,8     &       \\
\hline
\multicolumn{13}{l}{$^a$ Removendo Harmatão} \\
\multicolumn{13}{l}{$^b$ Incluindo Harmatão} \\
\multicolumn{13}{l}{$^c$ Dimensão de $\mu g / m^3$ } \\
\multicolumn{13}{l}{$^d$ Média para seis cidades: São Paulo, Rio de Janeiro, 
Belo Horizonte, Curitiba, Recife e Porto Alegre. } \\
\hline
\end{tabular}


    \caption{}
  \end{table} 
\end{landscape}

Na tabela \ref{table:RFcH_descriptive} encontra-se a média, desvio padrão da média, 
mediana, mínimo e máximo das concentrações na área residencial. 

A concentração média da massa total ($83\pm 18 \mu g / m^3$) é aproximadamente
3 vezes maior que a relatada para São Paulo ($28\pm 13 \mu g / m^3$) reportada 
em estudo que mediu as concentrações elementares de $MP_{2,5}$ em 6 cidades 
brasileiras (São Paulo, Rio de Janeiro, Belo Horizonte, Curitiba, Recife e 
Porto Alegre) \cite{andrade2012}. 

Em Kwabenya, região periférica e semi-rural de Acra 11 $km$ a noroeste de Nima, 
medidas realizadas durante 2006 e 2007 resultaram em 40,8 $g / m^3$ de massa de 
$MP_{2,5}$ e 1,9 $g / m^3$ (ou $4,6\%$) de $BC$ \citep{aboh2009}.

Em outro estudo realizado em Aishaiman (região industrial), também em Acra, 
mas a 22 $km$ a nordeste de Nima, entre Fevereiro e Agosto (portanto não 
incluindo os meses de incidência do Harmatão) reportou 21,6 $g / m^3$ de massa 
de $MP_{2,5}$ e 2,07 $g / m^3$ (ou $9,5\%$) de $BC$ \citep{ofosu2012}.

O $BC$ é um indicador de fontes de combustão, principalmente de veículos pesados.
Cidades onde veículos representam a principal fonte poluídora o $BC$ pode chegar 
a $50\%$ da massa total de $MP_{2,5}$.  

Neste estudo $BC$ representou $3,7 \%$ ($3,13\pm 0,05 \mu g / m^3$) da massa 
total, enquanto que em São Paulo esse valor foi de $35,7 \%$ \citep{andrade2012}.
Apesar de muito baixa quando comparada com São Paulo, a porcentagem de $BC$ 
encontrada em Nima concorda com outros estudos de Acra, mesmo Nima tendo alto
tráfego de veículo e frota envelhecida. Isso explica-se, pois, partículas 
oriundas de poeira do solo local e do Harmatão ofuscam a contribuição de outros
elementos, no caso o $BC$. 

\begin{table}[H]
  \centering
 % \begin{scriptsize}
    \input{../outputs/descriptive_RFcH}
 % \end{scriptsize}
  \caption{Estatística descritiva para $MP_{2,5}$ na área residencial
            \label{table:RFcH_descriptive}}
\end{table}

\begin{table}[H]
  \centering
 % \begin{scriptsize}
    \input{../outputs/descriptive_RFsH}
 % \end{scriptsize}
  \caption{Estatística descritiva para $MP_{2,5}$ na área residencial
           removendo-se harmatão \label{table:RFsH_descriptive}}
\end{table}

\begin{table}[H]
  \centering
 % \begin{scriptsize}
    \input{../outputs/descriptive_TFcH}
 % \end{scriptsize}
  \caption{Estatística descritiva para $MP_{2,5}$ na avenida
            \label{table:TFcH_descriptive}}
\end{table}

\begin{table}[H]
  \centering
 % \begin{scriptsize}
    \input{../outputs/descriptive_TFsH}
 % \end{scriptsize}
  \caption{Estatística descritiva para $MP_{2,5}$ na avenida
           removendo-se harmatão \label{table:TFsH_descriptive}}
\end{table}


Já na avenida \ref{table:TFcH_descriptive} o BC
representa 4.7 \% da massa total.

%\begin{table}[H]
% \centering
%  \begin{scriptsize}
%    \input{../outputs/tabela_descritiva_com_harmatan.tex}
%  \end{scriptsize} 
%  \caption{Média, desvio padrão e mediana da massa total e ultrapassagens das 
%           recomendações da Organização Mundial de Sáude (OMS) para média diária de 
%           25 $\mu g/m^3$ para $MP_{2,5}$ e 50 $\mu g/m^3$ para $MP_{10}$
%           \label{table:descritiva}}
%\end{table}

% Comparar com outras cidades
% histograma do elemento
% cidades: polônia, china, méxico, Kenia , india, 

%%%%
\subsection{Comparação dos resultados com os da EPA-US}

As comparações dos nossos resultados com os da US-EPA (feita as cegas, já que não sabíamos que parte das amostras já tinham sido analisadas na US-EPA) tiveram ótimas concordâncias para os elementos com concentrações acimas do limite de detecção, validando nossa método de calibração.  

Entre as 2898 amostras enviadas para serem analisadas na USP, 92 foram previamente 
analisadas por Fluorescência de Raios X na EPA-US
(United States Environmental Protection Agency). 
Ao fazermos as análises no Lapat, não sabíamos
dessa informação, pois o Prof. Dr. Majid Ezzati executou um teste a cega das 
nossas medidas para verificação da exatidão e precisão dos resultados. 
Só depois que enviamos os resultados das nossas análises, nos foi informando, 
que parte das amostra já tinham sido analisadas no EPA. Os resultados do Lapat
e EPA tiveram concordância. 

Os elementos com concentrações muito acima do limite de detecção tiveram ótima
correlação, com pode ser observado no gráfico da figura \ref{fig:epa} 

\begin{figure}[H]
  \centering
    \includegraphics[width=0.3\textwidth]{../outputs/EPA_Si.pdf}
    \includegraphics[width=0.3\textwidth]{../outputs/EPA_Fe.pdf}
    \includegraphics[width=0.3\textwidth]{../outputs/EPA_P.pdf}
  \caption{Comparação das concentrações com análise da USEPA \label{fig:epa}.}
\end{figure}

Verificação da qualidade dos resultado obtidos por XRF.
