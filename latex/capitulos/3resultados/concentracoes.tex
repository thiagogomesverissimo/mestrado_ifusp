%%%%
\section{Concentrações Ambientais}

Em estudos do aerossol atmosférico usando modelos receptores, 
há duas direções para considerar. 
A primeira, é a avaliação das concentrações da massa total no ar, que possibilita 
estimar de maneira genérica a carga de poluente ambiental durante o período da medida. 
Na segunda, a composição química das partículas dará suporte a estimativa 
dos perfis de fontes. Inicialmente, vamos fazer a análise ambiental 
e posteriormente a avaliaçao dos perfis de fontes.  

No projeto global, envolvendo 11 pontos de amostragem em Acra, 
coletou-se 2898 amostras, todas analisadas na refletância e 
\textbf{Fluorescência de Raios X (XRF)} do 
\textbf{Laboratório de Análise dos Processos Atmosféricos - LAPAt} 
do \textbf{Instituto de Astronomia, Geofísica e Ciências Atmosféricas - IAG-USP}.

Os resultados gerais foram apresentados em 2011 no
\textbf{23th Congress of the International Society for Environmental 
Epidemiology} \citep{zhou2011} e publicados em \cite{zhou2013} e \cite{zhou2014}. 

\begin{table}[H]
  \centering
  \begin{scriptsize}
  \input{../outputs/samples}
  \end{scriptsize}
  \caption{Quantificação total das amostras analisadas no \textbf{LAPAt} porcentagem
          refletância e \textbf{XRF-ED}}
\end{table}

Nesta pesquisa, em particular, aprofundou-se os estudos exclusivamente no bairro de Nima, 
que teve 791 amostras válidas coletadas em dois pontos de amostragem paralela entre 11 de 
Novembro de 2006 e 15 de Agosto de 2008.

Será referido como \textbf{residencial} a rua só com residências e como \textbf{avenida} 
o local de amostagrem com presença de comércio e alto tráfego de veículos.

Na tabela \ref{table:RFcH_descriptive} encontra-se a média, desvio padrão da média, 
mediana, mínimo e máximo das concentrações na área residencial. 

A concentração média da massa total ($83\pm 18 \mu g / m^3$) é aproximadamente
3 vezes maior que a relatada para São Paulo ($28\pm 13 \mu g / m^3$) reportada 
em estudo que mediu as concentrações elementares de $MP_{2,5}$ em 6 cidades 
brasileiras (São Paulo, Rio de Janeiro, Belo Horizonte, Curitiba, Recife e 
Porto Alegre) \cite{andrade2012}. 

O $BC$ é um indicador de fontes de combustão, principalmente de veículos pesados e 
em cidade onde veículos representam a principal fonte poluídora o $BC$ pode chegar 
a $50\%$ da massa total de $MP_{2,5}$. 
Mesmo Nima tendo tendo alto tráfego de veículo o $BC$ representou $3,7 \%$ 
($3,13\pm 0,05 \mu g / m^3$) da massa total, enquanto que em São Paulo esse valor foi de 
$35,7 \%$ \citep{andrade2012}, 

Em Kwabenya, 11 $km$ a noroeste de Nima, medidas de 2009 realizadas por 
\citep{aboh2009}, reportou  40,8 $g / m^3$ e 1,9 $g / m^3$ (ou ) de BC.

\begin{table}[H]
  \centering
 % \begin{scriptsize}
    \input{../outputs/descriptive_RFcH}
 % \end{scriptsize}
  \caption{Estatística descritiva para $MP_{2,5}$ na área \textbf{residencial}
           $\mu g / m^3$ \label{table:RFcH_descriptive}}
\end{table}

A baixa concentração de BC na massa de $MP_{2,5}$ explica ferro (Fe) 



\citep{ofosu2012} Tema 28 km Fevereiro até Agosto 2008

 conduziu um estudo em 
e obteve média diária PM10: 179 ug/m3 e 4 ug/m3 de BC.
O Harmatão foi presente tanto em MP10 quanto em MP2.5.





\begin{table}[H]
  \centering
 % \begin{scriptsize}
    \input{../outputs/descriptive_TFcH}
 % \end{scriptsize}
  \caption{Estatística descritiva para $MP_{2,5}$ na \textbf{avenida}
           $\mu g / m^3$ \label{table:TFcH_descriptive}}
\end{table}

A 
A tabela \ref{table:RFcH_descriptive} traz as médias elementares de $MP_{2,5}$ na área 
\textbf{residencial}.
O \textbf{Black Carbon} representa $3,73 \%$ da massa total.


\begin{figure}[H]
  \centering
  \begin{subfigure}[b]{0.45\textwidth}
    \includegraphics[width=\textwidth]{../outputs/massa_temporal_RFcH.pdf}
    \caption{Residencial}
  \end{subfigure}%
  \begin{subfigure}[b]{0.45\textwidth}
    \includegraphics[width=\textwidth]{../outputs/massa_temporal_TFcH.pdf}
    \caption{Avenida}
  \end{subfigure}
  \caption{Massa total de $MP_{2,5}$ na área \textbf{residencial} \label{fig:massa_temporal_fino}}
\end{figure}


Média, desvio padrão e mediana da massa total e ultrapassagens das 
           recomendações da Organização Mundial de Sáude (OMS) para média diária de 
           25 $\mu g/m^3$ para $MP_{2,5}$ e 50 $\mu g/m^3$ para $MP_{10}$


Na tabela \ref{table:descritiva} estão as médias para a avenida e área residencial
bem como a porcentagem de ultrapassagem do padrão diário da 
Organização Mundial de Sáude (OMS).
O indíce de ultrapassagens foi alto para todos os casos, mas principalmente na avenida,
acima de (90\%) tanto para $MP_{10}$ quanto para $MP_{2,5}$.

Os padrões de qualidade do ar são fixados em níveis que quando ultrapassados 
podem afetar a saúde da população. 
Na tabela \ref{table:pm10standards} há uma comparação dos padrões de $MP_{10}$ 
no Brasil (CONAMA 03/90), Gana (EPA-Gana) e os recomentados pela Organização
Mundial de Sáude.
Outra recomendação da OMS é que a concentração não ultrapasse $25 \mu g/m^3$ 
em mais que 1\% das amostragens durante um ano. 

\begin{table}[H]
  \centering
  \begin{scriptsize}
      \begin{tabular}{cccc}
     \hline
   Tipo & Brasil & Gana (residencial e rural) & Organização Mundial de Saúde \\
     \hline
   diaria & 150 & 150 &  50 \\
     anual &  50 & 100 &  20 \\
      \hline
  \end{tabular}
  \end{scriptsize}
  \caption{Padrões para $MP_{10}$ no Brasil (CONAMA 03/90), Gana (EPA-Gana) e 
          Organização Mundial de Sáude \label{table:pm10standards}}
\end{table}

%http://www.mma.gov.br/port/conama/processos/C1CB3034/Minuta_SMCQ.pdf,,,,
\begin{table}[]
\centering
\caption{My caption}
\label{my-label}
\begin{tabular}{lllll}
                              & PQI & PQ2 & Padrão Primário & Padrão Secundário \\
MP10-24horas                  & 120 & 100 & 75              & 50                \\
MP10- Média Anual aritmética  & 40  & 35  & 30              & 20                \\
MP2.5-24horas                 & 60  & 50  & 37              & 25                \\
MP2.5- Média Anual aritmética & 20  & 17  & 15              & 10               
\end{tabular}
\end{table}

\begin{table}[]
\centering
\caption{My caption}
\label{my-label}
\begin{tabular}{lll}
Tipo de medida               & Padrão Primário & Padrão Secundário \\
MP10-24horas                 & 150             & 150               \\
MP10- Média Anual aritmética & 50              & 50               
\end{tabular}
\end{table}

\begin{table}[]
\centering
\caption{My caption}
\label{my-label}
\begin{tabular}{ll}
oms Tipo de medida & Padrão \\
MP10-24horas       & 50     \\
MP10- Média Anual  & 20     \\
MP2.5-24horas      & 25     \\
MP2.5- Média Anual & 10    
\end{tabular}
\end{table}

% http://www.epa.gov.gh/ghanalex/policies/EPAguidelines%20Report.pdf
\begin{table}[]
\centering
\caption{My caption}
\label{my-label}
\begin{tabular}{llll}
Tipo de média                 & áreas sensíveis & área residencial e rural & área industrial e comercial \\
Média annual geométrica       & 70              & 100                      & 200                         \\
média mensal durante Harmatão & 100             & 200                      & 500                         \\
média diária                  & 110             & 150                      & 260                        
\end{tabular}
\end{table}


%TODO: Fazer sem o harmatão e ver se BC aumenta.  
Já na avenida \ref{table:TFcH_descriptive} o \textbf{Black Carbon} 
representa 4.7 \% da massa total.

\begin{table}[H]
  \centering
  \begin{scriptsize}
    \input{../outputs/descriptive_TFcH}
  \end{scriptsize}
  \caption{Tabela com estística descritiva para $MP_{2,5}$ na \textbf{avenida}
          \label{table:TFcH_descriptive}}
\end{table}

No gráfico da figura \ref{fig:plot_RFcH_massa} percebe-se que no período 
do harmatão tanto o padrão nacional de Gana quanto a recomendação da OMS 
são ultrapassados.



%\begin{table}[H]
% \centering
%  \begin{scriptsize}
%    \input{../outputs/tabela_descritiva_com_harmatan.tex}
%  \end{scriptsize} 
%  \caption{Média, desvio padrão e mediana da massa total e ultrapassagens das 
%           recomendações da Organização Mundial de Sáude (OMS) para média diária de 
%           25 $\mu g/m^3$ para $MP_{2,5}$ e 50 $\mu g/m^3$ para $MP_{10}$
%           \label{table:descritiva}}
%\end{table}

%%%%
\subsection{Comparação dos resultados com dados da USEPA}

Verificação da qualidade dos resultado obtidos por XRF.
Entre as 2898 amostras analisadas por \textbf{ED-XRF}, existiam 92 que foram 
previamente analisadas pela \textbf{United States Environmental Protection 
Agency (USEPA)}, porém nós não sabíamos, pois o Prof. Dr. Majid Ezzati quis 
fazer um teste a cega das nossas medidas. 

Os elementos com concentrações muito acima do limite de detecção tiveram ótima
concordância, com pode ser observado no gráfico da figura \ref{fig:epa} 

\begin{figure}[H]
  \centering
    \includegraphics[width=0.3\textwidth]{../outputs/EPA_Si.pdf}
    \includegraphics[width=0.3\textwidth]{../outputs/EPA_Fe.pdf}
    \includegraphics[width=0.3\textwidth]{../outputs/EPA_P.pdf}
  \caption{Comparação das concentrações com análise da USEPA \label{fig:epa}.}
\end{figure}
