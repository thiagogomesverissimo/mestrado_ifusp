%%%%
\newpage
\section{Concentrações Ambientais}

Concetrações da massa total de MP medidas usando metodologia de modelos 
receptores são usadas para avaliação do impacto no 
meio ambiente, materiais e saúde, usando como parâmetro a legislação, que
define limites que não deveriam ser atigindos, e sugestões de órgãos
internacionais, como da OMS, que estipula padrões de concentrações tomando como
critério a preservação da saúde humana. Em estudos mais detalhados, é possível 
aprofundar a análise com identificação e quantificação de espécies químicas, 
que além de revelar os elementos significativos na massa, possibilitam ajustes 
de métodos estatísticos para identificar e estimar perfis das principais 
fontes poluídoras. Assim, nesta seção serão expostos inicialmente os 
resultados das concentrações para comparação com limites de legislações e 
com padrões sugeridos pela OMS, e na próxima seção, serão aplicados métodos 
estatísticos multivariados, AF e PMF, para levantamento das fontes. 

A campanha de amostragem nos quatro bairros resultou em um total de 2898 
amostras que analisadas por refletância e XRF-ED no LAPAt. Concentrações médias, 
análises estatísticas gerais e levantamento de fontes, para os onze locais em 
conjuto, isto é, sem considerar as especifidades de cada bairro, foram 
apresentados em 2011 no \textit{23th Congress of the International Society 
for Environmental Epidemiology} \citep{zhou2011} e publicadas em 
\cite{zhou2013} e \cite{zhou2014}. 

Nesta pesquisa em particular, aprofundou-se os estudos exclusivamente no bairro
de Nima, em dois locais, Sam Road e Nima Road, com 791 amostras válidas obtidas
entre 11 de novembro de 2006 e 15 de agosto de 2008. Daqui em diante, 
medidas realizadas na Sam Road serão referidas como área residencial e na 
Nima Road como avenida, para melhor refletir as características de ambas.

%%%%
\newpage
\subsection{Material Particulado Inalável $MP_{10}$}

Os padrões de qualidade do ar são fixados em níveis que quando 
ultrapassados podem afetar o meio ambiente ou a saúde da população, porém, 
como são definidos na legislação, dependem também de aspectos políticos, 
sociais e econômicos da realidade de cada país, motivo pelo qual cada nação
tem limites de concetração e espécies legisladas diferentes.

O padrão de qualidade do ar para $MP_{10}$ ambiental vigente em Gana referendado
pela EPA-GH \citeyearpar{epa2015} é apresentado na tabela 
\ref{table:padroesgana}. Nota-se limites específicos para os meses em que 
ocorre o Harmantão, refletindo o efeito do fenômeno no MP local. 
Regiões consideradas como zonas industriais ou de comércio têm limites mais
permissivos, revelando uma priorização do Governo com o desenvolvimento 
econômico do País em detrimento da saúde e bem estar da população.

\begin{table}[H]
\centering
  % http://www.epa.gov.gh/ghanalex/policies/EPAguidelines%20Report.pdf
\begin{tabular}{cccc}
\hline
                              &   regiões  &        zonas       &         zonas               \\
                              & sensíveis$^1$  & residenciais e rurais & industriais e comerciais      \\
Tipo da média                 & $\mu / m^3$ & $\mu / m^3$ & $\mu / m^3$      \\
\hline
diária                    & 110             & 150                      & 260         \\       
anual geométrica          & 70              & 100                      & 200                   \\
mensal (durante Harmatão) & 100             & 200                      & 500                     \\
\hline
\multicolumn{4}{l}{$^1$ A determinação se uma região é sensível ou não é feita
                   pela EPA de Gana, bairros com hospitais, por exemplo, são classificados como sensíveis.} \\
\hline
\end{tabular}

\caption{Padrões de Qualidade do Ar para $MP_{10}$ Ambiental em Gana
         \cite{epa2015} \label{table:padroesgana}}
\end{table}

Na tabela \ref{table:pm10standards} compara-se os padrões de $MP_{10}$ entre 
Brasil, Gana e OMS, com a observação de que em Gana a média anual é calculada 
geometricamente e dependendo do valores envolvidos, a torna menor que a 
aritmética. 

\begin{table}[H]
\centering
  \begin{tabular}{cccc}
\hline
              & Brasil & Gana (residencial e rural) & OMS \\
Tipo da média & $\mu / m^3$ & $\mu / m^3$ & $\mu / m^3$          \\
\hline
diária   & 150              & 150              &  50             \\
anual    &  50 (aritmética) & 100 (geométrica) &  20 (aritmética) \\
\hline
\end{tabular}

  \caption{Padrões para média anual de $MP_{10}$ no Brasil \citep{conama1990}, 
           Gana \citep{epa2015} e OMS \citep{who}. \label{table:pm10standards}}
\end{table}

O gráfico da figura \ref{fig:massa_temporal_mp10} mostra as concetrações medidas
(em escala logaritímica para melhor vizualização dos dados) 
nos dois sítios em função do tempo, sinalizando os padrões de médias 
diárias das tabelas acima. 

Observa-se, que para verificação do atendimento ou não de padrões diários, 
mensais ou anuais previstos na legislação seria necessário remover as amostras 
de 2006 e 2008, mantendo somente as 2007, pois foi o único ano deste experimento
com medidas em todos meses, além de outras exigências da norma, como o 
número máximo de filtros faltantes possível para considerar essas comparações
válidas. O mesmo procedimento deveria ser realizado para comparação com 
limites Brasileiros ou da OMS, seguindo as respectivas normas. 
Entretanto, a intenção é verificar
relativamente os níveis de poluição durante o período de 
coleta, sendo a confrontação com parâmetros legais realizada somente para dar
noção relativa e qualitativa dos resultados obtidos na campanha, 
e não para observância estrita de atendimento ou não da lei.

Sendo assim, o índice de ultrapassagens foi 16,24 \% na área 
residencial e 19,60 \% na avenida, quando considerado o padrão de 150 
$\mu g / m^3$ praticado pela EPA-GH e no Brasil. Em relação
ao padrão de média diária da OMS, 50 $\mu g / m^3$, mais restritivo, 
O índice de ultrapassagens foi alto para os dois sítios 59,90 \% 
na área residencial e 90,95 \% na avenida, sendo que nos meses do Harmatão 
os dois padrões são ultrapassados todos dias. 

\begin{figure}[H]
  \centering
  \begin{subfigure}[b]{0.45\textwidth}
    \includegraphics[width=\textwidth]{../outputs/massa_temporal_RIcH.pdf}
    \caption{Área residencial (Sam Road)}
  \end{subfigure}%
  \begin{subfigure}[b]{0.45\textwidth}
    \includegraphics[width=\textwidth]{../outputs/massa_temporal_TIcH.pdf}
    \caption{Avenida (Nima Road)}
  \end{subfigure}
  \caption{Concentrações de $MP_{10}$ ao longo da campanha de amostragem.
           \label{fig:massa_temporal_mp10}}
\end{figure}

A separação dos dias que ocorrem ventos do Harmatão dos outros dias é importante,
pois além de ser previsto na lei, torna a comparação dos resultados de 
concentrações com outros países que não sofrem do fenômeno mais justa. 
O critério de identificação dos dias de Harmatão, entretanto, não é simples, 
pois o fenômeno 
não é ininterrupto durante os meses em que ocorre. Classificando todos dias
de novembro até março como Harmantão, corre-se o risco de perde informações 
importantes de fontes locais com atividades nesse período, compromente análises
estatísticas de AF e PMF. 
As velocidades e direções do vento medidas na estação meteorologia mais próxima
(aeroporto) também não podem ser usadas, pois mesmo nos dias de Harmatão, 
não há ventos nordeste frequentes e intensos, já que o fenômeno ocorre em 
altas altitudes. 
Considerando o fato dos silicatos serem os principais constituintes da crosta
terrestre, \citet{aboh2009} sugere como critério para identificação dos dias 
de ocorrência do Harmatão os de concentrações de silício no $MP_{10}$ maiores 
que 10 $\mu g/m^3$ entre novembro e março. 

Quando desconsidera-se o Harmatão, segundo critério de \citet{aboh2009},
não há mais ultrapassagens do padrão de 150 $\mu g / m^3$ na área residencial 
e na avenida o índice despenca de 19,60 \% para 0,88 \%. 
Para o padrão OMS de 50 $\mu g / m^3$ na área residencial as 
ultrapassagens diminuem para 31,25 \% (antes 59,90 \%) e na avenida para 
84,21 \% (antes 90,95 \%).

\begin{table}[H]
  \centering
  \input{../outputs/descriptive_inalavel_harmatao}
  \caption{Estatística descritiva das concentrações de $MP_{10}$ conjunta
           (área residencial e avenida) somente para os dias de ocorrência 
           de vento do Harmatão. 54 amostras na área residencial e 59 na avenida 
          \label{table:descriptive_inalavel_harmatao}}
\end{table}

A tabela \ref{table:descriptive_inalavel_harmatao} apresenta a média, 
desvio padrão da média, mediana, mínimo e máximo para as concentrações de 
MP Inalável, $MP_{10}$ para dias de ocorrência do Harmatão, classificados
segundo sugestão de \citet{aboh2009}. A massa total média foi 269,2 $\pm$ 23,0
$\mu g/ m^3$, tendo como elemento mais representativo o Si, 11,3 \% da massa total. 
A ambiguidade (talvez estratégica) da lei define um intervalo de média diária 
de $MP_{10}$ no Harmantão de 100 à 500 $\mu g/ m^3$, dependendo de como a região
é classificada: sensível, residencial, rural, industrial ou comercial, abrindo
brechas que permitem definir regiões baseado em interesses puramente econômicos,
pois limites de 500 $\mu g/ m^3$ mesmo com a poeira pesade do Harmatão serão
dificilmente alcançados. A mediana, que representa melhor um conjunto de dados
com valores no extremo, foi 154,7 $\mu g/ m^3$, e indica que apesar da média 
de 269,2 $\mu g/ m^3$ e máximo 1255,8 $\mu g/ m^3$, no geral apenas alguns
dias tiveram concentrações extremamentes altas. 

Excluíndo-se as concentrações de $MP_{10}$ medidas nos dias de 
ocorrência do Harmantão obtém-se as estatísticas descritivas apresentadas na 
tabela \ref{table:descriptive_inalavel_sH}.

\begin{table}[H]
  \centering
    \input{../outputs/descriptive_inalavel_sH}
  \caption{Estatística descritiva das concentrações de $MP_{10}$ conjunta 
           (Sam Road e Nima Road) excluíndo-se os dias do Harmantão
            \label{table:descriptive_inalavel_sH}}
\end{table}

A média aritmética da massa total de $MP_{10}$ durante a campanha de amostragem 
foi de 57,1 $\pm$ 2,5 $\mu g/ m^3$, abaixo do padrão de qualidade do no país 
para qualquer tipo de zona, e também próxima da mediana, 53,7 $\mu g/ m^3$, 
pois agora os dias de Harmatão, eventos extremos, foram excluídos.  

Comparando com RMSP, em um estudo de 2008 realizado por \citet{souza2014}
que quantificou e caracterizou a composição química das partículas na RMSP,
reportou 64 $\mu g / m^3$ para média de $MP_{10}$ durante o período do estudo, 
com variações entre 30 e 122 $\mu g / m^3$. Por outro lado, o acompanhamento 
da Companhia Ambiental Do Estado De São Paulo (CETESB), que possui 
rede de monitoramento de qualidade do ar em todo estado \citep{cetesb2014}, 
a média de $MP_{10}$ tem despencado nas últimas décadas, saindo de 58 
$\mu g / m^3$ em 2003 para 40 $\mu g / m^3$ em 2012. Em 2013, a estação da rede
que registrou maior média foi a de Parelheiros, 44 $\mu g / m^3$, e a menor 
Ibirapuera, 29 $\mu g / m^3$. Assim, quando excluímos o efeito do Harmatão, 
as concetrações de massa total de $MP_{10}$ de Acra são similares as encontradas
na RMSP no mesmo período.

Em outro estudo também realizado em Acra por \citet{aboh2009} no distrito de 
Kwabenya, região periférica e semi-rural, a 11 $km$ a noroeste de Nima, com 
medidas no mesmo período, 2006 e 2007, obteve 179 $\mu g / m^3$ 
na média da massa total de $MP_{10}$ quando considerou todas amostras. 
Entretanto, quando o autor removeu as medidas dos dias de Harmatão, 
a média diminui para 77 $\mu g / m^3$,
valor um pouco maior do que os 57,1 $\mu g/ m^3$ encontrados em Nima.

\begin{table}[H]
  \centering
    \input{../outputs/inalavel_2sitios}
  \caption{Estatística descritiva da área residencial (Sam Road) e avenida (Nima) 
           \label{table:inalavel_2sitios}}
\end{table}

Por fim, a tabela \ref{table:inalavel_2sitios} apresenta as médias em separado
para a área residencial (Sam Road) e a avenida (Nima). As diferenças 
entre o dois locais são mínimas, mas quando há presença de Harmatão, 
elas são levementes camufladas, pois enquanto que na presença do fenômeno, 
a massa média de $MP_{10}$ na Sam Road equivale a 0,84 \% da Nima Road, 
sem ele, essa equivalência é de 0,70 \%. Com a exclusão do Harmatão, as 
diferenças entre os dois pontos são melhores expressadas, pois o mesmo ocorre 
com os demais elementos, sendo a avenida, como esperado devido ao tráfego de 
veículos e ao comércio, a que carrega as maiores concentrações.

%%%%
\newpage
\subsection{Material Particulado Fino ($MP_{2,5}$)}

\begin{figure}[H]
  \centering
  \begin{subfigure}[b]{0.45\textwidth}
    \includegraphics[width=\textwidth]{../outputs/massa_temporal_RFcH.pdf}
    \caption{Área residencial (Sam Road)}
  \end{subfigure}%
  \begin{subfigure}[b]{0.45\textwidth}
    \includegraphics[width=\textwidth]{../outputs/massa_temporal_TFcH.pdf}
    \caption{Avenida (Nima Road)}
  \end{subfigure}
  \caption{OMS não ultrapasse 25 $\mu g/m^3$ em mais que 1\% das amostragens 
           durante um ano 
          \label{fig:massa_temporal}}
\end{figure}

sem harmatão
ultrapassagens na sam road: 48.78049
ultrapassagens na NIma road: 86.88525

com harmatão
ultrapassagens na sam road: 66.49746
ultrapassagens na NIma road: 92

BC sem Harmantão: 10.49758
BC com Harmatão: 4.206399

mp2.5 no mp10 com harmatão: 73.19706
mp2.5 no mp10 sem harmatão: 58.66797

%The average PM 2.5 /PM 10 aerosol mass fraction for the period of study
%was 0.28

\begin{table}[H]
  \centering
    \input{../outputs/descriptive_fino_sH}
  \caption{Estatística descritiva das concentrações de  $MP_{10}$ na área 
           residencial \label{table:descriptive_fino_sH}}
\end{table}

Para $MP_{2,5}$ Acrescenta-se ainda outra 
recomendação da OMS de que a concentração de  não ultrapasse 25 $\mu g/m^3$ 
em mais que 1\% das amostragens durante um ano.

As recomendações da Organização Mundial de Sáude (OMS) para média diária de 
25 $\mu g/m^3$ para $MP_{2,5}$ e 50 $\mu g/m^3$ 

Acra Kwabenya \citep{aboh2009}

Acra Ashaiman \citep{ofosu2012}

México cidade do méxico 2004-2005 \citep{diaz2014}

Egito Cairo (2010-2011) \citep{boman2013}

China  Pequim (2008-2009) \citep{yang2011}

Quênia Nairobi (2008-2010) \citep{gaita2014}

Brasil 2007-2008 \citep{andrade2012urban} 
Média para seis cidades: São Paulo, Rio de Janeiro, Belo Horizonte, Curitiba, 
Recife e Porto Alegre.

Croácia Rijeka (2013-2015) \citep{ivovsevic2015}

Argentina Córdoba (2010-2011) \citep{achad2014}

As composições elementares médias de $MP_{2,5}$ observada em Nima na maioria 
dos elementos quando considerados eventos do Harmatão 
(tabela \ref{table:fino_in_the_world}) só não superam as relatadas
em Pequim \citep{yang2011}. Por outro lado, removendo-se o Harmat as concentrações eram mais baixos
comparando a Hanói no Vietnã (Cohen et al. 2010), em um
área de amostra com fontes intensas devido a termelétrica de carvão
geração de energia, mas comparável a outros estudos em torno
áreas urbanas no Brasil (Dallarosa et al 2008;. Castanho e
Artaxo 2001; Miranda e Tomaz 2008)


\begin{landscape}
  \begin{table}[H]
    \centering
    %  & \multicolumn{2}{c}{maio 2010} & \multicolumn{2}{c}{novembro 2010} & \multicolumn{2}{c}{abril 2011} \\
% Z & medido & ajustado & medido & ajustado & medido & ajustado \\

\begin{tabular}{ccccccccccccc}
\hline
& \multicolumn{2}{c}{Acra Nima} & \multicolumn{2}{c}{Acra kwabenya} & México & kenia & China & India & Brasil & Croácia & Argentina & Acra \\
& sem H & com H & sem H & com H & mexico & kenia & china & india & brasil & croacia & argentina & ashaiman \\
\hline
$MP_{2,5}$    & 29,7       & 79,8       & 22,9             & 96,5         &        & 30    &       & 41,78 &        &         & 50,055    & 21,6     \\
Na       & 336,0      & 270,9      &                  &              & 12     &       &       & 1920  &        & 117     &           & 743      \\
Mg       & 108,1      & 436,4      &                  &              &        &       & 935   & 650   &        & 22      & 73        & 94,4     \\
Al       & 719,6      & 2897,1     & 342              & 1430         &        &       &       & 1520  & 43,9   & 44      & 383       & 806      \\
Si       & 1453,1     & 7011,3     & 425              & 4500         &        &       &       & 460   & 125,3  & 110     & 1509      & 1159     \\
P        & 14,3       & 25,3       &                  &              &        &       &       &       & 10,9   & 2,8     &           &          \\
S        & 604,6      & 793,1      & 442              & 524          & 600    & 1300  &       & 247   & 496,6  & 789     & 336       & 391      \\
Cl       & 457,7      & 603,2      &                  &              & 370    &       &       & 2210  & 66,6   & 54      &           & 145      \\
K        & 857,6      & 1544,8     & 260              & 731          & 96     & 730   &       & 1550  & 225,3  & 194     & 628       & 487      \\
Ca       & 353,2      & 1248,1     & 41,7             & 432          & 130    & 70    &       & 2300  & 64     & 88      & 308       & 287      \\
Ti       & 47,2       & 205,3      & 12,6             & 100          & 9      & 8,7   &       & 45    & 5,5    & 3,4     & 22        & 59       \\
V        & 1,8        & 4,4        &                  &              & 10     & 3,2   &       &       & 1,53   & 3,4     & 0,5       & 2,9      \\
Mn       & 9,3        & 36,0       & 3,5              & 19           & 3      & 12    &       &       & 11,71  & 4,4     & 12        & 27,4     \\
Fe       & 445,6      & 1852,9     & 109              & 845          & 100    & 130   &       & 438   & 108,3  & 93      & 301       & 987      \\
Zn       & 30,7       & 41,0       & 5,5              & 9,5          & 12     & 100   &       & 73    & 29,7   & 14      & 17        & 164      \\
Br       & 23,5       & 27,4       & 5,7              & 6,8          & 5      & 36    &       &       & 3,75   & 2,6     &           & 32,4     \\
Pb       & 15,5       & 19,3       & 2,1              & 3,5          & 9      & 76    &       & 3     & 8,39   & 6,8     &           & 43,9     \\
BC       & 3,1        & 3,4        & 1,7              & 2,5          &        & 4,8   &       &       &        &         &           & 2,07    \\
\hline
\end{tabular}


    \caption{Médias elementares e média da massa de $MP_{2,5}$ encontradas
             em Nima neste estudo comparadas com outras regiões do mundo:
             Kwabenya (Acra) \citep{aboh2009},
             Ashaiman (Acra) \citep{ofosu2012},
             Cidade do México (México) \citep{diaz2014},
             Cairo (Egito) \citep{boman2013},
             Pequim (China) \citep{yang2011},
             Nairóbi (Quênia)  \citep{gaita2014},
             Brasil $^d$ \citep{andrade2012urban},
             Rijeka (Croácia) \citep{ivovsevic2015} e
             Córdoba (Argentina) \citep{achad2014}.
             \label{table:fino_in_the_world}}
  \end{table} 
\end{landscape}

Na tabela \ref{table:RFcH_descriptive} encontra-se a média, desvio padrão da média, 
mediana, mínimo e máximo das concentrações na área residencial. 

A concentração média da massa total ($83\pm 18 \mu g / m^3$) é aproximadamente
3 vezes maior que a relatada para São Paulo ($28\pm 13 \mu g / m^3$) reportada 
em estudo que mediu as concentrações elementares de $MP_{2,5}$ em 6 cidades 
brasileiras (São Paulo, Rio de Janeiro, Belo Horizonte, Curitiba, Recife e 
Porto Alegre) \cite{andrade2012}. 

Em Kwabenya, região periférica e semi-rural de Acra 11 $km$ a noroeste de Nima, 
medidas realizadas durante 2006 e 2007 resultaram em 40,8 $g / m^3$ de massa de 
$MP_{2,5}$ e 1,9 $g / m^3$ (ou $4,6\%$) de $BC$ \citep{aboh2009}.

Em outro estudo realizado em Aishaiman (região industrial), também em Acra, 
mas a 22 $km$ a nordeste de Nima, entre Fevereiro e Agosto (portanto não 
incluindo os meses de incidência do Harmatão) reportou 21,6 $g / m^3$ de massa 
de $MP_{2,5}$ e 2,07 $g / m^3$ (ou $9,5\%$) de $BC$ \citep{ofosu2012}.

O $BC$ é um indicador de fontes de combustão, principalmente de veículos pesados.
Cidades onde veículos representam a principal fonte poluídora o $BC$ pode chegar 
a $50\%$ da massa total de $MP_{2,5}$.  

Neste estudo $BC$ representou $3,7 \%$ ($3,13\pm 0,05 \mu g / m^3$) da massa 
total, enquanto que em São Paulo esse valor foi de $35,7 \%$ \citep{andrade2012}.
Apesar de muito baixa quando comparada com São Paulo, a porcentagem de $BC$ 
encontrada em Nima concorda com outros estudos de Acra, mesmo Nima tendo alto
tráfego de veículo e frota envelhecida. Isso explica-se, pois, partículas 
oriundas de poeira do solo local e do Harmatão ofuscam a contribuição de outros
elementos, no caso o $BC$. 

\begin{table}[H]
  \centering
    \input{../outputs/descriptive_fino_sH}

  \caption{Estatística descritiva para $MP_{2,5}$ na área residencial
            \label{table:RFcH_descriptive}}
\end{table}



Já na avenida \ref{table:TFcH_descriptive} o BC
representa 4.7 \% da massa total.

%\begin{table}[H]
% \centering
%  \begin{scriptsize}
%    \input{../outputs/tabela_descritiva_com_harmatan.tex}
%  \end{scriptsize} 
%  \caption{Média, desvio padrão e mediana da massa total e ultrapassagens das 
%           recomendações da Organização Mundial de Sáude (OMS) para média diária de 
%           25 $\mu g/m^3$ para $MP_{2,5}$ e 50 $\mu g/m^3$ para $MP_{10}$
%           \label{table:descritiva}}
%\end{table}

% Comparar com outras cidades
% histograma do elemento
% cidades: polônia, china, méxico, Kenia , india, 

\citet{dotse2012} mediu BC em Ashaiman
PM 2.5 mass concentrations determined averaged 23.26 : g/m 3 (3.85 - 46.43 : g/m 3 ) and that of PM 10 was 96.56
: g/m 3 (37.10-293.06 : g/m 3 ).

The results of the current study showed higher concentrations
of PM 2.5 at SPA site (winter) than those found in the same
season in Ghent (25 μg/m 3 ) and in Barcelona (26 μg/m 3 ).
In the summer the concentrations were comparable (Ghent,
12 μg/m 3 ; Barcelona, 16 μg/m 3 ). PM 10 mass concentration
was higher at SPA in the winter than other sites.

\begin{table}[H]
  \centering
    \input{../outputs/fino_2sitios}
  \caption{}
\end{table}


%%%%
\subsection{Comparação dos resultados com os da EPA-US}

As comparações dos nossos resultados com os da US-EPA (feita as cegas, já que não sabíamos que parte das amostras já tinham sido analisadas na US-EPA) tiveram ótimas concordâncias para os elementos com concentrações acimas do limite de detecção, validando nossa método de calibração.  

Entre as 2898 amostras enviadas para serem analisadas na USP, 92 foram previamente 
analisadas por Fluorescência de Raios X na EPA-US
(United States Environmental Protection Agency). 
Ao fazermos as análises no Lapat, não sabíamos
dessa informação, pois o Prof. Dr. Majid Ezzati executou um teste a cega das 
nossas medidas para verificação da exatidão e precisão dos resultados. 
Só depois que enviamos os resultados das nossas análises, nos foi informando, 
que parte das amostra já tinham sido analisadas no EPA. Os resultados do Lapat
e EPA tiveram concordância. 

Os elementos com concentrações muito acima do limite de detecção tiveram ótima
correlação, com pode ser observado no gráfico da figura \ref{fig:epa} 

\begin{figure}[H]
  \centering
    \includegraphics[width=0.3\textwidth]{../outputs/EPA_Si.pdf}
    \includegraphics[width=0.3\textwidth]{../outputs/EPA_Fe.pdf}
    \includegraphics[width=0.3\textwidth]{../outputs/EPA_P.pdf}
  \caption{Comparação das concentrações com análise da USEPA \label{fig:epa}.}
\end{figure}

Verificação da qualidade dos resultado obtidos por XRF.


