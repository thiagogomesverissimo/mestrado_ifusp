%%%%
%\section{Conclusão}
 
Essa pesquisa contribuiu para enriquecer o conhecimento acerca de fontes poluidoras do ar em Acra, capital de Gana, além de servir como aporte para levantamentos de fontes poluidoras em cidades com características similares, isto é, aquelas que passaram ou estão passando por um intenso e rápido processo de transição de uma sociedade agrária para outra urbana e industrializada.%
%
%a questão não é apenas de crescimento, mas a mudança do perfil do modo de vida e da economia.
%  

O uso combinado dos métodos estatísticos multivariados Análise de Fatores (AF) e Positive Matrix Factorization (PMF) permitiu identificar e estimar os perfis de fontes. Para o bom desempenho destes modelos, conseguiu-se particulares avanços em relação aos métodos analíticos empregados, especialmente quanto às metodologias para calibrá-los e definir as incertezas das medidas. Isso trouxe resultados positivos para o PMF, que pondera suas estimativas pelas incertezas das concentrações. Consideramos que o laboratório experimentou uma substancial melhoria na qualidade das medidas de concentrações químicas, analisadas por Fluorescência de Raios X (XRF) e das concentrações de Black Carbon (BC), medido por Refletância e intercalibrada por Thermal Optical Transmitance (TOT). 

Usar uma série de alvos padrões, em combinação com o ajuste funcional da calibração do sistema XRF, mostrou-se uma estratégia eficaz. Isso permitiu a medição de todos elementos químicos (10<Z<83), mesmo quando não se dispunha de alvos padrões para alguns deles. O ajuste por Mínimos Quadrados Matricial ofereceu incertezas menores que as fornecidas pelo fabricante dos alvos padrões, melhorando substancialmente a precisão analítica na determinação dos elementos químicos.

Verificou-se, a partir do acompanhamento de calibrações regular do equipamento de XRF, que a sua eficiência decai com o uso, fenômeno natural, resultado de possível perda na eficiência do detector ou no tubo de raios X. Nas três calibrações que realizamos durante período de análise das amostras (mais de 3000), registramos um decréscimo da eficiência (medida pelo fator de resposta), da ordem de 9\% para as linhas K e 7\% para as linhas L. Iremos completar esta avaliação, incorporando outras calibrações e uma estimativa baseada em horas de uso efetivo do equipamento, mas esses valores permitem dimensionar a necessidade de calibrações em função da exatidão necessária do equipamento. Ou seja, em um ano de uso intensivo do equipamento, sua resposta deslocou-se em ~10\%.
%
%Na conclusão deve-se focar nos resultados síntese que obtivemos - numéricos. Isso está faltando em sua conclusão e procurei completar.
%Também não convém divagar sobre o que não fizemos, e até ao que fizemos mas não incorporamos. O sistema já foi calibrado depois que trocou o tubo. Não foi você quem fez?
%
As comparações, para um sub-conjunto de amostras, dos nossos resultados de XRF versus os da US-EPA (feita as cegas, já que desconhecíamos totalmente que estaria sendo realizada) tiveram ótima concordância para os elementos com concentrações acimas do limite de detecção, validando nosso método de calibração em relação a essa importante agência de controle.%
%
%Ou você usa em português EPA-EUA (aí o mais correto ainda seria APA-EUA) ou em inglês US-EPA. EPA-US não acerta nem o inglês e nem o português.
%   

Ao empregarmos técnica de refletância para medida de Black Carbon pudemos utilizar os mesmos filtros analisados por XRF, o que deu melhor precisão aos dados, já que eliminou a incerteza por duplicidade de amostragem e equipamento. Ela ainda diminuiu significativamente os custos da pesquisa. Utilizando o método absoluto TOT em 52 amostras, para intercalibrar as medidas de refletância, foi possível a quantificação absoluta do BC em todas amostras. O emprego de mínimos quadrados matriciais no ajuste da função de calibração, mostrou-se uma ferramenta robusta para tanto, provendo a incerteza para os valores calculados por essa função, necessários para o uso de PMF.

No período do Harmatão (quando vento de nordeste traz poeira do deserto do Saara) as concentrações de $MP_{2,5}$ e $MP_{10}$ elevaram-se em um fator 10. O padrão diário de $MP_{10}$ em Gana foi ultrapassado em 16,24 \% dos dias na área residencial e 19,60 \% na avenida, enquanto a diretriz da OMS, mais restritiva, foi ultrapassada em 59,90 \% dos dias na área residencial e 90,95 \% na avenida, principalmente nos períodos de ocorrência do Harmatão. A média geométrica anual em 1987 foi de 68 $\mu g/m³$ na área residencial e 89 $\mu g/m³$ na avenida, não ultrapassando esse padrão do paí. As correspondentes médias aritméticas foram de 115 e 132 $\mu g/m³$, o que ficou em torno de 5 vezes a correspondente diretriz da OMS. Isso reflete a total inadequação do padrão local, tanto em relação ao tipo de média empregada (os efeitos sobre a saúde não respondem geometricamente) quanto por sua dimensão pouco atual em relação aos danos à saúde que deveriam limitar.

Mar (Na,Cl), solo (Mg,Al,Si,Ca,Ti,V,Mn,Fe), emissões veiculares (BC,Zn,K,Pb), queima de biomassa (P,S,K, BC) e queima de lixo sólido e outros materiais a céu aberto (Br,Pb) foram as principais fontes encontras para $MP_{2,5}$. Para $MP_{2,5-10}$ as principais fontes encontradas foram: mar (Na,Cl), solo (Mg,Al,Si,Ca,Ti,V,Mn,Fe), partículas envelhecidas de emissões veiculares, queima de biomassa e solo (S, K, Zn, Br, Pb + solo) e poeira de estrada (Zn + solo).

A poeira do solo, fonte de maior peso, representa um problema grave de poluição em Acra, pois há muitos terrenos descampado, que também acabam tornando-se depósitos de lixo a céu aberto. Somente avenidas principais e rodovias são pavimentadas. Apesar, desta ser uma fonte característica de $MP_{2,5-10}$, pois é formada por processos mecânicos, em Acra ela também apareceu como principal fonte no $MP_{2,5}$, tamanha sua presença na atmosfera. Além disso, verificou-se, como em outros experimentos, que o fenômeno natural do Harmatã introduz um grande incremento de poeira do solo (vinda do Saara), no período do inverno.

O peso da biomassa seria uma consequência de que mais da metade da população de Acra queimava lenha ou carvão para cozimento de alimentos no período do estudo, enquanto no país esse percentual ultrapassava os 70\%.

Acra é uma cidade litorânea e teve a fonte mar muito bem caracterizada nas análises de AF e PMF, fato esse esperado e que fortaleceu a validade da metodologia experimental e teórica usada.

Por fim a grande movimentação de veículos antigos em Acra, fenômeno comum em cidades de países em desenvolvimento que crescem sem planejamento, poluí intensamente o ar, direta e indiretamente. Qual seja, pela exaustão dos resíduos de combustão nos motores, evaporação de combustíveis, desgaste de pneus e freios, ou por levantamento de partículas do solo ou participação na formação de aerossol secundário. 

Estes resultados representam um caracterização valiosa do aerossol atmosférico na região de Acra, sendo importante no contexto do projeto principal que foi desenvolvido com a Faculdade de Saúde Pública de Harvard e com a Universidade de Gana, e que avaliou outras cidades da África Subsariana (SSA). 

Nossos resultados, além de terem contribuído para a metodologia de análise de espécies do aerossol atmosférico, demostram a necessidade de políticas públicas de combate a poluição do ar, que pediria medidas perfeitamente tangíveis, como a pavimentação das vias, cobertura do solo com vegetação, incentivo ao uso de gás de cozinha, melhora no transporte público, as quais ajudariam a diminuir os altos índices de poluição observados.
