%%%%
%\section{Conclusão}

%A presença de solo foi grande tanto no PM10 quando no PM2.5
%Pesquisa de base como esta servem para dar sustentação científica 
%ao debate relacionados a legislação de padrões;.

Essa pesquisa contribuiu para enriquecer o conhecimento acerca de fontes 
poluidoras do ar em Acra, capital de Gana, ou mesmo em cidades com características similares, ou seja, aquela que passaram ou estão passando por um intenso e rápido processo de urbanização nas últimas décadas.  

O uso combinado dos  métodos estatísticos multivariados Análise de Fatores (AF) e Positive Matrix Factorization permitiu identificar e estimar os perfis das fontes. Todavia, foi necessário avaliar
a qualidade experimental dos resultados das concentrações químicas, analisadas por Fluorescência de Raios X (XRF) e das concentrações de Black Carbon (BC), medido por Refletância e Thermal Optical Transmitance (TOT). 

O uso de alguns alvos padrões seguido de ajuste funcional da calibração do sistema XRF se mostrou uma estratégia eficaz, pois permitiu a medição de elementos químicos com alvos padrões inexistentes. O ajuste dos Mínimos Quadrados Matricial obteve incertezas menores que as garantidas pelo fabricante dos alvos padrões. Assim, a calibração além de permitir identificação de elementos sem alvo padrão, também melhorou a precisão dos elementos com alvo padrão. 

Uma vez que o PMF pondera sua estimativa pelas incertezas das concentrações a qualidade das incertezas interfere na determinação da contribuição das fontes.

Notou-se que conforme o uso do equipamento de XRF a eficiência diminui com tempo, mas ainda não se identificou o motivo, que pode ser queda de eficiência do detector ou queda de eficiência do tubo de raios X. Nova calibração será realizada, pois o tubo de raios X foi trocado, e poderemos descobrir o se o tubo perde eficiência com o uso comparando a nova calibração com as anteriores. 

As comparações dos nossos resultados com os da US-EPA (feita as cegas, já que não sabíamos que parte das amostras já tinham sido analisadas na US-EPA) tiveram ótimas concordâncias para os elementos com concentrações acimas do limite de detecção, validando nossa método de calibração.   

Nas medidas de Black Carbon (BC), a técnica de refletância pode ser realizada nos mesmos filtros analisados por XRF, diminuindo significativamente os custos da pesquisa. Entretanto, a refletância apresentou problemas para filtros super carregados. Utilizando o método absoluto TOT para inter-calibrar as medidas de refletância, possibilitou a quantificação do BC em todas amostras. 

No período do Harmatão (quando vem vento de nordeste trazendo poeira do deserto do Saara) as concentrações de MP2,5 e MP10 se elevam em um fator 10 e destaca-se como fonte principal em análises de AF e PMF. 

Os limites diários e anuais recomendados pela Organização Mundial de Saúde (OMS) foram ultrapassados muitas vezes, principalmente nos períodos de ocorrência do Harmatão.

Sal marinho (Na,Cl), solo (massa,Fe,Ti, Mn,Si,Al,Ca,Mg), emissões veiculares (BC,Pb,Zn) e queima de biomassa (K,P,S) foram as principais fontes encontras.  

O solo, fonte de maior peso, representa um problema grave de poluição em Acra. Somente grandes avenidas principais e rodovias são pavimentadas e há muitos terrenos descampados, formando depósitos de lixo a céu aberto. Apesar, do solo ser uma fonte característica de MP10, pois é formado por processos mecânicos, em Acra ele aparece também como principal fonte no MP2,5, tamanha sua presença na atmosfera.    

Mais da metade da população usa queima de lenha ou carvão para cozimento de alimentos. Em Acra, a situação é um pouco melhor que no resto do país. 

Acra é uma cidade litorânea e teve a fonte Mar muito bem caracterizada nas análises de AF e PMF, fato esse que valida toda metodologia experimental e teórica usada, pois caso não encontrássemos Mar em uma cidade litorânea seria sinal de algum problema na metodologia empregada. 
A grande movimentação de veículos antigos circulando em Acra, fenômeno comum em cidades de países em desenvolvimento que crescem sem planejamento, poluí o ar direta e indiretamente, ou seja, por combustão nos motores, pneus, freios ou por levantamento de partículas do solo. 

Estes resultados representam um caracterização importante do aerossol atmosférico na região de Acra e será muito importante no contexto do projeto principal que avalia outras cidades da África Subsariana (SSA). 

Nossos resultados demostram a necessidade de políticas públicas combate a poluição e medidas tangíveis como pavimentação das vias, cobertura do solo com vegetação, incentivo ao uso de gás de cozinha, melhora no transporte público, ajudariam a diminuir os altos índices de poluição observados.
