%%%%
%\section{Conclusão}

Essa pesquisa contribuiu para enriquecer o conhecimento acerca de fontes poluidoras do ar em Acra, capital de Gana, além de servir como aporte para levantamento de fontes poluidoras em cidades com características similares, isto é, aquelas que passaram ou estão passando por um intenso e rápido processo de urbanização.  

O uso combinado dos métodos estatísticos multivariados Análise de Fatores (AF) e Positive Matrix Factorization (PMF) permitiu identificar e estimar os perfis das fontes. Todavia, foi necessário avaliar a qualidade experimental dos resultados das concentrações químicas, analisadas por Fluorescência de Raios X (XRF) e das concentrações de Black Carbon (BC), medido por Refletância e Thermal Optical Transmitance (TOT). 

O uso de alguns alvos padrões seguido de ajuste funcional da calibração do sistema XRF se mostrou uma estratégia eficaz, pois permitiu a medição de elementos químicos com alvos padrões inexistentes. O ajuste por Mínimos Quadrados Matricial ofereceu incertezas menores que as garantidas pelo fabricante dos alvos padrões, melhorando a precisão dos elementos com alvo padrão disponível. Uma vez que o PMF pondera sua estimativa pelas incertezas das concentrações a qualidade das mesmas interfere na determinação da contribuição e perfis das fontes.

Observou-se a partir de calibrações frequentes do equipamento de XRF que conforme o uso, a eficiência diminui, fenômeno natural, resultado de possível perda na eficiência do detector ou no tubo de raios X. Nova calibração ainda será realizada, pois o tubo de raios X foi recentemente trocado, e a partir da comparação com calibrações anteriores poderá se identificar o motivo da diminuição da eficiência (tubo ou detector).

As comparações dos nossos resultados com os da EPA-US (feita as cegas, já que não sabíamos que parte das amostras tinham sido previamente analisadas pela XRF da EPA-US) tiveram ótimas concordâncias para os elementos com concentrações acimas do limite de detecção, validando nossa método de calibração.   

Nas medidas de Black Carbon (BC), a técnica de refletância pode ser realizada nos mesmos filtros analisados por XRF, diminuindo significativamente os custos da pesquisa. Entretanto, a refletância apresentou problemas para filtros super carregados. Utilizando o método absoluto TOT para intercalibrar as medidas de refletância, possibilitou a quantificação do BC em todas amostras. 

No período do Harmatão (quando vem vento de nordeste trazendo poeira do deserto do Saara) as concentrações de $MP_{2,5}$ e $MP_{10}$ se elevam em um fator 10. Os padrões diários e anuais recomendados pela Organização Mundial de Saúde (OMS) foram ultrapassados muitas vezes, principalmente nos períodos de ocorrência do Harmatão, mas a média anual do país não foi ultrapassada.

Mar (Na,Cl), solo (Mg,Al,Si,Ca,Ti,V,Mn,Fe), emissões veiculares (BC,Zn,K,Pb), queima de biomassa (P,S,K) e queima de lixo sólido e outros materiais a céu aberto (Br,Pb) foram as principais fontes encontras para $MP_{2,5}$. Para $MP_{2,5-10}$ as principais fontes encontradas foram: mar (Na,Cl), solo (Mg,Al,Si,Ca,Ti,V,Mn,Fe), partículas envelhecidas de emissões veiculares e queima de biomassa (S, K, Zn, Br, Pb) e poeira de estrada (Ca, Zn).

O solo, fonte de maior peso, representa um problema grave de poluição em Acra, pois há muitos terrenos descampado, que acabam se tornando depósitos de lixo a céu aberto. Somente avenidas principais e rodovias são pavimentadas. Apesar, do solo ser uma fonte característica de $MP_{2,5-10}$, pois é formado por processos mecânicos, em Acra ele aparece também como principal fonte no $MP_{2,5}$, tamanha sua presença na atmosfera.    

Mais da metade da população usa queima de lenha ou carvão para cozimento de alimentos. Em Acra, a situação é um pouco melhor que no resto do país. 

Acra é uma cidade litorânea e teve a fonte mar muito bem caracterizada nas análises de AF e PMF, fato esse que valida toda metodologia experimental e teórica usada, pois caso não encontrássemos mar em uma cidade litorânea seria sinal de algum problema na metodologia empregada. 
A grande movimentação de veículos antigos circulando em Acra, fenômeno comum em cidades de países em desenvolvimento que crescem sem planejamento, poluí o ar direta e indiretamente, ou seja, por combustão nos motores, pneus, freios ou por levantamento de partículas do solo. 

Estes resultados representam um caracterização importante do aerossol atmosférico na região de Acra e será muito importante no contexto do projeto principal que avalia outras cidades da África Subsariana (SSA). 

Nossos resultados demostram a necessidade de políticas públicas combate a poluição e medidas tangíveis como pavimentação das vias, cobertura do solo com vegetação, incentivo ao uso de gás de cozinha, melhora no transporte público, ajudariam a diminuir os altos índices de poluição observados.
