%%%%
\section{Meteorologia}

Faremos a avaliação da meteorologia local usando dados da estação meteorológica
de Acra cadastrada na \textit{NOAA} e com dados de re-análise global com grade
em Acra. 

%%%%
\subsection{NOAA}
Utilizando os dados da NOAA, pode-se verificar a distribuição de frequência de
direção dos ventos para encontrar a direção predominante de origem dos ventos e 
ainda verificar se a distribuição muda conforme as estãções do ano.  

%TODO: fazer gráfico velocidade do vento + poluicao nos dia medidos
%TODO: fazer Grafico com limite (linha horizontal) da EPA ou Ghna lei da serio temporal.

Oeste e sudoeste foram as direções predominantes do vento durante todo o período 
de amostragem (Setembro de 2006 à Junho de 2008).
 
\begin{figure}[H]
\begin{center}
  \includegraphics[scale=0.4]{../outputs/completo2006_2008.pdf}
  \caption{Distribuição da frequência do vento em (\%) entre
           Setembro de 2006 e Junho de 2008.}
\end{center}
\end{figure}


No inverno e outono a mais predominânia de ventos sul que na primevera 
e no verão. 

\begin{figure}[H]
\begin{center}
  \includegraphics[scale=0.30]{../outputs/autumn2006.pdf}
  \includegraphics[scale=0.30]{../outputs/winter2006_2007.pdf}
  \includegraphics[scale=0.30]{../outputs/spring2007.pdf}
  \includegraphics[scale=0.30]{../outputs/summer2007.pdf}
  \includegraphics[scale=0.30]{../outputs/autumn2007.pdf}
  \includegraphics[scale=0.30]{../outputs/winter2007_2008.pdf}
  \includegraphics[scale=0.30]{../outputs/spring2008.pdf}
\end{center}
\caption{Distribuição da frequência do vento em (\%) por estação do ano entre
         Setembro de 2006 e Junho de 2008.}
\end{figure}

Entre Dezembro e Fevereiro, período de ocorrência do Harmatão, aparece 
ventos oriundos da direção nordeste, ainda que com frequência inferior 
aos de sudoeste.

\begin{figure}[H]
\begin{center}
  \includegraphics[scale=0.40]{../outputs/harmattan2006_2007.pdf}
  \includegraphics[scale=0.40]{../outputs/harmattan2007_2008.pdf}
\end{center}
\caption{Distribuição da frequência do vento em (\%) nos mês de ocorrência
         do Harmatão, entre Dezembro e Fevereiro.}
\end{figure}



