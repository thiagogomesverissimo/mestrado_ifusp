%%%%
\section{massa}


Os padrões de qualidade do ar são fixados em níveis que possam propiciar uma margem de segurança.

quando ultrapassados do valor permitido poderão afetar a saúde da população. 
%TODO: fazer Grafico com limite (linha horizontal) da EPA ou Ghna lei da serio temporal.
% Resolução do CONAMA Nº03/90 é estabelecido dois tipos de padrões de qualidade do ar, os primários e os secundários
Legislação vigente de Gana poluição do Ar:

\begin{table}[]
\centering
\caption{Brazil,  WHO, Ambient Air Quality-EPA Ghana}
\label{my-label}
\begin{tabular}{llll}
  Poluente                                   & Gana & Brasil & OMS \\
  NOx-24h $\mu g/m^3$                        & 150  & 190    & 200 \\
  PM10-24h $\mu g/m^3$                       & 70   & 150    & 50  \\
  SOx-24 hr $\mu g/m^3$                      & 150  & 100    & 20  \\
  Partículas Totais em Suspensão $\mu g/m^3$ & 230  & 150    & -  
\end{tabular}
\end{table}

A moda grossa - subtração do Pm10 por pm2.5 - é provinda em grande parte
de poeira mineral XX\% +/-. pm2.5 também tem grande participação de poera XX\%, 
mas as fontes antropogênicas ainda dominam.

figura abaixo explicita as concentrações no período de amostragem.

%\begin{figure}[H]
%\begin{center}
%  \includegraphics[scale=0.30]{../outputs/RFcH_massa.pdf}
%  \caption{}
%\end{center}

A OMS recomenda que a concentração média anual de material particulado 
fino não ultrapasse os 10 µg/m3. No Recife foi de 7,29 µg/m3. 
Outra recomendação é que a concentração não ultrapasse 25 µg/m3 em mais que 1\% 
das amostragens durante um ano. 
No nosso caso, tivemos 205 amostragens em um ano, houve ultrapassagens deste padrão no período,
 significando que em ambos os quesitos o Recife ficou dentro do padrão recomendado pela OMS. Destaque-se, contudo, que essa concentração ficou significativamente abaixo daquilo que foi observado nas demais capitais onde se realiza o projeto.

Verifica-se, no gráfico 5, que os dias com maior incidência de MP estão situados principalmente nos meses de inverno, o que se associa ao aumento de pessoas com problemas respiratórios nesses meses.



%%%%
\subsection{Comparação com dados da EPA}
, I plotted elemental concentrations measured in USP against those measured by EPA. You can find them in the attached file. As we can see,most data points sit along the diagonal line, which suggest s there are good agreements between elemental concentrations measured in USP and EPA.

\subsection{Concentrações elementares}

A tabela 1 mostra as concentrações médias dos elementos químicos, 
BC e massa total do MP2,5, para as 309 amostras válidas analisadas nesta campanha.
A tabela abaixo nos da o número de amostras seguimentada por
moda e região de amostragem. 
Também mostra o número de casos quando remove-se os dias de ocorrência do harmatão.

A figura 2 mostra a concentração da massa total ao longo do período de amostragem e a figura 3 a concentração de Black Carbon durante o mesmo período.

\begin{table}[H]
 \centering
  %\label{my-label}
  \input{../outputs/tabela_descritiva_com_harmatan.tex}
  \caption{Estatística descritiva incluindo-se os dias com harmatão}
\end{table}

\begin{table}[H]
  \centering
  %\label{my-label}
  \input{../outputs/tabela_descritiva_sem_harmatan.tex}
  \caption{Estatística descritiva excluíndo-se os dias com harmatão}
\end{table}

%%%%
\section{Black Carbon}
a


