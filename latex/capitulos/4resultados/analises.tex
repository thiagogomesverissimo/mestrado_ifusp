%%%%
\section{massa}

A moda grossa - subtração do Pm10 por pm2.5 - é provinda em grande parte
de poeira mineral XX\% +/-. pm2.5 também tem grande participação de poera XX\%, 
mas as fontes antropogênicas ainda dominam.

%%%%
\section{Fluorescência de Raiox X}

%%%%
\subsection{Calibração}

Calibrações:

16/nov/2010

\begin{figure}[H]
  \caption{}
  \begin{subfigure}[b]{0.5\textwidth}
    \includegraphics[width=\textwidth]{../outputs/Knov2016Calibration.pdf}
    \caption{k}
  \end{subfigure}%
  \begin{subfigure}[b]{0.5\textwidth}
    \includegraphics[width=\textwidth]{../outputs/Lnov2016Calibration.pdf}
    \caption{l}
  \end{subfigure}
\end{figure}


maio/2010

08/abr/2011


%%%%
\subsection{Limite de Detecçção}

\begin{figure}[H]
  \caption{}
  \begin{subfigure}[b]{0.5\textwidth}
    \includegraphics[width=\textwidth]{../outputs/limitDetectionK.pdf}
    \caption{k}
  \end{subfigure}%
  \begin{subfigure}[b]{0.5\textwidth}
    \includegraphics[width=\textwidth]{../outputs/limitDetectionL.pdf}
    \caption{l}
  \end{subfigure}
\end{figure}


%%%%
\subsection{Comparação com dados da EPA}


\subsection{Concentrações elementares}
A tabela abaixo nos da o número de amostras seguimentada por
moda e região de amostragem. 
Também mostra o número de casos quando remove-se os dias de ocorrência do harmatão.

\begin{table}[H]
 \centering
  %\label{my-label}
  \input{../outputs/tabela_descritiva_com_harmatan.tex}
  \caption{Estatística descritiva incluindo-se os dias com harmatão}
\end{table}

\begin{table}[H]
  \centering
  %\label{my-label}
  \input{../outputs/tabela_descritiva_sem_harmatan.tex}
  \caption{Estatística descritiva excluíndo-se os dias com harmatão}
\end{table}

%%%%
\section{Black Carbon}
a


