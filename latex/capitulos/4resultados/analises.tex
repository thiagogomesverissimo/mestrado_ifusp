%%%%

\section{massa}

A moda grossa - subtração do Pm10 por pm2.5 - é provinda em grande parte
de poeira mineral XX\% +/-. pm2.5 também tem grande participação de poera XX\%, 
mas as fontes antropogênicas ainda dominam.


\section{Fluorescência de Raiox X}

A tabela abaixo nos da o número de amostras seguimentada por
moda e região de amostragem. 
Também mostra o número de casos quando remove-se os dias de ocorrência do harmatão.

\begin{table}[H]
 \centering
  %\label{my-label}
  \input{../outputs/tabela_descritiva_com_harmatan.tex}
  \caption{Estatística descritiva incluindo-se os dias com harmatão}
\end{table}

\begin{table}[H]
  \centering
  %\label{my-label}
  \input{../outputs/tabela_descritiva_sem_harmatan.tex}
  \caption{Estatística descritiva excluíndo-se os dias com harmatão}
\end{table}

Calibração

\begin{figure}[H]
\begin{center}
  \includegraphics[scale=0.4]{../outputs/limitDetectionK.pdf}
  \caption{Distribuição da frequência do vento em (\%) entre
           Setembro de 2006 e Junho de 2008.}
\end{center}
\end{figure}

\begin{figure}[H]
\begin{center}
  \includegraphics[scale=0.4]{../outputs/limitDetectionL.pdf}
  \caption{Distribuição da frequência do vento em (\%) entre
           Setembro de 2006 e Junho de 2008.}
\end{center}
\end{figure}

%%%%
\section{Black Carbon}
a


