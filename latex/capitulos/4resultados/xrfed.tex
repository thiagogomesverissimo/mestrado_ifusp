%%%%
\section{Fluorescência de Raiox X}


Foram consideradas as 
incertezas do método analítico (EDX, Refletância e gravimetria), 
assim como um erro percentual fixo de 8\%, atribuído ao método de amostragem em si 
(erro médio observado em amostragens em paralelo com o amostrador Harvard).

%%%%
\subsection{Calibração}

Com os alvos padrões que tínhamos no LAPAT plotamos o gráfico
da ... e foi possível ajustar uma curva nos pontos. 
Assim, calculou-se o fator de resposta para elementos que 
não tínhamos os alvos padrões. 

Os alvos da Micromatter tem incerteza na massa de 5%.
No cálculo do fator de respota, a propagação do erro dos alvos de calibração
e a incerteza percentual estão na tabela. 

percebe-se que a incerteza percentual ajustada é menor que 
a incerteza percentual medida. 

A incerteza é crítica para o PMF. 

Usando \textbf{Ajuste dos Mínimos Quadrados Matricial} foi possível
reduzir a incerteza percentual.  

\begin{table}[H]
 
  \begin{scriptsize} %small footnotesize scriptsize
  %\label{my-label} 
  \caption{}
  \input{../outputs/edxCalibrationnov2010K.tex}
  \end{scriptsize}
\end{table}

\begin{table}[H]
  
  \begin{scriptsize} %small
  %\label{my-label} 
  \caption{}
  \input{../outputs/edxCalibrationnov2010L.tex}
  \end{scriptsize}
\end{table}


\begin{table}[H]
  
  \begin{footnotesize} %small
  %\label{my-label} 
  \caption{}
  \input{../outputs/comparaCalibrationK.tex}
  \end{footnotesize}
\end{table}

\begin{table}[H]
  
  \begin{footnotesize} %small
  %\label{my-label} 
  \caption{}
  \input{../outputs/comparaCalibrationK.tex}
  \end{footnotesize}
\end{table}


\begin{figure}[H]
  \caption{Calibração da Fluorescência de Raiox X - Maio de 2010}
  \begin{subfigure}[b]{0.5\textwidth}
    \includegraphics[width=\textwidth]{../outputs/CalibrationK2010MaiAkerr.pdf}
    \caption{Linha K}
  \end{subfigure}%
  \begin{subfigure}[b]{0.5\textwidth}
    \includegraphics[width=\textwidth]{../outputs/CalibrationL2010MaiAkerr.pdf}
    \caption{Linha L}
  \end{subfigure}
\end{figure}

\begin{figure}[H]
  \caption{Calibração da Fluorescência de Raiox X - Novembro de 2010}
  \begin{subfigure}[b]{0.5\textwidth}
    \includegraphics[width=\textwidth]{../outputs/CalibrationK2010NovAkerr.pdf}
    \caption{Linha K}
  \end{subfigure}%
  \begin{subfigure}[b]{0.5\textwidth}
    \includegraphics[width=\textwidth]{../outputs/CalibrationL2010NovAkerr.pdf}
    \caption{Linha L}
  \end{subfigure}
\end{figure}

\begin{figure}[H]
  \caption{Calibração da Fluorescência de Raiox X - Abril de 2011}
  \begin{subfigure}[b]{0.5\textwidth}
    \includegraphics[width=\textwidth]{../outputs/CalibrationK2011AbrAkerr.pdf}
    \caption{Linha K}
  \end{subfigure}%
  \begin{subfigure}[b]{0.5\textwidth}
    \includegraphics[width=\textwidth]{../outputs/CalibrationL2011AbrAkerr.pdf}
    \caption{Linha L}
  \end{subfigure}
\end{figure}

O queda de desempenho no uso da fluorescência de Raios X pode ser observada 
na figura \ref{fig:compara_calibracao}. 
Possíveis motivos para queda de desempenho: desgaste do tubo ou desgaste do detector. 

\begin{figure}[H]
  \caption{Calibrações de Fluorescência de Raiox X em 3 períodos \label{fig:compara_calibracao}}
  \begin{subfigure}[b]{0.5\textwidth}
    \includegraphics[width=\textwidth]{../outputs/CalibrationKcomparacao.pdf}
    \caption{Linha K}
  \end{subfigure}%
  \begin{subfigure}[b]{0.5\textwidth}
    \includegraphics[width=\textwidth]{../outputs/CalibrationLcomparacao.pdf}
    \caption{Linha L}
  \end{subfigure}
\end{figure}

%%%%
\subsection{EPA}

verificação da qualidade dos resultado obtidos por XRF.

there are good agreements between
>>>>>>elemental concentrations measured in USP and EPA. 
inter-comparisons

%%%%
\subsection{Limite de Detecçção}

O limite de detecção é cálculado somo sendo a concentração correspondente: 

$N_{LD} = 3 x \sqrt{NB}$

$NB$ é o número de contagens medidas sob o pico do respectivo elemento e $N_{LD}$ 
é o número mínimo de contagens para se distiguir um pico, esse valor foi convertido para
$\mu g/cm^2$. 

Para se ter um ideia do limite de detecção em termos das concentrações atmosféricas 
, multiplicamos esse valor pela área de deposição e 
dividimos pelo volume típico amostrado $\mu g/m^3$. Assim, esse limite de detecção 
é aproximado. 

O LD muda conforme quantidade de material coletado, aqui fizemos um filtro supercarregado
e outro branco. O branco seria o menor LD possível para esse tipo de filtro Teflon. 
O LD muda se mudar o tipo de filtro: material e espessura , pois muda o NB. 

Lembrete, usa-se LD K do sódio até Mo. E L para frente do Mo.  
  

No caso das concentrações diárias, obtidas por EDX.  

Os valores faltantes foram preenchidos com o valor da metade do Limite de Detecção 
(LD/2, supondo que a concentração não detectada de um elemento que aparece 
frequentemente nas demais amostras esteja entre 0 e LD, com igual 
probabilidade de ocorrência para cada valor, LD/2 seria a média e, portanto, 
o valor mais provável entre estes valores "invisíveis"); 

para sua incerteza tomou-se 5LD/6 \citep{polissar1998}.

\begin{figure}[H]
  \caption{}
  \begin{subfigure}[b]{0.5\textwidth}
    \includegraphics[width=\textwidth]{../outputs/limitDetectionK.pdf}
    \caption{k}
  \end{subfigure}%
  \begin{subfigure}[b]{0.5\textwidth}
    \includegraphics[width=\textwidth]{../outputs/limitDetectionL.pdf}
    \caption{l}
  \end{subfigure}
\end{figure}

