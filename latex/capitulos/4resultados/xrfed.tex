%%%%
\section{Fluorescência de Raiox X}

%%%%
\subsection{Calibração}

\begin{table}[H]
 
  \begin{scriptsize} %small footnotesize scriptsize
  %\label{my-label} 
  \caption{}
  \input{../outputs/edxCalibrationnov2010K.tex}
  \end{scriptsize}
\end{table}

\begin{table}[H]
  
  \begin{scriptsize} %small
  %\label{my-label} 
  \caption{}
  \input{../outputs/edxCalibrationnov2010L.tex}
  \end{scriptsize}
\end{table}


\begin{table}[H]
  
  \begin{footnotesize} %small
  %\label{my-label} 
  \caption{}
  \input{../outputs/comparaCalibrationK.tex}
  \end{footnotesize}
\end{table}

\begin{table}[H]
  
  \begin{footnotesize} %small
  %\label{my-label} 
  \caption{}
  \input{../outputs/comparaCalibrationK.tex}
  \end{footnotesize}
\end{table}


\begin{figure}[H]
  \caption{Calibração da Fluorescência de Raiox X - Maio de 2010}
  \begin{subfigure}[b]{0.5\textwidth}
    \includegraphics[width=\textwidth]{../outputs/CalibrationK2010MaiAkerr.pdf}
    \caption{Linha K}
  \end{subfigure}%
  \begin{subfigure}[b]{0.5\textwidth}
    \includegraphics[width=\textwidth]{../outputs/CalibrationL2010MaiAkerr.pdf}
    \caption{Linha L}
  \end{subfigure}
\end{figure}

\begin{figure}[H]
  \caption{Calibração da Fluorescência de Raiox X - Novembro de 2010}
  \begin{subfigure}[b]{0.5\textwidth}
    \includegraphics[width=\textwidth]{../outputs/CalibrationK2010NovAkerr.pdf}
    \caption{Linha K}
  \end{subfigure}%
  \begin{subfigure}[b]{0.5\textwidth}
    \includegraphics[width=\textwidth]{../outputs/CalibrationL2010NovAkerr.pdf}
    \caption{Linha L}
  \end{subfigure}
\end{figure}

\begin{figure}[H]
  \caption{Calibração da Fluorescência de Raiox X - Abril de 2011}
  \begin{subfigure}[b]{0.5\textwidth}
    \includegraphics[width=\textwidth]{../outputs/CalibrationK2011AbrAkerr.pdf}
    \caption{Linha K}
  \end{subfigure}%
  \begin{subfigure}[b]{0.5\textwidth}
    \includegraphics[width=\textwidth]{../outputs/CalibrationL2011AbrAkerr.pdf}
    \caption{Linha L}
  \end{subfigure}
\end{figure}

O queda de desempenho no uso da fluorescência de Raios X pode ser observada 
na figura \ref{fig:compara_calibracao}. 
Possíveis motivos para queda de desempenho: desgaste do tido ou desgaste do detector. 

\begin{figure}[H]
  \caption{Calibrações de Fluorescência de Raiox X em 3 períodos \label{fig:compara_calibracao}}
  \begin{subfigure}[b]{0.5\textwidth}
    \includegraphics[width=\textwidth]{../outputs/CalibrationKcomparacao.pdf}
    \caption{Linha K}
  \end{subfigure}%
  \begin{subfigure}[b]{0.5\textwidth}
    \includegraphics[width=\textwidth]{../outputs/CalibrationLcomparacao.pdf}
    \caption{Linha L}
  \end{subfigure}
\end{figure}

%%%%
\subsection{Limite de Detecçção}

\begin{figure}[H]
  \caption{}
  \begin{subfigure}[b]{0.5\textwidth}
    \includegraphics[width=\textwidth]{../outputs/limitDetectionK.pdf}
    \caption{k}
  \end{subfigure}%
  \begin{subfigure}[b]{0.5\textwidth}
    \includegraphics[width=\textwidth]{../outputs/limitDetectionL.pdf}
    \caption{l}
  \end{subfigure}
\end{figure}

