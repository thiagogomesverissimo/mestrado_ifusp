%%%%
\section{Resultados gerais}

O projeto global envolvendo 11 pontos de amostragem em Acra, 
reuniu 2898 amostras, todas analisadas na refletância e \textbf{ED-XRF}
do \textbf{LAPAt}. 
Os resultados gerais foram apresentados em 2011 no
\textbf{23th Congress of the International Society for Environmental 
Epidemiology} \citep{zhou2011} e publicados em \citep{zhou2013} e \citep{zhou2014}. 

Essa pesquisa aprofundou as análises exclusivamente no bairro de Nima, 
com 791 amostras válidas entre 11 de Novembro de 2006 e 15 de Agosto de 2008.

\begin{table}[H]
  \centering
  \begin{scriptsize}
  \input{../outputs/samples}
  \end{scriptsize}
  \caption{Quantificação total das amostras analisadas por \textbf{ED-XRF} e refletância}
\end{table}

Os dois pontos de amostragem em Nima serão referidos como \textbf{residencial},
para medidas na rua só de residências e \textbf{avenida} para medidas na avenida
com presença de comércio e alto fluxo de veículos.

\begin{table}[H]
 \centering
  \begin{scriptsize}
    \input{../outputs/tabela_descritiva_com_harmatan.tex}
  \end{scriptsize} 
  \caption{Média, desvio padrão e mediana da massa total e ultrapassagens das 
           recomendações da Organização Mundial de Sáude (OMS) para média diária de 
           25 $\mu g/m^3$ para $MP_{2,5}$ e 50 $\mu g/m^3$ para $MP_{10}$
           \label{table:descritiva}}
\end{table}

Na tabela \ref{table:descritiva} estão as médias para a avenida e área residencial
bem como a porcentagem de ultrapassagem do padrão diário da 
Organização Mundial de Sáude (OMS).
O indíce de ultrapassagens foi alto para todos os casos, mas principalmente na avenida,
acima de (90\%) tanto para $MP_{10}$ quanto para $MP_{2,5}$.

Os padrões de qualidade do ar são fixados em níveis que quando ultrapassados 
podem afetar a saúde da população. 
Na tabela \ref{table:pm10standards} há uma comparação dos padrões de $MP_{10}$ 
no Brasil (CONAMA 03/90), Gana (EPA-Gana) e os recomentados pela Organização
Mundial de Sáude.
Outra recomendação da OMS é que a concentração não ultrapasse $25 \mu g/m^3$ 
em mais que 1\% das amostragens durante um ano. 

\begin{table}[H]
  \centering
  \begin{scriptsize}
    \input{../outputs/standard_brazil_ghana_OMS_pm10}
  \end{scriptsize}
  \caption{Padrões para $MP_{10}$ no Brasil (CONAMA 03/90), Gana (EPA-Gana) e 
          Organização Mundial de Sáude \label{table:pm10standards}}
\end{table}

A tabela \ref{table:RFcH_descriptive} traz as médias das medidas elementares 
realizadas por \textbf{ED-XRF} e de \textbf{Black Carbon} para $MP_{2,5}$ na área 
\textbf{residencial}.
O \textbf{Black Carbon} representa 3.7 \% da massa total.

\begin{table}[H]
  \centering
  \begin{scriptsize}
    \input{../outputs/descriptive_RFcH}
  \end{scriptsize}
  \caption{Tabela com estística descritiva para $MP_{2,5}$ na área \textbf{residencial}
           \label{table:RFcH_descriptive}}
\end{table}

%TODO: Fazer sem o harmatão e ver se BC aumenta.  
Já na avenida \ref{table:TFcH_descriptive} o \textbf{Black Carbon} 
representa 4.7 \% da massa total.

\begin{table}[H]
  \centering
  \begin{scriptsize}
    \input{../outputs/descriptive_TFcH}
  \end{scriptsize}
  \caption{Tabela com estística descritiva para $MP_{2,5}$ na \textbf{avenida}
          \label{table:TFcH_descriptive}}
\end{table}

No gráfico da figura \ref{fig:plot_RFcH_massa} percebe-se que no período 
do harmatão tanto o padrão nacional de Gana quanto a recomendação da OMS 
são ultrapassados.

\begin{figure}[H]
\begin{center}
  \includegraphics[width=0.6\textwidth]{../outputs/plot_RFcH_massa.pdf}
  \caption{Massa total $MP_{2,5}$ na área \textbf{residencial} \label{fig:plot_RFcH_massa}}
\end{center}
\end{figure}

%%%%
\subsection{Comparação dos resultados com dados da USEPA}

Entre as 2898 amostras analisadas por \textbf{ED-XRF}, existiam 92 que foram 
previamente analisadas pela \textbf{United States Environmental Protection 
Agency (USEPA)}, porém nós não sabíamos, pois o Prof. Dr. Majid Ezzati quis 
fazer um teste a cega das nossas medidas. 

Os elementos com concentrações muito acima do limite de detecção tiveram ótima
concordância, com pode ser observado no gráfico da figura \ref{fig:epa} 

\begin{figure}[H]
  \centering
    \includegraphics[width=0.3\textwidth]{../outputs/EPA_Si.pdf}
    \includegraphics[width=0.3\textwidth]{../outputs/EPA_Fe.pdf}
    \includegraphics[width=0.3\textwidth]{../outputs/EPA_P.pdf}
  \caption{Comparação das concentrações com análise da USEPA \label{fig:epa}.}
\end{figure}
