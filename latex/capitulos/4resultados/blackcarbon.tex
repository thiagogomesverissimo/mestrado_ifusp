%%%%
\section{Black Carbon}

O padrão de calibração usado até então no laboratório usando Monarca 21 cobre refletância de 30\% a 100\%. 
Os alvos de Ghana, por serem extremamente carregados, chegaram a quase 100\% o que nos impossibilitou de usar 
essa curva de calibração.  



Para tentar contornar e recalibrar o refletância foi produzido 27 filtros (cetesb) com camera de ressuspensão
usando ASTM -N762 Black Carbon, como amostrador dicotomico, com refletância de 3,1 até 95,9\% (mais um de 100\%).
A massa foi medida em balança microanalítica +/- 1ug.
Discordância entre os alvos padrões antigos e novos foram entre 2 e 5, mas chegou 
até 18 em alvos poucos carregados. Problemas com alvos padrões BC são cohecidos e 
neste caso, provavelmente associado com diferentes distriições de tamanho das partículas.

Experimento realizado em túneis (rodoanel - saturou - e janio quadros) em 2010 são paulo também colotou amostras com 100\%
em alvos de policarbonato (amostrador Partisol)  e quartzo em paralelo usando amostrador Minivol . 
O filtros de quartzo foram analisados usando thermal/optical transmittance (TOT) (método absoluto) o que
possibilitou uma calibração informal entre o TOT e os filtros de policarbonato na repletometro. 
Os filtros do rodoanel foram saturado e os do janio quados teve resultados de 4.9 to 57,1\% (mais o branco de 100\%)
o que possibilitou a comparação e o ajuste dos dois métodos.  

A intercalibração de TOT em refletância Janio quadros feita em 4/Maio/2011 até 13/Maio/2011.

quartzo TOT foi medido pelo Dr. Pierre Herckes
Department of Chemistry and Biochemistry in the Arizona State University


Em gana também foi feito amostragem em paralelo com 100 filtros de quartzo e teflon. 
Os dados mostram uma combinação linear entre BC obtido da refletência e TOT. 

impor o ponto 0 no ajuste.

Na refletância, considerou-se a incerteza como a desvio padrão da medidas de 10 alvos brancos de laboratórios, 
assim incorporamos a incerteza dos brancos e da variabilidade do equipamento  (0,1). 

A incerteza do ajuste foi feita usando ajuste dos minimos quadrados, considerando
variâncias e co-variâncias. 
Obrigar a linha a passar em zero, ou seja, em 100\% de reflêtancia há zero de massa. 

A dependência linear entre Black Carbon (ASTM -N762) do log da refletância indica
que a reflectance é um bom método para avaliar a massa no filtro. 
O problema é na refletância é que ela depende do tipo de BC e de filtro. 

Inter-calibrou-se a curva obtida pela refletância com um equipamento 
Sunset para determinação de carbono orgânico e elementar, 
por processo Térmico/Transmitância Óptica (EPA, 2012).

%incerteza na medida do black carbon: calculado com os brancos e tirar o sd. 
erro absoluto , a mesma para todos .

\begin{figure}[H]
\begin{center}
  \includegraphics[width=0.5\textwidth]{../outputs/TOTrefletanciaCalibration.pdf}
  \caption{}
\end{center}
\end{figure}
  
\begin{figure}[H]
\begin{center}
  \includegraphics[width=0.5\textwidth]{../outputs/CalibracaoRefletancia2007Lapat.pdf}
  \caption{Lapat }
\end{center}
\end{figure}

\begin{table}[H]
  \centering
  %\label{my-label} 
  \caption{RFsH}
  \input{../outputs/TOTcalibration.tex}
\end{table}

\begin{table}[H]
  \centering
  %\label{my-label} 
  \caption{RFsH}
  \input{../outputs/CalibracaoRefletancia2007.tex}
\end{table}




