\section{Análise de Fatores}

A tabela abaixo nos da o número de amostras seguimentada por
moda e região de amostragem. 
Também mostra o número de casos quando remove-se os dias de ocorrência do harmatão.

\begin{table}[H]
 \centering
  %\label{my-label}
  \input{../outputs/tabela_descritiva_com_harmatan.tex}
  \caption{Estatística descritiva incluindo-se os dias com harmatão}
\end{table}

\begin{table}[H]
  \centering
  %\label{my-label}
  \input{../outputs/tabela_descritiva_sem_harmatan.tex}
  \caption{Estatística descritiva excluíndo-se os dias com harmatão}
\end{table}

% RFsH,RGsH,RIsH,TFsH,TGsH,TIsH,RFcH,RGcH,RIcH,TFcH,TGcH,TIcH 

As fontes poluídoras para $MP_{2,5}$ na região residencial são representadas 
pelos seguintes elementos:

\input{../outputs/loadings_RFsH.tex}

As principais fontes encontradas utilizando-se
\textit{Análise de Fatores} foram solo, queima de biomassa, mar e veículo.
Porém, os elementos que representam cada fonte variam por região 
(residencial ou avenida) e moda (fino ou grosso). 

Ao incluirmos os dias de ocorrência do Harmatão a fonte solo
virá predominande por causa da poeira e fica praticamte impossível
detectar as outras fontes, inclusive a variância explicada pelas
outros fatores são baixas, assim só apresentaremos aqui os resultados 
removendo-se os dias do Harmatão. Mas a análise de Fatores do Harmatão
estão todas no Apêndice I.


\begin{table}[H]
  \centering
  %\label{my-label}
  \input{../outputs/briefFA_RFsH}
  \caption{RFsH}
\end{table}

Já o $MP_{2,5-10}$ na região residencial:

\begin{table}[H]
  \centering
  %\label{my-label}
  \input{../outputs/briefFA_RGsH}
  \caption{RGsH}
\end{table}

As principais fontes encontradas para $MP_{2,5}$ (fino) na via
com maior tráfego:

\begin{table}[H]
  \centering
  %\label{my-label}
  \input{../outputs/briefFA_TFsH}
  \caption{TFsH}
\end{table}

Já o $MP_{2,5-10}$ na região de tráfego:

\begin{table}[H]
  \centering
  %\label{my-label}
  \input{../outputs/briefFA_TGsH}
  \caption{TGsH}
\end{table}

Separação da borracha do cobre.

%%# Fontes encontradas em Harvard:
%%#   Solid waste burning: Br. 
%%#   Road dust & vehicle: Al, Si, Ca, Fe, Zn, BC.  
%%#   Crustal: Al, Si, Mg, Ti, Mn, Fe.
%%#   Aged biomass particles: K, Cl, S, BC
%%   Fresh biomass burning: K, Cl, S, BC
%%#   Sea salt: Na, Cl, S
%%# Sal marinho, solo, emissões veiculares e combustão de biomassa
%%# Zn/Cu podem representa veículos por causa do mencanismo interno do carro. 
%%# Zn: Freio, pneu, peças.
%
%%\begin{figure}[H]
%%\centering
%%\includegraphics[width=\textwidth]{../outputs/AFCH_factor_scores.pdf}
%%\caption{AFCH factor scores}
%%\label{fig:AFCH_factor_scores}
%%\end{figure}
%
%%Aumento do silicio no harmatan.
%
%%Zheng: 
%%Possíveis fontes: ressuspensão da ruas não pavimentadas
%%biomassa: K,Cl,S
%
%%massa,Al,Si,K,Ca,Fe
%%K e cl se relacionam
%%K e S no período não harmathan 
%%Na e Cl se descorrelacionam no harmathan
%%Br, Zn, Pb no não harmathan (gasolina-Br e Pb) Pneu:Zn
