\section{Análise de Fatores}

A tabela abaixo nos da a distribuição geral das amostras em moda e local de amostragem:

\begin{table}[H]
 \centering
 \caption{My caption}
 %\label{my-label}
 \input{../outputs/amostras_por_local.tex}
\end{table}

Depois da separação entre grosso e fino temos:
\begin{table}[H]
 \centering
 \caption{My caption}
 %\label{my-label}
 \input{../outputs/tabela_subset_com_harmata.tex}
\end{table}

Estatística descritiva incluindo-se os dias com harmatão:
\begin{table}[H]
 \centering
 \caption{My caption}
 %\label{my-label}
 \input{../outputs/tabela_descritiva_com_harmatan.tex}
\end{table}

Removendo-se os dias de ocorrência do harmatão da tabela acima, temos:
\begin{table}[H]
 \centering
 \caption{My caption}
 %\label{my-label}
 \input{../outputs/tabela_subset_sem_harmata.tex}
\end{table}

O $MP_{2,5}$ é o que causa maior problemas de saúde. Somando-se as amostras dois pontos de coleta em Nina, área residencial e tráfego de veículos, podemos fazermos uma estimativa das fontes usando-se \textit{Análise de Fatores} 

\input{../outputs/loadings_TFcH.tex}
\input{../outputs/loadings_TFsH.tex}

\input{../outputs/loadings_RFcH.tex}

%Resultados da Análise de Fatores para RIcH:
\input{../outputs/loadings_RIcH.tex}

%Resultados da Análise de Fatores para TIcH:
\input{../outputs/loadings_TIcH.tex}

%Resultados da Análise de Fatores para RGcH:
\input{../outputs/loadings_RGcH.tex}

%Resultados da Análise de Fatores para TGcH:
\input{../outputs/loadings_TGcH.tex}

%Resultados da Análise de Fatores para RFsH:
\input{../outputs/loadings_RFsH.tex}



%Resultados da Análise de Fatores para RIsH:
\input{../outputs/loadings_RIsH.tex}

%Resultados da Análise de Fatores para TIsH:
\input{../outputs/loadings_TIsH.tex}

%Resultados da Análise de Fatores para RGsH:
\input{../outputs/loadings_RGsH.tex}

%Resultados da Análise de Fatores para TGsH:
\input{../outputs/loadings_TGsH.tex}

%%# Fontes encontradas em Harvard:
%%#   Solid waste burning: Br. 
%%#   Road dust & vehicle: Al, Si, Ca, Fe, Zn, BC.  
%%#   Crustal: Al, Si, Mg, Ti, Mn, Fe.
%%#   Aged biomass particles: K, Cl, S, BC
%%   Fresh biomass burning: K, Cl, S, BC
%%#   Sea salt: Na, Cl, S
%%# Sal marinho, solo, emissões veiculares e combustão de biomassa
%%# Zn/Cu podem representa veículos por causa do mencanismo interno do carro. 
%%# Zn: Freio, pneu, peças.
%
%%\begin{figure}[H]
%%\centering
%%\includegraphics[width=\textwidth]{../outputs/AFCH_factor_scores.pdf}
%%\caption{AFCH factor scores}
%%\label{fig:AFCH_factor_scores}
%%\end{figure}
%
%%Aumento do silicio no harmatan.
%
%%Zheng: 
%%Possíveis fontes: ressuspensão da ruas não pavimentadas
%%biomassa: K,Cl,S
%
%%massa,Al,Si,K,Ca,Fe
%%K e cl se relacionam
%%K e S no período não harmathan 
%%Na e Cl se descorrelacionam no harmathan
%%Br, Zn, Pb no não harmathan (gasolina-Br e Pb) Pneu:Zn
