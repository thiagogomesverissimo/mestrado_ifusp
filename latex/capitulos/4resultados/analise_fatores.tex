%%%%
\section{Análise de Fatores}

Ao incluirmos os dias de ocorrência do Harmatão, apenas 1 fator
predomina na análise, ficando os outros fatores com autovalores
menores do que 1 e degenerados. Neste cenário, a fonte 
\textbf{poeira de solo} predomina e fica praticamte impossível
detectar as outras fontes. Assim, só apresentaremos aqui os resultados 
removendo-se os dias do Harmatão. Mas a Análise de Fatores com os
dias do Harmatão inclusos estão no estão no Apêndice I.

Diversas parametrizações foram testadas na \textit{Análise de Fatores}
e aquelas que resultaram em fatores que se associavam com fontes poluídoras
reais foram escolhidas. A seguir estão os resultados para 
\textit{Análise de Fatores} nos dois pontos de amostragem 
(residencial e avenida) e nas duas modas
fina ($MP_{2,5}$) e grossa ($MP_{2,5-10}$).

% RFsH,RGsH,RIsH,TFsH,TGsH,TIsH,RFcH,RGcH,RIcH,TFcH,TGcH,TIcH 

%%%%

\subsection{$MP_{2,5}$ na região residencial e avenida}

Para \textit{$MP_{2,5}$ na região residencial} o resultado da 
\textit{Análise de Fatores}  explicou $84,0\%$ da variância total 
dos dados com 4 fatores retidos.
A comunalidade, indicação do quão o ajuste explicou a variabilidade por 
cada espécie, foi maior que $0,7$ para a maioria dos elementos,
com exceção do bromo (Br), zinco (Zn) e chumbo (Pb) que tiveram comunalidade 
de $0,43$, $0,49$ e $0,62$, respectivamente.

\begin{table}[H]
  %\label{my-label} 
  \caption{\textbf{Análise de Fatores com rotação varimax - 4 fatores retidos} 
             para $MP_{2,5}$ na região residencial.
           (\textcolor{red}{h} : Comunalidade; 
           \textcolor{red}{S=1-h} : Singularidade; 
           \textcolor{red}{C} : Complexidade.)}
  \input{../outputs/loadings_RFsH4.tex}
\end{table}

O fator predominante, isto é, que explica grande parte da variância, 
tem autovalor $9,57$ e explica $53,0\%$ da variância total.
Os elementos com maiores \textit{loadings} neste fator são essecialmente 
metais (Al, Si, Ti, V, Fe, Mn, Ca, Mg) e massa (\textit{mass}), e proveem 
principalmente de ressuspensão de solo.
Neste fator, aparecem ainda fósforo (P), potássio (K), cloro (Cl), indicando 
possível sobreposição de fontes, ou mesmo solo contaminado.
%TODO: citar speciate

O segundo fator predominante, de autovalor $1,91$, agrega fósforo (P), 
potássio (K) e enxofre (S), somente. Eles caracteriza a queima de biomassa, 
comumente usada em \textit{Nima}.
%TODO: citar artigo sobre elementos de queima de biomassa.

O terceiro fator predominante, de autovalor $1,88$, contém elementos
característicos de mar, sódio (Na), cloro (Cl), bromo (Br) e enxofre (S), 
sendo o sódio (Na) de maior loading $0,81$. 
Depois de um tempo de residência na atmosfera, o Cl do sal marinho envelhecido 
é substituído por $SO_4^{2-}$ como resultado da reação com acido sulfurico e 
ácido nitrico \citep{mcinnes1994}. 

%TODO: citar artigo sobre elementos MAR

O quarto e último fator retido, de autovalor $1,78$, é composto 
por black carbon (BC), chumbo (P) e zinco (Zn). 
Acra tem um dos maiores lixões de eletrônicos do mundo, recebendo
grande parte dos eletrônicos descartados como lixo na Europa e que são derretidos
para obtenção do cobre (Cu). 

O lixão está localizado em \textit{Agbogbloshie}, 4 kilômetros a sudoeste de Nima. 
O vento predominante em Nima é de sudoeste, o que nos indica grandes chances 
desse fator representar uma fonte de queima de lixo eletrônico.
%TODO: citar artigo Agbogbloshie

\citep{asante2012} mediu níveis de:
Al, Co, Cu, Zn, Cd, In, Sb, Ba, and Pb no solo do e-waste.

Além disso, black carbon (BC) e zinco (Zn) também estão associados a veículos, 
pois o black carbon (BC) é resultado de combustão incompleta e o zinco (Zn) é
liberado no desgate das pastilhas de freio e dos pneus dos automóveis. 
%TODO: citar artigo sobre poluentes de veículos

Entretanto o chumbo (Pb) não pode estar associado a veículo, pois 
foi banido da gasolina em Gana em 2003 devido a acordo internacional. 
Mas nota-se que o loading de $0,30$ do chumbo (Pb) no primeiro fator revela 
que ainda pode haver contaminação do solo por chumbo (Pb).
%TODO: procurar artigo sobre isso. https://en.wikipedia.org/wiki/Ghana_Environmental_Protection_Agency

Comparando com região metropolitana São Paulo, em uma medida de 2006, a média de Chumbo foi $11 +/- 0.7 ng/m^3$  
(64 amostras) e frota veicular 7 milhões. 
%TODO: citar artigo da simone.

O zinco (Zn) é o principal composto de baterias de eletrônicos, assim também 
pode estar associado ao \textit{e-waste} de \textit{Agbogbloshie}.

\begin{table}[H]
  %\label{my-label} 
  \caption{\textbf{Análise de Fatores com rotação varimax - 4 fatores retidos} 
             para $MP_{2,5}$ na avenida.
           (\textcolor{red}{h} : Comunalidade; 
           \textcolor{red}{S=1-h} : Singularidade; 
           \textcolor{red}{C} : Complexidade.)}
  \input{../outputs/loadings_TFsH4.tex}
\end{table}

Ainda para $MP_{2,5}$, mas na avenida com intenso movimento de veículos,
a Análise de Fatores tamém explicou $84,0\%$ da variância total 
dos dados com 4 fatores retidos.

O fator predominante, de autovalor $9,87$, ressuspensão de solo, continua 
como os mesmos elementos (Al, Si, Ti, V, Fe, Mn, Ca, Mg, P, K, Cl, massa).

O segundo fator predominante, de autovalor $1,87$, além dos elementos que 
apareceram no barro residencial - fósforo (P), potássio (K) e enxofre (S) -
também contém black carbon (BC) com alto loading $(0,72)$. 
Esse fator continua caracterizando queima de biomassa.

O terceiro fator predominante, autovalor $1,75$, composto por
potássio (K), enxofre (S), bromo (Br), chumbo (P) e zinco (Zn). 
Esse fator está associado a emissões veiculares. 

O quarto e último fator predominante, autovalor $1,65$, 
Nódio (Na), cloro (Cl) e bromo (Br) e enxofre (S).



A associação entre fatores e fontes poluídoras pode ser resumida
assim: \textit{solo, queima de biomassa, mar e lixo.}

\begin{table}[H]
  \centering
  \caption{Associação de fonte de poluídoras na \textit{Análise de Fatores}
         para $MP_{2,5}$ na região residencial}
  \input{../outputs/briefFA_RFsH4.tex}
\end{table}

%%%%
\subsection{$MP_{2,5-10}$ na região residencial e avenida}

\begin{table}[H]
  %\label{my-label} 
  \caption{\textbf{Análise de Fatores com rotação varimax - 4 fatores retidos} 
             para $MP_{2,5}$ na avenida.
           (\textcolor{red}{h} : Comunalidade; 
           \textcolor{red}{S=1-h} : Singularidade; 
           \textcolor{red}{C} : Complexidade.)}
  \input{../outputs/loadings_TFsH4.tex}
\end{table}

\begin{table}[H]
  %\label{my-label} 
  \caption{\textbf{Análise de Fatores com rotação varimax - 4 fatores retidos} 
             para $MP_{2,5}$ na avenida.
           (\textcolor{red}{h} : Comunalidade; 
           \textcolor{red}{S=1-h} : Singularidade; 
           \textcolor{red}{C} : Complexidade.)}
  \input{../outputs/loadings_TFsH4.tex}
\end{table}



%%# Fontes encontradas em Harvard:
%%#   Solid waste burning: Br. 
%%#   Road dust & vehicle: Al, Si, Ca, Fe, Zn, BC.  
%%#   Crustal: Al, Si, Mg, Ti, Mn, Fe.
%%#   Aged biomass particles: K, Cl, S, BC
%%   Fresh biomass burning: K, Cl, S, BC
%%#   Sea salt: Na, Cl, S
%%# Sal marinho, solo, emissões veiculares e combustão de biomassa
%%# Zn/Cu podem representa veículos por causa do mencanismo interno do carro. 
%%# Zn: Freio, pneu, peças.
%
%%\begin{figure}[H]
%%\centering
%%\includegraphics[width=\textwidth]{../outputs/AFCH_factor_scores.pdf}
%%\caption{AFCH factor scores}
%%\label{fig:AFCH_factor_scores}
%%\end{figure}
%
%%Aumento do silicio no harmatan.
%
%%Zheng: 
%%Possíveis fontes: ressuspensão da ruas não pavimentadas
%%biomassa: K,Cl,S
%
%%massa,Al,Si,K,Ca,Fe
%%K e cl se relacionam
%%K e S no período não harmathan 
%%Na e Cl se descorrelacionam no harmathan
%%Br, Zn, Pb no não harmathan (gasolina-Br e Pb) Pneu:Zn
