\chapter{Análise multivariada dos dados}

\section{Visão Geral}
Vamos divivir a análise de fatores nos seguintes cenários:

\input{../outputs/amostras_por_local.tex}
\input{../outputs/tabela_subset_com_harmata.tex}
\input{../outputs/tabela_subset_sem_harmata.tex}
\input{../outputs/tabela_descritiva_com_harmatan.tex}
\input{../outputs/loadings_RFsH.tex}

\subsection{AFCH: Residencial $MP_{2.5}$ \textbf{com} Harmathan}



%# Fontes encontradas em Harvard:
%#   Solid waste burning: Br. 
%#   Road dust & vehicle: Al, Si, Ca, Fe, Zn, BC.  
%#   Crustal: Al, Si, Mg, Ti, Mn, Fe.
%#   Aged biomass particles: K, Cl, S, BC
%   Fresh biomass burning: K, Cl, S, BC
%#   Sea salt: Na, Cl, S
%# Sal marinho, solo, emissões veiculares e combustão de biomassa
%# Zn/Cu podem representa veículos por causa do mencanismo interno do carro. 
%# Zn: Freio, pneu, peças.



%\begin{figure}[H]
%\centering
%\includegraphics[width=\textwidth]{../outputs/AFCH_factor_scores.pdf}
%\caption{AFCH factor scores}
%\label{fig:AFCH_factor_scores}
%\end{figure}

Aumento do silicio no harmatan.

Zheng: 
Possíveis fontes: ressuspensão da ruas não pavimentadas
biomassa: K,Cl,S

massa,Al,Si,K,Ca,Fe
K e cl se relacionam
K e S no período não harmathan 
Na e Cl se descorrelacionam no harmathan
Br, Zn, Pb no não harmathan (gasolina-Br e Pb) Pneu:Zn



