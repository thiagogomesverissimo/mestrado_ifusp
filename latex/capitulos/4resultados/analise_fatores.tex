%%%%
\section{Análise de Fatores}

A tabela abaixo nos da o número de amostras seguimentada por
moda e região de amostragem. 
Também mostra o número de casos quando remove-se os dias de ocorrência do harmatão.

\begin{table}[H]
 \centering
  %\label{my-label}
  \input{../outputs/tabela_descritiva_com_harmatan.tex}
  \caption{Estatística descritiva incluindo-se os dias com harmatão}
\end{table}

\begin{table}[H]
  \centering
  %\label{my-label}
  \input{../outputs/tabela_descritiva_sem_harmatan.tex}
  \caption{Estatística descritiva excluíndo-se os dias com harmatão}
\end{table}

Ao incluirmos os dias de ocorrência do Harmatão, apenas 1 fator
predomina na análise, ficando os outros fatores com autovalores
menores do que 1 e degenerados. Neste cenário, a fonte 
\textbf{poeira de solo} predomina e fica praticamte impossível
detectar as outras fontes. Assim, só apresentaremos aqui os resultados 
removendo-se os dias do Harmatão. Mas a Análise de Fatores com os
dias do Harmatão inclusos estão no estão no Apêndice I.

Diversas parametrizações foram testadas na \textit{Análise de Fatores}
e aquelas que resultaram em fatores que se associavam com fontes poluídoras
reais foram escolhidas. A seguir estão os resultados para 
\textit{Análise de Fatores} nos dois pontos de amostragem 
(residencial e avenida) e nas duas modas
fina ($MP_{2,5}$) e grossa ($MP_{2,5-10}$).

% RFsH,RGsH,RIsH,TFsH,TGsH,TIsH,RFcH,RGcH,RIcH,TFcH,TGcH,TIcH 

%%%%
\subsection{$MP_{2,5}$ na região residencial}

Na \textit{$MP_{2,5}$ na região residencial} o resultado da 
\textit{Análise de Fatores}  explicou $83,0\%$ da variância total 
dos dados com 4 fatores retidos.
A comunalidade, valor que nos indica o quão o ajuste explicou
a variabilidade dos elementos, foi maior que 0,7 para a maioria dos elementos,
com exceção do chumbo (Pb), bromo (Br) e zinco (Zn), que tiveram comunalidade 
de 0,62, 0,50 e 0,42, respectivamente.

\begin{table}[H]
  %\label{my-label} 
  \caption{\textbf{Análise de Fatores com rotação varimax - 4 fatores retidos} 
             para $MP_{2,5}$ na região residencial.
           (\textcolor{red}{h} : Comunalidade; 
           \textcolor{red}{S=1-h} : Singularidade; 
           \textcolor{red}{C} : Complexidade.)}
  \input{../outputs/loadings_RFsH4.tex}
\end{table}

O fator predominante, isto é, que explica grande parte da variância, 
tem autovalor $9,09$.
Os elementos com maiores \textit{loadings} neste fator são essecialmente 
metais (Al, Si, Ti, V, Fe, Mn, Ca, Mg) e provém principalmente de ressuspensão de solo.
fósforo (P), potássio (K), cloro (Cl) possuem \textit{loadings} 
altos neste fator, indicando sobreposição de fontes, ou mesmo solo contaminado.
Por fim, a massa (\textit{mass}) também está no primeiro fator.
%TODO: citar speciate

O segundo fator predominante, de autovalor 2,04, agrega fósforo (P), 
potássio (K) e enxofre (S), somente. Eles caracteriza a queima de biomassa, 
comumente usada em \textit{Nima} 
%TODO: citar artigo sobre elementos de queima de biomassa.

O terceiro fator predominante, de autovalor 2,02, contém elementos
característicos de mar, sódio (Na), cloro (Cl) e bromo (Br) com 
contaminação de enxofre (S). 
%TODO: citar artigo sobre elementos MAR

O quarto e último fator retido, de autovalor 1,83, é composto 
por black carbon (BC), chumbo (P) e zinco (Zn). 
Acra tem um dos maiores lixões de eletrônicos do mundo, recebendo
grande parte dos eletrônicos descartados como lixo na Europa e que são derretidos
para obtenção do cobre (Cu). O lixão está localizado em \textit{Agbogbloshie}, 
4 kilômetros a sudoeste de Nima. O vento predominante em Nima é de sudoeste, 
o que nos indica grandes chances desse fator representar uma fonte de queima de 
lixo eletrônico.
%TODO: citar artigo Agbogbloshie

Além disso, black carbon (BC) e zinco (Zn) também estão associados a veículos, 
pois o black carbon (BC) é resultado de combustão incompleta e o zinco (Zn) é
liberado no desgate das pastilhas de freio e dos pneus dos automóveis. 
%TODO: citar artigo sobre poluentes de veículos

Entretanto o chumbo (Pb) não pode estar associado a veículo, pois 
foi banido da gasolina em Gana em 2003 devido a acordo internacional.
%TODO: procurar artigo sobre isso. https://en.wikipedia.org/wiki/Ghana_Environmental_Protection_Agency

O zinco (Zn) é o principal composto de baterias de eletrônicos, assim também 
pode estar associado ao \textit{e-waste} de \textit{Agbogbloshie}.

No intuito de melhor investigar fontes que se associam a esses fatores e aumentar
a comunalidade do zinco (Zn) o ajuste foi refeito para 5 fatores.


\begin{table}[H]
  %\label{my-label} 
  \caption{\textbf{Análise de Fatores com rotação varimax - 5 fatores retidos} 
            para $MP_{2,5}$ na região residencial.
           (\textcolor{red}{h} : Comunalidade; 
           \textcolor{red}{S=1-h} : Singularidade; 
           \textcolor{red}{C} : Complexidade.)}
  \input{../outputs/loadings_RFsH5.tex}
\end{table}

Acrescentando-se um quinto fator o zinco (Zn) aparece como elemento predominate 
e quase único do novo fator. Há um peso (\textit{loading}) não desprezível de 
black carbon (BC) $(0.19)$. 

A comunalidade do chumbo (Pb), bromo (Br) e zinco (Zn) aumentaram e o zinco (Zn)
que estava mal explicado no ajuste de 4 fatores (0,42) foi muito bem explicado
com o novo fator (0,93), indicando que black carbon (BC) e chumbo (Pb) representam 
uma fonte e zinco (Zn) e um pouco black carbon (BC) foram outra fonte.  

Portanto, a associação entre fatores e fontes poluídoras pode ser resumida
assim: \textit{solo, queima de biomassa, mar e lixo.}

\begin{table}[H]
  \centering
  \caption{Associação de fonte de poluídoras na \textit{Análise de Fatores}
         para $MP_{2,5}$ na região residencial}
  \input{../outputs/briefFA_RFsH5.tex}
\end{table}

%%%%
\subsection{$MP_{2,5-10}$ na região residencial}


%%# Fontes encontradas em Harvard:
%%#   Solid waste burning: Br. 
%%#   Road dust & vehicle: Al, Si, Ca, Fe, Zn, BC.  
%%#   Crustal: Al, Si, Mg, Ti, Mn, Fe.
%%#   Aged biomass particles: K, Cl, S, BC
%%   Fresh biomass burning: K, Cl, S, BC
%%#   Sea salt: Na, Cl, S
%%# Sal marinho, solo, emissões veiculares e combustão de biomassa
%%# Zn/Cu podem representa veículos por causa do mencanismo interno do carro. 
%%# Zn: Freio, pneu, peças.
%
%%\begin{figure}[H]
%%\centering
%%\includegraphics[width=\textwidth]{../outputs/AFCH_factor_scores.pdf}
%%\caption{AFCH factor scores}
%%\label{fig:AFCH_factor_scores}
%%\end{figure}
%
%%Aumento do silicio no harmatan.
%
%%Zheng: 
%%Possíveis fontes: ressuspensão da ruas não pavimentadas
%%biomassa: K,Cl,S
%
%%massa,Al,Si,K,Ca,Fe
%%K e cl se relacionam
%%K e S no período não harmathan 
%%Na e Cl se descorrelacionam no harmathan
%%Br, Zn, Pb no não harmathan (gasolina-Br e Pb) Pneu:Zn
