%%%%
\section{Positive Matrix Factorization}

\begin{figure}[H]
  \centering
  \includegraphics[width=1\textwidth]{../outputs/windRose_horaria.pdf}
  \caption{ \citep{carslaw2012} \label{fig:windRose_horaria}}
\end{figure}

\begin{figure}[H]
  \centering
  \includegraphics[width=1\textwidth]{../outputs/windRose_mensal.pdf}
  \caption{ \citep{carslaw2012} \label{fig:windRose_mensal}}
\end{figure}

Nota-se também uma perceptível diferença entre os dois períodos climáticos, pois no Verão temos maior quantidade de radiação solar, fortalecendo a formação de brisa marinha e, consequentemente, deslocando mais para o leste a distribuição de frequência nas direções dos ventos.


\begin{figure}[H]
  \centering
  \includegraphics[width=0.5\textwidth]{../outputs/windRoseNoaaHarvard.pdf}
  \caption{Rosa do ventos entre
           Setembro de 2006 e Junho de 2008. Utilisou-se dados 
           do \textbf{Kotoka International Airport} de Acra \label{fg:rosaCompleta}}
\end{figure}

Fonte de alimentos em Gana: \ref{table:cookfuel}
\begin{table}[H]
 \centering
  \input{../outputs/census_cookfuel.tex}
  \caption{Fontes de energia usadas para cozimento de alimentos em 
           Gana \citep{ghanacensus2013} \label{table:cookfuel}}
\end{table}
No caso de Accra a discriminação de fontes é complexa pois
há uma mistura de elementos do deserto com fontes antropogênicas.

Apesar dos altos índices de poluição em Accra pouca pesquisa tem sido 
realizada na tentativa de enteder a composição química do ar na região
(tanto antropogênica quando natural - poeira mineral do deserto). 
% Aço: 

\subsection{$MP_{2,5}$ na região residencial e avenida}

Os perfis dos fatores para $MP_{2,5-10}$ nos dois sítios de amostragem, 
residencial e avenida, estão na tabela \ref{table:grosso_profiles_percent_species}.

A contribuição por fator e a respectiva associação com fontes poluídoras
estão no gráfico da figura \ref{figure:grosso_pmf_contribution_pizza}. 

\citep{asante2012} analisou nível de elementos traços nas urinas de trabalhadores 
do e-waste de Agbogbloshie e Fe, Sb, and Pb tiveram concentrações alta se comparada
a um pessoa de referência, que não tem contato com \textbf{e-waste}.

\begin{figure}[H]
\begin{center}
  \includegraphics[width=0.5\textwidth]{../outputs/RGsH_pmf_contribution_pizza4.pdf}
  \caption{PMF - 4 fatores retidos para $MP_{2,5-10}$ na região residencial}
\end{center}
\end{figure}



\citep{kaku2016}


\citep{prospero2002} mostra países afetados por haramtão. 

\citep{engelbrecht2009a} e \citep{engelbrecht2009b} estudaram a composição 
da poeira do deserto.


\begin{table}[H]
  \centering
  %\label{my-label} 
  \caption{RFsH}
  \input{../outputs/RFsH_profiles_percent_species4.tex}
\end{table}

\begin{table}[H]
  \centering
  %\label{my-label} 
  \caption{RGsH}
  \input{../outputs/RGsH_profiles_percent_species4.tex}
\end{table}

\begin{table}[H]
  \centering
  %\label{my-label} 
  \caption{TGsH}
  \input{../outputs/TGsH_profiles_percent_species4.tex}
\end{table}


\begin{table}[H]
  \centering
  %\label{my-label} 
  \caption{TFsH}
  \input{../outputs/TFsH_profiles_percent_species4.tex}
\end{table}

