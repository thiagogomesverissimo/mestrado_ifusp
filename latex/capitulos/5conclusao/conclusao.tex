\chapter{Conclusão}

Em geral, como podemos verificar na Figura-8 e figura-4, há uma  predominância dos ventos sudeste, que somada à pouca precipitação, pode arrastar para o amostrador os poluentes provenientes do porto de Recife e outras fontes presentes na área urbana tais como o centro urbano, e a  circulação expressiva de veículos na cidade. Além disso, ainda na região sudeste, temos duas grandes rodovias relativamente próximas, a BR-101 e a BR-232, que devido intenso fluxo de veículos suspendem partículas do solo e liberam poluentes provenientes da combustão de seus motores que podem ser dispersos e transportados pelos ventos ao amostrador (figuras 2 e 3).


possibilidades de transporte, para o ponto de amostragem, dos resíduos de possíveis fontes emissoras ou geradoras  de MP2,5 .

A maior incidência de ventos deu-se no quadrante leste sul, sendo a direção principal (mais frequente Fig.4) a ESE, ou seja, os ventos principais na região de Recife vieram de ESE no período do experimento.

Neste quadrante localizam-se o lixão

Existe uma grande possibilidade do material particulado fino (MP2,5) amostrado ser proveniente tanto dos veículos de forma direta, da combustão dos motores, quanto de forma indireta no levantamento de partículas do solo, ou da conversão gás-partículas. Ao mesmo tempo, o amostrador também acaba recebendo a carga de outras fontes presentes na zona urbana

Nossos resultados demostram a necessidade de políticas públicas para o 
melhoramento ao acesso a fontes de combustíveis limpas tanto em área rural 
quanto urbana, bem como o controle do trafego de emissões veiculares.

Em combinação com dados meteorológicos e o mapeamento de algumas fontes emissoras locais, isso possibilitou avaliar as principais fontes locais de PM2,5, bem como estimar seus impactos nas amostragens, particularmente naquelas mais carregadas e ponderar e confrontar resultados oferecidos por essas diferentes metodologias.

Ultrapassagem das referências de qualidade do ar propostas pela OMS no período de amostragem. 

Os resultados obtidos neste trabalho, representam um registo histórico valioso, 
e deve ser muito importante no contexto do projeto principal, que irá correlacionar os dados de saúde e níveis de MP2,5, bem como identificar a correspondente responsabilidade das fontes emissoras locais.

Para realizar a associação entre o perfil dos fatores obtidos, com fontes locais, usou-se o conhecimento disponível sobre a composição de emissão de fontes conhecidas, bem como  informações sobre sua localização e o possível transporte para o amostrador, com base em dados de de ventos e outros parâmetros meteorológicos.

Os fatores obtidos apresentam uma certa mesclagem de fontes, descrita particularmente para o PMF,

A tabela 2 mostra as 6 fontes extraídas por PMF para a base de dados de concentrações obtidas para o Recife, associadas às seguintes fontes regionais: ...
Usando-se o método de PMF, foi possível distinguir 6 grupos de fontes atuando na região do... 

% pois oceano e área urbana (além das BR101 e 202 próximas da estação amostradora) são arrastados conjuntamente pelos ventos de leste e sudeste, frequentes na região (Figura 3)

% No período chuvoso a fonte solo teve impacto reduzido, distribuindo-se os efeitos entre as demais fontes, sendo que em todas as concentrações

Em função das amostragens de 48h e da dinâmica climática e de circulação locais, 
ocorreu uma certa mixagem de fontes em alguns fatores. 

Uma vez que o PFA pondera sua estimativa pelas incertezas das concentrações, 
isso interfere na determinação da contribuição das fontes.

piora das condições de dispersão dos poluentes. 

Estes resultados representam um caracterização importante do aerossol atmosférico na região do Recife, bem como de sua dinâmica associada ao clima.

percebe-se que aproximadamente 50\% das correntes de ventos possuem velocidades que variam de 1 a 3 m/s e 51% das mesmas são provenientes do sudeste. Em consequência, é possível ter havido o arraste de  poluentes do aterro de Jabotão até a base de amostragem.
Assim, também, o amostrador deve ter recebido as emissões advindas de ampla área da região urbana. 
