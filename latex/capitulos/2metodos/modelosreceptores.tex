%%%%
\section{Modelos Receptores}

Modelo receptor é uma abordagem matemática para determinar e 
quantificar o efeito das fontes poluidoras do ar em amostras.
Análise de Fatores (AF) e Positive Matrix Fatorization (PMF) são técnicas 
matemáticas de análise multivariada usadas em modelos receptores e 
permitem a identificação de fontes quando os seus perfis são desconhecidos.

Outro modelo receptor é Balanço Químico de Massa (CMB), usado quando se conhece 
o perfil das fontes locais e permite verificar o impacto das fontes 
nas concetrações medidas no amostrador. O CMB não foi usado nesta pesquisa, 
por falta de perfis de fontes adequados para a região.

A análise multivariada é uma técnica estatística utilizada para 
reduzir a dimensão de dados matriciais e permite manter a quantidade de 
informação contida inicialmente nos dados, armazenada na variância e covariância. 
Dâ-se o nome de fator às novas variáveis reduzidas, que são ortogonais entre si,  
sendo o fator uma variável latente, pois não é uma grandeza diretamente
medida, mas sim obtida a partir de outras variáveis observáveis. 

As dimensões (variáveis) reduzidas de um conjunto de dados analíticos 
complexos podem ser interpretados como fonte(s) poluídora(s) como detalhado por
\citet{wang2012} e \citet{mansha2012}, em que o número de fatores extraídos 
dependerá dos limites de detecção dos equipamentos usados, número de amostras, 
resolução da amostragem, espécies medidas, entre outros, sendo o conhecimento 
do pesquisador, juntamente com informações das potenciais fontes poluidoras 
próximas ao ponto receptor, dados meteorológicos do período de coleta, 
inventário de emissões e outras informações imprescindíveis para fazer a
associação fator-fonte(s). 

Assim, o principal desafio em análises multivariada 
é identificar se o fator de fato existe e tem significado físico 
enquanto fonte(s) poluídora(s) ou não, sendo que número reduzido de amostras, 
baixa resolução da amostragem e \textbf{outliers} 
(eventos infrequentes de curta duração e com alta concentração) representam 
um desafio para modelos multivariados.

%%%%
\subsection{Conservação da massa}

O princípio de conservação da massa é usado no modelo receptor, pois os 
poluentes emitidos pelas fontes em uma bacia aérea devem chegar no 
receptor. Aplicando-se a equação de conservação da massa para uma amostragem 
em que se coletou $i$ amostras e se mediu $j$ poluentes, tem-se:

\begin{equation}
  \label{eq:conservacaomassa}
  x_{ij} = \sum_{p=1}^{P} g_{ip}f_{pj} %+ \epsilon_{ij}
\end{equation} 

sendo,
\begin{itemize}
  \begin{spacing}{1.0}
    \item $x_{ij}$ = concentração na amostra receptora $i$ da espécie $j$;
    \item $f_{pj}$ = fração da espécie $j$ emitida pela fonte $p$;
    \item $g_{ip}$ = contribuição da fonte $p$ para amostra $i$;
    %\item $\epsilon_{ij}$ = erro do modelo empregado.
  \end{spacing}
\end{itemize}

%%%%
\subsection{Análise de Fatores}

A técnica Análise de Fatores (AF) consiste na redução da dimensionalidade 
dos dados, para uma nova base contendo a menor dimensão possível, 
relativamente à dimensão dos dados originais, baseando-se para tal na 
decomposição da matriz de correlação ou covariância em autovalores e autovetores. 
A redução é feita de forma que as novas variáveis 
(fatores) sejam combinação linear das anteriores e que representem a máxima 
fração possível da variância contida naquelas variáveis.

Popularizada após a publicação do trabalho de Lourenz em 1956, que chamou 
a técnica de Análise de Funções Ortogonais Empíricas, 55 anos após a 
descoberta da mesma por Karl Pearson \citep{bartholomew2011}.

Dadas as variáveis observáveis ($x_1,\dots,x_j$) com 
respectivas médias $\mu_1,\dots,\mu_j$ a equação da AF pode ser escritas como: 
 
\begin{equation}
  \label{eq:af}
  x_i-\mu_i = l_{i1} F_1 + \cdots + l_{ip} F_p + \varepsilon_i 
\end{equation}

Sendo F os fatores extraídos e $l_{ip}$ o peso ou \textit{loading} 
da variável $i$ no fator $p$.

Deve-se evitar amostras medidas em paralelo na mesma rodada de AF devido a 
multicolinearidade e consequente piora ou não solução do ajuste.  
A comunalidade é igual a soma dos \textit{loading} ao quadradro 
para cada variável e indica como ela foi explicada pelo ajuste, sendo que o zero
indica que a variável não foi nada explicada pelos fatores extraídos 
e um que foi completamente explicada. 
O quanto da variabilidade global dos dados iniciais foi explica pelo modelo está 
contida na variância total explicada, sendo que os ajustes com valores 
acima de 80\% são considerados bons. 
Reteu-se ainda fatores com autovalores maiores que um e não degenerados 
(o gráfico \textit{scree} facilita a investigação para identificação de fatores 
degenerados, pois exibe os autovalores associados aos fatores). 

Para melhor interpretação dos resultados da AF, aplicou-se a rotação do tipo 
Varimax, que projeta os fatores em um novo sistema de eixos maximizando 
as variâncias \citep{kaiser1958}.

%%%%
\subsection{Positive Matrix Factorizarion}

Positive Matrix Factorizarion (PMF) é outro método multivariado usado
em modelos receptores e permite determinar um conjunto de fatores $p$, 
o perfil $f$ das espécies de cada fonte e a contribuição da massa $g$ 
de cada fator para cada amostra individual \citep{norris2014}. 

O PMF resolve a equação de conservação de massa usando mínimos 
quadrados, diferentemente da AF, que reduz a dimensão dos 
dados decompondo a matriz de correlação em autovalores e autovetores e 
tentará encontrar solução matemática para qualquer número de fatores, tendo
eles significado físico ou não. 

Supondo uma campanha da amostragem com $i$ amostras válidas e 
$j$ espécies medidas, a matriz das concentrações $c_{ij}$ 
pode ser decomposta em $g_{ik}$ e $f_{kj}$ conforme \ref{eq:pmf}. 
O PMF procura um par das matrizes $g_{ik}$ e $f_{kj}$ que
minimizam a função objeto $Q$ dada pela equação \ref{eq:pmfobject}. 

\begin{equation}
  c_{ij} = \sum_{k=i}^p g_{ik}f_{kj} + e_{ij}
  \label{eq:pmf}
\end{equation}

e

\begin{equation}
  Q = \sum_{i=1}^n \sum_{j=1}^m  \left[ \frac{e_{ij}} {u_{ij}} \right] ^2
  \label{eq:pmfobject}
\end{equation}

Onde,
\begin{itemize}
  \begin{spacing}{1.0}
	\item $c_{ij}$: Matriz de concentração;
	\item $u_{ij}$: Matriz de incertezas (experimentais e analíticas);
	\item $p$: Número de fatores informado pelo usuário;
	\item $g_{ik}$: Contribuição dos fatores nas amostras (\textit{Factor Score};
	\item $f_{kj}$: Perfil da fonte ou assinatura da fonte (\textit{Factor Loadings});
	\item $e_{ij}$: Matriz dos resíduos escalados pelas incertezas;
  \end{spacing}
\end{itemize}

O método PMF tem sido muito utilizado em pesquisas de poluição 
atmosférica, pois inclui algoritmo robusto e estável desenvolvido por 
\citet{paatero1994} que impede o aparecimento de valores negativos no 
perfil e na contribuição de fontes e diferentemente da AF, pondera o ajuste 
pelo inverso das incertezas nas concentrações, diminuindo assim, o peso de 
espécies com incertezas altas (equação \ref{eq:pmfobject}).
Daí a importância de uma adequada definição para as incertezas dos parâmetros 
medidos, como feito para XRF-ED e para intercalibração de BC usando MQM.

%%%%
\subsubsection{Função Objeto}

Função objeto, em matemática, é uma função que precisa ser minimizada 
ou maximizada usando métodos numéricos para equações não lineares, pois não 
tem solução analítica. A matriz de resíduo ${e_{ij}}$ da equação \ref{eq:pmf}
pode ser substítuida em \ref{eq:pmfobject} obtendo-se:

\begin{equation}
  Q = \sum_{i=1}^n \sum_{j=1}^m  \left[ \frac{c_{ij} - \sum_{k=i}^p g_{ik}f_{kj}} {u_{ij}} \right] ^2
  \label{eq:pmfobjectfull}
\end{equation}

A solução para \ref{eq:pmfobjectfull} foi inicialmente implementada usando 
o método de Gauss-Newton \citep{paatero1994}, mas na versão atual do software 
(PMF EPA 5.0) é usado o Método do Gradiente Conjugado \citep{norris2014}, 
cujo inicia $g_{ik}$ e $f_{kj}$ randomicamente os ajustando até se chegar no 
menor $Q$ possível.

