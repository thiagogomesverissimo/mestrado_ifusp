%%%%
\section{Modelos Receptores}

Modelo receptor é uma abordagem matemática para determinar e 
quantificar o efeito das fontes poluidoras do ar. 

Modelos receptores buscam quantificar a participação de fontes poluidoras em amostras. 
O Balanço Químico de Massas (CBM), por exemplo, necessita do conhecimento prévio do 
perfil das fontes locais, cuja combinação linear proverá o rateio de amostragens 
entre tais fontes. 

Já a Análise de Fatores Principais (PFA) ou Positive Matrix Fatorization (PMF)
permitem o rateio de fontes quando os seus perfis são desconhecidos. 
Entretanto, necessitamos de conhecê-los quando buscamos associar estes 
resultados a realidade concreta da bacia aérea estudada. 

%TODO: no site da cetesb tem um esquema legal de modelo receptor
%TODO: traçar um panomara da evolução histórica dos modelos receptores.

\subsection{Análise multivariada}

O processo de identificação das fontes de poluição do ar é 
complexo, havendo diversos caminhos, os quais as técnicas estatísticas
multivariadas são historicamente usadas.
 
Análise multivariada é uma das técnicas estatística utilizadas para 
resolver reduzir a dimensão dos dados coletados nos modelos receptores, 
usando a variância. 

As dimensões (variáveis) reduzidas de um conjunto de dados analíticos 
complexo poderão ser interpretados como fonte(s) poluídora(s).

As técnicas de redução da dimensão, como análise multivariada, 
permitem manter a quantidade de informação contida inicialmente nos dados. 
Essa informação está armazenada na variância e covariância (ou correlação) 
inicial dos dados. 

Dâ-se o nome de Fator as novas variáveis reduzidas. 
O Fator é um tipo de variável latente, pois não é uma grandeza diretamente
medida, mas sim obtida a partir de outras variáveis observáveis. 

O número de fatores extraídos dependerá do conhecimento do(a) pesquisador(a)
sobre a região, possíveis fontes poluidoras, número de amostras, 
resolução da amostragem e espécies medidas.

O conhecimento do pesquisador é imprescindível para fazer o relacionamento 
entre fator e fonte(s), afim de avaliar o significado físico das fontes. 
Para tal, informações das possíveis fontes poluidoras próximas ao ponto 
receptor, dados meteorológicos do período de coleta, inventário de emissões 
e outras informações que possam ajudar a fazer esse relacionamento 
devem ser usadas.

Número reduzido de amostras, baixa resolução da amostragem, \textit{outliers} 
(eventos infrequentes de curta duração e com alta concentração) representam 
um desafio para modelos multivariados.

%%%%
\subsection{Fomulação matemática dos Modelos Receptores}

Os poluentes que são emitidos pelas fontes em uma bacia aérea devem chegar no 
receptor, podendo-se, assim, usar o princípio de conservação de massa no Modelo Receptor. 
Vamos fazer uma adaptação da equação de conservação de massa para o contexto 
de poluição do ar. 

Se em uma amostragem coletamos $j$ amostras e medimos $i$ poluentes, podemos 
escrever a equação da conservação de massa como segue: 

\begin{equation}
  x_{ij} = \sum_{p=1}^{P} g_{ip}f_{pj} + \epsilon_{ij}
\end{equation} 

sendo,
\begin{itemize}
  \item $x_{ij}$ = concentração na amostra receptora $i$ da espécie $j$;
  \item $f_{pj}$ = fração da espécie $j$ emitida pela fonte $p$;
  \item $g_{ip}$ = contribuição da fonte $p$ para amostra $i$;
  \item $\epsilon$ = erro do modelo empregado.
\end{itemize}

%%%%
\subsection{Balanço Químico de Massa}

O quanto dos perfis de fontes medidas explica as concentrações dos elementos? 
A técnica Balanço Químico de Massa (\textit{BQM}) busca responder essa questão.

%TODO: você deve terminar de dizer o que é o CMB antes de dizer suas condições de uso e limitações. Desse modo o leitor pode saber porque isso ocorre. Cabe falar do CMB, já que não realizou nenhuma avaliação com ele.

O \textit{BQM} apresenta problema quando fontes importantes são omitidas do
cálculo ou alguma fonte incluída não representa uma fonte real.

Considerações no uso do \textit{BQM}: 
\begin{itemize}
  \item composição química das fontes constantes no tempo;
  \item as espécies não são reativas;
  \item todas fontes devem ser incluídas;
  \item número de fontes menor que o número de espécies.
\end{itemize}

%TODO: falar um pouco do speciate. 

Critérios para validar os resultados do \textit{BQM}:

\begin{itemize}
  \item $t-statistic$ > 2;
  \item $R^2$ >= 0,8;
  \item Porcentagem da massa explicada entre 80\% e 100\%; 
  \item ${\chi}^2$ < 4
  \item concentração modelada/concentração medida entre 0,2 e 2,0.
\end{itemize}
%ajuste o formato matemático dos itens acima.

%%%%
\subsection{Análise de Fatores}
%TODO: padronize a denominação dada à metodologia: Análise de Fatores Principais (AFP); Análise de Componentes Principais (ACP). Uma é geral. A outra é um método para resolvê-la.

A ACP é um técnica multivariada que permite-nos encontrar os padrões 
sistemáticos da variação dos dados. 
Do ponto de vista de análise dos dados, a ACP é usada para estudar um 
tabela de observações e variáveis com a principal ideia de transformar 
as variáveis observadas em um conjunto de novas variáveis, 
as componentes principais, que são ortogonais entre si 
e explicam a variação inicial do dados.
%TODO: o que faz a componentes principais? Não está claro  no que disse. O que de diferente faz a AF? Quais são as equações básicas deste modêlo?

Os resultados do PCA e AF são similares, apesar de PCA ser um modelo
descritivo dos dados e AF ser um modelo estrutura. 
Os loadins no AF levemente menores que os da PCA, pois PCA tenta
explicar toda a variância da matriz de correlação, enquanto que a AF
explica apenas a variância comum. 

PCA é uma representação onde as componentes são uma combinação linear 
ortogonal que maximizam a variância total.
AF é uma representação onde os fatores são uma combinação linear ortogonal
que maximiza a porção compartilhada da variância. 

Há evidencias que a variável latente (fator) existe?

No PCA não perdemos variância no processo de extração, pois 
o número de fatores extraídos é igual ao número de espécies medidas.

PCA e AF são ambas usadas para redução da dimensão. 
Na AF o objetivo é identificar o menor número de fatores que explica os 
dados observados. Em PCA as dimensões só são reduzidas, mas a 
interpretação das componentes é feita de forma similar no fatores da AF.

Um problema comum na Análise de Fatores é a Multicolinearidade. 
Caso o determiante da matriz de correlação seja maior que $0,00001$ 
há multicolinearidade e os dados devem ser checados. 
% essas explicações somente fazem sentido se você apresentar as equações do modelo e daquilo que está tratando.

\textit{Factor Loading} é a correlação entre a variável observada e o fator
extraído (similar ao coeficiente padronizado da regressão Linear Múltipla). 

\textit{Comunalidade} é a influência total numa variável de todos os fatores 
relacionadas com ela. Matematicamente, é igual a soma dos \textit{Factor Loading}
desses fatores ao quadradro. 
$0$ indica que a variável não foi nada explicada pelos fatores e $1$ indica 
que foi completamente explicada pelos fatores. 

\textit{Singularidade} é a porção da variabilidade de uma variável que não 
pode ser explicada pelas variáveis latentes relacionadas a essa variável. 
Matematicamente, é o resto da subtração da \textit{Comunalidade} por 1,
ou seja, é a porcentagem da variabilidade que não pode ser prevista pelo modelo.

\textit{Total da Variância Explicada} é o quanto da variabilidade dos dados foi 
modelada pelas variáveis latentes. 

\textit{Rotação} maximiza os \textit{Factor Loading} para melhor 
entendimento dos resultados. 

\textit{Rotação Varimax} pressupõe que os fatores não são correlacionados.

Equação da Análise de Fatores
equação para fatores absulutos, ou seja, é a recomposição dos fatores que 
são resultados da análise dos dados a partir da normalização dos dados 
originais pela média e pelo desvio padrão.

\cite{keiding1986}

Onde, sj é o desvio padrão da variável j para todas as N amostras; Bjp é o valor dos 'factor loadings' rotacionados para a variável j num fator p; sMP é o desvio padrão do material particulado para todas as N amostras; e Cp é o valor do 'factor loading' referente ao material particulado num fator p.

\begin{eqnarray}
FA = \frac{L_{ij}\sigma_i}{\sigma_{PM}L_{PM_j}}
\end{eqnarray}

\begin{itemize}
  \item L = loadings
  \item i = espécies
  \item j = fatores extraídos
\end{itemize}

Factor Scores: 
\begin{equation}
FatorScore1 = coeficiente_{elemento1 fator1}*valor_elemento1 + coeficiente_{elemento2 fator1}*valor_elemento2 ...
\end{equation} 

%%%%
\subsection{\textit{PMF} - Positive Matrix Factorizarion}

O \textit{Positive Matrix Factorizarion (PMF)} é outro método multivariado usado
em modelos receptores para resolver a equação da conservação da massa 
(citar equação).

%TODO: Além de colocar a equação, antes de discutir o código numérico computacional, você deve discutir estatística e cientificamente o que faz o método. Apenas na sequência é que cabe discutir as facilidades oferecidas pelo programa computacional.


O \textit{PMF} também é um tipo de Análise Fatorial
mas ao invés de fazer redução da dimensão dos dados decompondo a matriz de 
correlação em autovalores e autovetores, resolve a equação de conservação 
usando a resolução por mínimos quadrados. Em pesquisas de poluição atmosférica 
o método \textit{PMF} tem ganhado espaço nos últimos anos, devido a 3 motivos 
principais:

\begin{itemize}
  \item Disponibilização de uma versão gratuita do software que implementa 
        o método \textit{PMF} pela \textit{EPA}, o \textit{PMF EPA 5.0} 
        \citep{Norris:2014}. 
        Ainda existe a versão proprietária e comercial, com mais recursos.   
  \item Incorporação de um algoritmo robusto desenvolvido por (citar paatero e tapper) 
        que impede o aparecimento de valores negativos no perfil e 
        contribuição de fontes.
  \item Ponderação pelas incertezas das concentrações, diminuído assim o peso 
        de espécies com incertezas altas.
\end{itemize}  

Em termos de reprodutibilidade da pesquisa científica o \textit{PMF EPA 5.0} 
nos fornece os seguintes recursos:

\begin{itemize}
  \item Fixação de \textit{Random Seed}.%o que significa isso? Porque este detalhe destaca-se dos demais?
  \item Arquivo \textit{XML} com as configurações usadas na rodada. 
\end{itemize} 

%%%%
\subsubsection{Conservação da Massa}
O PMF é uma solução para a equação da conservação da massa, assim, se 
consiredarmos que:
\begin{itemize}
  \item $c_{ij}$ matriz de concentração
  \item $u_{ij}$ matriz de incertezas (experimentais e analíticas)
  \item $i$ as amostras válidas
  \item $j$ as espécies medidas
\end{itemize}

EPA PMF v3.0.2.2 (EPA, 2008), uma ferramenta de análise multivariada que permite determinar um conjunto de p fatores, o perfil das espécies f de cada fonte e a contribuição da massa g de cada fator para cada amostra individual, de modo que a matriz de dados obtidos, em j amostragens e com i espécies químicas determinadas é dada por:

Pode-se escrever a equação da conservação da massa no 
contexto do \textit{PMF} \ref{equation:pmf}: 

\begin{equation}
  c_{ij} = \sum_{k=i}^p g_{ik}f_{kj} + e_{ij}
  \label{equation:pmf}
\end{equation}

Onde,
\begin{itemize}
  \item $p$: O número de fatores informado pelo usuário.
  \item $g_{ik}$: Contribuição dos fatores nas amostras (\textit{Factor Score}).
  \item $f_{kj}$: Perfil da fonte ou assinatura da fonte, ou seja, 
        a distribuição das espécies nos fatores. (\textit{Factor Loadings}).
  \item $e_{ij}$: Matriz dos resíduos escalados pelas incertezas.
\end{itemize}

Onde eij é o residuo para cada amostra/espécie.
O modelo procura o melhor par das matrizes g e f, que minimiza a função Q, que pondera as diferenças entre valores observados e estimados pelas incertezas dos valores observados (uij):

O objetivo do \textit{PMF} é encontrar $g_{ik}$ e $f_{kj}$. 
($k$ é um fator genérico), pois a matriz $c_{ij}$ é decomposta em 
$g_{ik}$ e $f_{kj}$. Assim, a pergunta que temos que fazer ao trabalha com 
o \textit{PMF} é: 

\textit{Quais $g_{ik}$ e $f_{kj}$ melhor reproduzem $c_{ij}$?}

O usuário informa a quantidade de fatores desejados $p$ e o \textit{PMF} 
sempre encontrará uma solução para essa quantidade de fatores. 
Entretanto, é necessário avaliar a qualidade do ajuste, seguindo os passos 
que serão apresentados logo a seguir, assim como interpretar o significado 
físico dos fatores. 

%%%%
\subsubsection{Função Objeto}

Uma função objeto, em matemática, é uma função que precisa ser minimizada 
ou maximizada usando métodos numéricos para equações não lineares, pois não 
tem solução analítica. 

No \textit{PMF} a função objeto $Q$ é calculada conforme a equação 
\ref{equation:pmf:object}, onde ${e_{ij}}$ é encontrado isolando-o na 
equação \ref{equation:pmf}.
%TODO: veja que uij é a incerteza que temos na determinação da concentração (método analítico e amostragem); esse é um ponto importante e forte realizado neste trabalho e que vai precisar ser bemd iscutido, inclusive nesta parte teórica.


\begin{equation}
  Q = \sum_{i=1}^n \sum_{j=1}^m  \left[ \frac{e_{ij}} {u_{ij}} \right] ^2
  \label{equation:pmf:object}
\end{equation}

O \textit{PMF} minimiza a função objeto $Q$ e converge quando encontra um $Q$ 
mínimo local ou global.% o que é local e global?


Como sempre estamos interessados no mínimo global, precisa-se verificar se a 
solução gerou um $Q$ mínimo local ou global. A estratégica para identificar
o tipo de mínimo é rodar o \textit{PMF} para diferentes valores de 
\textit{Random Seed} e acompanhar a estabilidade de $Q$.

A equação \ref{equation:pmf:object} foi inicialmente implementada usando 
o \textit{método de Gauss-Newton}  (citar fonte 1997 Paatero Tapper), mas na 
versão atual do software \textit{PMF EPA 5.0} é usado o 
\textit{Método do Gradiente Conjugado}, que necessita de menos recursos 
computacionais. 

Os valores de $g_{ik}$ e $f_{kj}$ são ajustados até encontrar menor $Q$. 
O \textit{PMF EPA 5.0} nos oferece dois valores para $Q$: $Q_{verdadeiro}$ e 
$Q_{robusto}$, onde o primeiro foi calculado considerando todos os valores 
de concentração e no segundo remove-se os \textit{outliers}.
Assim quando há poucos \textit{outliers}, $Q_{verdadeiro}$ e $Q_{robusto}$ 
são próximos. Incertezas muitos altas também causam $Q_{verdadeiro}$ e 
$Q_{robusto}$ similares.

O \textit{PMF EPA 5.0} começa com os valores da matriz $f_{kj}$ randômicos, 
que são então modificados, até a encontrar a melhor solução, ou seja, 
a do menor $Q_{robusto}$. 
Devido ao inicio randômico é recomendado uma rodada de pelo menos 100 iterações 
como solução final, escolhendo-se a do menor $Q_{robusto}$, para termos assim 
mais esperança de encontrarmos um $Q$ mínimo global e não local. 

%TODO: você está fazendo uma disucssão randomica sobre o método. Separe primeiro os fundamentos básicos do método no início, com suas equqções fundamentais. Depois selecione o que é relevante para a metodologia de solução numérica da equação. O manual de uso do método deve ser referido e quem for trabalhar com o modelo irá tratar de aprender seus detalhes. Isso vale para o que fez a seguir, que representa um conjunto de sugestões arbritrárias dos produtores do modelo e que podem direcioná-lo sem uma base concreta para tanto. O mais importante é você discutir como montou suas incertezas. Você pode até comentar algumas destas sugesões mas não convém adotá-las a prióri.


%%%%
\subsubsection{Pré-processamento do ajuste}
Requisitos conceituais e técnicos para realização do ajuste.

%%%%
\paragraph{Signal Noise (S/N)}

Dependendo do método de medida, conhecimento dos processos químicos 
atmosféricos e limite de detecção é necessário diminuir o peso de algumas 
espécies no modelo, diminuindo as incertezas. 
Além do conhecimento da espécie, o \textit{Signal Noise (S/N)} é um bom 
indicador de quais amostra deve-se aumentar as incerteza.

\begin{itemize}
  \item Se, $c_{ij} >  u_{ij}$, então $ S/N = (c_{ij} - u_{ij})/u_{ij}$.
  \item Se, $c_{ij} <  u_{ij}$, então $S/N = 0 $.
\end{itemize}

O \textit{S/N} indica se a variabilidade nas medidas é real ou faz parte do 
ruído dos dados. 
Espécie com a concentração menor que a incerteza, não apresentam ruído, e devem 
ser removidas da análise. Espécies com valores muito próximos da incerteza, 
portanto com $S/N$ próximo de zero (< 0,5) também devem ser removidas da 
análise. 
Espécie nas quais a concentração é pelo menos duas vezes o valor da incerteza, 
isto é, $S/N$ é maior ou igual à 1, são as ideais para análise, 
e não altera-se as incertezas. 

Quando $S/N$ está entre 0,5 e 1, diminuímos o peso da espécie aumentando a 
incerteza em um fator 3.  

%%%%
\paragraph{Manipulação do Software EPA}

Os dados de concentração, $c_{ij}$, e incertezas, $u_{ij}$, devem seguir 
alguns pré-requisitos antes da realização do ajuste:

\begin{itemize}
  \item Não é permitido valores negativos em $c_{ij}$ ou em $u_{ij}$.
  \item células vazias não são aceitas.
  \item nomes das colunas(espécies) e linhas(amostras) devem ser únicos.
  \item é ideal (não obrigatório) que os dados já estejam classificados 
        pela data em ordem crescente.
\end{itemize}

%%%%
\paragraph{Inspeção dos dados de entrada}

É necessário fazer ainda inspeções ou alterações antes de fazer o ajuste.
\begin{itemize}
  \item Espera-se uma relação tipicamente linear entre 
        \textbf{Concentração} $\times$ \textbf{Incerteza}.  
        Investigar o motivo caso isso não ocorra. 
  \item Adequação das incertezas baseado no \textit{Signal Noise (S/N)} e 
        conhecimento da espécie. 
  \item Marcar a Massa Total como \textit{Total Variable}
  \item Inspecionar gráficos de dispersão entre espécies nas quais se espera 
        correlação, anti-correlação ou não correlação. 
  \item Inspecionar gráficos de séries temporais, para:
    \begin{itemize}
      \item Identificar padrões temporais em espécies individuais ou em grupo 
            de espécies.
      \item Remover \textit{outliers} 
    \end{itemize}
\end{itemize}

%%%%
\subsubsection{Fazendo o ajuste}

Segue-se uma séries de conceitos que deverão ser entendido antes da realização 
do ajuste. 

%%%%
\paragraph{Ambiguidade Rotacional}

A comparação fator por fator da contribuição dos fatores, $g_{ik}$, 
pode ser plotada em um tipo de gráfico chamado \textit{G-Space}. 
O gráfico \textit{G-Space} auxilia na verificação da existência de 
\textit{Ambiguidade Rotacional} na solução. 
Dada a natureza da \textit{Análise Fatorial}, onde os fatores não devem estar 
correlacionados, no gráfico \textit{G-Space} fatores com pontos fora da 
proximidades dos eixos tem maior \textit{Ambiguidade Rotacional}. 

Amostras com contribuição zero em ambos os eixos proporcionam maior 
estabilidade na solução e portanto menor \textit{Ambiguidade Rotacional}. 

%%%%
\subsubsection{Pós-processamento do ajuste}
Métodos para avaliação da estabilidade da solução.

%DICA PARA PLOTAR OS HISTOGRAMAS DOS RESÍDUOS:
%Os resíduos estão na escala das incertezas: o ideal é que o eixo x vá de -3 até 3 (bin:0.5 eixo y: porcentagem)
 
% % % % % % % % % % % % % % % % % % % % % % % % % % % % % % % %
%COMPARAÇÃO ENTRE AS RODADAS DE DIFERENTES Q
%Dois diagnósticos para avaliar diferenças entre as rodadas:
% 1)analise residual entre as rodadas 
% 2)o sumário do fator distribuição dos elementos comparados com a rodada do menor Qrobusto. 
% (compara resultados do menor $Q_{robusto}$ com outras Q)
% Mas detalhes de comparação entre as rodadas é dada na comparação dos resíduos.). 
%Cálculo residual entre as rodadas: compara os residuos entre a baserun mais a diferenca quadrada entre os residuos (na escala das incertezas) para cada par de base run:
% \begin{equation}
%  d_{jkl} = \sum\limits_{l} (r_{ijk} - r_{ijl})^2
% \end{equation}
% r: resíduo na escala das incertezas
% i: amostra
% j: espécie
% k,l: duas diferentes rodadas consideradas.
% %Teremos assim uma matriz onde podemos checar rodadas diferenças significativas, a matriz está no arquivo *_diagnostics. O que olhar? Comparar um fator com um outro, deveria ser contantes...
% 2.5)Os valores das espécies pareadas para cada rodada podem ser comparados adcionando os d da equação anterior:
% \begin{equation}
%   D_{kl} = \sum\limits_{j} d_{jkl}
% \end{equation}
% %a matriz D está no arquivo *_diagnostics. Prestar a atenção na estabilidade por elemento.Grande variações indicam que duas rodadas resultaram em solucoes diferentes de fato em vez de meras rotações uma da outras.
% 2.6) Variação do Q nos fatores: No arquivo *_run_comparation podemos ver o Q por fator, assum veremos o quão estável está o Q no fator, encontrando assim fatores com Q instáveis 
% 
%perfil: azul: concentração de cada espécie no distribuida no fator.
% vermelho: porcentagem de cada especie distribuida no fator: 
% o grafico esta normalizado entap a media de  todas contribuicoes para cada é 1. 
% O interessante aqui é verificar todas rodada afim de ver a estabilidade da solução. (ou seja, ver se os fatores ou especies estao vatriand muuito entre as rodadas). Fazer um stack plotr grafico com todos fatroes.  
%
%no grafico do contribuition podemo multiplicar pela massa total para plotaR nas unidades. 

%Qesperado = (número de amostras*número de elementos) - (n. de fatores (n.elementos + n. n.amostras))
%
%Q/ Qesperado (é por espécie) = soma dos quadrados dos residuos (na escala correta) de todas espécies dividido pelo número da espécie.   

%o gráfico de  Q/ Qesperado é um caminho de como enteder os resíduos na solucao pmf, e ver qual especie ou amostra nao é bemajustada (com valor maior que 2). Assim, podemos rodar o modelo novamente com mais fontes, para ver se os elementos não ajustados na verdade não correspondem a outro fator. na temporal os pontos maior que dois pode indicar impactos de fontes unicas e devem ser investigado.

%PARA COMPARA COM SAIDAS DO ACP OU CMB
%As saidas F (ug) e G ($1/m^3$) estao normatizadas (preciso conferir se quando marcamos a massa total ele nao faz isso), mas precisam ser renomartizados para as unidades originais: F (ug/ug) e G ($ug/m^3$), mas se marcarmos a MAssa total o programa faz isso, será? como:
%F sai em ug (deveria ser ug/ug) 
%G sai rm 1/m3 (ug/m3)
%
%Pegue um fator. 
%divida as espécies da Matriz F pela Massa (do fator considerado).
%multiplique o cada fator da matriz G pela Massa (do fator considerado)
%
 
%Exemplo:
%F:
%$Al_{fator1}$ = valor do Al no fator 1 de F / valor da massa do fator1 de F
%
%G:
%amostra1.fator1 = valor da amostra1 no fator1 de G x Valor da massa do fator 1 de F.  


% % % % % % % % % % % % % % % % % % % % % % % % % % % % % % % % % % % % % % % % % % % % % % % %
%% % % % % % % % % % % % % % % % % % % % % % % % % % % % % % % % % % % % % % % % % % % % % % % %
%\subsection{Estabilidade da Solução}
%
%\subsection{BS - Bootstrap}
%Para avaliar a estabilidade da solução, o \textit{PMF EPA 5.0} cria conjuntos de dados "similarers" aos dados originais, na técnica \textit{BS}, $g_{ik}$ e $f_{kj}$ são calculados para esses conjuntos de dados e comparados com os encotrados para a base original.
%Métodos de estimativa do erro: DISP,BS e BS-DISP.
%
%A estabilidade da solução do PMF pode ser estimada usando 3 métodos:
%#Métodos de Avaliação do erro para verificar a estabilidade da solução
%3)DISP displacement: sensitividade da solução escolhida a pequenas mudanças. O DISP (deslocamento) avalia o efeito da ambiguidade rotacional. 
%Explora o ambiguidade rotacional no PMF avaliando o maior intervalo de valores de perfis de fontes sem um aumento significativo do Q-value. 
%Altera o Qrobusto.? 
%Os valores do perfil do fator são rearranjados ao máximo (dQmax=base-modificado).
%O modelo gerara resultado para 4 valores de dQmax=4,8,15 e 25. 
%Mas vamos usar o 4 (Q minimo local) pois fornece um intervalo robusto para o dataset. 
%Dois arquivos são gerados nesse momento: \_DISPest.dat e \_DISP.txt, sendo o txt mais amigavel. 
%output: para cada dQmax *\_DISPres1, *\_DISPres2, *\_DISPres3, *\_DISPres4. (mas vamos usar o menor)
%
%4)BS(inicialização): analise de bootstrap: investiva se um subset de amostras podem influenciar disproporcionalmente a solução. Inclui efeitos de erros aleatórios 
%e inclui parcialmente erros da ambiguidade rotacional. 
%Estratégia de análise: O usuário especifica quantos subconjuntos ele quer e o BS vai rodar para esse subconjuntos. 
%O fatores desses subconjuntos serão correlacionados com os do Base run (mapeamento). 
%O resultado será em forma de porcentagem dos fatores do base run mapeados no BS, os que não forem mapeados, investigar melhor. 
%***block size: Número de amostras de cada subconjunto (há um método para estimar Politis and white).
%Number de bootstraps: 100 para robustes (numeros de subconjutos que ele vai testar).
%Minimo R: pearson, minimo R que vamos aceitar como fatores correlacioandos. O DDP (discrete difference percentiles) podem ser usados 
%para reportarmos a estimativa do erro do bootstrap. (DDP o Intervalo de confianca 95\% em relacao A BASE RUN.)
% Olhar sempre a matriz de mapping, nela os fatores devem ter sido mapeados para eles mesmos, ao menos 80 por cento dos fatores devem ser
% mapeados por eles mesmos, o que indica que as incerteza do bootstrap e o numro de fatores sao apropriados.
% é interessante plotar um gráfico da varibilidade (em concentracao e em porcentagem) - Box-and-whiskers plot) e é
% bom colocar um ponto do baserun, as especies que estao fora do interquartil range devemos olhar com  cautela. 
%output: *profile_boot, contém o número de rodade BS mapeada para base run. 
%
%
%5)BS-DISP: abordagen hibrida.Estima efeitos associados a  erros aleatórios e da ambiguidafde rotacional, é uma combinação 
%dos dois métodos BS e DISP, sendo que cada subconjunto BS rodado acima é feito um DISP. Assim cada DISP cria espaços rotacioandos,
%ai novamente o BS é rodado aleatóriamente em diversas direções. Assim esse método apresenta a incerteza randômica e a incerteza rotacional. 
%Há espécie chaves para cada fator. 
%Estratégia de análise:  
%Olhar CI e dQmax. Se a solução nao tem uma ambiguidade rotacional e não tem swap (troca de fatores) os resultado da solução podem ser usados. 
%Se houver swaps na solução, diminuir número de FATORES? 
%cAbecalho: k (casos no arquivo)=1 + n. bootstrap.; o segundo valor é o quanto mudou o Q, se for mais que 1 por cento refazer analise.; 
%numero de cassos com diminuicao de Q; numero de casos com swap factor no mekhor fit; numero de swaps no DISP.Olhar arquivo de saida: baseErrorEstimationSummary.
%output: São gerados 4 arquivos: (*\_BSDISP1, *\_BSDISP2,*\_BSDISP3, *\_BSDISP4) para cada dQmax (mas vamos usar o menor)
%
%4)Fpeak
%output: *_fpeak, contém perfis e contribuição para cada rodada Fpeak
%
%5)Constrained 
%output: *_Constrained
%
%6)Profiles (\_profile unidade de massa, porcentagem da especie e gracao da conc da especie  )
%
%7)Contributions (\_contrib normalizado e em funo da massa total)
%
%8)Residual (\_resid), Run Comparison (run\_compararios) e  Summary , Input, Base Runs (diag com os 3)
%
%
%9)\_ErrorEstimationSummary provides a summary of the base run and the error estimations that have been done using BS, DISP, and BS-DISP. 

%10) Ferramentas de Rotação:
%
%Um infinito número de soluções são geradas, a rotação tenta diminuir o número de soluções. informações das fontes, nos ajudam a descartar possíveis soluções.
%
%Aqui não é bem uma rotação, mas sim uma transformação linear.
%\begin{equation}
%G* = GT
%\end{equation} 
%
%\begin{equation}
%F* = T^{-1}F
%\end{equation}
%
%A matriz T é pxp.
%p número de fatores.
% No PMF não é possível ter valores negativos (restrição não negativa), isso já nos assegura que existe uma pequena ambiguidade rotacional na solucao, pois uma rotacao pura só é possivel se nenhum dos elementos das novas matrizes são negativos. Caso nenhuma rotacao seja possível, a solucao é unica.
%rotacao pura: com uma matriz T especifica 
%Assim, rotaçÕes aproximadas que permitem algum aumento no Q e impedem qualquer elemento ba solucao de tornar negativo são uteis. 
%
%Para determinar se a solucao foi enconttrada, é necessário inspecinar o G-space plot para pares de fatores.
%
%comparar os gráficos do fpeak com o baserun.
%
%Precauções ao usar o \textit{PMF}:
%\begin{itemize}
%  \item É possível ter um bom ajuste resulte, mas com uma solução sem significado físico. 
%  \item Rotação ambígua, isto é, a unicidade da solução não eh garantida.
%\end{itemize}
