%%%%
\section{Black Carbon}

O Black Carbon 
% https://www3.epa.gov/blackcarbon/

Concentrações de Black Barbon (BC) foram obtidos por  refletância, 
onde o filtro é iluminado por uma lâmpada de tungstênio e a luz refletida é 
detectada por um sensor, assim, quanto menor a intensidade dessa luz,
 maior é a quantidade de black carbon presente na amostra. 

Foi utilizado um refletômetro 
\textbf{Smoke Stain Refletometer, DiffusionSystem} modelo M43D. 

Inter-calibrou-se a curva obtida pela refletância com um equipamento 
Sunset para determinação de carbono orgânico e elementar, 
por processo Térmico/Transmitância Óptica (EPA, 2012).

A análise de refletância  é utilizada para se determinar a concentração 
de Black Carbon (BC) presente em amostras de poluição do ar. 

Black Carbon é encontrado  predominantemente na fração fina do aerossol, 
sendo normalmente partículas provenientes de combustão incompleta de biomassa, 
biocombustíveis ou combustíveis fósseis. 

A técnica baseia-se na propriedade do composto (BC) possuir alta seção de 
choque de absorção de luz na região do visível. 

A radiação emitida por uma lâmpada de tungstênio reflete nas amostras, 
detectando-se o percentual de luz refletida. Este sistema foi calibrado 
com amostras padrões, o que permitiu a determinação da concentração de BC, 
segundo a expressão:

Black carbon 
%$ ug.m-3 = 81,95 - 71,83 x log(R)+15,43 x log(R)^2  x A/ V $

Sendo R a refletância (\%), A a área do filtro e V o volume de ar amostrado ($m^3$). 
Utilizou-se um refletômetro Smoke Stain Refletometer, da Diffusion System, modelo M43D (fig.10).

%\begin{figure}[H]
%\begin{center}
%  \includegraphics[width=\textwidth]{../../inputs/images/repletometro.jpg}
%  \caption{Refletômetro e peças auxiliares de fixação e teste do filtro}
%\end{center}
