\section{Fluorescência de Raios X}

Avaliação quali-quantitativa da amostras. Técnica não destrutiva, simultânea. 
Instrumental: sem pré-tratamento químico. 

Tipos de fluorescências de Raios X:

\begin{itemize}
  \item Dispersão de Energia
  \item dDispersão de comprimento de onda
  \item Reflexão total
\end{itemize}

A dispersão por comprimento de onda é baseada na lei de Bragg. Os dispersivos 
em energia usam semicondutor capaz de discriminar energia próximas. 

Os raios X excitador pode ser um tudo ou fonte radioativa emissoras de raios x. 
(no nosso caso não seria dedido a baixa inetnsidade)

Correção do efeito matriz: interações do raios x característicos com os 
elementos da amostras. (absorção do raiox x ou reforço)

fases do edx;

outras técnicas de excitação: partículas aceleradas: elétrons, prótons ou íons,
 alfa e beta negativa.

Energia de ligação pode ser aproximada pela teoria atomica de bohr.

\begin{math}
E = \frac{me^4(Z-b)^2}{8w^2h^2n^2}
\end{math}

E = energia de ligação eletrônica (joules),
m = massa de repouso do elétron = 9,11.10 -31 kilogramas,
e = carga elétrica do elétron = 1,6.10 -19 coulombs,
Z = número atômico do elemento emissor dos raios X,
b = constante de Moseley, com valores iguais a 1 e 7,4, para as camadas K e L,
respectivamente.
w = o = permitividade elétrica no vácuo = 8,8534.10 -12 coulombs.newton -l .metro -2 ,
h = constante de Planck = 6,625.10 -34 joules.s, e
n = n o quântico principal do nível eletrônico (n = 1 para camada K, n = 2 para
camada L, etc.),

amplitude do pulso eletrônico produzido no detector.

bragg: critais de difração com distancias interplanares conhecidas 

kalfa, kbeta: representa transições. kalfa: L para K. Kbeta: M para K

Fazer diagrama das transições .Lembrando que há transições proíbidas. 
Algumas transições tem energias tão próximas que é impossível seprara.

%Notação http://en.wikipedia.org/wiki/Siegbahn_notation
%http://www.amptek.com/pdf/characteristic_xrays.pdf que não é a recomendada pela: IUPAC

rendimento: raios-x efetivamente emitidos em relação as vacancias produzidas. 

espectrômetro de raios X por dispersão de energia

pulsos ele-trônicos proporcionais às energias dos raios X

o mais empregado é o detector de silício ativado com lítio, Si(Li),
 


\section{Limite de Detecção}



