%%%%
\section{Amostragem}
%TODO: citar artigo do harvardinho (Impactor Desing)
% http://maps.google.com/maps/ms?ie=UTF8&hl=en&msa=0&msid=116003586198857296821.00046d7e7367b947abe12&z=12


As amostras foram coletadas em dois locais no bairro de Nima. 
Um na parte residencial do bairro (+5° 35' 2.00", -0° 11' 58.80") 
e o outro na avenida principal (+5° 34' 54.00", -0° 11' 56.30"). 

As amostras foram coletadas em filtro de teflon do tipo \textit{PTFE} de 
37 $mm$ de diametro  e poros de 0,2 $\mu m$.

As medidas de PM10 e PM2.5 foram feitas no impactador Harvard com $D_{50}$ 
de $10 \mu m$ \cite{marple1987}. 
Na medidas de PM2.5 o impactador Harvard com $D_{50}$ de $2,5 \mu m$ 
foi acoplado com um \textit{inlet} responsável por filtrar somente PM2.5.
Filtros brancos de campo e laboratórios foram separados para avaliar possíveis
contaminações no tranporte e manipulação das amostras. 

As medidas começaram em Novembro de 2006 e encerraran-se em Agosto de 2008
com coletas diárias. Descontando-se amostras perdidas, dias com falta
de eletricidade e outros impedimentos diversos das medidas, 
coletou-se 858 amostras em Nima.  
O tempo de amostragem foi de 48 horas, o que trouxe dificuldades para 
as análises multivaridas. 

Os filtros foram pesados antes e depois de irem a campo em uma microbalança  
\textit{(Mettler Toledo MT5)} do laboratório da 
\textit{Harvard School of Public Health}. Cada pesagem foi realizada duas vezes 
e o valor final foi a média das duas medidas.
A sala de pesagem contém controle de umidade ($39 \pm 2 \%$) e 
temperatura ($20,5 \pm 0,2 ^{\circ} C$). 
A eletricade estática foi minimizada por um fonte de polônio. 



Os dois pontos de amostragem em Nima distam X mestros e as amostragem foram de 48 horas.
A campanha de amostragem começou em x/x/x e terminou em x/x/x.


