%%%%
\section{Amostragem}

% http://maps.google.com/maps/ms?ie=UTF8&hl=en&msa=0&msid=116003586198857296821.00046d7e7367b947abe12&z=12

A descrição dos pontos de medição e sua localização geográfica é apresentado 
na Fig XX, rua BLA. 
A X km do aeroporto e vizinhança de casas construídas de forma precarizada.

A principal fonte de poluição de automóvel vem de uma estrada pavimentada que 
dá acesso ao centro e o outro ponto um conjunto residencial improvisado.

O local de medição está localizado a 165 m desta estrada em linha reta. 
inspeção visual indica que queimadas a céu aberto ocorrem frequentemente, 
pois moradores queimam galhos, folhas, detritos domésticos e também de plástico 
a partir de fios de cobre. 
Esta é uma estrada arterial de 2,8 km alinhados na direção norte-sul e recebe o 
tráfego de muitas pequenas ruas, atravessando áreas de comércio. 

A campanha de amostragem ocorreu em Acra (capital de Gana) 
entre 11 de Novembro de 2006 e 15 de Agosto de 2008. 
Foram coletadas 879 amostras de 48 horas no topo de duas residências 
no bairro de Nima.

Os dois pontos de amostragem distam entre si 280 metros, sendo um na rua
\textbf{Sam Rd} com característica residencial
(+5$\degree$ 35$'$ 2$''$,-0$\degree$ 11$'$ 58$''$)
e o outro na avenida \textbf{Al-Waleed bin Talal Highway} 
(+5$\degree$ 34$'$ 54$''$, -0$\degree$ 11$'$ 56.3$''$) com comércios e
alta movimentação de veículos, pois conecta diversos bairros ao centro
\ref{fig:nima}. 

\begin{figure}[H]
\begin{center}
  \includegraphics[width=0.6\textwidth]{../inputs/images/zheng/nima_mapa.pdf}
  \caption{Mapa de Nima. NM-1 ponto de amostragem na área residencial e 
           NM-2 ponto de amostragem na avenida. Porcentagem do uso da queima
           de biomassa para preparação de alimentos em residências usando dados
           do censo de 2000 \citep{ghanacensus2003} \label{fig:nima_mapa}}
\end{center}
\end{figure}

A coleta foi diária, mas houve dias sem medidas devido a falta de eletricidade,
problemas no amostrador, filtros danificados, ausência do operador, entre outros. 

Concentrações dos poluentes variam ao longo do dia na atmosfera
e quanto menor o tempo de amostragem obtém-se melhor resolução 
para identificação das fontes. Porém, amostras com concentrações menores 
são mais difíceis de se medir devido ao limite de detecção dos equipamentos
\citep{calzolai2015}. Doze ou menos horas de amostragem permitiria captar 
a variabilidade das fontes de Acra (por ser uma região muito poluída).
Mas o tempo de amostragem foi de 48 horas e foi definido antes da 
entrada da USP no projeto.

As amostras  foram coletadas em filtro de Teflon do tipo 
\textbf{PTFE} de 37 $mm$ de diâmetro, com orifícios de 0,2 $\mu m$ de diâmetro. 

%TODO: qual método foi utilizado para medir a área?
A área de deposição nos filtros foi de $7,32 (\pm 0,366) cm^2$ .

Impactadores são responsáveis pela coleta e classificação 
do material particulado, utilizando-se para tal a inércia das
partículas.
Para $MP_{10}$ utilizou-se impactador Harvard com $D_{50}$ de $10 \mu m$ 
com fluxo de $10,0 L/min$ \citep{marple1987}. 

Nas medidas de $MP_{2,5}$, também utilizou-se impactador Harvard, 
mas com $D_{50}$ de $2,5 \mu m$ acoplado com um \textbf{inlet} 
responsável fazer a filtragem de $MP_{2,5}$.

Os volumes amostrados foram medidos por um integrador de volume
com 5\% de incerteza.

Filtros brancos de campo e laboratórios foram separados para avaliar 
possíveis contaminações no tranporte e manipulação das amostras. 

O laboratório da \textbf{Harvard School of Public Health} foi
utilizado para pesagem das amostras.

Os filtros foram pesados antes e depois da amostragem, usando uma balança 
microanalítica \textbf{(Mettler Toledo MT5)} com precisão de $1 \mu g$, 
seguindo procedimentos padrões de controle de umidade ($39 \pm 2 \%$), 
temperatura ($20,5 \pm 0,2 ^{\circ} C$) e eliminação de cargas eletrostáticas 
(fonte de polônio). 
Cada medida de pesagem foi realizada duas vezes e o valor final foi a média das 
duas medidas. 
