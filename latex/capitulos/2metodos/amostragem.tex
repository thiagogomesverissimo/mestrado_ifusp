%%%%
\section{Amostragem}
%TODO: citar artigo do harvardinho (Impactor Desing)

Dados da amostragem:%normalmente começa-se localizando as estações de amostragem, indicando frequência da coleta e número de amostras coletadas.

\begin{itemize}
  \item filtro de PTFE (teflon) de $37mm$ de diâmetro;
  \item $MP_{10}$ medido com Harvard Impactor com $D_50$, diâmetro 
        aerodinâmico: 10 $\mu m$;
em 5LPM (+/-10 \%)%isso e o que vem a seguir não está claro. Também deve ser D50.

  \item $MP_{2,5}$ também usou o Harvard Impactor combinado com inlet 
         seletivo de $MP_{2,5}$ com PUF (espuma de polyurethane) com $D_50$ 
          de 2.5 at 5LPM (+/-10 \%);
  \item Duas placas impactadoras consecticvas (com óleos) servindo como 
        superfície de impacto.
        

\end{itemize}




.




