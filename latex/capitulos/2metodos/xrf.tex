%%%%
\newpage
\section{Fluorescência de Raios X}

Para quantificar a composição elementar (número atômico $ 10 < Z < 82$) das 
amostras foi utilizada a técnica não-destrutiva de Fluorescência de Raios X 
(XRF), um método analítico quali-quantitativo, multielementar, 
que mede os raios X característicos emitidos pelos átomos da amostra, 
depois de também serem excitados por raios X. Permite análise simultânea dos 
elementos químicos e não exige pré-tratamento dos alvos.

Há basicamente 3 etapas envolvida na técnica de medida de raios X 
característicos: excitação da amostra, emissão de raios X pelos átomos da amostra
e detecção. A excitação pode ocorrer por feixe de raios X (ou raios gamas) 
produzido em fontes radioativas, por partículas aceleradas 
(elétrons, prótons, alfas etc) ou 
por tubos geradores de raios X quando submetidos a diferença de potencial
\citep{jenkins1988}. 

No caso da XRF, a excitação ocorre por um feixe de raios X incidente, que  
expulsa os elétrons das camadas mais internas do átomo (K,L e M) 
produzindo vacâncias. Para tal, a energia do feixe incidente deve ser maior 
que a energia de ligação dos elétrons nessas camadas. Um átomo com vacância é 
instável e rapidamente elétrons das camadas mais externas preenchem as vacâncias,
liberando fotóns e estabilizando o átomo, sendo a energia destes fótons 
correspondentes às energias de transição entre camadas do átomos, 
característica de cada elemento químico. O processo de excitação do átomo 
é ilustrado classicamente na figura \ref{fig:shimadzu_atomo}.

\begin{figure}[H]
  \centering
  \includegraphics[width=0.5\textwidth]{../inputs/images/shimadzu_atomo.jpg}
  \caption{Ilustração clássica do fenômeno de fluorescência de raios X no átomo. 
           Figura acompanha o manual da Shimadzu da série de equipamentos
           EDX 700 \label{fig:shimadzu_atomo}}
\end{figure}

As transições dos elétrons entre os níveis quânticos K, L e M encontram-se 
tipicamente na faixa dos raios X, tendendo a ultravioleta (UV) e luz visível,
conforme ocorram em transições atômicas de menor energia no átomo.

Transições da camada L para K são do tipo $K_{\alpha}$, de M para K 
são $K_{\beta}$ e de M para L são $L_{\alpha}$ ou $L_{\beta}$. 
As camadas L e M possuem ainda subníveis de energia, o que resulta em diversas
combinações de transições, sendo algumas delas proibidas, e outras 
com diferenças de energia indistinguíveis para os detectores utilizados 
neste método analítico (XRF-ED).

A notação desenvolvida por Siegbahn \citep{jenkins1991}, 
com as principais transições possíveis sintetizada na figura \ref{fig:siegbahn},
permite identificarmos melhor os subníveis de origen e destino, por exemplo, 
a transição de $M_{IV}$ para $L_{III}$ é uma transição do tipo $L_{\beta_1}$,  . 

\begin{figure}[H]
  \centering 
  \includegraphics[width=0.5\textwidth]{../inputs/images/Siegbahn.jpg}
  \caption{Transições de elétrons entre os subníveis das camadas K, L e M. 
           Figura extraída de \citet{jenkins1991} \label{fig:siegbahn}}
\end{figure}

Na prática, dependendo do modo de detecção dos raios-X, agrupa-se transições 
dos subníveis e trabalha-se com as denominações de linhas K e L apenas. As linha
$K_{\alpha}$ são as mais intensas, oferecendo melhor limite de detecção
para um elemento. Mas elas também são as mais energéticas, podendo ultrapassar a
faixa de sensibilidade do detector empregado. Nestes casos as linhas L passam a 
ser analisadas. No modelo de XRF usado nesta pesquisa, analisou-se as linhas K, 
desde o sódio (Na) até molibdênio (Mo), e as linha L, do molibdênio (Mo)
até o chumbo (Pb).  
 
%%%%
\subsection{Tipos de XRF}

Há essencialmente dois tipos principais de equipamentos de XRF que se 
diferenciam pelo modo como os raios X são detectados: fluorescência de raios 
X dispersivo em comprimento de onda (XRF-WD) e fluorescência de raios X 
dispersivo em energia (XRF-ED).

Na XRF-WD os raios X da amostra sofrem difração em um cristal, quantificando-se
os elementos químicos pela contagem de fótons nos ângulo de difração $\theta$, 
característicos dos elementos, segundo a lei de Bragg:

\begin{equation}
	\label{eq:bragg}
	2d sen(\theta) = n \lambda
\end{equation}

Onde, $d$ a distância entre planos do cristal, $\theta$ o ângulo de incidência 
em relação ao plano considerado, $\lambda$ o comprimento de onda 
(e, portanto, a energia) da radiação incidente e $n$ um inteiro.
Esse tipo de equipamento permite uma grande resolução dos comprimentos de onda 
(energia) característicos dos elementos, facilitando detectá-los e, geralmente, 
com bom limite de detecção (LD).
Entretanto apresentam diversos problemas práticos quando empregados em filtros 
de aerossol atmosférico. Em função disso há uma preferência pelos sistemas 
dispersivos em energia nesse campo de análise. 

A XRF-ED usa detectores de semicondutores capazes de discriminar energias 
próximas com alta resolução temporal, viabilizando a detecção simultânea dos 
elementos químicos através da amplitude do pulso eletrônico produzido no 
detector, proporcional à energia do fóton incidente. O sistema eletrônico do 
equipamento faz a conversão analógica-digital da intensidade do pulso, 
acumulando a contagem por energia em um multicanal. O detector mais empregado é 
o de silício ativado com lítio Si(Li). 

Neste sistema o LD das análises é particularmente limitado pelos raios-X de 
excitação que sofrem reflexão na amostra e também chegam ao detector. 
Formam assim uma contagem de fundo (\textit{background}) que concorre com 
aquela proveniente dos picos característicos. 
Fluorescência de raios X polarizada (XRF-EDP) e fluorescência de raios X por 
reflexão total (XRF-T) são dois sistemas que têm sido empregados 
para reduzir o \textit{background} e melhorar significativamente o LD.

Na XRF-EDP excita-se a amostra com raios X polarizados e se ajusta o ângulo 
de detecção a 90 $\degree$ deste feixe \cite{dzubay1974}. Nesta direção a 
reflexão do feixe incidente polarizado é pequena, obtendo-se grande redução 
na intensidade do fundo e, consequentemente, no LD, tipicamente fator 2 a 10
dependendo do elemento e das condições de irradiação comparadas 
(Meel K. V., 2009).

%Meel, 2009
%Meel, Katleen Van, 2009. Application of high-energy polarized-beam energy-dispersive X-ray fluorescence for industrial and environmental purposes. Doctoral Thesis, Universiteit Antwerpen, Faculteit Wetenschappen, Departement Chemie. Orientador: René Van Grieken; p. 154.  

A XRF-T emprega a propriedade da radiação eletromagnética incidente abaixo do
ângulo crítico sobre uma superfície. Incidindo-se um feixe que resvale a amostra
com um ângulo bem pequeno, este excitará uma fina camada da superfície, 
penetrando muito pouco no suporte. Desta forma um detector posicionado 
perpendicularmente à superfície receberá muitos fótons característicos gerados 
pela amostra e pouco fundo refletido no suporte, melhorando o LD
\citep{yoneda1971} e \citep{aiginger1974}.

%%%%
\subsection{Características da XRF-ED utilizada}

Neste trabalho foi utilizado uma XRF-ED da marca Shimadzu modelo EDX 720HS, 
apresentado na figura \ref{fig:xrfed_iag},
pertencente ao Laboratório de Poluição atmosférica Experimental (LAPAE) 
da Faculade de Medicina da USP e instalado no LAPAt. 

\begin{figure}[H]
  \centering
  \includegraphics[width=0.5\textwidth]{../inputs/images/xrf-ed-IAG-USP.jpg}
  \caption{ED-XRF Shimadzu modelo EDX 720HS - LAPAt \label{fig:xrfed_iag}}
\end{figure}

Um tubo de ródio (Rh) submetido a uma diferença de potencial 
de 50 kV foi utilizado para geração do feixe de raios X.
O detector de silício ativado com lítio Si(Li) possui sensibilidade
para medida de fótons com energia entre 1 e 20 keV acoplado a um sistema 
eletrônico com multicanal de 2048 canais capaz de quantificar simultaneamente 
os elementos desde o Na até o Pb.
Um filtro de alumínio Al foi posto entre o feixe e a amostra para remover
a radiação da linha L dos raios X do tubo de ródio, ~2,6 keV, melhorando o 
limite de detecção dos elementos com energia igual ou menor que esta. 
O diâmetro do feixe de raios X era definido por um colimador de 10 mm, 
garantindo a irradiação de uma área representativa e homogênea da amostra. 

O tempo vivo de irradiação de cada amostra foi $\pm$ 960 minutos, ajustando-se 
a corrente para manter o tempo morto em 20\%. Desejava-se desta forma fixar a 
taxa de contagem, obtendo-se ao final o mesmo número total de fótons contados 
em cada espectro. Esse mecanismo permite melhorar o LD dos elementos presentes 
em amostras menos carregadas. Isso nem sempre foi possível já que a corrente 
máxima no tubo gerador de raios X era de 1000 $\mu A$. 

Na figura \ref{fig:xrfed_software}, extraída do software que acompanha o 
equipamento da Shimadzu, pode-se verificar em tempo real características da
análise como:  voltagem e corrente no tubo, filtro, informação sobre o vácuo na
câmera, dentre outros dados que ajudam o pesquisador a conferir se 
a análise está sendo realizada como planejado. 

\begin{figure}[H]
  \centering
  \includegraphics[scale=0.4]{../inputs/images/edx_iag_monitor.png}
  \caption{Software da ED-XRF Shimadzu 720HS, tela de verificação 
           do estado do equipamento. \label{fig:xrfed_software}}
\end{figure}

O EDX 720HS permite análise automática de amostras encaixadas em carrosséis 
de 8 ou 16 posições. O LAPAt contava então apenas com o carrossel de 16 
posições. Foi necessário adquirir um com 8 posições, cujos receptáculos admitiam
o maior diâmetro dos filtros de PTFE, que não podem ser cortados e remontados 
como os de policarbonatos (figura \ref{fig:carrossel8}).

\begin{figure}[H]
  \centering
  \includegraphics[width=0.5\textwidth]{../inputs/images/carrossel8.jpg}
  \caption{Carrossel de 8 posições para do XRF-ED da Shimadzu 720HS
           - LAPAt. \label{fig:carrossel8}}
\end{figure}

Para eliminar as ondulações típicas que ocorrem nos filtros de PTFE, 
projetou-se um suporte de aço inox que comprimia sua moldura, mantendo plana 
a superfície a ser analisada. Ao mesmo tempo, o peso deste suporte impedia que 
o filtro "voasse" quando era feito ou quebrado o vácuo na câmera 
(figura \ref{fig:suporte8}).

\begin{figure}[H]
  \centering
  \includegraphics[width=0.5\textwidth]{../inputs/images/suporte8.jpg}
  \caption{Suporte para filtros de PTFE, projetado no LAPAt para carrossel de 
           8 posições e produzido pela oficina da FAP no 
           Instituto de Física da USP. \label{fig:suporte8}}
\end{figure}

%%%%
\subsection{Calibração do ED-XRF}

A massa por unidade de área, depositada sobre os filtros tipicamente usados em 
pesquisas de aerosol atmosférico, permite tratá-los como amostras finas,
ou seja, os efeitos de matriz, interações dos raios X característicos com os 
elementos da amostras causando absorção ou reforço do número de fótons de 
raios X característicos, são pequenos diante da precisão do método,
podendo-se desconsiderá-los. Nesta aproximação, pode-se relacionar de modo 
simples o número de fótons contados sob o pico característico de um elemento, 
presente no espectro obtido, com a sua massa na amostra irradiada:

\begin{equation}
  \label{eq:contagem}
  N(Z) = R(Z) \cdot I \cdot \Delta t  \cdot \frac{m(Z)}{A} ,
\end{equation}

onde $R(Z)$ é chamado fator de resposta, $I$ é a corrente de excitação do tubo 
de raios X, determinando, portanto, o fluxo de raios X que chegam à amostra, 
$\Delta t$ é o tempo vivo de irradiação e $m(Z)/A$ é a densidade 
(massa por unidade de área) do elemento Z na amostra. 
Nota-se pressupõe-se distribuição uniforme da massa na superfície do filtro.

O termo $R(Z)$ depende da seção de choque de cada elemento com o feixe de 
raios X incidente (incluindo modificações por eventuais filtros moduladores 
de suas características), da curva de eficiência do detector de raios X, 
da eficiência de operação do tubo de raios X e da geometria do sistema, 
incluindo os diâmetros de colimadores selecionados. Mantendo estes parâmetros 
fixos, pode-se calcular $R(Z)$ a partir da equação \ref{eq:contagem}, 
irradiando-se alvos padrões com densidades ($d(Z) = m(Z)/A$) conhecidas:

\begin{equation}
  \label{eq:fator_de_resposta}
  R(Z) = \frac{N(Z)}{d(Z) \cdot I \Delta t}
\end{equation}

Considerando que a incerteza da corrente e do tempo vivo são desprezíveis 
perto da incerteza da densidade e da contagem, a incerteza no fator de resposta
pode ser calculada por propagação de erro destas variáveis:

\begin{equation}
  \label{eq:erro_fator_de_resposta}
  \sigma_{R(Z)}^2 = {R(Z)}^2 \cdot \left( \left(\frac{\sigma_{N(Z)}}{N(Z)}\right)^2 + 
                                      \left(\frac{\sigma_{d(Z)}}{d(Z)}\right)^2 
                                   \right)
\end{equation}

Dados ambientais são reportados em medidas de concentração ($\mu g/m^3$),
razão da massa $m(Z)$ pelo volume amostrado .
Conhecendo-se o fator de resposta $R(Z)$, pode-se calcular a massa $m(Z)$ depositada na amostra, usando a equação \ref{eq:contagem}, tomando como valor de $A$ a área de deposição no filtro de aerossol:

\begin{equation}
  \label{eq:xrfedmassa}
  m(Z) = \frac{N(Z) \cdot A}{ R(Z) \cdot I \cdot \Delta t}
\end{equation}

Empregando-se novamente a expressão da propagação de erro para variáveis 
independentes, a incerteza na massa será:

\begin{equation}
  \label{eq:erro_massa}
  \sigma_{m(Z)}^2 = {m(Z)}^2 \left( \left(\frac{\sigma_{N(Z)}}{N(Z)}\right)^2 + 
                                  \left(\frac{\sigma_A}{A}\right)^2 + 
                                  \left(\frac{\sigma_{R(Z)}}{R(Z)}\right)^2 
                             \right)
\end{equation}


Vê-se, entretanto, que para cada espécie química de interesse, seria necessário ao menos um alvo de calibração para calcular seu $R(Z)$. Alvos padrões finos, com densidade elementar conhecida, podem ser comprados ou produzidos, dependendo da precisão desejada. Neste projeto foram comprados alvos padrões da \textbf{Micromatter}, que promete uma incerteza de 5\%.

Mas esta empresa não produz alvos com proporção estequiométrica quantificada para todas espécies químicas.%
%
%Você sabe o que isso significa?
%
 Assim, trouxemos para o \textbf{ED-XRF} o conceito proposto por \citep{tabacniks2000}
para o sistema \textbf{PIXE (Particle Induced X-Ray Emission)}, que permite obter uma calibração abrangendo elementos de número atômico 
$ 10 < Z < 83$ %
%
%você sabe explicar o porquê destes limites em Z?
%
nas análises de \textbf{ED-XRF}. 
Empregou-se, ainda, uma metodologia estatística robusta para estimativa das incertezas para a calibração.

Essa preocupação em definir adequadamente as incertezas deve-se, particularmente, ao fato de que elas ponderam o peso das variáveis na modelagem por PMF.

No gráfico da figura \ref{fg:edxrfcalib} estão plotados os $R(Z)$ 
dos alvos padrões da \textbf{Micromatter} irradiados em Maio de 2010. 

\begin{figure}[H]
  \begin{subfigure}[b]{0.45\textwidth}
    \includegraphics[width=\textwidth]{../outputs/plot_R_maio2010K.pdf}
    \caption{Linha K}
  \end{subfigure}%
  \begin{subfigure}[b]{0.45\textwidth}
    \includegraphics[width=\textwidth]{../outputs/plot_R_maio2010L.pdf}
    \caption{Linha L}
  \end{subfigure}
  \caption{Fatores de respostas $R(Z)$ dos para alvos padrões da 
           \textbf{Micromatter} irradiados em Maio de 2010. 
           \label{fg:edxrfcalib}}
\end{figure}

Nota-se que é possível fazer um ajuste polinomial sobre esses dados, o que tanto melhora a precisão para todos os $R$, quanto fornece seu valor para elementos que não possuem alvos padrões.

Como o fator de resposta reflete o arranjo experimental, mudanças físicas
nesse arranjo alteram o valor de R, assim como a progresiva fadiga do detector
ou do tubo. Portanto, a calibração deve ser realizada periodicamente.



Amostra espessas apresentam o efeito matriz, ou seja, interações dos 
raios X característicos com os elementos da amostras, causando 
absorção do raios X ou mesmo reforço de raios X. 
Para os filtros usados na amostragem, Teflon, será considerado filtro
fino, ou seja, fino o suficiente para desconsiderarmos o efeito matriz.

Alvos padrões, com densidade elementar conhecida, podem ser 
comprados ou produzidos, dependendo da precisão desejada.
Neste projeto foram comprados alvos padrões da \textbf{Micromatter}
com incerteza de 5\%. 
Não há alvos para todas espécies possíveis de se encontrar na atmosfera. 

Assim, usou-se um método de calibração proposto por \citep{tabacniks2000}
para sistema \textbf{PIXE (Particle Induced X-Ray Emission)} com pequenas
adaptações no contexto da \textbf{ED-XRF}.   
A calibração permitiu abrangência de todos elementos de número atômico 
$ 10 < Z < 83$ nas análises de \textbf{ED-XRF}. 
Desensolve-se ainda, uma metodologia para estimativa das incertezas
para a calibração.   

O \textbf{PIXE (Particle Induced X-Ray Emission)} é
outro método comumente usado em análises ambientais. 
Usa feixe de íons (prótons ou alfas) para excitação dos 
átomos das amostras.

\begin{figure}[H]
\begin{center} 
  \includegraphics[width=0.5\textwidth]{../inputs/images/arranjopixe.png}
  \caption{Arranjo experimental básico para análise de método PIXE 
           \citep{tabacniks2000} \label{fig:arranjopixe}}
\end{center}
\end{figure}

Dado o arranjo experimental da figura \ref{fig:arranjopixe} e
considerando o filtro fino (alguns $\mu m$),
\citep{tabacniks2000} chega na equação \ref{eq:npixe} para 
quantidade de raios X $N$ contadas no detector. 

\begin{equation}
  \label{eq:npixe}
  N(Z) = \frac{\Omega}{4\pi} \sigma \zeta T t_z \frac{Q}{qe}
\end{equation}

Sendo $Z$ a espécie química, o número de raios X detectados 
$N(Z)$ é proporcional à densidade (massa ou átomos por área) $t_Z$ 
e a carga coletada $Q$.
$\zeta$ é a eficiência do detector, $\sigma_x$ é a secção de choque, 
$\Omega$ é o ângulo sólido, $T$ é a transmitância para raios-X em 
caso de uso de absorvedores (colocados entre a amostra e o detector), 
$q$ é o estado de carga da partícula incidente e 
$e$ é a carga do elétron \citep{tabacniks1983}.

Para resolver a equação \ref{eq:npixe}, parâmetros do arranjo experimental
($\Omega$, $\sigma$, $\zeta$ e $T$) deveriam ser conhecidos. 

Na prática, é mais comum é irradiar alvos de calibração com $t_z$ conhecidos,
e encontrar uma variável única proporcional aos parâmetros do arranjo experimental.
Da-se o nome de fator de resposta $R(Z)$ a essa variável.

Assim, adaptando a nomenclatura da equação \ref{eq:npixe} para o contexto 
da \textbf{ED-XRF} e incluindo o fator de resposta $R(Z)$ chegamos na equação 
\ref{eq:contagem}.

\begin{equation}
  \label{eq:contagem}
  N(Z) = R(Z) I\Delta t \frac{m(Z)}{A}
\end{equation}

$R(Z)$ é o fator de resposta, $I\Delta t$ é a carga efetiva expressa
em termos da corrente e do tempo efetivo de análise.
A densidade $t_z$ está representada pela razão da massa $m(Z)$ pela 
área de deposição $A$ (suposição: a massa está uniformemente distríbuida 
na superfície do filtro).

Isolando-se $R(Z)$ em \ref{eq:contagem} obtém-se a equação para cálculo 
do fator de respota \ref{eq:fator_de_resposta}.
Irradiando-se alvos padrões - que possuem densidades $m(Z)/A$ conhecidas - 
mede-se $N(Z)$ e $I \Delta t$. Assim, são calculados fatores de repostas $R(Z)$ 
para os respectivos alvos padrões. 

\begin{equation}
  \label{eq:fator_de_resposta}
  R(Z) = \frac{N(Z) A}{m(Z)I \Delta t}
\end{equation}

Considerando que a incerteza da corrente e do tempo vivo são
desprezíveis perto da incerteza da massa, da contagem e da área, 
a incerteza do fator de resposta pode ser calculada usando a expressão
de propagação de erro para variáveis independentes conforme equação 
\ref{eq:erro_fator_de_resposta}.

\begin{equation}
  \label{eq:erro_fator_de_resposta}
  \sigma_{R(Z)}^2 = {R(Z)}^2 \left( \left(\frac{\sigma_{N(Z)}}{N(Z)}\right)^2 + 
                                  \left(\frac{\sigma_A}{A}\right)^2 + 
                                  \left(\frac{\sigma_{m(Z)}}{m(Z)}\right)^2 
                             \right)
\end{equation}

Dados ambientais são reportados em medidas de concentração,
razão da massa $m(Z)$ pelo volume amostrado ($\mu g/m^3$).
Conhecendo-se o fator de resposta $R(Z)$, pode-se calcular a massa $m(Z)$ 
pela equação \ref{eq:xrfedmassa}, que nada mais é que o isolamento de $m(Z)$ na 
\ref{eq:contagem}. 

\begin{equation}
  \label{eq:xrfedmassa}
  m(Z) = \frac{N(Z) A}{ R(Z)I \Delta t}
\end{equation}

Empregando-se novamente a expressão da propagação de erro para variáveis independentes, 
a incerteza na massa pode ser calculada pela equação \ref{eq:erro_massa}.

\begin{equation}
  \label{eq:erro_massa}
  \sigma_{m(Z)}^2 = {m(Z)}^2 \left( \left(\frac{\sigma_{N(Z)}}{N(Z)}\right)^2 + 
                                  \left(\frac{\sigma_A}{A}\right)^2 + 
                                  \left(\frac{\sigma_{R(Z)}}{R(Z)}\right)^2 
                             \right)
\end{equation}


Entretanto, para cada espécie química de interesse, seria necessário 
o alvo de calibração correspondente.

Nem sempre é possível adiquirir alvos de calibração para todos elementos
químicos e a incerteza garantida pela \textbf{Micromatter}) é de 5\%. 

No gráfico da figura \ref{fg:edxrfcalib} estão plotados os $R(Z)$ 
dos alvos padrões da \textbf{Micromatter} irradiados em Maio de 2010. 

\begin{figure}[H]
  \begin{subfigure}[b]{0.45\textwidth}
    \includegraphics[width=\textwidth]{../outputs/plot_R_maio2010K.pdf}
    \caption{Linha K}
  \end{subfigure}%
  \begin{subfigure}[b]{0.45\textwidth}
    \includegraphics[width=\textwidth]{../outputs/plot_R_maio2010L.pdf}
    \caption{Linha L}
  \end{subfigure}
  \caption{Fatores de respostas $R(Z)$ dos para alvos padrões da 
           \textbf{Micromatter} irradiados em Maio de 2010. 
           \label{fg:edxrfcalib}}
\end{figure}

Nota-se que é possível fazer um ajuste polinomial nos dados, o que 
permite encontrar $R$ para elementos que não possuem alvos padrões.

O fator de resposta reflete o arranjo experimental. Mudanças físicas
nesse arranjo alteram o valor de R, assim como desgates no detector
ou tubo. Portanto, a calibração deve ser realizada com frequência.

%%%%
\subsection{Estimativa das incertezas}

Utilizou-se o \textbf{Ajuste dos Mínimos Quadrados Matricial} 
para a estimativa do erro do ajuste \citep{helene2006}.

Dada as variáveis $Y_i$ e $X_i$ relacionadas polinomialmente, 
conforme \ref{eq:polinomio}.

\begin{equation}
  \label{eq:polinomio}
  \begin{split}
    y_1 = a + b x_1 + c{x_1}^2 + d{x_1}^3 + ...\\
    y_2 = a + b x_2 + c{x_2}^2 + d{x_2}^3 + ... \\
    ...
  \end{split}
\end{equation}

A representação matricial da equação \ref{eq:polinomio} pode 
ser escrita como \ref{eq:polinomioMatriz}.

\begin{equation}
  \label{eq:polinomioMatriz}
  [Y] = [A][X]
\end{equation}

Os coeficientes ajustados $[Ã]$ são dados pela equação \ref{eq:coeficientesajustados},
que depende da matriz de covariância dos coeficientes $[V_{Ã}]$, 
dada pela equação \ref{eq:matrizcovariancia}.

\begin{equation}
  \label{eq:coeficientesajustados}
  [Ã] = [V_{Ã}] ([X]^T {[V_Y]}^{-1} [Y])
\end{equation}

\begin{equation}
  \label{eq:matrizcovariancia}
  [V_{Ã}] = ([X]^T [V_Y]^{-1} [X])^{-1}
\end{equation}

Com o coeficientes ajustados $[Ã]$ pode-se calcular os 
$[\tilde{Y}]$ ajustado \ref{eq:polinomioajustado}.

\begin{equation}
  \label{eq:polinomioajustado}
  [\tilde{Y}] = [Ã][X]
\end{equation}

Por fim, a incerteza é dada pela diagonal da matriz de covariância 
de $[\tilde{Y}]$, $[V_{\tilde{Y}}]$ em \ref{eq:matrizcovarianciaY}.

\begin{equation}
  \label{eq:matrizcovarianciaY}
  [V_{\tilde{Y}}] = [X] [V_{Ã}]^{-1} [X]^{-1}
\end{equation}

%%%%
\subsection{Fontes de erro no branco}

A massa depositada no filtro amostrado $m(Z)_{medido}$ para um certo 
elemento $Z$, é composta pela massa coletada na amostragem $m(Z)$ 
mais a massa do filtro branco $m_{B}(Z)$. 

Um conjunto de 10 amostras brancas (campo e laboratório) foram analisadas, 
para eliminação da contaminação dos próprios filtros, assim como de 
transporte e manipulação.

A equação \ref{eq:contagem} pode ser escrita para representar essa situação
\ref{eq:contagem_medida}. 

\begin{equation}
  \label{eq:contagem_medida}
  N(Z)_{medido} = R(Z) I\Delta t \left( \frac{m(Z)}{A} + \frac{m_B(Z)}{A} \right)
\end{equation}  

Escrevendo a equação \ref{eq:contagem} para os brancos temos \ref{eq:contagembranco}:

\begin{equation}
  \label{eq:contagembranco}
  N_B(Z) = R(Z) I_B\Delta t_B \frac{m_B(Z)}{A}
\end{equation}

Isolando-se $m_B(Z)$ em \ref{eq:contagembranco} e substituindo em 
\ref{eq:contagem_medida}, encontramos \ref{eq:contagemcorrigida}:
 
\begin{equation}
  \label{eq:contagemcorrigida}
  N(Z) = N(Z)_{medido} - I\Delta t \left( \frac{N_B}{I_B \Delta t_B} \right)
\end{equation}


A incerteza da contagem corrigida \ref{eq:contagemcorrigida}, 
usando-se propagação de erros para variáveis independentes fica
\ref{eq:erro_contagemcorrigida}.

\begin{equation}
  \label{eq:erro_contagemcorrigida}
  \sigma_{N(Z)}^2 = \sigma_{N(Z)_{medido}}^2 + \left( \frac{I \Delta t}{I_B \Delta t_B} \right)^2 \sigma_{N_B}^2
\end{equation}

Usa-se a média da razão $\frac{N_B}{I_B \Delta t_B}$ dos $n$ alvos brancos
selecionados para o experimento. 

%TODO: incluir discussão sobre incerteza do Otaviano

%Temos duas fontes possíveis de erro.

%1) Erro estatístico. Quando se tem N brancos, a concentração do elemento considerado será a média sobre os N dados. O erro será o desvio padrão (p não é o desvio padrão da média), porque é a chance de termos esse erro em cada filtro analisado e não sobre a média. Para considerar o sentido do desvio padrão (68,3\% de chance de pertencer ao amostral) ainda precisa haver um fator de correção quando N for pequeno. Seria bom corrigir para N<=10. Dai em diante pode usar fator 1, porque a correção seria pequena.

%$\sigma_{pc} = fatorr x \sigma_p$

%\begin{table}[]
%\centering
%\caption{My caption}
%\label{my-label}
%\begin{tabular}{lllllllllll}
%N     & 2    & 3    & 4    & 5    & 6    & 7    & 8    & 9    & 10   & 20   \\
%Fator & 1,84 & 1,32 & 1,20 & 1,14 & 1,12 & 1,11 & 1,09 & 1,08 & 1,06 & 1,03
%\end{tabular}
%\end{table}

%2) Erro da integração do espectro. Esse erro é calculado pelo erro na integração de cada espectro de branco e, obviamente, tem que ser calculado sem branco, só com o fator de resposta (ou seja, a rotina que calcula a concentração - em µg/cm - tem que ser igual para o alvo comum e para o branco; mas o alvo comum tem que ter uma complementação no cálculo para subtrair o branco e adicionar o erro, devido ao branco). Havendo N alvos brancos, com erro  i de integração para cada um deles, o erro transferido para a média deve ser:

%\sigma_e = \sqrt{\frac{\Sigma \sigma_i}{N} }

%3) Erro total no branco. O erro final em cada elemento do branco é obtido considerando que 1 e 2 (acima) são independentes, logo:

%$\sigma_{r} = \sqrt{\sigma_e^2 + \sigma_pc^2} $

%%%%
\subsection{Integração dos Espectros}

Realizou-se a integração de todos os espectro obtidos na \textbf{ED-XRF} no
WinQxas \textbf{Quantitative X-Ray Analysis System for Widows},
programa desenvolvido para integração numérica de espectros. 

O \textbf{WinQxas} foi elaborado sob o patrocínio da Agência Internacional 
de Energia Atômica (IAEA) \citep{capote2000}.

Obtém-se os parâmetros iniciais da relação linear entre canal e energia,
conhecendo-se ao menos dois picos no espectro, normalmente Ferro e Cálcio, 
no caso de poluição do ar ambiental. 

Os parâmetros iniciais entre a largura do pico à meia altura (FWHM),
dependente da energia (E), do nível geral de ruído (NOISE) no espectro 
e de um termo (FANO), dado pela relação \ref{eq:fwhm}. 

\begin{equation}
  \label{eq:fwhm}
   FWHM^2 = NOISE^2 + 2,35 x FANO  x E
\end{equation}

A largura à meia altura (FWHM) para o Ferro $K\alpha$ foi 138.32 eV
Cálcio 129.14

Os parâmetros iniciais foram determinados a partir de espectros com picos bem definidos.

Na figura \ref{fig:winqxas} há dois exemplos de espectros abertos no \textbf{WinQxas}, 
um de uma amostra branca e outro de uma amostra carregada. 
Os picos característicos de elementos químicos encontrados estão indicados na figura.

\begin{figure}[H]
  \centering
  \begin{subfigure}[b]{0.7\textwidth}
    \includegraphics[width=\textwidth]{../inputs/images/winqxas/GHA41editado.pdf}
    \caption{Espectro de amostra carregada (GHA41) - Acra Nima}
  \end{subfigure}
  \begin{subfigure}[b]{0.7\textwidth}
    \includegraphics[width=\textwidth]{../inputs/images/winqxas/GPE770editado.pdf}
     \caption{Espectro de amostra branca (GPE770)- Acra Nima}
  \end{subfigure}
  \caption{Espectros amostra branca e carregada \label{fig:winqxas}}
\end{figure}

O fundo do espectro, gerado pela radiação emitida no tubo de ródio
devido a aceleração e desaceleração de elétrons, dificulta a distinção 
dos picos e altera o limite de detecção. 

Na análise feita nos espectros, feita um a um, deve-se ter atenção 
para identificação e correção dos seguintes eventos: 

\begin{itemize}
  \item pico soma: quando fotóns do diferente elementos são contados
        juntos pelo detector. 
  \item pico escape: quando o fóton incidente excita o silício do detector
        e é contato com a energia de incidência menos a energia usada para excitar o silício. 
\end{itemize}

Calibração do multicanal (ajuste linear entre canal e energia 
encontrado a partir de dois elemento conhecidos do espectros.): 

E = 0,0101 Canal - 0,1335 (keV)

%%%%
\subsection{Limite de Detecção}

O limite de detecção é a densidade mínima para haver 
detecção da espécie no tipo de filtro e \textbf{ED-XRF} considerados. 

Irradiando-se uma amostra branca obtém-se o número de contagens 
medidas ($N_{fundo}$) sob o pico (fundo do espectro) de todo elementos.

A medida realizada pelo detector segue uma \textbf{distribuição de Poisson}
pois realiza contagens aleatórias em torno do valor médio da energia 
no pico. O desvio padrão da contagem $N$ é portanto $\sqrt{N}$.

Dada a natureza do detector, o número mínimo de contagens para se 
distiguir um pico é $N_{LD}$, calculado usando \ref{eq:limitedeteccao} 
(distribuição de Poisson).

\begin{equation}
  \label{eq:limitedeteccao}
  N_{LD} = 3 x \sqrt{N_{fundo}}
\end{equation}

Pode-se calcular o limite de detecção em termos da massa 
elementar usando \ref{eq:xrfedmassa}.
O limite de detecção muda conforme a quantidade de material coletado,
pois em amostras carregadas, a radiação gerada no tubo aumenta, e 
portanto aumenta o fundo, aumentando o limite de detecção. 
Assim, o ideal é calcular o limite de detecção
em amostra branca e carregada.

O limite de detecçãodo branco foi usado para preechimento dos valores faltantes.  
Os valores faltantes foram preenchidos com o valor da metade do 
limite de detecção $(LD/2)$, pois supoe-se que a concentração não 
detectada de um elemento que aparece frequentemente nas demais amostras 
esteja entre 0 e o limite de detecção, com igual probabilidade de ocorrência 
para cada valor. Assim, $LD/2$ seria a média e, portanto, o valor mais 
provável entre estes valores não detectados. 

Para a incerteza das concentrações, usou-se o valor proposto por
\citep{polissar1998} de $5LD/6$.
