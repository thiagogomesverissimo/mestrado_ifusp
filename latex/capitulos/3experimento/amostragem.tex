%%%%
\section{Amostragem}
%TODO: citar artigo do harvardinho (Impactor Desing)
% http://maps.google.com/maps/ms?ie=UTF8&hl=en&msa=0&msid=116003586198857296821.00046d7e7367b947abe12&z=12

Foram coletadas, entre Novembro de 2006 e Agosto de 2008, 858 amostras válidas, 
de 48 horas, em dois sítios de \textbf{Nima}, sendo um localizado perto da rodovia
(+5° 34' 54.00", -0° 11' 56.30") e outro parte residencial do 
bairro (+5° 35' 2.00", -0° 11' 58.80").

A coleta foi diária, mas houve dias sem coleta devido a falta de eletricade e
filtros danificados no transporte ou na troca no amostrador. 
O tempo de amostragem de 48 horas, foi definido antes da entrada \textbf{USP} 
no projeto e trouxe dificuldades nas análises multivaridas, pois o
ciclo de 48h para a amostragem dificulta a captação das especificidades 
de cada uma das fontes.

As amostras de foram coletadas em filtro de teflon do tipo 
\textit{PTFE} de 37 $mm$ de diâmetro orifícios de 0,2 $\mu m$ de diâmetro. 

Para $MP_{10}$ utilizou-se impactador Harvard com $D_{50}$ de $10 \mu m$ 
com fluxo de $10,0 L/min$ \cite{marple1987}. 
Nas medidas de $MP_{2,5}$, também utilizou-se impactador Harvard, 
mas com $D_{50}$ de $2,5 \mu m$ acoplado com um \textit{inlet} 
responsável fazer a filtragem de $MP_{2,5}$.

Filtros brancos de campo e laboratórios foram separados para avaliar 
possíveis contaminações no tranporte e manipulação das amostras. 

O laboratório da \textit{Harvard School of Public Health} foi
utilizada para pesagem das amostras.
Os filtros foram pesados antes e depois da coleta, usando uma balança 
microanalítica \textit{(Mettler Toledo MT5)} com precisão de $1 \mu g$, 
seguindo procedimentos padrões de controle de umidade ($39 \pm 2 \%$), 
temperatura ($20,5 \pm 0,2 ^{\circ} C$) e eliminação de cargas eletrostáticas 
(fonte de polônio).
Cada pesagem foi realizada duas vezes e o valor final foi a média das 
duas medidas.

As concentrações foram calculadas a partir dos volumes 
medidos por um integrador de volume.



