\section{Tratamento dos dados}

Nesta seção está contida todas modificações, pressupostos e considerações 
realizadas nos dados antes da análises estatísticas.
As informações da amostragem em \textbf{Nima}, assim como os resultados obtidos
estão distribuídos nas seguintes tabelas: 

\begin{itemize}
  \item Concentrações elementares obtidas por análise \textbf{EDXRF};
  \item Concentração de Black Carbon para $MP_{2,5}$, obtido por refletância;
  \item Concentrações de massa obtidas na análise gravimétrica ;
  \item Tabela de informações da amostragem com os campos:
  \begin{itemize}
    \item Horário de trocas dos filtros no amostrador;
    \item Volume medido no totalizador;
    \item Anotações de possíveis avarias ou contaminações do filtro.
          durante o processo de troca
  \end{itemize}
\end{itemize}

Modificações realizadas nos dados:
\begin{itemize}
\item Adicionada uma coluna para erro na massa total de 5\%.
\item Substituição dos faltantes pelo limite de detecção nas 
      concentrações elementares
\end{itemize}

Amostras:
\begin{table}[H]
  \centering
  \input{../outputs/samples}
  \caption{Diagnósticos das amostras}
\end{table}

As filtros de $MP_{2,5}$ e $MP_{10}$ coletados na duas regiões nos permitem 
estudar as fontes predominantes de poluição do ar a partir de 4 
conjuntos de dados principais: 

\begin{itemize}
  \item $PM_{2.5}$ no bairro com ruas não pavimentadas
  \item $PM_{10}$ no bairro com ruas não pavimentadas
  \item $PM_{2.5}$ na região urbana
  \item $PM_{10}$ na região urbana
\end{itemize}

Nos meses em que ocorre o Harmatão, entre Dezembro e Março, há grande aumento da
poluição do ar, principalmente na concentração de $PM_{10}$, devido a poeira 
provinda do deserto do Saara. 

No período do Harmatão, devido as altas concentrações, as fontes locais sejam 
ocultadas nas análises multivariadas, assim, além dos 4 conjuntos de dados 
supracitados, as análises multivariadas serão também realizadas nos mesmos 4 
conjuntos de dados, mas excluindo-se as amostras do período do Harmatão, 
resultando em 8 bancos de dados disponíveis para análises multivariadas. 

Agrupando-se as amostras de $PM_{10}$ dos dois locais de amostragem com e sem 
o período do Harmatão e fazendo o mesmo para $PM_{2.5}$, resulta-se em mais 4 
banco de dados disponíveis para análise multivariada, elevando nossa bases de
dados para 12. Além disso, afim de fazer estudo específico do período do 
Harmatão, agrupou-se amostras $PM_{2.5}$ somente do período do Harmatão. 

Fez-se o mesmo para $PM_{10}$. Totalizando 14 banco de dados para as análises 
multivariadas, com mais de 100 casos cada.  

Detalhamento da nomenclatura: 
\begin{itemize}
\item Região de amostragem: 
	\begin{itemize}
	\item \textbf{R} Residencial, ruas não pavimentadas. 
	\item \textbf{T} Tráfego intenso, avenida pavimentadas, comércio urbano.
	\item \textbf{J} Agrupamento das duas regiões acima.
	\end{itemize}
\item Tipo de Filtro:
	\begin{itemize}
	\item \textbf{F} Fino
	\item \textbf{I} Inalável
	\end{itemize}
\item Indicação se os filtros incluem ou não os meses do Harmatão:
	\begin{itemize}
	\item \textbf{cH} Com Harmatão, ou seja, o conjunto de filtros 
        incluem filtros amostrados nos meses do Harmatão.
	\item \textbf{sH} Sem Harmatão, ou seja, o conjunto de filtros \textbf{não} 
        incluem filtros amostrados nos meses do Harmatão.
	\item \textbf{eH} Exclusivamente Harmatão, ou seja, o conjunto de filtros 
        incluem \textbf{somente} filtros amostrados nos meses do Harmatão.
	\end{itemize}
\end{itemize}
