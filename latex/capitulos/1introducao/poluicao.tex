%%%%
\section{Poluição do Ar}

Poluição do ar resulta de uma complexa mistura de emissões naturais e 
antropogênicas. Estima-se que a poluição do ar seja responsável por 
$3,2$ milhões de mortes por ano no mundo todo \citep{lim2013}.%
%falta uma conexão aqui entre o geral - gases e partículas e apenas partículas, posicionando que nosso trabalho concentra-se sobre partículas e porque isso.

A inalação de material particulado $MP$ exerce um papel importante na 
exacerbação de doenças respiratórias, incluindo enfisema pulmonar e asma. 
Aqueles de pequenas dimensões e massa, têm maior facilidade de penetração no sistema 
respiratório. 

A deposição no sistema respiratório humano ocorre em função do diâmetro da 
partícula.
As partículas grossas $MP_{2,5-10}$ sofrem maior retenção nas vias superiores (nasofaringe) e
as mais finas $MP_{2,5}$ têm maior facilidade em chegam nos alvéolos pulmonares e 
nos bronquíolos.
Nas áreas danificadas ocorre o comprometimento das trocas gasosas podendo, 
também, acarretar problemas cardiovasculares.
O próprio corpo consegue remover partículas através do macrófago alveolar 
e do sistema linfático \citep{arbex2012}.

%Os vírus são $MP_{2,5}$ e bactérias são $MP_{10}$ 
%o que significa que vírus tem maior penetração.

%%%%
\subsection{Material Particulado $MP$}

Material Particulado $MP$ ou Aerossóis Atmosféricos são partículas
sólidas, líquidas ou mistas em suspensão em um gás, as quais têm diâmetro 
aerodinâmico compreendido  entre $0,001-100\mu m$. 
A parte considerada inalável para humanos tem diâmetro menor que 10 $\mu m$
e se comporta praticamente como um gás.
As partículas maiores que 10 $\mu m$ têm dificuldade em penetrar 
no sistema respiratório porque a força de arraste do ar inalado através do nariz tende a não vencer a força da gravidade \citep{seinfeld1998}.%
%use como referência o Seinfeld mais novo - 2006, se constar lá.

O $MP$ pode ser classificado por tamanho, formação 
(primária ou secundária), pela forma de remoção da atmosfera, 
composição química ou forma geométrica da partículas \citep{seinfeld1998}.%
%veja se a versão nova dá conta disso - 2006

A divisão por tamanho tipicamente separa o $MP$ menor de $10 \mu m$ em quatro modas:
ultra-finas, núcleos de Aitken, acumulação e grossa. 

\begin{itemize}
  \item \textbf{ultra-finas:} formada pela nucleação homogênea de vapores de baixa volatilidade;
  \item \textbf{nucleação ou núcleos de Aitken:} 
        gerada a partir da condensação de vapores quentes ou durante o processo de 
        transformação de gás em partícula. Partículas formadas por 
        nucleação são removidas da atmosfera por coagulação.   
  \item \textbf{moda acumulação:} 
         partículas na moda de acumulação são formadas 
         a partir do crescimento das partículas ultra finas e dos núcleos de condensação (núcleos de Aitken), através do processo de 
         coagulação ou de condensação de vapores. 
         Partículas formadas por acumulação
         são removidas da atmosfera por deposição seca ou úmida.%
         %na figura você colocou apenas chuva. Esse é  o principal processo de remoção desta fração
  \item \textbf{moda grossa ($MP_{2,5-10}$):} 
        são oriundas de processos mecânicos como fragmentação, 
        movimentação e manuseio. Partículas grossas são removidas da atmosfera 
        por sedimentação.
\end{itemize}

A figura adaptada a partir de proposta de \citep{finlayson1999} ilustra os processos de 
formação e remoção de cada moda.%
%confira isso, mas acho que a figura original era mais simples, por isso seria adaptação. Veja o artigo.

\begin{figure}[H]
\begin{center}
  \includegraphics[width=0.8\textwidth]{../inputs/images/modas_aerossol.png}
  \caption{Esquema da distribuição de tamanho do aerossol atmosférico 
           \citep{finlayson1999} \label{fig:modas_aerossol}}
\end{center}
\end{figure}

O material particulado fino $MP_{2,5}$ engloba as partículas 
ultra-finas, a moda de nucleação e a moda de acumulação.  

O $MP_{2,5-10}$ tende a ficar retido na parte superior do sistema respiratório, 
enquanto que $MP_{2,5}$ tem maior facilidade para penetrar e atingir 
os alvéolos pulmonares e corrente sanguínea, 
podendo comprometer significativamente a saúde humana. 

O $MP$ pode ser emitido por diferentes fontes antropogênicas ou naturais e, 
também, pode ser gerado secundariamente na atmosfera, através de 
reações fotoquímicas, por exemplo. 

O $MP_{2,5}$ representa uma maior ameça a saúde humana, pois além 
do alto poder de penetração no sistema respiratório,
é gerado pelo processo de conversão gás-partícula, resultando muitas vezes em materiais tóxicos.

No $MP_{2,5}$ encontra-se principalmente íons $SO_4^=$, 
íons $ NO_3^-$, carbono elementar, carbono orgânico, 
elementos traços como metais 
(cádmio, níquel, vanádio, zinco, cromo, ferro, mercúrio), 
sulfatos e nitratos \citep{finlayson1999}. 

Compostos como nitrato e sulfato de amônio são poluentes secundários,
pois óxido de nitrogênio $NO_x$ e amônia $NH_3$ formam nitrato de amônio. 
Dióxido de enxofre $SO_2$ e amônia $NH_3$ formam o sulfato de amônio. 

Na moda grossa $MP_{2,5-10}$ há poluentes formados por processos mecânicos, 
como poeira de solo, fuligem, polens, metais alcalinos, entre outros.%
%esse trecho está repetitivo. Precisa compor com o que já falou quando tratou da figura.
%precisa organizar uma conclusão que pode contemplar alguns dos pontos que anotou abaixo. Pode também relacionar com a pesquisa, justificando porque mediu-se MP e não algum gás específico.

%%%%
%\subsection{Meio Ambiente}
%Ao depositar em folhas, as partículas impedem a absorção de luz. Também obstruem os estômatos e, dependendo da composição atacam quimicamente o vegetal. Podem contaminar o solo e serem absorvidos pela raiz.
%Efeitos na visibilidade.
%Efeitos na formação de nuvens.


