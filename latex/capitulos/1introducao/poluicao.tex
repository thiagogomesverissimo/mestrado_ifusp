
%%%%
\section{Poluição do Ar}

Estima-se que a poluição do ar é responsável por 
3,2 milhões de mortes por ano no mundo \cite{lim2013}. 

é o poluente que tem apresentado correlações mais consistentes com
 danos à saúde originando-se principalmente da queima de combustíveis

A inalação de material particulado exerce um papel importante na 
exacerbação de doenças respiratórias, incluindo enfisema pulmonar e asma. 
Suas pequenas dimensões e massa, facilitam a penetração do MP no sistema 
respiratório, trazendo danos aos diminutos sacos de ar (alvéolos) que 
compõem os pulmões. Nas áreas danificadas ocorre o comprometimento das 
trocas gasosas podendo, também, acarretar problemas cardiovasculares
\cite{arbex2012}.

Poluição do ar resulta de uma complexa mistura de emissões naturais e 
antropogênicas. No caso de Accra a discriminação de fontes é complexa pois
há uma mistura de elementos do deserto com fontes antropogênicas.

Apesar dos altos índices de poluição em Accra pouca pesquisa tem sido 
realizada na tentativa de enteder a composição química do ar na região
(tanto antropogênica quando natural - poeira mineral do deserto). 






%%%%
\section{Aerossol Atmosférico}

Os Material Particulado $MP$ ou  Aerossóis Atmosféricos são partículas
sólidas ou líquidas em suspensão em um gás, as quais tem diâmetro 
aerodinâmico compreendido  entre $0,001-100\mu m$, 
A parte considerada inalável tem diâmetro menor que 10 $\mu m$
e se comporta praticamente como gás \cite{seinfeld1998}.

Essa fração do MP costuma ser subdividida em MP fino, com diâmetro menor que 2,5 , e MP grosso, entre  2,5 m e 10 . O MP grosso tende a ficar retido na parte superior do sistema respiratório (os maiores que 10 têm dificuldade em penetrar no sistema respiratório porque seu arraste pelo ar inalado não vence a força da gravidade), enquanto o MP fino (MP2,5) tem maior facilidade para penetrar e atingir os alvéolos pulmonares, podendo comprometer significativamente a saúde humana. O MP pode ser emitido por diferentes fontes antropogênicas ou naturais e, também, pode ser gerado secundariamente na atmosfera por reações químicas. Fontes potenciais como queimadas,  indústrias e os veículos automotivos, particularmente, emitem a maior parte dos poluentes urbano industriais.
 	


O Material Particulado pode ser classificado por tamanho, formação 
(primária ou secundária), composição química, remoção da atmosfera, 
forma da partículas (espérica, espiga).
%TODO: colocar a figura clássica do seinfeld de distribuição

Formação e remoção de partículas na atmosfera:
\begin{itemize}
  \item \textbf{moda nucleação:} núcleo de condensação são criados devido a 
        condensação de vapores quentes ou durante o processo de 
        transformação de gás em partícula. Partículas formadas por 
        nucleação são removidas da atmosfera por aglomeração. 
  \item \textbf{moda acumulação:} partículas na moda de acumulação são criadas 
         a partir de núcleos de condensação através da coagulação ou 
         condensação de vapores. Partículas formadas por acumulação
         são removidas da atmosfera por deposição seca ou úmida.
  \item \textbf{moda grossa:} são oriundas de processos mecânicos como fragmentação, 
        movimentação e manuseio. Partículas grossas são removidas da atmosfera 
        por sedimentação.
\end{itemize}

%TODO: Qual mode é mais numerosa? usar aquele trabalho do Seinfeld da Fátima?

Na moda fina encontra-se principalmente: íons $SO_4^=$, 
íons $ NO_3^-$, carbono elementar, carbono orgânico, compostos orgânicos elementar
condensados, metais (cádmio, níquel, vanádio, zinco, cromo, ferro, mercúrio), 
sulfatos, nitratos, nitrato amônia. 

O oxido de nitrogênio (NOx) e amônia NH3 formam  nitrato de amônio e 
o dióxido de enxofre (SO2) e amônia NH3 formam o sulfato de amônio, 
portanto são secundário. 

Na moda grossa encontra-se principalmente: poeira de solo, fuligem, 
pólen, Si, Al, ferro, K, Ca e metais alcalinos.

%%%%
\section{Saúde}

A deposição no sistema respiratório humano ocorre em função do diâmetro da partícula.
As partículas mais finas chegam nos bronquíolos e as maiores só alcançam os alvéolos.
As maiores ficam no nariz (traqueobrônquica) nasofaringe. 
Os vírus são $MP_{2,5}$ e bactérias são $MP_{10}$. 

O próprio corpo consegue remover partículas através do Macrófago alveolar 
e do sistema linfático. 

%%%%
\section{Meio Ambiente}
Ao depositar em folhas, as partículas impedem a absorção de luz. 

Efeitos na visibilidade.

Efeitos na formação de nuvens.


