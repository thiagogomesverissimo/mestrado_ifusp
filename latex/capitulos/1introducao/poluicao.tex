\section{Poluição do Ar}

Poluição do ar urbana é uma complexa mistura de emissões e naturais e 
antropogenicas.  

O que é material particulado?

Tipos de classificações de Material Particulado: tamanho, por formação 
(primária ou secundária), 
composição química, remoção da atmosfera. (colocar a figura clássica do seinfeld de distribuição), forma da partículas (espérica, espiga, ...) ?

nucleação: condensação de vapores quentes (formando núcleo de condensação) ou durante transformação de gás em partícula. Remoção: aglomeração. 

acumulação: formação: from nucleação através coagulação ou condensação de vapores.remoção: deposição seca ou úmida.

grossas: processos mecânicos: fragmentação, movimentação, manuseio. remoção: sedimentação.

Qual faixa é mais numerosa? usar aquele trabalho do Seinfeld da Fátima?
Qual a composição química por faixa?

fina: íons $SO_4^=$ e $ NO_3^-$, carbono elementar, carbono orgânico, compostos orgânicos elementar condensados?, metais (cádmio, níquel, vanádio, zinco, cromo, ferro, mercúrio), sulfatos, nitratos, nitrato amônia?  
o óxido de nitrogênio (NOx) e amônia NH3 formam  nitrato de amônio e 
o dióxido de enxofre (SO2) amônia NH3 formam o sulfato de amônio, portanto são secundário. 

Qual a diferença da composição entre ambiente rural e urbano.
Artigo que a fátima deu seria interessante... 

grossa: solo, fuligem, pólen. origem mineral: Si, Al, ferro, K, Ca e metais alcalinos.

nomenclatura adotada

relação com sistema respiratório humano (deposição em função do diâmetro da partícula).
as partículas mais finas (PM2.5) chegam nos bronquíolos e as maiores (ainda no PM2.5) só alcançam os alvéolos. e como o corpo as elimina? Macrófago alveolar ou o sistema linfático a expulsam? 
As maiores ficam no nariz (traqueobrônquica) nasofaringe . 
Os vírus são PM2.5 e bactérias são PM10, lega né?

ao depositar em folhas, as partículas impedem a absorção de luz. 


visibilidade

formação de nuvens.

danos a saúde (saldiva e WHO) e ao ambiente.

Legislação nacional e internacional.

fontes naturais e antropogênicas. 
móveis e estacionárias. SPECIATE. As fontes depende da região.

