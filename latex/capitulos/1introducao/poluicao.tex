
%%%%
\section{Poluição do Ar}

Poluição do ar resulta de uma complexa mistura de emissões naturais e 
antropogênicas. No caso de Accra a discriminação de fontes é complexa pois
há uma mistura de elementos do deserto com fontes antropogênicas.

Apesar dos altos índices de poluição em Accra pouca pesquisa tem sido 
realizada na tentativa de enteder a composição química do ar na região
(tanto antropogênica quando natural - poeira mineral do deserto). 




%%%%
\section{material particulado?}

O Material Particulado pode ser classificado por tamanho, formação 
(primária ou secundária), composição química, remoção da atmosfera, 
forma da partículas (espérica, espiga).
%TODO: colocar a figura clássica do seinfeld de distribuição

Formação e remoção de partículas na atmosfera:
\begin{itemize}
  \item \textbf{moda nucleação:} núcleo de condensação são criados devido a 
        condensação de vapores quentes ou durante o processo de 
        transformação de gás em partícula. Partículas formadas por 
        nucleação são removidas da atmosfera por aglomeração. 
  \item \textbf{moda acumulação:} partículas na moda de acumulação são criadas 
         a partir de núcleos de condensação através da coagulação ou 
         condensação de vapores. Partículas formadas por acumulação
         são removidas da atmosfera por deposição seca ou úmida.
  \item \textbf{moda grossa:} são oriundas de processos mecânicos como fragmentação, 
        movimentação e manuseio. Partículas grossas são removidas da atmosfera 
        por sedimentação.
\end{itemize}

%TODO: Qual mode é mais numerosa? usar aquele trabalho do Seinfeld da Fátima?

Na moda fina encontra-se principalmente: íons $SO_4^=$, 
íons $ NO_3^-$, carbono elementar, carbono orgânico, compostos orgânicos elementar
condensados, metais (cádmio, níquel, vanádio, zinco, cromo, ferro, mercúrio), 
sulfatos, nitratos, nitrato amônia. 

O oxido de nitrogênio (NOx) e amônia NH3 formam  nitrato de amônio e 
o dióxido de enxofre (SO2) e amônia NH3 formam o sulfato de amônio, 
portanto são secundário. 

Na moda grossa encontra-se principalmente: poeira de solo, fuligem, 
pólen, Si, Al, ferro, K, Ca e metais alcalinos.

%%%%
\section{Saúde}

A deposição no sistema respiratório humano ocorre em função do diâmetro da partícula.
As partículas mais finas chegam nos bronquíolos e as maiores só alcançam os alvéolos.
As maiores ficam no nariz (traqueobrônquica) nasofaringe. 
Os vírus são $MP_{2,5}$ e bactérias são $MP_{10}$. 

O próprio corpo consegue remover partículas através do Macrófago alveolar 
e do sistema linfático. 

%%%%
\section{Meio Ambiente}
Ao depositar em folhas, as partículas impedem a absorção de luz. 

Efeitos na visibilidade.

Efeitos na formação de nuvens.


