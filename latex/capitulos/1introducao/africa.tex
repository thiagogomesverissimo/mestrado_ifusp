%%%%
\section{África}

Os maiores níveis de poluição de ar no mundo ocorre
em cidades da ásia, oriente médio e áfrica \cite{brauer2012}.   

Embora a poluição do ar urbana em cidades nos países em 
desenvolvimento guarde algumas similaridades com as dos 
países desenvolvidos, como a poluição cauda pelo tráfego, 
há ainda grandes diferenças, já que ainda é comum encontrar
a queima de biomassa como fonte de energia para cozimento em
cidades de países em desenvolvimento \cite{brauer2012}.  



%%%%
\subsection{África Subsariana}
A África Subsariana (SSA) é atualmente a região no mundo que tem a maior taxa de transição 
da população rural - ainda predominante - para cidade \citep{MONTGOMERY2008}.

Pesquisas recentes tem avaliado a poluição do ar em favelas 
(regiões periféricas) \citep{SCLAR2005} e \citep{RILEY2007}. 

Mesmo assim, as cidades da \textit{SSA} ainda não possuem sistemas de 
monitoramento sistemático de poluição do ar e suas implicações na saúde 
\citep{EZZATI2004}. 
Além disso há poucas pesquisas acadêmicas dos níveis de poluição do ar nos 
países da \textit{SSA}.

Diferente dos países industrializado, onde as principais fontes de poluição 
são os setores da industria e do transporte, nos países da \textit{SSA} a 
queima de biomassa assume a primazia, sendo comum o seu uso no cozimento 
de alimentos, tanto em regiões urbanas quanto rurais \citep{SMITH2004}. 
Outras características regionais ampliam as diferenças na qualidade do ar 
e no perfil de fontes observado:

\begin{itemize}
  \item população predominantemente rural;
  \item muitas vias não pavimentadas;
  \item maior taxa de crescimento populacional urbano do mundo;
  \item não possuem sistemas de monitoramento sistemático de Poluição do Ar;
  \item é comum o uso da queima de biomassa para o cozimento de alimentos  
        (comercial e doméstico), tanto em regiões urbanas quanto rurais.
\end{itemize}


%%%%
\section{Gana}

Os últimos dois censos publicados em Gana datam
de 2000 \cite{censu2003} e 2010 \cite{censu2013}.

A população de gana em 2010 era de $24,7$ milhões, 
estando $49\%$ no meio rural e $61\%$ no meio urbano, 
sendo que as mulheres representam $51,2\%$ da polulação
total \cite{censu2013}.  

A população teve uma aumento de aproximadamente 
$30\%$ em 10 anos, já que era de $18,9$ milhões 
em 2000 \cite{censu2003}.

Acra é a capital de Gana. 

A Região Metropolitana de Acra (RMA) agrega outras 9 cidades
e tem uma população de 4 milhões pessoas. Se caracteriza por 
ser essencialmente insdustrializada, com $90,5 \%$ 
da polulação urbana \cite{censu2013}.  

A densidade populacional em RMA é de 1205 $habitantes/km^2$, 
enquato que na Região Metropolitana de São Paulo é de 
2476 $habitantes/km^2$ \cite{ibge2011}

População por faixa etária:  

\begin{figure}[H]
\begin{center}
  \includegraphics[scale=0.4]{../outputs/ghanaidade.pdf}
  \caption{Pirâmide etária Gana. Dados de \cite{censu2013}}
\end{center}
\end{figure}

A economia de Gana, antes essencialmente agricultura é agora liderado pela contabilidade serviço para cerca de 51\% da produção nacional. 

A agricultura representa cerca de 30\% , enquanto trilha indústria, com apenas 19\% da produção nacional total. (embora cerca de 55\% dos empregados estão envolvidos no setor)

Ghana‟s economy which until 2006 was dominated by agriculture is now led by service accounting for about 51\% of national output. 

Agriculture accounts for about 30\% (although about 55\% of employed are engaged in the sector).


%TODO: falar sobre EPA. https://en.wikipedia.org/wiki/Ghana_Environmental_Protection_Agency

%%%%
\section{Acra}

Tema Metropolis District is one of the ten (10) districts in the Greater Accra Region of Ghana
% https://en.wikipedia.org/wiki/Greater_Accra_Region

Acra é a capital  de Gana e está localizada no Golfo Guiné. Ela tem uma área 
total de mais de 2500 $km^2$ com elevações que variam de 0 até 60 metros do nível 
do mar. 
%TODO: citar fonte

A população da região metropolitana de Acra estava em torno de 2 milhões em 2005, 
último censo disponível.

%TODO: acho que caberia ampliar as informações sobre Acra: 1) o que há sobre a frota veicular?; 2) quais são as atividades econômicas locais? 3) qual o PIB da cidade e de Ghana? 4) qual o PIB per capita?

A população da região metropolitana de Acra estava em torno de 2 milhões em 2005, 
último censo disponível.
%TODO: (citar censo) e colocar dados da issue 34.

Importação ilegal e reclicagem imprópria caracterizam Agbogbloshie.

\cite{asante2012} analisou nível de elementos traços nas urinas de trabalhadores 
do e-waste de Agbogbloshie e Fe, Sb, and Pb tiveram concentrações alta se comparada
a um pessoa de referência. 


%%%%
\section{Nima}

\begin{figure}[H]
  \caption{Fotos do bairro de Nima}
  \includegraphics[scale=0.35]{../inputs/images/zheng/nima.pdf}
\end{figure}

%TODO: acho que essas fotos são de Nima e de uma zona rural. É importante ter um mapa que localize Ghana, destaque Acra e neste último destaque Nima. Quais as características deste bairro? Qual o contraste com os demais?

\begin{figure}[H]
\begin{center}
  \includegraphics[width=0.9\textwidth]{../outputs/accra_sources.pdf}
  \caption{sources}
\end{center}
\end{figure}


%%%%
\section{Projeto Internacional}

%TODO: Conferir o título do trabalho internacional de Harvard.
Este trabalho enquadra-se no projeto de pesquisa internacional 
\textit{Air Pollution in Accra Neighborhoods: Spatial, Socioeconomic, and Temporal Patterns} 
coordenado por pesquisadores da \textit{Harvard School of Public Health} nos Estados Unidos, 
com participação da Universidade de Ghana. 
Coordenado pelo Dr Majid Ezzati, a época professor da \textit{Harvard School of Public Health} 
nos Estados Unidos (atualmente no Imperial College London, na Inglaterra), com participação 
da Universidade de Ghana. 

Os países da \textit{África Subsariana (SSA)} possuem a maior taxa de crescimento 
populacional urbano do mundo \cite{united2006world}, mas ainda possuem uma população 
predominantemente rural. 
Apesar disso, ainda é escasso o conhecimento do impacto que esse novo modelo de vida 
tem causado para saúde e meio ambiente na SSA \cite{cohen2004urban}. 

Pioneiro nas pesquisas de poluição do ar ambiental em países da SSA, o objetivo 
global desse projeto internacional é fazer avaliações iniciais e sistemáticas 
dos níveis de poluição do ar em Acra (capital de Gana e maior cidade da SSA), 
inexistentes até então, e relacioná-los com as condições socioeconômicas 
específicas das diferentes áreas estudadas naquela capital.

O grupo de Harvard \citep{ARKU2008} conduziu um levantamento nos níveis de 
poluição, bem como da distribuição espacial e temporal de alguns poluentes 
em duas regiões periféricas de Acra \citep{DIONISIO2010}:



\begin{itemize}
  \item Jamestown/Ushertown: região entre a Costa e o centro comercial local.
  \item Nima: Centro comercial de Acra, cercada com Rodovia.
\end{itemize} 

Nos dois bairros há poucas ruas pavimentadas, com exceção das principais 
avenidas.
