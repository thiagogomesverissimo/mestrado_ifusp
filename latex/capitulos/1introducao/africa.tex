%%%%
\section{África}

A poluição do ar ambiental nas cidades dos países em desenvolvimento
guarda algumas similaridades com as cidades dos países desenvolvidos, já que 
ambas possuem algumas fontes poluidoras que são características de meios urbanos, 
tais como, tráfego de veículos automotivos, indústrias, geração de 
eletricidade por usina hidrelétrica ou termoelétrica, entre outras. 

No entanto, em cidades de países de industrialização tardia, soma-se a esta 
poluição, comum a meios urbanos, agravantes decorrentes a incapacidade do 
Estado em atender toda a população nas demandas urbanas de infraestrutura, 
como por exemplo, no provimento da energia necessária para realização de 
atividades cotidianas, tais como, transporte, iluminação ou alimentação. 
As alternativas comuns encontradas pela população destas cidades são: queima da 
biomassa para cozimento de alimentos, uso de querosene para iluminação 
noturna e contínuo uso de frota veicular com tecnologias ultrapassadas
\citep{brauer2012}.

Esses fatores citados acima e a urbanização rápida e descontrolada ocorrida na
última década fazem com que cidades da Ásia, África e Oriente 
Médio possuam os maiores níveis de poluição do ar ambiental do mundo 
\citep{brauer2012}.

Em 2015, a África contava com $1.186,178$ habitantes, ficando atrás 
apenas da Ásia, que possuía $4.393,296$ habitantes no mesmo ano. 
A África é o terceiro maior continente em extensão, com área territorial 
de 30 milhões de quilômetros quadrados, abrigando 54 países independentes 
\citep{UN}.

Existe uma barreira natural formada pelo \textbf{Deserto do Saara}
separando norte e sul da África. Separação não só geográfica, como
social, étnica e econômica. 

A África Subsariana \textbf{(SSA)}, localizada ao sul do 
\textbf{Deserto do Saara}, conta com os países maios pobres do mundo, mas que 
nas últimas décadas iniciaram um processo intenso de urbanização \citep{UN}. 
   	
%%%%
\subsection{África Subsariana \textbf{(SSA)}}

A África Subsariana \textbf{(SSA)} é atualmente a região no mundo com a maior 
taxa de transição da população rural - ainda predominante - para cidade
\citep{MONTGOMERY2008}. 

Até 2003, nenhuma cidade da \textbf{SSA} possuía sistemas de monitoramento 
sistemático de poluição do ar, somente medidas esporádicas realizadas
por universidades, mas que revelaram concentrações de poluentes altíssimas e 
acimas dos recomendados pela Organização Mundial de Saúde \textbf{OMS},
fato esse que levou pesquisadores do mundo inteiro a 
realizarem estudo ambientais no continente \citep{EZZATI2004}. 
Segundo \cite{aboh2009} ainda há pouco estudos de aerossol atmosférico 
em países africanos, mas o interesse da comunidade acadêmica cresceu
nos último anos.
 
O aerossol atmosférico africano também desperta interesse, pois afeta o clima 
(radiação, umidade, vento, etc) do mundo.

Diferente dos países industrializados, onde as principais fontes de poluição 
são os setores da industria e do transporte, nos países da \textbf{SSA} a 
queima de biomassa assume a primazia, sendo comum o seu uso no cozimento 
de alimentos, tanto em regiões urbanas quanto rurais \citep{SMITH2004}. 

Estas são algumas características regionais da \textbf{SSA} que ampliam as 
diferenças do perfil de fontes poluidoras do ar comparada com cidades 
de países desenvolvidos: população predominantemente rural,
vias não pavimentadas, altas taxa de crescimento populacional, população jovem,
inexistência de sistemas de monitoramento sistemático em larga escala do meio 
ambiente, queima de biomassa em residências e comércio para o preparação
de alimentos, entre outros. 

%%%%
\subsection{Gana}

Gana situa-se no continente africano, 5 graus a norte do Equador. 
Faz fronteira a oeste com a Costa do Marfim, ao norte com Burkina
e ao leste pelo Togo. 

Com clima equatorial, possui praticamente duas estações climáticas:

\begin{itemize}
  \item estação seca, caracterizada pelo ar seco e pela poeira do harmatão. 
       Começa no meados de Novembro se estende até metade de Março.
 \item estação chuvosa, começa em Abril e acaba em outubro, tendo maior
       intensidade entre Abril e Julho.
\end{itemize}

As chuvas ocorrem em dois períodos, entre Maio e Junho e entre Agosto e Setembro.
Um vento quente e seco provindo de nordeste sopra entre final de Dezembro
e início de Fevereiro e é chamado de Harmatão. O Harmatão vem carregado de 
poeira do \textbf{Deserto do Saara} \citep{breuning2005}.

O \textbf{United States Department of Commerce National Oceanic 
and Atmospheric (NOAA)} mantém um banco de dados com parâmetros 
meteorológicos do mundo inteiro enviado por estações meteorológicas 
cadastradas \citep{noaa}. 

Para avaliação do perfil dos parâmetros meteorológicos locais
utilizou-se dados horários de Setembro de 2006 à Junho de 2008 
coletados na estação meteorológica do aeroporto de Acra 
(\textbf{Kotoka International Airport}) cadastrado na \textbf{NOAA}.

Plotando-se o gráfico da figura \ref{fg:rosaCompleta}, 
que mostra a distribuição de frequência de direção dos ventos bem como a 
intensidade, verifica-se que a direção predominante de origem dos ventos 
é sudoeste. 

\begin{figure}[H]
  \centering
  \includegraphics[width=0.5\textwidth]{../outputs/windRoseNoaaHarvard.pdf}
  \caption{Rosa do ventos entre
           Setembro de 2006 e Junho de 2008. Utilizou-se dados 
           do \textbf{Kotoka International Airport} de Acra \label{fg:rosaCompleta}}
\end{figure}

No gráfico da figura \ref{fg:rosaCompleta} a observação horária da direção e 
intensidade dos ventos, identifica-se um regime de brisa marinha bem 
definido, com ventos de sul quando sol a pino, mas se deslocando a oeste com a 
entrada da noite. 

Nota-se também uma perceptível diferença entre os dois períodos climáticos, 
pois no Verão temos maior quantidade de radiação solar, fortalecendo a 
formação de brisa marinha e, consequentemente, deslocando mais para o 
oeste a distribuição de frequência nas direções dos ventos.
\begin{figure}[H]
  \centering
  \includegraphics[width=1\textwidth]{../outputs/windRose_horaria.pdf}
  \caption{ \citep{carslaw2012} \label{fig:windRose_horaria}}
\end{figure}

\begin{figure}[H]
  \centering
  \includegraphics[width=1\textwidth]{../outputs/windRose_mensal.pdf}
  \caption{ \citep{carslaw2012} \label{fig:windRose_mensal}}
\end{figure}

A distribuição de frequência por estação do ano pode ser vista em 
\ref{fg:windFrequencyStation}. 
No inverno, quando há ocorrência do Harmatão, esperaria-se maior
frequência de ventos de nordeste, mas no gráfico do inverno é quase 
impercepitível esse aumento. \citep{breuning2005} mostra que na 
verdade o vento do Harmatão passa por Gana em altas altitudes, por isso
não influência os ventos locais. 

\begin{figure}[H]
  \centering
  \includegraphics[width=1\textwidth]{../outputs/windRoseNoaaHarvard.pdf}
  \caption{ \citep{carslaw2012} \label{fig:windRose_full}}
\end{figure}

%\begin{figure}[H]
%  \centering
%  \includegraphics[width=0.4\textwidth]{../outputs/windFrequencySpring.pdf}
%  \includegraphics[width=0.4\textwidth]{../outputs/windFrequencySummer.pdf}
%  \includegraphics[width=0.4\textwidth]{../outputs/windFrequencyFall.pdf}
%  \includegraphics[width=0.4\textwidth]{../outputs/windFrequencyWinter.pdf}
%  \caption{Distribuição da frequência do vento em (\%) entre
%           Setembro de 2006 e Junho de 2008 por estação do ano. Utilisou-se dados 
%           do \textbf{Kotoka International Airport} de Acra \label{fg:windFrequencyStation}}
%\end{figure}

Os últimos dois censos demográficos realizados em Gana datam
de 2000 \citep{ghanacensus2003} e 2010 \citep{ghanacensus2013}. Os
dados resultantes podem ser encontrados no portal de dados abertos
do Governo Federal de Gana \citep{opendataghana}.

A população de Gana em 2000 correspondia a $18,9$ milhões 
de habitantes e em 2010 subiu para $24,7$ milhões, tendo assim
um aumento de $30\%$ no intervalo dos dois censos demográficos.

No censo demográfico de 2010, indica que $49\%$ da população está 
no meio rural e $61\%$ no meio urbano. As mulheres representam 
$51,2\%$ da polulação.

A pirâmede etâria de Gana \ref{fig:piramedegana} indica população jovem, 
e que a expectativa média de vida está não é muito maior que 50 anos. 

\begin{figure}[H]
\begin{center}
  \includegraphics[width=0.5\textwidth]{../outputs/piramide_etaria.pdf}
  \caption{Pirâmide etária Gana plotada com dados do censo 
           demográfico de 2010 \citep{ghanacensus2013} \label{fig:piramedegana}}
\end{center}
\end{figure}

A economia de Gana, antes essencialmente dominada pela agricultura, 
agora está distribuída entre: industria $19\%$, agricultura $30\%$ 
e serviços $51\%$ \citep{ghanacensus2013}.
Na industria, recebe destaque fabricação e exportação de aparelhos digitais, 
automóveis e navios. Em termos de matéria primária, há grande exportação de 
hidrocarbonetos e minerais \citep{ghanacensus2013}.

O produto interno bruto (PIB) per capita anual de Gana em 2010 foi
de \$ 1.323,09 USD e no Brasil no mesmo ano \$ 11.124,09 USD.
O gráfico da figura \ref{fib:pib} apresenta o \textbf{PIB} de Gana e do Brasil 
calculado pelo Banco Mundial \citep{bancomundial}.

\begin{figure}[H]
\begin{center}
  \includegraphics[width=0.5\textwidth]{../outputs/PIBGhanaBrazil.pdf}
  \caption{\textbf{PIB} do Brasil e de Gana. Plotado com dados do 
           Banco Mundial \citep{bancomundial} \label{fig:pib}}
\end{center}
\end{figure}

A responsabilidade do controle, fiscalização e monitoramento das 
atividades poluídoras em Gana é realizado pela 
\textbf{Ghana Environmental Protection Agency (EPA Ghana)}, que é 
hierarquicamente subordinada ao 
\textbf{Ministério de Meio Ambiente, Ciência, Tecnologia e Inovação} do 
Governo Federal de Gana.

%%%%
\subsection{Região Metropolitana de Acra \textbf{(RMA)}}

Acra é uma cidade litorânea e é a capital de Gana. Está localizada 
no Golfo Guiné tendo área total de mais de 2500 $km^2$, com elevações que 
variam de 0 até 60 metros do nível do mar \citep{ARKU2008}.

A Região Metropolitana de Acra \textbf{(RMA)} agrega outras 9 cidades
além de Acra e conta com uma população total de 4 milhões de habitantes. 
Se caracteriza por ter sua economia baseada exclusivamente na industria 
e em serviços, $90,5\%$ da polulação ocupando o meio urbano \citep{ghanacensus2013}.

Em 2010, havia aproximadamente 1000 fazendas urbanas com produção de vegetais 
para consumo local em Acra, onde as irrigações são feitas por água não tratada, 
provinda de córregos locais. Altos indíces de metais pesados 
(Fe, Mn, Cu, Zn, Pb, Ni, Cr, Cd, Co) foram encontrados nas plantações 
dessas fazendas \citep{lente2014}, pois os resíduos das residências são despejados 
diretamanente nos córregos.

A densidade populacional em \textbf{(RMA)} é de 1205 $habitantes/km^2$, 
enquanto que na Região Metropolitana de São Paulo \textbf{(RMSP)} é de 
2476 $habitantes/km^2$ \citep{ibge2011}. 

\textbf{Driver and Vehicle Licensing Authority (DVLA)} é o
departamento do governo de Gana que cuida dos registros de automóveis, 
mas há muitos automóveis circulando que não são registrados. 
Em contava em 2009, Gana contava com 1,12 milhões de veículos legais. 
Na tabela \ref{table:dvla} percebe-se que a frota dobrou em um década.

\begin{table}[H]
 \centering
  \input{../outputs/frota_ghana.tex}
  \caption{Frota veícular de Gana \citep{dvla} \label{table:dvla}}
\end{table}

Segundo a EPA de Gana \citep{epagh} as fontes de poluição do ar ambiental 
majoritárias em Acra, são:

\begin{itemize}
  \item Emissões veiculares, principalmente emissões de veículos antigos sem manutenção;
  \item Emissões industriais;
  \item Queima de lixo e outros materiais a céu aberto;
  \item Poeira de ressupensão de solo, pois há muitas vias ainda não pavimentadas;
  \item Vento seco do \textbf{Deserto do Saara}, o harmatão.
\end{itemize}
sendo o harmatão a maior fonte de poeira do mundo.
%%%%
\subsection{Nima}


%%%%%%%%%%5

Accra is a diverse city, and you will find up-scale restaurants and clubs with prices similar to Europe – but you will also find local areas where you can fill up your tummy for a few dollars or less.

If you want to see another side of Accra, follow this walking route and experience the local life in some of the impoverished parts of Accra, and pick some local street food on the way.

 

Nima is known to be an “unsafe” part of Accra – but from personal experience I can tell you that having a stroll in this part of town is far from a high risk activity. Now, I will not recommend you to wander off in this are alone at night since I do not know this area inside out (I´m a newbie after all), but during the day you will be fine.

How do I know?

I happen to settle in Nima three weeks ago, so I roam the streets here every day! People are curious, since there are no other white people staying in this area – but they are friendly.

Nima is one of the unplanned settlements in Accra, and the area lack a proper sanitation system. Most people fetch their water in water station, and have to pay to use public toilets which are scattered around the area. Despite this is regarded a poor area of town, it is buzzing with life and business from early morning till late night.

 

Start your tour from Nima Police Station by the Ring Road, and walk up Nima Road. When you get to the roundabout, continue straight. In case you get lost – everyone will be able to direct you back to Nima roundabout! You are now entering Nima Market, and you will meet the chaos of cars, vendors and customers who are busy to close their deals. To avoid the worst part of the street, branch right through the yellow overhead banner, and walk through the main market in Nima.

The roadside in Nima market is a good place to grab a snack and a sachet of ice cold water.

 

After working your way through spices, grains, beans, chickens, veggies and hundreds of people you will hit the Nima Road again, and you continue straight.  When you reach the big drain, branch left and walk down the street, then branch right over the bridge where the road turn slightly left. Walk straight up the road, and branch off the fourth road to your left to enter Newtown.

This is a busy road with a lot of small shops selling everything from clothes, shoes, mattresses, electronics and beauty products.

Similar to Nima you will find small food stalls along the street. After passing Bennett Memorial Clinic you can branch second left onto Hill Street which will lead you back to the roundabout, or you can take the third left if you want to pass buy the Mallem Atta Market.

 

If you eat vegetarian you may struggle to find street food which is guaranteed no meat. Fried yams and plantains served with chili sauce is one option, the sweet doughnut-like bread ball boo-froot is another option. You will also find fresh bread with Blueband, boiled eggs served with chili sauce – and you can try to find freshly made akkala, a fried ball made from grinded black eyed beans (I think they are free of meat broth).

Everything else will probably contain meat or broth of some kind. If you see a stack of eggs and instant noodles you can buy yourself a delicious meal of indomie, the noodles are prepared as you wait and mixed with vegetables, eggs – and if you like some canned beef or sardines. You will find different kind of stews everywhere – groundnut soup, palm soup, okra stew and fish stew, and they are served with a starchy ball of banku (made from corn and/or cassava), kenkey (made from fermented maize) or fufu (mixed cassava and plantains).

Other dishes you may find are red-red (bean stew), jollof rice (rice booked in tomato sauce) and waakaye (rice and beans). Also look out for waagashi, a local cheese which is grilled or fried. From the grills you get fresh grilled fish and meat.

If you are not comfortable walking on your own in the area, there is a walking tour on request in Nima with a guy called Charles, he is a jovial and knowledgeable guy who knows a lot about the history of Nima – and this may be a good choice if you just want a guide or also want to get deeper into the area and pass by some local families.



%%%%%%%%%%%
Nima é um dos bairros mais pobres de Acra, Gana. 
Embora careça de serviços básicos como moradia, eletricidade, rede de esgoto, 
tratamento de água, é a região que mais recebe migrantes das partes rurais e
imigrantes de países vizinhos, possuindo assim, diversidade cultural e religiosa.

O bairro analisado foi \textbf{Nima}, que tem por características
ser um bairro dormitório de Acra. O principal meio de transporte
dos trabalhadores de Nima para a zona industrial e de serviços de Acra é 
feito por vans. Muitas vias não são pavimentadas e a intensa movimentação
de veículos causa ressuspensão de poeira do solo, já que há vias não
pavimentadas.

O carvão e a lenha são as principais fontes de energia para cozimento 
em Nima. Na figura \ref{fig:nima} há imagens mostrando como as cozinhas
são equipadas. 

\begin{figure}[H]
  \centering
    \includegraphics[width=0.7\textwidth]{../inputs/images/zheng/nima.pdf}
    \caption{Fotos de residências de Nima, por Raphael Arku \label{fig:nima}}
\end{figure}

Foi feito um levantamento das principais fontes poluídora de Acra.
Na figura \ref{fg:acrasources} estão assinalados os dois pontos de 
amostragem em Nima, fontes poluídoras levantadas, bem como a aeroporto
(onde foram coletados os dados meteorológicos).

\begin{figure}[H]
  \centering
  \includegraphics[width=0.9\textwidth]{../outputs/accra_sources.pdf}
  \caption{Levantamento de algumas fontes poluídora de Accra \label{fg:acrasources}}
\end{figure}

Acra é mundialmente conhecida por receber ilegalmente lixo 
eletrônico dos países desenvolvidos, que são então derretidos, de forma
imprópria, para obtenção de cobre pela população local. 
O depósito de lixo eletrônico, conhecido como \textbf{e-waste}, 
fica no bairro \textbf{Agbogbloshie}, $4 km$ a sudoeste de \textbf{Nima}
\citep{asampong2015}.

\begin{figure}[H]
  \centering
  \includegraphics[width=0.5\textwidth]{../inputs/images/ewaste_jack_caravano.jpg}
  \caption{Foto do bairro de Agbogbloshie em Acra. Autorizado por Jack Caravanos, 
           Professor da \textbf{School of Public Health} em \textbf{Hunter College, CUNY}
           Nova Iorque, Estado Unidos da América. \label{fig:ewaste}}
\end{figure}

\citep{ARKU2008} e \citep{DIONISIO2010} foram pioneiros em conduzir  
levantamento dos níveis de poluição, composição elementar e distribuição espacial 
e temporal de poluentes.
