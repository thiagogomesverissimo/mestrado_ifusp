\section{África Subsariana}

\section{Acra}

Acra é a capital  de Gana e está localizada no Golfo Guiné. Ela tem uma área 
total de mais de 2500 $km^2$ com elevação que varia de 0 até 100 pés do nível 
do mar. 

Período chuvovo: Abril-Julho e Setembro-Outubro. 
Ocorrêcnia do Harmathan: Novembro-Março com ventos para direção sudeste.

O grupo de Harvard \citep{ARKU2008} conduziu um levantamento nos niveis de 
poluição, bem como da distribuição espacial e temporal de alguns poluentes 
em duas regiões periféricas de Acra: 
\citep{DIONISIO2010}

\begin{itemize}
  \item Jamestown/Usshertown: região entre a Costa e o centro comercial local.
  \item Nima: Centro comercial de Acra, cercada com Rodovia.
\end{itemize} 

Nos dois bairros há poucas ruas pavimentadas, com exceção das principais 
avenidas. 

Pesquisas recentes tem avaliado a poluição do ar em favelas 
(regiões periféricas) \citep{SCLAR2005} e \citep{RILEY2007}. 

A África Subsariana (SSA) é atualmente região no mundo que tem a maior taxa de 
transição da polulação rural - ainda predominante - para cidade 
\citep{MONTGOMERY2008}.

Mesmo assim, as cidades da \textit{SSA} ainda não possuem sistemas de 
monitoramento sistemático de poluição do ar e suas implicações na saúde 
\citep{EZZATI2004}. 
Além disso há poucas pesquisas acadêmicas dos níveis de poluição do ar nos 
países da \textit{SSA}.

Diferente dos países industrializado, onde as principais fontes de poluição 
são os setores da industria e do transporte, os países da \textit{SSA} tem como 
principal fonte poluidora a queima de biomassa, sendo comum o uso no cozimento 
de alimentos, tanto em regiões urbanas quanto rurais \citep{SMITH2004}.
