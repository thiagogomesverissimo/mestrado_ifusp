\section{África Subsariana}

A engloba países...

\section{Acra}

Acra é a capital  de Gana e está localizada no Golfo Guiné. Ela tem uma área 
total de mais de 2500 $km^2$ com elevação que varia de 0 até 100 pés do nível 
do mar. 

Período chuvovo: Abril-Julho e Setembro-Outubro. 
Ocorrêcnia do Harmathan: Novembro-Março com ventos para direção sudeste.

O grupo de Harvard \citep{ARKU2008} conduziu um levantamento nos niveis de 
poluição, bem como da distribuição espacial e temporal de alguns poluentes 
em duas regiões periféricas de Acra: 

\begin{itemize}
  \item Jamestown/Usshertown: região entre a Costa e o centro comercial local.
  \item Nima: Centro comercial de Acra, cercada com Rodovia.
\end{itemize} 

Nos dois bairros há poucas ruas pavimentadas, com exceção das principais 
avenidas. 

Pesquisas recentes tem avaliado a poluição do ar em favelas (regiões periféricas) 
%\citep{SCLAR2005} e \citep{RILEY2007}. 

A África Subsaariana (SSA) é a região no mundo que tem a maior taxa de transição da 
polulação rural - ainda predominante - para cidade. %\citep{MONTGOMERY2008}.

Mesmo assim, as cidades da SSA ainda não possuem sistemas de monitoramento sistemático de 
poluição do Ar e seus riscos na saúde % \citep{EZZATI2004}. 
Além disso há poucos estudos dos níveis de poluição do Ar nos países da SSA 
(Procurar alguns artigos de trabalhos recentes SSA).

Na SSA o combustível sólido é para produção da energia.
Diferente dos países industrializado, os quais tem como fontes de principais fontes 
poluição poluição a industria e o transporte, os países da África SSA tem como principais 
fontes de não-combustão e queima de biomassa. Na SSA é comum o uso da queima de biomassa 
para o cozimento de alimentos, tanto em regiões urbanas quanto rurais. % \citep{SMITH2004}
