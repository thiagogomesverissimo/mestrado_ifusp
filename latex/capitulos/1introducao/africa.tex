\section{Poluição do ar em países da africa subsariana}

Buscar referências sobre:
 - Explicação do Harmathan usando céludas de halley mostrando ocorrência no 
   verão inverno.
 - Direção do vento de SOLO, comandado pela brisa terr-mar.
 - Pb no solo do deserto?

- Procurar dados do aumento da população da area metropolitana de acra 
  (tabela e gráfico)
- http://graphic.com.gh/General-News/report-on-2010-population-census-out.html
- http://www.ghanaweb.com/GhanaHomePage/general/statistics.php
- Porcentagem de uso de queima de biomassa para alimetação: 
http://www.statsghana.gov.gh/nada/index.php/catalog/13 E 
  %\citep{BOADI2006}
- Outros dados interessantes da população ganense: http://data.gov.gh

\section{O caso de Accra}

Acra é a capital  de Gana e está localizada no golfo guiné. Ela tem uma área 
total de mais de 2500 $km^2$ com elevação que varia de 0 até 100 pés do nível do mar. 

poluição do ar urbana é uma complexa mistura de emissões e naturais e antropogenicas. 
A composição quimica varia.  Periodo chuva: abril-julho e set-out. harmathan: nov-março na
direção sudeste.

Primeiro artigo do grupo de Harvard, conduziu um levantamento nos niveis de poluição, 
bem como distribuição espacial e temporal de alguns poluentes em duas regiões periféricas
de accra: JT/UT(entre a costa e a area comercial) e Nima(centro de Acra- mora muito 
migrantes - e está cercada no norte por uma rodovia: Central ring road). Há poucas ruas 
pavimentadas nos dois casos, normalmente as principais avenidas. % \citep{ARKU2008}. 

Pesquisas recentes tem avaliado a poluição do ar em favelas (regiões periféricas) 
%\citep{SCLAR2005} e \citep{RILEY2007}. 

A África Subsaariana (SSA) é a região no mundo que tem a maior taxa de transição da 
polulação rural - ainda predominante - para cidade. %\citep{MONTGOMERY2008}.

Mesmo assim, as cidades da SSA ainda não possuem sistemas de monitoramento sistemático de 
poluição do Ar e seus riscos na saúde % \citep{EZZATI2004}. 
Além disso há poucos estudos dos níveis de poluição do Ar nos países da SSA 
(Procurar alguns artigos de trabalhos recentes SSA).

Na SSA o combustível sólido é para produção da energia.
Diferente dos países industrializado, os quais tem como fontes de principais fontes 
poluição poluição a industria e o transporte, os países da África SSA tem como principais 
fontes de não-combustão e queima de biomassa. Na SSA é comum o uso da queima de biomassa 
para o cozimento de alimentos, tanto em regiões urbanas quanto rurais. % \citep{SMITH2004}
