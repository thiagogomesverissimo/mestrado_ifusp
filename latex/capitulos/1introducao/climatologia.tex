\section{Climatologia}

Ocorrência do Harmatão entre Novembro e Março, com ventos para direção sudeste.
Período chuvovo: Abril-Julho e Setembro-Outubro.

Faremos a avaliação da meteorologia local usando dados da estação meteorológica
de Acra cadastrada na \textit{NOAA} e com dados de re-análise global com grade
em Acra. 

%%%%
\subsection{NOAA}

Usou-se os dados horários coletados na estação meteorológica do aeroporto 
de Acra e mantido pelo
\textit{United States Department of Commerce National Oceanic 
and Atmospheric (NOAA)}.

Pode-se verificar a distribuição de frequência de
direção dos ventos para encontrar a direção predominante de origem dos ventos e 
ainda verificar se a distribuição muda conforme as estãções do ano.  

%verifica-se que os ventos mais frequentes são advindos da região leste-sudeste.

%TODO: fazer gráfico velocidade do vento + poluicao nos dia medidos
%TODO: fazer Grafico com limite (linha horizontal) da EPA ou Ghna lei da serio temporal.

Oeste e sudoeste foram as direções predominantes do vento durante todo o período 
de amostragem (Setembro de 2006 à Junho de 2008).
 
\begin{figure}[H]
\begin{center}
  \includegraphics[scale=0.4]{../outputs/completo2006_2008.pdf}
  \caption{Distribuição da frequência do vento em (\%) entre
           Setembro de 2006 e Junho de 2008.}
\end{center}
\end{figure}


\begin{figure}[H]
\begin{center}
  \includegraphics[scale=0.4]{../outputs/windRoseNoaaHarvard.pdf}
  \caption{Rosa dos ventos}
\end{center}
\end{figure}


No inverno e outono a mais predominânia de ventos sul que na primevera 
e no verão. 

\begin{figure}[H]
\begin{center}
  \includegraphics[scale=0.30]{../outputs/windFrequencySpring.pdf}
  \includegraphics[scale=0.30]{../outputs/windFrequencySummer.pdf}
  \includegraphics[scale=0.30]{../outputs/windFrequencyFall.pdf}
  \includegraphics[scale=0.30]{../outputs/windFrequencyWinter.pdf}
\end{center}

Nota-se também uma perceptível diferença entre os dois períodos climáticos, pois no Verão temos maior quantidade de radiação solar, fortalecendo a formação de brisa marinha e, consequentemente, deslocando mais para o leste a distribuição de frequência nas direções dos ventos.


\caption{Distribuição da frequência do vento em (\%) por estação do ano entre
         Setembro de 2006 e Junho de 2008.}
\end{figure}

Entre Dezembro e Fevereiro, período de ocorrência do Harmatão, aparece 
ventos oriundos da direção nordeste, ainda que com frequência inferior 
aos de sudoeste.




\begin{figure}[H]
\begin{center}
  \includegraphics[scale=0.40]{../outputs/windFrequencyHarmatao.pdf}
\end{center}
\caption{Distribuição da frequência do vento em (\%) nos mês de ocorrência
         do Harmatão, entre Dezembro e Fevereiro.}
\end{figure}

\begin{figure}[H]
\begin{center}
  \includegraphics[scale=0.4]{../outputs/windRoseHarmatao.pdf}
  \caption{Rosa dos ventos}
\end{center}
\end{figure}



%%%%
\subsection{Carta sinótica}

%TODO: inserir detalhamento do harmatam:
%http://master.iag.usp.br/pr/ensino/sinotica/aula16/

No hemisfério norte os ventos são de nordeste e no hemisfério sul são de sudeste. 


Usando-se dados da \textit{European Centre for Medium-Range Weather Forecasts (ECMWF)}
para construção da carta sinóptica para o período, podemos perceber as 
\textit{Ondas de leste Africanas} ou \textit{ventos alísios}.
%TODO: inserir figuras do grads

Baixamos os seguintes dados do (ECMWF):
\begin{itemize}
  \item temperatura a 2 metro do mar;
  \item direção e velocidade vento 10 metros do mar;
  \item pressão atmosférica reduzida a nível médio do mar;
\end{itemize}

%TODO: equação da Conversão de altura em metros em hPa


