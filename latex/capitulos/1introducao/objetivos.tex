\section{Objetivos}

Os objetivos 

\begin{itemize}
  \item Identificar e quantificar as fontes majoritárias de poluição do ar ambiental em Acra;
  \item Aperfeiçoar a metodologia numérica empregada na quantificação de elementos químicos, 
        buscando particularmente a determinação mais consistente das incertezas nos valores medidos;
  \item Comparar resultados dos métodos estatísticos multivariados
        \textbf{Positive Matrix Factorization (PMF)} e \textbf{Análise de Fatores (AF)};
  \item Apresentar uma proposta de calibração do equipamento de Fluorescência de Raios X
        em dispersão em energia para melhorar a qualidade das medidas;
  \item Apresentar uma proposta de calibração de medidas de \textbf{Black Carbon} por refletância
        utilizando o método absoluto \textbf{Thermal/Optical Transmittance (TOT)}.
\end{itemize}


Esta pesquisa de \textbf{Mestrado} faz parte de um projeto de porte 
internacional denominado \textbf{Energy, air pollution, and health 
in developing countries} coordenado pelo \textbf{Dr. Majid Ezzati}, 
a época professor da \textit{Harvard School of Public Health} nos 
Estados Unidos e atualmente no \textbf{Imperial College London}, 
na Inglaterra.  

A interação entre grupos de pesquisa de diferente áreas como, saúde, 
estatística, medicina, físicos, químicos, entre outros, é fundamental
para a temática em questão, poluição do ar, tema naturalmente 
interdisplinar, não sendo possível pesquisá-lo isoladamente. 

A entrada da \textbf{Universidade de São Paulo (USP)} no projeto se deu pela
vasta experiência com medidas usando fluorescência de raios X, dominando a técnica.
