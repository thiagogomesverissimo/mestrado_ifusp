\section{Objetivos}

O objetivo central desse estudo foi analisar as concentrações de Material 
Particulado em diferentes locais de Acra e levantar as principais fontes 
poluídoras, usando para tal também informações meteorológicas. 

%The core objective of this study was to analyze the spatio-
%seasonal patterns of BC concentrations in different urban
%microenvironments affected by both local and regional
%pollution and their linkages to meteorological variables. This
%characterization is well-aligned with the recommendations
%made by the General Assembly of the World Medical
%Association (WMA) held in Durban (South Africa) in
%October 2014, which includes: a) monitoring and limiting
%the concentrations of BC in urban areas, b) increasing public
%awareness on the health damage caused by BC particle
%%emitted by diesel vehicles, and c) developing strategies to
reduce people's exposure to BC in aircraft passenger cabins,
trains and homes (WMA, 2014).

Faz parte de um projeto \textit{Energy, air pollution, and health 
in developing countries} coordenado por Dr. Majid Ezzati, 
a época professor da Harvard School of Public Health nos 
Estados Unidos e atualmente no Imperial College London, 
na Inglaterra.  

A interação entre grupos de pesquisa de diferente áreas como, saúde, 
estatística, medicina, físicos, químicos, entre outros, é fundamental
para a temática em questão, poluição do ar, tema naturalmente  
interdisplinar, não sendo possível pesquisá-lo isoladamente. 

A entrada da \textbf{Universidade de São Paulo (USP)} no projeto se deu pela
vasta experiência com medidas usando fluorescência de raios X, dominando a técnica.

\begin{itemize}
  \item Identificar e quantificar as fontes majoritárias de poluição do ar ambiental em Acra;
  \item Aperfeiçoar a metodologia numérica empregada na quantificação de elementos químicos, 
        buscando particularmente a determinação mais consistente das incertezas nos valores medidos;
  \item Comparar resultados dos métodos estatísticos multivariados
        \textbf{Positive Matrix Factorization (PMF)} e \textbf{Análise de Fatores (AF)};
  \item Apresentar uma proposta de calibração do equipamento de Fluorescência de Raios X
        em dispersão em energia para melhorar a qualidade das medidas;
  \item Apresentar uma proposta de calibração de medidas de \textbf{Black Carbon} por refletância
        utilizando o método absoluto \textbf{Thermal/Optical Transmittance (TOT)}.
\end{itemize}




