\section{Objetivos}

O objetivo desta pesquisa é identificar e quantificar 
as fontes de poluição do ar, bairro de Acra,  
Caracterizar o Material Particulado em Acra, capital de Gana, na África.
Aperfeiçoar a metodologia numérica empregada na quantificação de elementos 
químicos e de black carbon, buscando particularmente a determinação mais
consistente das incertezas nos valores medidos.

Para buscar identificar fontes emissoras de PM2,5, e estimar seus impactos, utilizou-se o método estatístico PMF (PAATERO, 1997)



Assim, faz-se cidades da \textbf{(SSA)} tem ganhado atenção em pesquisas sobre 
poluição e seus efeitos na sáude. 

%%%%
\section{Projeto Internacional}

Este trabalho enquadra-se no projeto de pesquisa internacional 
\textit{Air Pollution in Accra Neighborhoods: Spatial, Socioeconomic, and Temporal Patterns} 
coordenado por pesquisadores da \textit{Harvard School of Public Health} nos Estados Unidos, 
com participação da Universidade de Ghana. 
Coordenado pelo Dr Majid Ezzati, a época professor da \textit{Harvard School of Public Health} 
nos Estados Unidos (atualmente no Imperial College London, na Inglaterra), com participação 
da Universidade de Ghana. 

A interação entre grupos de pesquisa de diferente áreas como, saúde, estatística,
medicina e aqueles capazes de quantificar, analisar processos e 
identificar fontes geradoras de poluentes é fundamental

Apesar disso, ainda é escasso o conhecimento do impacto que esse novo modelo de vida 
tem causado para saúde e meio ambiente na SSA. 

Pioneiro nas pesquisas de poluição do ar ambiental em países da SSA, o objetivo 
global desse projeto internacional é fazer avaliações iniciais e sistemáticas 
dos níveis de poluição do ar em Acra (capital de Gana e maior cidade da SSA), 
inexistentes até então, e relacioná-los com as condições socioeconômicas 
específicas das diferentes áreas estudadas naquela capital.

Poucos estudos sobre poluição atmosférica foram realizados em \textbf{Gana}, 
sendo \citep{ARKU2008} e \citep{DIONISIO2010} pioneiros em conduzirem 
levantamento nos níveis de poluição, composição elementar e distribuição espacial 
e temporal de poluentes. 

