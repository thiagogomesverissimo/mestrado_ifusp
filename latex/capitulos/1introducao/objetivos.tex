\section{Objetivos}
%Acho que está aproveitando mal a explicação sobre nossa participação no trabalho. Acho que no próprio projeto está melhor. Acerte isso. 

O objetivo desse estudo foi analisar as concentrações ambientais e fontes de 
material particulado fino e grosso em dois locais de Acra, um com característica
residencial e o outro em zona comercial e de alto tráfego de veículos. 
Para tal, além das coletas dos filtros e posterior análise laboratorial, 
foi levantado dados econômicos, políticos, sociais, ambientais e 
meteorológicos do país, que deu suporte na interpretação dos resultados 
quantitativos, olhando-os sob um prisma crítico e contextualizado, permitindo
inclusive comparações com a realidade brasileira no âmbito das causas e 
consequências devido à poluição do ar.

Dada a natureza intrinsecamente interdisciplinar do tema, tornando qualquer 
pesquisa pobre quando realizada isoladamente, envolve minimamente as seguintes 
áreas de conhecimento: saúde, estatística, física, química, meteorologia e 
geologia. Assim, a interação entre grupos de pesquisa dessas diferentes áreas 
é fundamental e largamente empregada pelos que trabalham com o tema.

Na Universidade de São Paulo (USP), o Laboratório de Análise dos Processos 
Atmosféricos (LAPAt), pertencente ao Instituto de Astronomia, Geofísica e 
Ciências Atmosféricas (IAG) agrega pesquisadores com reconhecida experiência em 
medidas de poluentes atmosféricos (gases e partículas) e devido a isto foi 
procurado em meados de 2010 pela Harvard School of Public Health (HSPH) para 
compor parceria nos estudos ambientais em cidades de países em desenvolvimento. 

O Dr. Majid Ezzati, à época professor da HSPH e atualmente no Imperial College 
London, na Inglaterra, estava à procura de um laboratório com competência 
técnica para analisar, rapidamente e com qualidade, 3000 amostras de teflon 
com MP coletadas em Gana, contatou para isso, o Dr. Paulo Saldiva, professor 
da Faculdade de Medicina da USP, que indicou então o LAPAt para realização das
medidas. Dentro de seis meses, medidas de fluorescência de raios X e 
refletância, foram realizadas em todas as amostras, sendo que destas, as referentes
ao bairro de Nima, em Acra, aproximadamente 800, foram objeto de pesquisa desse
Mestrado.  

Portanto, este estudo, intitulado 
\textit{Análise do Aerossol Atmosférico em Acra, Capital de Gana}, 
faz parte no projeto internacional 
\textit{Energy, air pollution, and health in developing countries}
coordenado pelo Dr. Majid Ezzati, e é um dos ramos da pesquisa desenvolvida em 
Gana, focada exclusivamente na capital do país, e tratará resumidamente 
dos seguintes tópicos:  

\begin{itemize}
  \begin{spacing}{1.0}
  \item Identificar e quantificar as fontes majoritárias de poluição do ar 
        ambiental em Acra;
  \item Aperfeiçoar a metodologia numérica empregada na quantificação de 
        elementos químicos, buscando particularmente a determinação mais 
        consistente das incertezas nos valores medidos;
  \item Apresentar uma proposta de calibração de medidas de BC por refletância
        utilizando o método absoluto Thermal/Optical Transmittance TOT.
  \item Apresentar uma proposta de calibração do equipamento de fluorescência 
        de raios X em dispersão em energia para melhorar a qualidade das medidas;
  \item Comparar resultados dos métodos estatísticos multivariados
        Positive Matrix Factorization (PMF) e Análise de Fatores (AF);
  \item Discussão dos resultados quantitativos face aos aspectos da realidade 
        de Gana, propondo sugestões de como lidar com a questão da qualidade do 
        ar no país, através de medida de possível aplicação.
  \end{spacing}
\end{itemize}
