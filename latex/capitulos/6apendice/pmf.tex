\section{Apêndice II - Software PMF}

Uma vez que apresentamos os conceitos necessários o entendimento do jargão 
usado no método \textit{PMF}, apresentamos as etapas escolhidas para realização 
do ajuste.

\begin{enumerate}
  \item Rodar o \textit{base run} com 20 iterações, encolhendo o número de 
        fatores variando de 3 até 10. 
        Verificar a quantidade de fatores com melhor significado físico. 
        As matrizes soluções $g_{ik}$ e $f_{kj}$ sairão nos arquivos: 
        \textit{profiles.csv} e \textit{contributions.csv}.
  \item Verificar a estabilidade do $Q_{verdadeiro}$ e $Q_{robusto}$ que 
        convergiram, se $Q$ não for estável, então não foi um bom ajuste.
  \item Regressão linear simples das concentrações das espécies ajustadas 
        versus medidas, que devem estar correlacionas. 
        Remover amostras ou aumentar as incertezas das amostras que não foram 
        bem ajustadas. Rodar novamente o \textit{base run}.  
  \item Série temporal das concentrações das espécies ajustadas sobreposta
        as medidas. 
        Identificar pontos não bem ajustados, devido a fontes infrequentes, 
        por exemplo.
        Removê-los ou aumentar a incerteza. Rodar novamente o \textit{base run}.
  \item Análise residual. Verificar se a distribuição do resíduo é normal 
        (usando o \textit{Teste de Kolmogorov–Smirnov}). 
        Quando não é normal há a indicação que o ajusto foi pobre para essa 
        espécie. 
        Pode-se diminuir o peso da espécie na análise aumentando sua incerteza, 
        ou até mesmo remover a espécie da análise.
  \item Verificação se $Q$ é mínimo global ou Local usando 10 valores 
        diferentes de \textit{Random Seed}.
  \item Avaliação da \textit{Ambiguidade Rotacional} usando os 
        gráficos \textit{G-Space}.
  \item Série temporal de $g_{ik}$. Avaliação do significado físico.
\end{enumerate}
