\section{Software PMF EPA 5.0}

Segue-se roteiro desenvolvido para utilização do software PMF EPA 5.0, 
bem como algumas dicas e detalhes destacadas durante o intenso uso desse 
programa realizado nessa pesquisa.  

Sendo $c_{ij}$ e $u_{ij}$ as matrizes de concentração e incerteza, 
respectivamente, ao se iniciar o programa, os seguintes itens devem ser 
satisfeitos antes de rodar análise:

\begin{itemize}
  \begin{spacing}{1.0}
  \item $c_{ij}$ e $u_{ij}$ não devem ter valores negativos;
  \item células vazias não são aceitas;
  \item nomes das colunas(espécies) e linhas(amostras) devem ser únicos;
  \item é ideal (mas não obrigatório) que os dados já estejam classificados 
        pela data em ordem crescente.
  \end{spacing}
\end{itemize}

É importante adequar as incertezas, dimunuindo ou aumetando as mesmas por um 
fator constante, pois o PMF pondera pelas mesmas, assim deve-se diminuir o peso
das medidas com incertezas muito altas. Para tal, é possível usar o 
Signal Noise (S/N), que indica se a variabilidade nas medidas é real ou faz 
parte do ruído dos dados, e é calculado da seguinte forma: 

\begin{itemize}
  \begin{spacing}{1.0}
    \item Se, $c_{ij} >  u_{ij}$, então $ S/N = (c_{ij} - u_{ij})/u_{ij}$.
    \item Se, $c_{ij} <  u_{ij}$, então $ S/N = 0 $.
  \end{spacing}
\end{itemize}

Espécie com a concentração menor que a incerteza, não apresentam ruído, e devem 
ser removidas da análise e espécies com valores muito próximos da incerteza, 
portanto com S/N próximo de zero (< 0,5) também devem ser removidas da 
análise. Espécie nas quais a concentração é pelo menos duas vezes o valor da 
incerteza, isto é, S/N é maior ou igual à 1, são as ideais para análise, 
e não altera-se as incertezas.
Quando S/N está entre 0,5 e 1, diminuímos o peso da espécie aumentando a 
incerteza em um fator 3, diretamento na planilha, ou na interface do PMF EPA.   

Após carregado as planilhas corretamente, já é possível rodar a análise, 
mas é aconselhável inspecionar e até alterar os dados se necessário. Segue
uma lista de itens para se averiguar após carregamento dos dados: 

\begin{itemize}
\begin{spacing}{1.0}
  \item Há relação tipicamente linear entre concentração e incerteza?  
        Se não, investigar o motivo;
  \item Adequação das incertezas baseado no S/N (bem como do conhecimento da 
        espécie e do limite de detecção do método de medida); 
  \item Marcar a massa total como \textit{Total Variable};
  \item Inspecionar gráficos de dispersão entre espécies nas quais se espera 
        correlação, anti-correlação ou não correlação; 
  \item Inspecionar gráficos de séries temporais para identificação de padrões 
        temporais e removeção de \textit{outliers}; 
  \end{spacing}
\end{itemize}

Por questões de reprodutibilidade da análise é importante fixar um seed antes 
de cada rodada. Devido ao inicio randômico é recomendado uma rodada de pelo 
menos 100 iterações para solução final e 10 para as rodadas testes. 

Cada rodada que converge fornece dois valores para a função objeto Q, 
$Q_{verdadeiro}$ e $Q_{robusto}$, sendo o primeiro calculado considerando 
todos os valores, mesmo os marcados para remoção, e o segundo os removendo.
Assim, quando há poucos \textit{outliers}, $Q_{verdadeiro}$ e $Q_{robusto}$ 
são próximos, mas incertezas muitos altas resultam em $Q_{verdadeiro}$ e 
$Q_{robusto}$ similares.

Escolhe-se a solução com menor $Q_{robusto}$, devendo-se testar se o ajuste 
resultou em Q mínimo global e não local. Pode-se rodar o software para 
diferentes valores de seed e acompanhar a estabilidade do Q, que se for mantida, 
significa que o Q é mínimo global.  

Realizadas as adaptações necessárias nos dados, segue-se com a realização 
do ajuste na sequência a seguir:

\begin{enumerate}
\begin{spacing}{1.0}
  \item Rodar o \textit{base run} com 20 iterações, encolhendo o número de 
        fatores variando de 3 até 10.
        Verificar a quantidade de fatores com significado físico;
  \item Verificar a estabilidade dos $Q_{verdadeiro}$ e $Q_{robusto}$ que 
        convergiram, se os Q's não forem estáveis, então não foi um bom ajuste
        (o mínimo não é global);
  \item Regressão linear simples das concentrações das espécies ajustadas 
        pelas medidas, que devem estar correlacionas; 
        Remover amostras ou aumentar as incertezas das amostras que não foram 
        bem ajustadas e rodar novamente o \textit{base run}.  
  \item Série temporal das concentrações das espécies ajustadas sobreposta
        as medidas. 
        Identificar pontos não bem ajustados, devido a fontes infrequentes, 
        por exemplo, remover ou aumentar a incerteza desses dias e rodar novamente 
        o \textit{base run};
  \item Análise residual. Verificar se a distribuição do resíduo é normal, 
        usando o teste de Kolmogorov–Smirnov. Quando não é normal, há a 
        indicação que o ajuste foi pobre para essa espécie. 
        Novamente, diminuir o peso da espécie aumentando sua incerteza, 
        ou remover a espécie da análise.
  \end{spacing}
\end{enumerate}
