\section{Apêndice II - Software PMF}

Além do conhecimento da espécie, o \textbf{Signal Noise (S/N)} é um bom 
indicador de quais amostra deve-se aumentar as incerteza.

\begin{itemize}
  \item Se, $c_{ij} >  u_{ij}$, então $ S/N = (c_{ij} - u_{ij})/u_{ij}$.
  \item Se, $c_{ij} <  u_{ij}$, então $S/N = 0 $.
\end{itemize}

O \textbf{S/N} indica se a variabilidade nas medidas é real ou faz parte do 
ruído dos dados. 
Espécie com a concentração menor que a incerteza, não apresentam ruído, e devem 
ser removidas da análise. Espécies com valores muito próximos da incerteza, 
portanto com $S/N$ próximo de zero (< 0,5) também devem ser removidas da 
análise. 
Espécie nas quais a concentração é pelo menos duas vezes o valor da incerteza, 
isto é, $S/N$ é maior ou igual à 1, são as ideais para análise, 
e não altera-se as incertezas. 

Quando $S/N$ está entre 0,5 e 1, diminuímos o peso da espécie aumentando a 
incerteza em um fator 3.  

Os dados de concentração, $c_{ij}$, e incertezas, $u_{ij}$, devem seguir 
alguns pré-requisitos antes da realização do ajuste:

\begin{itemize}
  \item Não é permitido valores negativos em $c_{ij}$ ou em $u_{ij}$.
  \item células vazias não são aceitas.
  \item nomes das colunas(espécies) e linhas(amostras) devem ser únicos.
  \item é ideal (não obrigatório) que os dados já estejam classificados 
        pela data em ordem crescente.
\end{itemize}

%%%%
\paragraph{Inspeção dos dados de entrada}

É necessário fazer ainda inspeções ou alterações antes de fazer o ajuste.
\begin{itemize}
  \item Espera-se uma relação tipicamente linear entre 
        \textbf{Concentração} $\times$ \textbf{Incerteza}.  
        Investigar o motivo caso isso não ocorra. 
  \item Adequação das incertezas baseado no \textbf{Signal Noise (S/N)} e 
        conhecimento da espécie. 
  \item Marcar a Massa Total como \textbf{Total Variable}
  \item Inspecionar gráficos de dispersão entre espécies nas quais se espera 
        correlação, anti-correlação ou não correlação. 
  \item Inspecionar gráficos de séries temporais, para:
    \begin{itemize}
      \item Identificar padrões temporais em espécies individuais ou em grupo 
            de espécies.
      \item Remover \textbf{outliers} 
    \end{itemize}
\end{itemize}

Devido ao inicio randômico é recomendado uma rodada de pelo menos 100 iterações 
como solução final, escolhendo-se a do menor $Q_{robusto}$, para termos assim 
mais esperança de encontrarmos um $Q$ mínimo global e não local. 
O \textbf{PMF} minimiza a função objeto $Q$ e converge quando encontra um $Q$ 
mínimo local ou global.% o que é local e global?

Como sempre estamos interessados no mínimo global, precisa-se verificar se a 
solução gerou um $Q$ mínimo local ou global. A estratégica para identificar
o tipo de mínimo é rodar o \textbf{PMF} para diferentes valores de 
\textbf{Random Seed} e acompanhar a estabilidade de $Q$.

Uma vez que apresentamos os conceitos necessários o entendimento do jargão 
usado no método \textbf{PMF}, apresentamos as etapas escolhidas para realização 
do ajuste.

\begin{enumerate}
  \item Rodar o \textbf{base run} com 20 iterações, encolhendo o número de 
        fatores variando de 3 até 10. 
        Verificar a quantidade de fatores com melhor significado físico. 
        As matrizes soluções $g_{ik}$ e $f_{kj}$ sairão nos arquivos: 
        \textbf{profiles.csv} e \textbf{contributions.csv}.
  \item Verificar a estabilidade do $Q_{verdadeiro}$ e $Q_{robusto}$ que 
        convergiram, se $Q$ não for estável, então não foi um bom ajuste.
  \item Regressão linear simples das concentrações das espécies ajustadas 
        versus medidas, que devem estar correlacionas. 
        Remover amostras ou aumentar as incertezas das amostras que não foram 
        bem ajustadas. Rodar novamente o \textbf{base run}.  
  \item Série temporal das concentrações das espécies ajustadas sobreposta
        as medidas. 
        Identificar pontos não bem ajustados, devido a fontes infrequentes, 
        por exemplo.
        Removê-los ou aumentar a incerteza. Rodar novamente o \textbf{base run}.
  \item Análise residual. Verificar se a distribuição do resíduo é normal 
        (usando o \textbf{Teste de Kolmogorov–Smirnov}). 
        Quando não é normal há a indicação que o ajusto foi pobre para essa 
        espécie. 
        Pode-se diminuir o peso da espécie na análise aumentando sua incerteza, 
        ou até mesmo remover a espécie da análise.
  \item Verificação se $Q$ é mínimo global ou Local usando 10 valores 
        diferentes de \textbf{Random Seed}.
  \item Avaliação da \textbf{Ambiguidade Rotacional} usando os 
        gráficos \textbf{G-Space}.
  \item Série temporal de $g_{ik}$. Avaliação do significado físico.
\end{enumerate}

Em termos de reprodutibilidade da pesquisa científica o \textbf{PMF EPA 5.0} 
nos fornece os seguintes recursos:

\begin{itemize}
  \item Fixação de \textbf{Random Seed}.%o que significa isso? Porque este detalhe destaca-se dos demais?
  \item Arquivo \textbf{XML} com as configurações usadas na rodada. 
\end{itemize} 
