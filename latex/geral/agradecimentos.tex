\newpage

\begin{center}
  \textbf{Agradecimentos} 
\end{center}

Ao Arthur, meu filho, e sua trupe, Julia, Martim e Francisco, pelos 
constantes momentos de brincadeiras, conforto e paz.

À Thais do Val, minha companheira e amiga, que me acompanhou durante  
essa dura jornada.  

Ao meu pai, José Petrucio, pelo incentivo (e insistência) aos estudos que me deu desde criança, e a minha mãe, Maria Aparecida, por sempre me acolher durante esses anos de USP. 

Ao meu orientador, Américo, pelos ensinamentos científicos e políticos ao longo da última década. 

Ao meus irmãos, Julio Cesar e Bruno, à minha irmã Hilda, que foram compreensíveis e pacientes em minhas ausências nos encontros de família (devido principalmente à pesquisa acadêmica).  

À professora Fátima pelos ensinamentos na sua disciplina e pelas dicas de  
análises estatísticas durante o curso. 

Ao Luis, Atenágoras, Luciana, Mariana e Gregori pela ajuda essencial nas medidas laboratoriais, diurnas e noturnas, além das discussões técnico-científicas do trabalho. 

À Rosana, do LAPAt, pelas irradiações realizadas e conversas sobre a situação 
política da USP.  

Ao Majid, Zheng, Kathie e Raphael por nos acolher na Harvard School of Public 
Health, nos EUA.

Ao Mozart e a Patrícia Ribeiro pelas inúmeras leituras e sugestões no texto. 

Ao Alberto, amigo e pesquisador moçambicano, pela leitura da dissertação e conferências das tabelas e fórmulas. 

Aos amigos que direta ou indiretamente me acompanharam nessa trajetória, entre eles: 
Ana Paula, Nakamura, Ricardo, Augusto, Normando, Deidson, Zé, Carlos, Márcio e Isa.

E a toda comunidade de Software Livre que fornece gratuitamente as ferramentas computacionais utilizadas na produção desse trabalho. 