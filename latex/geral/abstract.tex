\clearpage
\vspace*{10pt}

\begin{center}
  \emph{\begin{large}Abstract\end{large}}\label{abstract}
\vspace{2pt}
\end{center}

\selectlanguage{english}
\noindent

Sub-Saharan Africa (SSA) cities have been intense developing process, 
resulting in generalized economical activities growing, 
specially industrial, as well as increase in the vehicular traffic 
and waste generation, among other changes directly affecting the environment 
and public health. 
Therefore, identifying the air pollution sources is an essential issue 
for public decisions to assure people rights to healthy life.

This Master work has been integrated to an international project 
called Energy, air pollution, and health in developing countries, 
under coordination of Dr. Majid Ezzati, then at the 
Harvard School of Public Health, grouping also researchers 
from the University of Ghana. 
The aims of this project were to evaluate the air pollution level 
at some developing countries, by this time devoted to Accra 
(the capital of Ghana and the main city of SSA), 
and two other cities of Gambia. Since then, no substantive 
study was performed there, connecting air pollution to the regional 
social-economical levels.

This Master project, provided the XRF and Black Carbon determination for all samples of the main project, and gave, else, support for meteorological and receptor modeling issues. But concerning the improving of the study of air quality and sources impact, the work focus Nima town, at Accra, the Ghana Capital. The characterization of species in the local atmospheric aerosol was used in Receptor Models to make factor to sources profile association, and respective apportionment in the local $PM_{2.5}$ and $PM_{10}$.

Between November/2006 and August/2008, 791 filters (sampled for 48 h) collected the local atmospheric aerosol, in two sites separated by 250 m. One was at the main avenue (Nima Road) and other in a residential street (Sam Road).

The $PM_{2.5}$ annual average concentration to 2007 was 61,57 $\pm$ 1,08 $\mu g/m^3$ near to the avenue and 44,91 $\pm$ 1,12 $\mu g/m^3$ in the residential area, surpassing ~5 times the Word Health Organization (WHO) guidelines to annual mean (10 $\mu g/m^3$). Another WHO guideline is not surpass 25 $\mu g/m^3$ in more than 1\% of the samples collected in one year - in each of these sites, 66,5\% and 92\% of the samples are above this limit.

X-Ray Fluorescence (XRF) provided the elemental concentrations, while 
reflectance, inter calibrated by Thermal Optical Transmittance (TOT), gave the Black 
Carbon (BC) levels. 

In this work we performed a methodology for the XRF calibration and for 
the inter calibration between TOT and reflectance, using Matrix Least Square Fitting that gives the uncertainties of 
fitted data and improves the precision of the adjusted values.

Factor Analysis (FA) and Positive Matrix Factorization (PMF) 
enabled the association between source and the determined factors, 
as well as, estimated the sources profile. Local meteorological data, like wind intensity and direction, 
and the identification of some heavy MP emission sources, 
helped the process of factors to sources association. 

During the winter period (January-March), Accra received the 
Harmattan wind, blowing from Sahara deserts, that increased
the concentrations from soil in 10 times. Therefore, the samples from 
this period were separately analyzed, providing better detection 
of the other source by PMF and FA.

The local main source detected by both methods showed coherency: 
sea salt (Na, Cl), soil (Fe, Ti, Mn, Si, Al, Ca, Mg), 
vehicular emissions (BC, Pb, Zn, K) and biomass burning (K, P, S, BC).

The 2000 and 2010 Ghana Population and Housing Census registers 
that more than half the population used wood and charcoal as cooking fuel. 

Only the avenues and main roads are paved, explaining the high 
load of soil in the samples, prevailing also in the $PM_{2.5}$.

The high traffic of old vehicles on Accra, that often occurs in developing countries growing with no planing, intensely contaminates the air, direct or indirectly. There is the motors combustion exhaustion, evaporative emissions, break and tires wear, besides soil particles resuspension and taking part of the secondary aerosol formation.

Reduction of air pollution levels in SSA cities, like Accra, 
requires public actions providing clean energy sources, health care, 
public transportation, urban planing and attention to they impact for 
the poor communities. Relatively simple providences, like roads paving, vegetation 
covering of the land, use of gas for cooking, public transportation, 
should decrease the high air pollution level in those cities.

\par
\vspace{1em}
\noindent\textbf{Keywords:} Atmospheric Aerosols, Sub-Saharan Africa SSA,
X-ray Fluorencence, Black Carbon, Receptor Models.

\selectlanguage{portuguese}
