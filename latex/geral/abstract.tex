\clearpage
\vspace*{10pt}

% Abstract
\begin{center}
  \emph{\begin{large}Abstract\end{large}}\label{abstract}
\vspace{2pt}
\end{center}

\selectlanguage{english}
\noindent

Sub-Saharan Africa (SSA) cities have been intense developing process, 
resulting in intense growing of economical activities in general, 
and industrial in especial, as well as increase in the vehicular traffic 
and waste, among other changes directly affecting the environment 
and public health. 
Therefore, identifying the air pollution sources is an essential issue 
for public decisions assuring people rights to healthy life.

This Master work has been integrated to an international project 
called Energy, air pollution, and health in developing countries, 
under coordination of Dr. Majid Ezzati, then at the 
Harvard School of Public Health, and also grouping researcher 
from the University of Ghana. 
The aims of this project were to evaluate the air pollution level 
at some developing countries, by this time devoted to Accra 
(the capital of Ghana and the main city of SSA), 
and two other cities of Gambia were, since then, no substantive 
study was performed connecting air pollution to the regional 
social-economical levels.

This Master project, in special, improved the studies dedicated 
to Nima area at Accra, the Ghana Capital, characterizing the 
local atmospheric aerosol in 879 samples collected between 
November/2006 and August/2008 (each one sampling during 48h), 
in two sites separated by 250 m. 
Receptor Models provided sources association to the model's 
given factors for $PM_{2.5}$ and $PM_{10}$.
 
The $PM_{2.5}$ annual average concentration was 76,4 (9) $\mu g/m^3$ 
near to the avenue, and 83,3 (18) $\mu g/m^3$ in the residential area, 
surpassing the Word Health Organization (WHO) guide lines (10 $\mu g/m^3$).

The other WHO guide line is not surpassing 25 $\mu g/m^3$ 
in more than 1\% of the samples collected in one year -  
66,5\% and 92\% of the samples, in each site, are above this limit.

X-Ray Fluorescence (XRF) provided the elemental concentrations while 
reflectance and Thermal Optical Transmittance (TOT) gave the Black 
Carbon (BC) levels. 

This work performed a methodology for the XRF calibration and for 
the inter calibration between the reflectance relative measurements 
and the TOT absolute detection of BC, 
using Matrix Least Square Fitting, giving the uncertainties of 
fitted data and improving the adjusted precision.


Factor Analysis (FA) and Positive Matrix Factorization (PMF) 
enabled the association between source and the determined factors, 
as well as, estimated the sources profile. 

Local meteorological data, like wind intensity and direction, 
and the identification of some heavy MP emission sources, 
helped the process of factors to sources association. 

During the winter period (January-March), Accra receives the 
Harmattan wind, blowing from Sahara deserts, that is increase
the concentrations from soil in 10. Therefore, the samples from 
this period were separately analyzed, providing better detection 
of the other source by PMF and FA.

The local main source detected by both methods are coherent: 
sea salt (Na, Cl), soil (mass, Fe, Ti, Mn, Si, Al, Ca, Mg), 
vehicular emissions (BC, Pb, Zn) and biomass burning (K, P, S). 
The 2000 and 2010 Ghana Population and Housing Census registers 
that more than half the population uses wood and charcoal as cooking fuel. 

Only the avenues and main roads are paved, explaining the high 
load of soil in the samples, including in the $PM_{2.5}$.

Reduction of air pollution levels in SSA cities, like Accra, 
requires public actions providing clean energy source, health care, 
public transportation, urban planing and attention to they impact for 
the poor communities. 

Relatively simple providences, like roads paving, vegetation 
covering of the land, use of gas for cooking, public transportation, 
should decrease the high air pollution level in those cities.

\par
\vspace{1em}
\noindent\textbf{Keywords:} Atmospheric Aerosols, Sub-Saharan Africa SSA, X-ray Fluorencence.

\selectlanguage{portuguese}
