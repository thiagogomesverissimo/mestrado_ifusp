\nomenclature{PMF}{
  Positive Matrix Factorizarion}

\nomenclature{AF}{
  Análise Fatorial}

\nomenclature{BC}{
  Black Carbon}

\nomenclature{BS}{
  Black Smoke}

\nomenclature{XRF}{
  Fluorescência de raios X}

\nomenclature{XRF-ED}{
  Fluorescência de raios X dispersivo em energia}

\nomenclature{XRF-WD}{
  Fluorescência de raios X dispersivo em comprimento de onda}

\nomenclature{XRF-EDP}{
   Fluorescência de raios X polarizada}

\nomenclature{XRF-T}{
   Fluorescência de raios X por reflexão total}

\nomenclature{SSA}{
  Sub-Saharan Africa}

\nomenclature{BQM}{
  Balanço Químico de Massa}

\nomenclature{US-EPA}{
  US Environmental Protection Agency - Agência de Proteção Ambiental do Estados
  Unidos}

\nomenclature{EPA-GH}{
  Ghana Environmental Protection Agency - Agência de Proteção Ambiental de Gana}

\nomenclature{NOAA}{
  National Oceanic and Atmospheric Administration - United States Department of Commerce}

\nomenclature{PIB}{
  Produto Interno Bruto}

\nomenclature{RMA}{
  Região Metropolitana de Acra}

\nomenclature{RMSP}{
  Região Metropolitana de São Paulo}

\nomenclature{DVLA}{
  Driver and Vehicle Licensing Authority}

\nomenclature{MP}{
  Material Particulado}

\nomenclature{MP$_{2,5}$}{
  Material Particulado Fino - diâmetro aerodinâmico menor ou igual à 2,5 $ \mu g/m^3$}

\nomenclature{MP$_{2,5-10}$}{
  Material Particulado Grosso - diâmetro aerodinâmico entre 2,5 $ \mu g/m^3$ e 10 $ \mu g/m^3$}

\nomenclature{MP$_{10}$}{
  Material Particulado Inalável - diâmetro aerodinâmico menor ou igual à 10 $ \mu g/m^3$}

\nomenclature{LAPAt}{
  Laboratório de Análise dos Processos Atmosféricos}

\nomenclature{OMS}{
  Organização Mundial de Saúde}

\nomenclature{IAG}{
  Instituto de Astronomia, Geofísica e Ciências Atmosféricas da USP}

\nomenclature{TOR}{
  Thermal Optical Reflectance}

\nomenclature{TOT}{
  Thermal Optical Transmittance}

\nomenclature{CETESB}{
  Companhia Ambiental Do Estado De São Paulo}

\nomenclature{IBGE}{
  Instituto Brasileiro de Geografia e Estatística}

\nomenclature{e-waste}{
  Electronic Waste (depósito de lixo eletrônico)}

\nomenclature{USP}{
  Universidade de São Paulo}

\nomenclature{HSPH}{
  Harvard School of Public Health}

\nomenclature{ECMWF}{
  The European Centre for Medium-Range Weather Forecasts}

\nomenclature{PTFE}{
  Politetrafluoretileno}

\nomenclature{MQM}{
  Mínimos Quadrados Matricial}

\nomenclature{EUA}{
  Estados Unidos da América}

\nomenclature{FMI}{
  Fundo Monetário Internacional}

\nomenclature{OC}{
   Carbono Orgânico}

\nomenclature{EC}{
   Carbono Elementar}

\nomenclature{LD}{
   Limite de Detecção}

\nomenclature{WinQxas}{
   Quantitative X-Ray Analysis System for Windows}

\nomenclature{IAEA}{
   Agência Internacional de Energia Atômica}

\nomenclature{ZCIT}{
   Zona de Convergência Intertropical}

\nomenclature{M71}{
   Black Carbon de Referência Monarch 71}

\nomenclature{GEPA}{
   Grupo de Estudos da Polução do Ar}

\nomenclature{LFA}{
   Laboratório de Física Atmosférica}

\nomenclature{IFUSP}{
   Instituto de Física da USP}

\nomenclature{IPT}{
   Instituto de Pesquisas Tecnológicas do Estado de São Paulo}


