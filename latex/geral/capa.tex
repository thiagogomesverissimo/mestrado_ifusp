\begin{titlepage}
\setlength{\voffset}{0pt}
\setlength{\hoffset}{0pt}
\centering
\Large{Universidade de São Paulo \\
Instituto de Física}

\vspace{\stretch{3}}
\LARGE{\bf (PARCIAL) Composição química e fontes de Material Particulado 
       em Nina, bairro periférico de Acra, capital de Gana.
}

\vspace{\stretch{1}}

\Large{ Thiago Gomes Veríssimo
}

\vspace{\stretch{5}}

\begin{flushright}

\begin{minipage}{.6\textwidth}
\large{Orientador: Américo Sansigolo Kerr
%\\ Co-orientador: Nome do co-orientador, se houver
}
\end{minipage}

\vspace{\stretch{1}}

\begin{minipage}{.6\textwidth}
\rule{\linewidth}{0.5mm}\\
\large{
Dissertação de mestrado apresentada ao Instituto de Física para a obtenção do 
título de Mestre em Física
}

\rule{\linewidth}{0.5mm}
\end{minipage}
\end{flushright}

\vspace{\stretch{1.5}}

\begin{flushleft}

\normalsize
% Comente este bloco no caso de um texto de qualificação ou relatório
% técnico
Comissão examinadora:\\
\vspace{\stretch{.4}}
\hspace{.03\textwidth}\begin{minipage}{.97\textwidth}
Professor 1 \\
Professor 2
\end{minipage}
\end{flushleft}

\vspace{\stretch{1}}

São Paulo\\
2015

\end{titlepage}

% Faz com que a página seguinte sempre seja ímpar (insere pg em branco)
%\cleardoublepage
