\clearpage
\vspace*{10pt}
% Abstract
\begin{center}
  \emph{\begin{large}Resumo\end{large}}\label{resumo}
\vspace{2pt}
\end{center}
\noindent

Cidades dos países da África Subsariana tem passado por um intenso processo de desenvolvimento, implicando em 
crescimento das atividade econômicas e do tráfego de veículos. 
Porém, ainda pouco se conhece acerca dos tipos e características das fontes poluidoras do ar nestas cidades. 
Este trabalho buscou identificar e caracterizar a contribuição de fontes de Material Particulado Ambiental em 
Nima, bairro periférico da capital da Gana, Acra. 
Coletou-se o material particulado entre 2006 e 2008. A composição química foi estimada por Fluorescência de 
Raios X, o Black Carbon por Refletância e \textit{Thermal Optical Transmitance} e a massa total por Análise 
Gravimétrica. Análise de Fatores e \textit{Positive Matrix Factorization} foram usadas para identificação e 
estimativa do peso das fontes. Os níveis de concentração de material particulados são bem maiores que os 
recomendados pela Organização Mundial de Saúde (OMS). 
Sal marinho, solo, emissões veiculares e combustão de biomassa foram as principais fontes encontradas, além 
de poeira do Saara no período do Harmatã (Dezembro a Janeiro). Vale ressaltar que mais que 50\% da população 
usa biomassa para cozimento de alimentos. A redução da poluição do ar em cidades da África Subsariana requerem 
políticas públicas relacionadas ao uso de energia, saúde, transporte e planejamento urbano, com devida atenção 
aos impactos nas comunidades pobres.

\par
\vspace{1em}
\noindent\textbf{Palavras-chave:}  Poluição do ar, África Subsariana, Material Particulado.
\newpage
