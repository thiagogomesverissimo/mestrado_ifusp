\clearpage
\vspace*{10pt}
% Abstract
\begin{center}
  \emph{\begin{large}Resumo\end{large}}\label{resumo}
\vspace{2pt}
\end{center}
\noindent 

VERISSIMO, T., G. 
\textbf{Análise do Aerossol Atmosférico em Acra, Capital de Gana.}
2016 150 p. Dissertação (Mestrado em Física) - Instituto de Física, Universidade
de São Paulo, São Paulo, 2016. \\

Cidades dos países da África Subsariana (SSA) têm passado por um 
intenso processo de urbanização, implicando em crescimento das atividades 
econômicas em geral e industriais em particular, assim como, o aumento do 
tráfego de veículos e da produção de lixo, dentre outras mudanças que 
afetam diretamente o meio ambiente e a saúde dos habitantes. 
Neste cenário, a identificação de fontes poluidoras do ar é essencial 
para a fundamentação de políticas públicas que visam assegurar o direito 
a uma boa qualidade de vida para a população.

Esta pesquisa de Mestrado esteve integrada a um projeto 
internacional denominado "Energy, air pollution, and health 
in developing countries", coordenado pelo Dr. Majid Ezzati, 
a época professor da Harvard School of Public Health, e integrando 
também pesquisadores da Universidade de Gana. 
Este projeto tinha por objetivo fazer avaliações dos níveis de 
poluição do ar em algumas cidades de países em desenvolvimento, 
voltando-se neste caso particular para Acra (capital de Gana e 
maior cidade da SSA), e duas outras cidades de Gambia, 
onde até então inexistiam estudos mais substantivos, 
relacionando-os com as condições socioeconômicas específicas 
das diferentes áreas estudadas.

Este projeto de Mestrado, em especial, aprofundou estudos
realizados exclusivamente na região de Nima, na capital de Gana, Acra. 
Caracterizou-se o aerossol atmosférico local e empregou-se modelos 
receptores para identificar o perfil e contribuição de fontes 
majoritárias do Material Particulado Atmosférico Fino $MP_{2,5}$ 
e Inalável $MP_{10}$. 

Foram coletadas 879 amostras (48 horas) entre novembro de 2006 e 
agosto de 2008 em dois locais distantes 250 metros entre si. 
Um na avenida principal local e outro em uma rua residencial. 

A concentração anual média de $MP_{2,5}$ encontrada na avenida 
foi de 76,4 (9) $\mu g/m^3$ e 83,3 (18) $\mu g/m^3$ na área residencial, 
superando as diretrizes de 10 $\mu g/m^3$ recomendadas pela 
Organização Mundial de Saúde (OMS). 
Outra recomendação da OMS é que a concentração não ultrapasse 
25 $\mu g/m^3$ em mais que 1\% das amostras em um ano. 

A porcentagem de ultrapassagem foi de 66,5\% e 92\% para 
a área residencial e avenida, respectivamente. 

As concentrações químicas elementares foram obtidas por 
Fluorescência de raios X (XRF) e o Black Carbon (BC) por 
refletância e Thermal Optical Transmitance (TOT). 

Neste trabalho desenvolvemos uma metodologia de calibração do XRF 
e de inter calibração entre refletância e TOT, 
baseada em Mínimos Quadrados Matricial, o que ofereceu a 
incerteza dos dados ajustados e boa precisão nos valores 
absolutos de concentrações ajustados.

Análise de Fatores (AF) e Positive Matrix Factorization (PMF) foram utilizadas
para associação entre fonte-fatores, bem como para estimar o perfil destas fontes. 
A avaliação de parâmetros meteorológicos locais, como direção e intensidade 
dos ventos e posicionamento de fontes significativas de emissão de MP 
auxiliaram no processo de associação dos fatores obtidos por esses modelos e 
fontes reais. 

No período do inverno em Gana, um vento  
provindo do deserto do Saara (a nordeste) denominado Harmatão, 
passa por Acra, aumentando em fator 10 a concentração dos poluentes 
relacionados à poeira de solo. Assim, as amostras dos dias de ocorrências 
do Harmatão foram analisadas separadamente, pois dificultavam a 
identificação de outras fontes por PMF e AF.

As fontes majoritárias indicadas por esses dois métodos (AF e PMF), 
mostraram-se concordantes: Sal marinho (Na, Cl), solo (massa, Fe, Ti, Mn, 
Si, Al, Ca, Mg), emissões veiculares (BC, Pb, Zn) e queima de 
biomassa (K, P, S). 

Segundo censos populacionais de Gana realizados em 2000 e 2010, 
mais da metade da população queima lenha ou carvão para cozimento 
de alimentos. 
Só as avenidas principais e rodovias são pavimentadas em Acra, 
o que explica o forte peso da fonte solo, predominando 
inclusive nas análises de $MP_{2,5}$.

A redução da poluição do ar em cidades da SSA, caso de Acra, 
requer políticas públicas relacionadas ao uso de energia, saúde, 
transporte e planejamento urbano, com devida atenção 
aos impactos nas comunidades pobres. 
Medidas como pavimentação das vias, cobertura do solo com vegetação, 
incentivo ao uso de gás de cozinha e incentivo ao transporte público, 
ajudariam a diminuir os altos índices de poluição do ar ambiental nessas cidades.

\par
\vspace{1em}
\noindent\textbf{Palavras-chave:}  Aerossol Atmosférico, África Subsariana, Fluorescência de Raios X.
%\newpage
