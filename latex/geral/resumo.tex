\clearpage
\vspace*{10pt}
% Abstract
\begin{center}
  \emph{\begin{large}Resumo\end{large}}\label{resumo}
\vspace{2pt}
\end{center}
\noindent

Cidades dos países da \textbf{África Subsariana (SSA)} tem passado por um 
intenso processo de desenvolvimento, implicando em crescimento das atividade 
econômicas e industriais, intesificação do tráfego de veículos e aumento da 
produção de lixo, dentre outras mudanças que afetam diretamente o meio 
ambiente e a saúde da população \cite{MONTGOMERY2008}.

Apesar disso, as pesquisas que avaliam os impactos econômicos, sociais e ambientais 
gerados por causa do desenvolvimento nas cidades da \textbf{SSA} são escassas.  
Pouco se conhece acerca dos tipos e características das fontes 
poluidoras do ar nestas cidades \cite{ARKU2008}.

Este trabalho enquadra-se no projeto de pesquisa internacional 
\textbf{Energy, air pollution, and health in developing countries} 
coordenado por pesquisadores da \textit{Harvard School of Public Health} 
nos Estados Unidos, com participação da \textbf{Universidade de Gana}. 
Coordenado pelo \textbf{Dr. Majid Ezzati}, a época professor da 
\textit{Harvard School of Public Health} nos Estados Unidos e 
atualmente no \textbf{Imperial College London}, na Inglaterra.

A pesquisa desse mestrado buscou identificar e caracterizar a 
perfil e contribuição de fontes majoritárias 
de Material Particulado Ambiental Fino ($MP_{2,5}$) e Inalável ($MP_{10}$) em 
Nima, bairro periférico da capital da Gana, Acra. 

Com coletas diárias entre Novembro de 2006 e Agosto de 2008 em 
dois sítios em \textbf{Nima}, 858 amostras foram 

%Neste artigo focalizamos a análise do MP2,5 no Recife, onde a concentração média observada para 309 amostras de 24h, desde junho de 2007 a julho de 2008, foi de 7,3 μg/m³, sendo 0,9 μg/m³ de Carbono Elementar (BC). A concentração química elementar das amostras foram obtidas por fluorescência de raios-X. A análise dos dados por “Positive Matrix Factorization” (PMF) indicou seis fatores preponderantes na área sendo associados a poeira do solo, veículos leves e aerossol marinho, fontes industriais e queima de biomassa e uma fonte com predominância de Cl. A análise pelo método dos Fatores Principais (PCA) determinou seis fatores principais, das quais extraiu-se assinaturas de 5 fontes, associadas a veículos leves, poeira de solo, queima de biomassa, fonte Cl e fonte Ni, utilizadas em um rateio de fontes por “Chemical Mass Balance” (CMB). Os resultados obtidos por esses diferentes métodos mostraram-se consistentes para as fontes majoritárias, apresentando algumas diferenças nas contribuições das fontes em comum e no perfil das demais fontes. Avaliou-se também o perfil de parâmetros meteorológicos locais, como pluviometria, direção e intensidade dos ventos, que foram empregados tanto para auxiliar na classificação das fontes determinadas, quanto para identificar períodos climáticos como peculiaridades que favoreceram níveis de concentração mais elevados. Os níveis de concentração de MP2,5 medidos no período foram aceitáveis, não obstante o elevado estado de urbanização da área metropolitana, provavelmente devido a condições favoráveis à dispersão da poluição do ar.


 A composição química elementar foi estimada por Fluorescência de 
Raios X, o Black Carbon por Refletância e \textit{Thermal Optical Transmitance} e a massa total por Análise 
Gravimétrica%quando tereminar a parte de discussão das metodologias analíticas e nossa contribuição para elas, provavelmente caberá acrescentar algo sobre isso
. Análise de Fatores e \textit{Positive Matrix Factorization} foram usadas para identificação e 
estimativa do peso das fontes. Os níveis de concentração de material particulados são bem maiores que os 
recomendados pela Organização Mundial de Saúde (OMS).%acho que aqui você deve apresentar um dimensionamento numérico rápido, indicando quantas vezes maiores (para a média anual) e dimensionar picos encontrados em relação aos casos episódicos definidos na legislação brasileira, por exemplo.
 
Sal marinho, solo, emissões veiculares e combustão de biomassa foram as principais fontes encontradas, além 
de poeira do Saara no período do Harmatã (Dezembro a Janeiro). Vale ressaltar que mais que 50\% da população 
usa biomassa para cozimento de alimentos. A redução da poluição do ar em cidades da África Subsariana requerem 
políticas públicas relacionadas ao uso de energia, saúde, transporte e planejamento urbano, com devida atenção 
aos impactos nas comunidades pobres.
fontes antropogênica versus fontes naturais (poeira do deserto).

x\% de amostras ultrapassaram os níveis de saúde recomendado.


\par
\vspace{1em}
\noindent\textbf{Palavras-chave:}  Poluição do ar, África Subsariana, Material Particulado.
\newpage
