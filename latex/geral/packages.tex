%%% Pacotes utilizados %%%

% Grande parte deste arquivo foi copiado de:
% http://nelas.github.io/mestre-em-latex/

% Uso de símbolo para graus celsius
\usepackage{gensymb}

% uso de citações tipo \citep
\usepackage{natbib}

\usepackage[font=small,skip=0pt]{caption}
\captionsetup[figure]{font=small,skip=0pt}

\usepackage{subcaption}

% Inserção de figuras ao longo do texto
\usepackage{graphicx} 
\usepackage{wrapfig}

% Fixar as figuras exatamente onde elas foram definidas no códido
\usepackage{float}

% O latex por padrão nãp preenche a página inteira
\usepackage{fullpage}

% Uso de epígrafes nos inícios dos capítulos
\usepackage{epigraph}

% Codificação dos arquivos *.tex
\usepackage[utf8]{inputenc}

% Suporte para português (hifenação e caracteres especiais)
\usepackage[brazil,english,portuguese]{babel}

% Mapear caracteres especiais no PDF
\usepackage{cmap}

% Codificação da fonte
\usepackage[T1]{fontenc}

% Usa a lmodern por padrão (caso cm-super não esteja instalada).
\usepackage{lmodern}

%% Microtipografia
% Utiliza recursos como espaçamento entre letras e entre linhas
\usepackage{microtype}

% Habilita protrusão e expansão, ignorando compatibilidade
\microtypesetup{activate={true,nocompatibility}}

% factor=1100 aumenta a protrusão (default 1000)
% stretch=10 diminui o valor máximo de expansão (default 20)
% shrink=10 diminui o valor máximo de encolhimento (default 20)
\microtypesetup{factor=1100, stretch=10, shrink=10}

% Tracking, espaçamento entre palavras, kerning
\microtypesetup{tracking=true, spacing=true, kerning=true}

% Remover tracking para Small Caps
\SetTracking{encoding={T1}, shape=sc}{0}

% Remove ligaduras para o 'f'. Se necessário, adicionar letras
% separadas por vírgulas
\DisableLigatures[f]{encoding={T1}}

% Documento em versão "final", suporte para outros idiomas
\microtypesetup{final, babel}

% Essencial para colocar funções e outros símbolos matemáticos
\usepackage{amsmath,amssymb,amsfonts,textcomp}

%% Layout
% Margens espelhadas
\usepackage[a4paper,twoside,margin=2cm,footskip=1cm]{geometry}

% Aumenta as margens internas para espiral
\geometry{bindingoffset=10pt}

% Só pra ajustar o layout
\setlength{\marginparwidth}{90pt}

% Para definir espaçamento entre as linhas
\usepackage{setspace}

% Espaçamento
\setstretch{1.5}

% Espaçamento do texto para o frame
\setlength{\fboxsep}{1em}

% Faz com que as margens tenham o mesmo tamanho horizontalmente
%\geometry{hcentering}

% Suporte a cores
\usepackage{color}

% Os argumentos declaram nomes novos, como Cyan e Crimson
\usepackage[usenames,dvipsnames,svgnames]{xcolor}

% Criar ambientes com 2 ou mais colunas
\usepackage{multicol}

% Tabelas com qualidade de publicação
\usepackage{booktabs}

% Notas de rodapé
\usepackage{footnote}

% Formatar as citações no texto e a lista de referências
\usepackage{natbib}

% Lista de Abreviaturas
\usepackage[notintoc,portuguese]{nomencl}
\makenomenclature
\renewcommand{\nomname}{Lista de Abreviaturas}

% Links do sumário para caṕítulos, referências e figuras
\usepackage{hyperref}

% Configurações dos links e metadados do PDF a ser gerado
\hypersetup{colorlinks=true, 
            linkcolor=black, 
            citecolor=black, 
            filecolor=black, 
            urlcolor=black,
            pdfauthor={Thiago Gomes Verissimo},
            pdftitle={Projeto de Mestrado IFUSP 2015},
            pdfsubject={Estudo da poluição do rr em Acra (Capital de Gana)},
            pdfkeywords={Gana, Poluição do Ar, África, Material Particulado},
            pdfproducer={LaTeX},
            pdfcreator={pdfTeX}}

% Adicionar bibliografia, índice e conteúdo na Tabela de conteúdo
% Não inclui lista de tabelas: notlot
% Não incluir lista de figuras: notlof
\usepackage[nottoc]{tocbibind}

%%%%%%%%%%% Remover linhas abaixo:

% Criar figura dividida em subfiguras
%\usepackage{subfig}
%\captionsetup[subfigure]{style=default, margin=0pt, parskip=0pt, hangindent=0pt, indention=0pt, singlelinecheck=true, labelformat=parens, labelsep=space}

% Caso queira guardar as figuras em uma pasta separada
% (descomente e) defina o caminho para o diretório:
%\graphicspath{{./figuras/}}

% Customizar as legendas de figuras e tabelas
%\usepackage{caption}

% Ative o comando abaixo se quiser colocar figuras de fundo (e.g., capa)
%\usepackage{wallpaper}
% Exemplo para inserir a figura na capa está no arquivo pre.tex (linha 7)
% Ajuste da posição da figura no eixo Y
%\addtolength{\wpYoffset}{-140pt}
% Ajuste da posição da figura no eixo X
%\addtolength{\wpXoffset}{36pt}

%% Tabelas
% Elementos extras para formatação de tabelas
%\usepackage{array}

% Para criar tabelas maiores que uma página
%\usepackage{longtable}

% adicionar tabelas e figuras como landscape
%\usepackage{lscape}



% Notas criadas nas tabelas ficam no fim das tabelas
% \makesavenoteenv{tabular}

% Conta o número de páginas
%\usepackage{lastpage}


%% Pontuação e unidades
% Posicionar inteligentemente a vírgula como separador decimal
%\usepackage{icomma}

% Formatar as unidades com as distâncias corretas
%\usepackage[tight]{units}

%% Cabeçalho e rodapé
% Controlar os cabeçalhos e rodapés
%\usepackage{fancyhdr}
% Usar os estilos do pacote fancyhdr
%\pagestyle{fancy}

%\fancypagestyle{plain}{\fancyhf{}}
% Limpar os campos do cabeçalho atual
%\fancyhead{}
% Número da página do lado esquerdo [L] nas páginas ímpares [O] 
% e do lado direito [R] nas páginas pares [E]
%\fancyhead[LO,RE]{\thepage}
% Nome da seção do lado direito em páginas ímpares
%\fancyhead[RO]{\nouppercase{\rightmark}}
% Nome do capítulo do lado esquerdo em páginas pares
%\fancyhead[LE]{\nouppercase{\leftmark}}
% Limpar os campos do rodapé
%\fancyfoot{}
% Omitir linha de separação entre cabeçalho e conteúdo
%\renewcommand{\headrulewidth}{0pt}
% Omitir linha de separação entre rodapé e conteúdo
%\renewcommand{\footrulewidth}{0pt}
% Altura do cabeçalho
%\headheight 13.6pt

% Dados do projeto
%\newcommand{\nomedoaluno}{Nome Completo do Aluno}
%\newcommand{\titulo}{Título original do projeto}

%% Inserir comentários no texto
% Marcar mudanças e fazer comentários
%\usepackage[margins]{trackchanges}
% Iniciais do autor
%\renewcommand{\initialsTwo}{bcv}
% Notas na margem interna
%\reversemarginpar

%% Comandos customizados

% Espécie e abreviação
%\newcommand{\subde}{\emph{Clypeaster subdepressus}}
%\newcommand{\subsus}{\emph{C.~subdepressus}}

%% Pacotes não implementados
% Para não sobrar espaços em branco estranhos
%\widowpenalty=1000
%\clubpenalty=1000

% Numeração em elementos pré-textuais é opcional (ativada por padrão).
% Para desativá-la comente a linha abaixo.
%\pagestyle{fancy}

% Numeração não deve aparecer na página de rosto.
%\thispagestyle{empty}
%\usepackage{url}
%\usepackage{layout}



%\setlength{\textfloatsep}{0pt}
%\setlength{\floatsep}{0pt}
%\setlength{\intextsep}{0pt}
%\setlength{\belowcaptionskip}{0pt}
%\setlength{\abovecaptionskip}{0pt}
%\setlength\intextsep{0pt}
%\floatsep: space left between floats (12.0pt plus 2.0pt minus 2.0pt).
%\textfloatsep: space between last top float or first bottom float and the text (20.0pt plus 2.0pt minus 4.0pt).
%\intextsep : space left on top and bottom of an in-text float (12.0pt plus 2.0pt minus 2.0pt).
%\dbltextfloatsep is \textfloatsep for 2 column output (20.0pt plus 2.0pt minus 4.0pt).
%\dblfloatsep is \floatsep for 2 column output (12.0pt plus 2.0pt minus 2.0pt).
%\abovecaptionskip: space above caption (10.0pt).

