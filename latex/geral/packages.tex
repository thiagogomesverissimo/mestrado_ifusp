%%% Pacotes utilizados %%%

\usepackage[toc,page]{appendix}

% Modificado a partir de http://nelas.github.io/mestre-em-latex/
% thanks guy!

% Remove 'Capítulo N' no início de cada capítulo
\usepackage{titlesec}
\titleformat{\chapter}[display]
  {\normalfont\bfseries}{}{0pt}{\Large}

% landscape em algumas páginas
\usepackage{pdflscape}

% Legendar tabelas junto com figuras dentro do ambiente minipage 
\usepackage{capt-of}

% Uso de símbolo para graus celsius
\usepackage{gensymb}

% uso de citações tipo \citep, \citet etc
% toda opções de citação: http://merkel.zoneo.net/Latex/natbib.php
\usepackage{natbib}

%
\usepackage[font=small,skip=0pt]{caption}
\captionsetup[figure]{font=small,skip=0pt}

% figuras lado-a-lado com subfigure
\usepackage{subcaption}

% Inserção de figuras ao longo do texto
\usepackage{graphicx} 

% Inserção de figuras com texte envolto
\usepackage{wrapfig}

% Fixar as figuras exatamente onde elas foram definidas no códido. Opção [H].
\usepackage{float}

% O latex por padrão nãp preenche a página inteira
\usepackage{fullpage}

% Uso de epígrafes nos inícios dos capítulos
\usepackage{epigraph}

% Codificação dos arquivos *.tex
\usepackage[utf8x]{inputenc}

% Suporte para português (hifenação e caracteres especiais)
\usepackage[brazil,english,portuguese]{babel}

% Mapear caracteres especiais no PDF
\usepackage{cmap}

% Codificação da fonte
\usepackage[T1]{fontenc}

% Usa a lmodern por padrão (caso cm-super não esteja instalada).
\usepackage{lmodern}

% Essencial para colocar funções e outros símbolos matemáticos
\usepackage{amsmath,amssymb,amsfonts,textcomp}

% Microtipografia. Espaçamento entre letras e entre linhas
\usepackage{microtype}
 % Habilita protrusão e expansão, ignorando compatibilidade
 \microtypesetup{activate={true,nocompatibility}}

 % factor=1100 aumenta a protrusão (default 1000)
 % stretch=10 diminui o valor máximo de expansão (default 20)
 % shrink=10 diminui o valor máximo de encolhimento (default 20)
 \microtypesetup{factor=1100, stretch=10, shrink=10}

 % Tracking, espaçamento entre palavras, kerning
 \microtypesetup{tracking=true, spacing=true, kerning=true}

 % Remover tracking para Small Caps
 \SetTracking{encoding={T1}, shape=sc}{0}

 % Remove ligaduras para o 'f'. Se necessário, adicionar letras
 % separadas por vírgulas
 \DisableLigatures[f]{encoding={T1}}

 % Documento em versão "final", suporte para outros idiomas
 \microtypesetup{final, babel}

% Margens espelhadas
\usepackage[a4paper,twoside,margin=2cm,footskip=1cm]{geometry}
 % Aumenta as margens internas para espiral
 \geometry{bindingoffset=10pt}

 % Só pra ajustar o layout
 \setlength{\marginparwidth}{90pt}

% Espaçamento entre as linhas
\usepackage{setspace}
\setstretch{1.5}
 % Espaçamento do texto para o frame
 %\setlength{\fboxsep}{1em}

% Faz com que as margens tenham o mesmo tamanho horizontalmente
\geometry{hcentering}

% Suporte a cores
\usepackage{color}

% Os argumentos declaram nomes novos, como Cyan e Crimson
\usepackage[usenames,dvipsnames,svgnames]{xcolor}

% Criar ambientes com 2 ou mais colunas
\usepackage{multicol}

% Tabelas com qualidade de publicação
\usepackage{booktabs}

% Notas de rodapé
\usepackage{footnote}

% Lista de Abreviaturas
\usepackage[notintoc,portuguese]{nomencl}
\makenomenclature
\renewcommand{\nomname}{Lista de Abreviaturas}

% Links do sumário para caṕítulos, referências e figuras
\usepackage{hyperref}

% Configurações dos links e metadados do PDF a ser gerado
\hypersetup{colorlinks=true, 
            linkcolor=black, 
            citecolor=black, 
            filecolor=black, 
            urlcolor=black,
            pdfauthor={Thiago Gomes Verissimo},
            pdftitle={Mestrado IFUSP 2016},
            pdfsubject={Análise do Aerossol Atmosférico em Acra, Capital de Gana},
            pdfkeywords={Gana, Poluição do Ar, África, Material Particulado},
            pdfproducer={LaTeX},
            pdfcreator={pdfTeX}}

% Adicionar bibliografia, índice e conteúdo na Tabela de conteúdo
% Não inclui lista de tabelas: notlot
% Não incluir lista de figuras: notlof
\usepackage[nottoc]{tocbibind}

% Criar figura dividida em subfiguras
%\usepackage{subfig}
%\captionsetup[subfigure]{style=default, margin=0pt, parskip=0pt, hangindent=0pt, indention=0pt, singlelinecheck=true, labelformat=parens, labelsep=space}

% Caso queira guardar as figuras em uma pasta separada
% (descomente e) defina o caminho para o diretório:
%\graphicspath{{./figuras/}}

% Ative o comando abaixo se quiser colocar figuras de fundo (e.g., capa)
%\usepackage{wallpaper}

% Exemplo para inserir a figura na capa está no arquivo pre.tex (linha 7)
% Ajuste da posição da figura no eixo Y
%\addtolength{\wpYoffset}{-140pt}

% Ajuste da posição da figura no eixo X
%\addtolength{\wpXoffset}{36pt}

%% Tabelas
% Elementos extras para formatação de tabelas
%\usepackage{array}

% Para criar tabelas maiores que uma página
\usepackage{longtable}

% Notas criadas nas tabelas ficam no fim das tabelas
% \makesavenoteenv{tabular}

% Conta o número de páginas
\usepackage{lastpage}

%Posicionar inteligentemente a vírgula como separador decimal
\usepackage{icomma}

% Formatar as unidades com as distâncias corretas
\usepackage[tight]{units}

%
\usepackage{url}

% Numeração não deve aparecer na página de rosto: \thispagestyle{empty}
\usepackage{layout}
