\section{Conclusão}

\begin{frame}
  \frametitle{Conclusão}
  \begin{itemize}
    \item Os níveis de concentração de material particulados são 10 vezes maiores que os recomendados pela OMS;
    \item Identificou-se as seguintes fontes principais:
         \begin{itemize}
           \item Bairro residencial (média 83.28 $ug/m^3$ e 66.5 \% de ultrapassgens): \\
                 \textcolor{blue}{veículo leve (46\%), solo (24\%), veículo pesado (13\%), biomassa (10\%) e mar (6\%)};
           \item Avenida principal (média 76.42 $ug/m^3$ e 92 \% de ultrapassgens): \\
                 \textcolor{blue}{solo (33\%), veículo leve (32\%), veículo pesado(17\%), biomassa(16\%), mar (3\%)}.
         \end{itemize}
  \end{itemize}
\end{frame}


\begin{frame}
  A redução e controle da poluição do ar em cidades da África Subsariana requerem políticas públicas de planejamento urbano, tais como:
  \begin{itemize}
    \item Estratégia de uso de Gás
    \item Pavimentação das vias
    \item Transporte coletivo
  \end{itemize}
com devida atenção aos impactos sociais das medidas.
\end{frame}

\begin{frame}
  \frametitle{Próximas Etapas}
  \begin{itemize}
    \item Avaliação do Material Particulado Grosso $MP_{2.5-10}$;
    \item Inclusão da análise detalhada dos dados meteorológicos;
  \end{itemize}
\end{frame}

\begin{frame}
  Muito Obrigado!
\end{frame}
