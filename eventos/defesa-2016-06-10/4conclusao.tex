\section{Conclusão}

\begin{frame}
  \frametitle{Conclusão}
  \begin{itemize}
    \item Avanços em relação aos métodos analíticos empregados, especialmente quanto às metodologias para calibrá-los e definição das incertezas;
    \item As comparações dos nossos resultados de XRF versus os da US-EPA tiveram ótima concordância;
    \item Com a refletância podesse utilizar os mesmos filtros analisados por XRF;
    \item O uso combinado de AF e PMF permitiu identificar e estimar os perfis de fontes;
    \item Observação no decréscimo na eficiência de 7\% a 9 \% na XRF;
        \item No período do Harmatão as concentrações de $MP_{2,5}$ e $MP_{10}$ elevam-se a um fator 10.
  \end{itemize}
\end{frame}

\begin{frame}
  \frametitle{Conclusão}
  \begin{itemize}
    \item A padrão anual do país não foi ultrapassado em 2007;
    \item As média total foi 5 vezes maior que o padrão anual da OMS;
    \item O padrão diário foi ultrapassado em 16,24 \% dos dias na área residencial e 19,60 \% na avenida. Na OMS, 59,90 \% e 90,95 \%, respectivamente;
    \item Mar (Na, Cl), solo (Mg, Al, Si, Ca, Ti, V, Mn, Fe), emissões veiculares (BC, Zn, K, Pb), queima de biomassa (P,S,K, BC) e queima de lixo sólido e outros materiais a céu aberto (Br,Pb) foram as principais fontes encontras para $MP_{2,5}$;
    \item Mar (Na,Cl), solo (Mg,Al,Si,Ca,Ti,V,Mn,Fe), partículas envelhecidas de emissões veiculares, queima de biomassa e solo (S, K, Zn, Br, Pb + solo) e poeira de estrada (Zn + solo) foram as principais fontes encontras para $MP_{2,5-10}$;
    \item Fonte mar muito bem caracterizada nas análises de AF e PMF;
  \end{itemize}
\end{frame}

\begin{frame}
  \frametitle{Conclusão}
  A redução e controle da poluição do ar em cidades da África Subsariana requerem políticas públicas de planejamento urbano. Entretanto, medidas de fácil implementação como as listadas abaixo ajudariam na redução dos níveis de poluição do ar:
  \begin{itemize}
    \item Estratégia de popularização do uso de gás de cozinha;
    \item Cobertura do solo com vegetação;
    \item Melhora no transporte público coletivo;
    \item Pavimentação das vias (em curto prazo, umedecimento diário);
  \end{itemize}
\end{frame}

\begin{frame}
  \frametitle{Conclusão}
   \begin{tcolorbox}[colback=blue!5,colframe=blue!40!black,title=Conclusão]
    Estes resultados representam um caracterização valiosa do aerossol atmosférico na região de Acra, sendo importante no contexto do projeto principal que foi desenvolvido com a Faculdade de Saúde Pública de Harvard e com a Universidade de Gana.
  \end{tcolorbox}
\end{frame}



\begin{frame}
  \frametitle{Conclusão}
  \begin{center}
    Muito Obrigado!
  \end{center}
\end{frame}
