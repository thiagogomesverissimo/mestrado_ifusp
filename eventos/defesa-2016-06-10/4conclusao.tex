\section{Conclusão}

\begin{frame}
	\frametitle{Conclusão}
	\begin{center}
		\Huge
		Conclusão
	\end{center}
\end{frame}

\begin{frame}
  \frametitle{Conclusão}
  \begin{itemize}
    \item Avanços em relação aos métodos analíticos empregados, especialmente quanto 
    às metodologias para calibrá-los e definição das incertezas;
    \item As comparações dos nossos resultados de XRF com os da US-EPA tiveram ótima concordância;
    \item Com a refletância intercalibrada com TOT foi possível medir BC em todas amostras (PTFE);
    \item No período do Harmatão as concentrações de $MP_{2,5}$ e $MP_{10}$ elevam-se a um fator 10.
    \item O padrão anual do país não foi ultrapassado em 2007;
   % \item A média total foi 6 vezes maior que o padrão anual da OMS;

  \end{itemize}
\end{frame}

\begin{frame}
  \frametitle{Conclusão}
  \begin{itemize}
    \item O padrão diário foi ultrapassado em 16 \% dos dias na área residencial e 19 \% na avenida. Na OMS, 59 \% e 90 \%, respectivamente;
     \item O uso combinado de AF e PMF permitiu identificar e estimar os perfis de fontes;
    \item Veículo, queima de lixo, solo, mar e queima de biomassa foram as principais fontes encontradas para $MP_{2,5}$;
     \item	Solo, solo-estrada (Zn+Pb), mar e partículas envelhecidas foram as principais fontes encontras para $MP_{2,5-10}$;
 %   \item Fonte mar muito bem caracterizada nas análises de AF e PMF;
  \end{itemize}
\end{frame}

\begin{frame}
  \frametitle{Conclusão}
     \begin{tcolorbox}[colback=blue!5,colframe=blue!40!black]
  A redução e controle da poluição do ar em cidades da África Subsariana requerem políticas públicas de planejamento urbano.
\end{tcolorbox}


	   Algumas medidas ajudariam na redução dos níveis de poluição do ar:

  \begin{itemize}
    \item Estratégia de popularização do uso de gás de cozinha;
    \item Cobertura do solo com vegetação;
    \item Pavimentação das vias (em curto prazo, umedecimento diário);
    \item \textcolor{red}{Maior controle da frota veicular};
    \item \textcolor{red}{Melhora no transporte público coletivo};
  \end{itemize}
\end{frame}

%\begin{frame}
%  \frametitle{Conclusão}
%   \begin{tcolorbox}[colback=blue!5,colframe=blue!40!black,title=Conclusão]
%    Estes resultados representam um caracterização valiosa do aerossol atmosférico na região de Acra, sendo importante no contexto do projeto principal que foi desenvolvido com a Faculdade de Saúde Pública de Harvard e com a Universidade de Gana.
%  \end{tcolorbox}
%\end{frame}


\begin{frame}
  \frametitle{Conclusão}
  \begin{center}
    Muito Obrigado!
  \end{center}
\end{frame}
