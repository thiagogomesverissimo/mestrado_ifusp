\section{Experimento}

\begin{frame}
  \frametitle{Amostragem}
  Pontos de amostragem em Nima.
  \begin{figure}[H]
    \centering
    \includegraphics[scale=0.35]{../../../inputs/images/zheng/nima_mapa.pdf}
  \end{figure}
\end{frame}

\begin{frame}
  \frametitle{Meteorologia}
  Distribuição das frequências de direção dos ventos, dados da NOAA.
  \begin{figure}[H]
    \centering
      \includegraphics[scale=0.3]{../../../outputs/ventos_dir.pdf}
      \includegraphics[scale=0.3]{../../../outputs/harmattan2007_2008.pdf}
      \includegraphics[width=1cm]{../../../inputs/images/rosa_ventos}
  \end{figure}
\end{frame}

\begin{frame}
  \frametitle{Análises}
  \begin{itemize}
    \item Gravimétrica (determinação da massa);  
    \item Refletância (determinação do Black Carbon);
    \item Fluorescência de Raios X (determinação da composição química inorgânica);
  \end{itemize}
\end{frame}

\begin{frame}
  \frametitle{Fluorescência de Raios X - \textit{ED-XRF}}
  Modelamento matemático usado na \textit{ED-XRF}:
  \begin{equation}
	  N_{ij} \propto \frac{m_{ij}}{A_i}I_i{\Delta}t_{i}
  \end{equation}
  Onde,  
  \begin{itemize}
    \item $N_{ij}$ = Contagem de fótons na amostra i para o elemento químico j;
    \item $I_{i}$ = Corrente (ampère) na amostra i;
    \item $\Delta t_i$ = Tempo vivo (segundos) que a amostra i foi irradiada;
    \item \textcolor{red}{$m_{ij}$} = Massa (grama) na amostra i para o elemento químico j;
    \item $A_i$ = Área ($cm^2$)irradiada da amostra i.
  \end{itemize}
\end{frame}

\begin{frame}
  \frametitle{Calibração: Ajuste do Fator de Resposta}
  Constante de proporcionalidade: Fator de Resposta:
  \begin{equation}
    R_j = \frac{A_i}{m_{ij}} \frac{N_{ij}}{I_i \Delta t_i}
  \end{equation}
  \begin{figure}[H]
  \centering
    \begin{minipage}[b]{0.40\linewidth}
      \includegraphics[scale=0.25]{../../../outputs/K2014abril}
    \end{minipage}
    \quad
    \begin{minipage}[b]{0.40\linewidth}
      \includegraphics[scale=0.25]{../../../outputs/L2014abril.png}
    \end{minipage}
  \end{figure}
\end{frame}

\begin{frame}
  \frametitle{Erro no Ajuste - Abordagem matricial mínimos quadrados}
   \begin{equation}
     [R] = A[Z]
   \end{equation}
   
   Sendo $\alpha$ o A ajustado, a covariância dos coeficientes $V_{\alpha}$:
   \begin{equation}
     V_{\alpha} = (Z^t V_R^{-1} Z)^{-1}
   \end{equation}

   O ajuste de A fica:
   \begin{equation}
     \alpha = V_{\alpha} Z^t V_R^{-1} R
   \end{equation}

    Calculando-se os novos valores de R a partir de $V_{\alpha}$:
   \begin{equation}
     [R_{adjusted}] = \alpha[Z]
   \end{equation} 

    A incerteza do ajuste é a raiz quadrada da diagonal da matriz de covariância de $R_{adjusted}$:
   \begin{equation}
     COV_{R_{adjusted}} = Z V_{\alpha} Z^t
   \end{equation} 
\end{frame}

\begin{frame}
  \frametitle{Comparação das análises com a da \textbf{EPA}}
  \begin{figure}[H]
   \centering
    \includegraphics[scale=0.42]{../../../inputs/images/zheng/epa_short_example.PDF}
  \end{figure}
  \begin{tiny}
    2900 amostras - \textbf{USP}. 
    95 de controle pela \textbf{EPA} (US Environmental Protection Agency).
  \end{tiny}
\end{frame}
\section{Metodologia}

\begin{frame}
  \frametitle{Modelo receptor}
  \textbf{Modelo Receptor} é uma abordagem matemática para quantificar o efeito das fontes 
  nas amostras. Determinar as fontes a partir do receptor. \\
  \textbf{Análise Multivariada} reduz as dimensões (variáveis) de um conjunto de dados 
  em um conjunto de dados analítico complexo que poderão ser interpretados como 
  tipo de fontes.
\end{frame}

\begin{frame}
  \frametitle{Conservação de massa}
  Fundamentação do modelo receptor: Conservação de massa. \\
  Todos modelos resolvem a mesma equação: 
  \begin{equation}
    x_{ij} = \sum_{p=1}^{P} g_{ip}f_{pj} + \epsilon_{ij}
  \end{equation} 
 
  \begin{itemize}
    \item $x_{ij}$ = concentração na amostra i da espécie j;
    \item $f_{pj}$ = fração da espécie j emitida na fonte p 
                    (perfil da fonte, assinatura da fonte ou \textit{Factor Loadings}); 
    \item $g_{ip}$ = contribuição da fonte p para amostra i (\textit{Factor Score});
    \item $\epsilon$ = Resíduo, depende do modelo empregado.
  \end{itemize}
\end{frame}

\begin{frame}
  \frametitle{Análise de Fatores}
  \begin{equation}
    z_{j} = l_{j1}F_1 + l_{j2}F_2 + l_{j3}F_3 + ... + \epsilon_{ij}
  \end{equation}
  \begin{enumerate}
    \item $z_{j}$ = concentração da espécie j normalizada;
    \item Cálculo da matriz de correlação/covariância
    \item Extração de autovalores e autovetores (ortogonais). 
    \item Transformada Linear nos novos eixos
    \item $l_{jk}$ = contribuição da Fonte $F_k$
  \end{enumerate}

\end{frame}


\begin{frame}
  \frametitle{Positive Matrix Factorizarion}

  Função objeto - Q -  é uma função que precisa ser minimizada. 
 
  \begin{equation}
    Q = \sum_{i=1}^n \sum_{j=1}^m  \left[ \frac{ x_{ij} - \sum_{p=1}^{P} g_{ip}f_{pj}} {u_{ij}} \right] ^2
  \end{equation}

  Diferente da Análise de Fatores, no PMF, a incerteza ($u_{ij}$) entra na conta.
\end{frame}
