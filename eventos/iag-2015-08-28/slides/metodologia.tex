\section{Metodologia}

\begin{frame}
  \frametitle{Modelo receptor}
  \textbf{Modelo Receptor} é uma abordagem matemática para quantificar o efeito das fontes 
  nas amostras. Determinar as fontes a partir do receptor. \\
  \textbf{Análise Multivariada} reduz as dimensões (variáveis) de um conjunto de dados 
  em um conjunto de dados analítico complexo que poderão ser interpretados como 
  tipo de fontes.
\end{frame}

\begin{frame}
  \frametitle{Conservação de massa}
  Fundamentação do modelo receptor: Conservação de massa. \\
  Todos modelos resolvem a mesma equação: 
  \begin{equation}
    x_{ij} = \sum_{p=1}^{P} g_{ip}f_{pj} + \epsilon_{ij}
  \end{equation} 
 
  \begin{itemize}
    \item $x_{ij}$ = concentração na amostra i da espécie j;
    \item $f_{pj}$ = fração da espécie j emitida na fonte p 
                    (perfil da fonte, assinatura da fonte ou \textit{Factor Loadings}); 
    \item $g_{ip}$ = contribuição da fonte p para amostra i (\textit{Factor Score});
    \item $\epsilon$ = Resíduo, depende do modelo empregado.
  \end{itemize}
\end{frame}

\begin{frame}
  \frametitle{Análise de Fatores}
  \begin{equation}
    z_{j} = l_{j1}F_1 + l_{j2}F_2 + l_{j3}F_3 + ... + \epsilon_{ij}
  \end{equation}
  \begin{enumerate}
    \item $z_{j}$ = concentração da espécie j normalizada;
    \item Cálculo da matriz de correlação/covariância
    \item Extração de autovalores e autovetores (ortogonais). 
    \item Transformada Linear nos novos eixos
    \item $l_{jk}$ = contribuição da Fonte $F_k$
  \end{enumerate}

\end{frame}


\begin{frame}
  \frametitle{Positive Matrix Factorizarion}

  Função objeto - Q -  é uma função que precisa ser minimizada. 
 
  \begin{equation}
    Q = \sum_{i=1}^n \sum_{j=1}^m  \left[ \frac{ x_{ij} - \sum_{p=1}^{P} g_{ip}f_{pj}} {u_{ij}} \right] ^2
  \end{equation}

  Diferente da Análise de Fatores, no PMF, a incerteza ($u_{ij}$) entra na conta.
\end{frame}
